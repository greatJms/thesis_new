\begin{figure}
	\centering

	\begin{multicols}{2}
\begin{subfigure}[t]{1.5\columnwidth}
	\begin{multicols}{2}
	\begin{lstlisting}[basicstyle=\small\ttfamily]
class A{} class B{}
class C{
  data = new Object;//O
  set(Object e){
    data = e;
  }
  get(){
    return data;
  }
}
class F{
  foo(){
    c1 = new C();//C1
    c2 = new C();//C2

    c1.set(new A());//A1
    c2.set(new B());//B1
    a = (A)c1.get();//query1
    b = (B)c2.get();//query2
    }
  }
}
main(){
  f1 = new F(); //F1
  f1.foo();
  f2 = new F(); //F2
  f2.foo();
}

\end{lstlisting}
\end{multicols}
\caption{Example code}
\label{fig:example}
\end{subfigure}
\begin{flushright}
\begin{subfigure}[t]{0.6\columnwidth}	
	\begin{center}
		\resizebox{0.8\columnwidth}{!}{
			\begin{tikzpicture}
			\node [shape=circle,draw=black] (f1) {{\tt
                            F1}};
			\node [shape=circle,draw=black,right = 0.5cm
                        of f1] (f2) {{\tt F2}};

			 \node [shape=circle,draw=black, below = 0.4cm of f1] (b1) {{\tt B1}};
			 \node [shape=circle,draw=black, left =
                         0.4cm of b1] (a1) {{\tt A1}};

			 \node [shape=circle,draw=black, below =
                         0.4cm of f2] (c1) {{\tt C1}};				
			 \node [shape=circle,draw=black, right =
                         0.4cm of c1] (c2) {{\tt C2}};

			\node [shape=circle,draw=black, below right =
                        0.4cm and   0.1cm of c1] (o) {{\tt O}};			

			\path [line] (f1) -- (c1);
			\path [line] (f1) -- (c2);

			\path [line] (f2) -- (c1);
			\path [line] (f2) -- (c2);

			\path [line] (c1) -- (o);
			\path [line] (c2) -- (o);
				
			\path [line] (f1) -- (a1);
			\path [line] (f1) -- (b1);

			\path [line] (f2) -- (a1);
			\path [line] (f2) -- (b1);
			\end{tikzpicture}
		}
	\end{center}
	\caption{\centering Object allocation graph}
        \label{fig:oag}
\end{subfigure}
~\\~\\~\\
\begin{subfigure}[t]{0.6\columnwidth}	
	\begin{center}
		\resizebox{0.6\columnwidth}{!}{
			\begin{tikzpicture}
				\node [block3] (n1) {{\tt\Large [0,$\infty$],[2,$\infty$]}};
				\node [block3,fill = gray,below = 0.8cm of n1] (n2) {{\tt\Large [0,$\infty$],[0,$\infty$]}};
				\node [block3,below = 0.8cm of n2] (n3) {{\tt\Large [2,$\infty$],[0,$\infty$]}};
				% \node [onlyText,left =0.5cm of n2]
                                % (lbr) {{\tt \Huge 1obj : \{ }};
				% \node [onlyText,right = -0.7cm of n2] (rbr) {{\tt\Huge \} }};
				\path [line] (n1) -> (n2);
				\path [line] (n2) -> (n3);
			\end{tikzpicture}
		}
	\end{center}
	\caption{\centering Object-sensitivity heuristic}
        \label{fig:oag_heuristic}
\end{subfigure}
% \begin{subfigure}[t]{0.6\columnwidth}	
% 	\begin{center}
% 		\resizebox{0.8\columnwidth}{!}{
% 			\begin{tikzpicture}
% 			\node [shape=circle,draw=black] (f1) {{\tt
%                             F1}};
% 			\node [shape=circle,draw=black,below = 0.4cm
%                         of f1] (f2) {{\tt F2}};
% 			\node [shape=circle,draw=black, right =
%                         0.4cm of f1] (c1) {{\tt C1}};				
% 			\node [shape=circle,draw=black, right =
%                         0.4cm of f2] (c2) {{\tt C2}};
% 			\node [shape=circle,draw=black, below right =
%                         0.1cm and   0.4cm of c1] (o) {{\tt O1}};			
% 			\node [shape=circle,draw=black, left =
%                         0.4cm of f1] (a1) {{\tt A1}};
% 			\node [shape=circle,draw=black, left = 0.4cm of f2] (b1) {{\tt B1}};
% 			\path [line] (f1) -- (c1);
% 			\path [line] (f1) -- (c2);
% 			\path [line] (f2) -- (c1);
% 			\path [line] (f2) -- (c2);
% 			\path [line] (c1) -- (o);
% 			\path [line] (c2) -- (o);
% 			\path [line] (f1) -- (a1);
% 			\path [line] (f1) -- (b1);
% 			\path [line] (f2) -- (a1);
% 			\path [line] (f2) -- (b1);
% 			\end{tikzpicture}
% 		}
% 	\end{center}
% 	\caption{\centering Object allocation graph}
%         \label{fig:oag}
% \end{subfigure}	
\end{flushright}
\end{multicols}
\caption{Example to illustrate an graph-based object-sensitivity heuristic}
\end{figure}


\section{Overview}
In this section, we describe how our graph-based heuristic works with an example program.
% Figure~\ref{fig:oag_heuristic} presents an example of our graph based
% object-sensitivity heuristic for object allocation graph.


%\subsection{Example program}
\subsection{Example Description}
\begin{comment}
Fig~\ref{fig:example} shows an example program to illustrates a
graph-based object-sensitivity heuristics which can be obtained from our framework.
For this example, we use allocation-site based heap abstraction that
present each objects with their allocation-site.
This example has a class {\tt C} which provides getter and setter method
to manipulate its field {\tt data}.
A class {\tt F} has a method {\tt foo} which allocates two
objects {\tt C1} and {\tt C2} to variable {\tt c1} and {\tt c2}, respectively.
The variables {\tt c1} and {\tt c2} call {\tt set} method with newly
allocated objects {\tt A1} and {\tt B1}, respectively.
There are two queries {\tt query1} and {\tt query2} asking the
safety of the down castings at line 18 and 19 are safe.
Obviously, the queries are safe in the real execution because {\tt get} method returns objects
{\tt A1} and {\tt B1} at line 18 and 19, respectively.
\end{comment}
Fig~\ref{fig:example} is an example program with two queries checking the down-casting safety.
This example has a main method which calls the method {\tt foo} with two different receiver objects {\tt F1} and {\tt F2}. 
A class {\tt C} provides getter and setter methods to manipulate its field {\tt data}.
A class {\tt F} has a method {\tt foo} which allocates two 
objects {\tt C1} and {\tt C2} to variable {\tt c1} and {\tt c2}, respectively.
These variables call {\tt set} method with newly allocated objects {\tt A1} and {\tt B1}.
There are two queries {\tt query1} and {\tt query2} asking the down-casting safety at line 18 and 19.
The safety holds because {\tt get} method returns objects {\tt A1} and {\tt B1} at line 18 and 19, respectively.



\begin{comment}
To prove the queries, we needs to context sensitively analyze the methods
called under receiver objects {\tt C1} and {\tt C2} .
Without context sensitivity, the methods {\tt get} and {\tt
  set} called from receiver
objects {\tt C1} and {\tt C2} are merged; thereby, {\tt get} method
returns both {\tt A} and {\tt B} at line 18 and 19.
Meanwhile, method {\tt foo} called from other objects {\tt C1} and {\tt C2}
need to be analyzed context-insensitively
because the method is not related with the queries;
applying context sensitivity to the method increases analysis
cost without precision gain.
\end{comment}
To prove the queries, we need to context-sensitively analyze the methods called under receiver objects {\tt C1} and {\tt C2}.
Without context sensitivity, an analyzer merges the methods {\tt get} and {\tt set} called from receiver objects {\tt C1} and {\tt C2}; 
eventually, the analyzer misjudges that {\tt get} method can return both {\tt A} and {\tt B} at line 18 and 19, and fails to prove the down-casting safety.
The context-sensitive analysis, however, is not necessary for the method {\tt foo} called from other objects {\tt F1} and {\tt F2}
because {\tt foo} is not related to the queries.
If we apply context sensitivity to this method, it only increases the analysis cost without any precision gain;
thereby, the context-insensitive analysis is enough for the method {\tt foo}.
By selectively applying context sensitive, we can achieve a cost-effective analysis.
%The context-insensitive analysis, however, is enough for the method {\tt foo} called from other objects {\tt C1} and {\tt C2}
%because {\tt foo} is not related to the queries.
%If we apply context sensitivity to this method, it only increases the analysis cost without any precision gain.
%Therefore, it requires a selective context-sensitive analysis to prove the queries in a cost-effective manner.

%\subsection{Overview of how our graph-based heuristic works}
\subsection{How Our Heuristic Works}
\begin{comment}
Fig~\ref{fig:oag_heuristic} is an example of graph-based 
object-sensitivity heuristic for OAG which
can be learned from our framework.
The heuristic is a singleton set of a feature which is a list that has a center
node (e.g., gray colored node) and each node contains a
pair of intervals.
Note that a heuristic can contain lots of features; our actual learned object-sensitivity heuristic contains 195 features.
In each node, the former interval present the
number of incoming edges while the other one corresponds to the number
of outgoing edges.
The feature presents a property of nodes in OAG that
have a predecessor having at least 2 outgoing edges and a successor
with at least 2 incoming edges.
\end{comment}
Our framework can automatically learn such heuristics, and
Fig~\ref{fig:oag_heuristic} illustrates a graph-based object-sensitivity heuristic example for OAG which our framework can learn.
%The heuristic is a singleton set of a feature list of which elements are a target node (e.g., gray colored node) and its neighbor nodes. 
It is a singleton set of a feature list of which elements are a target node (e.g., gray colored node) and its neighbor nodes. 
Note that a heuristic can contain a number of features; our learned object-sensitivity heuristic used in our evaluation contains 195 features.
In each node, the former interval presents the number of incoming edges while the other corresponds to the number of outgoing edges.
Fig~\ref{fig:oag_heuristic} presents nodes, which have a predecessor with at least 2 outgoing edges and a successor
with at least 2 incoming edges, in OAG.

\begin{comment}
The heuristic takes an OAG, which is obtained from a cheap
pre-analysis (e.g., context-insensitive analysis), and build an object-sensitivity strategy for the main analysis.
Fig~\ref{fig:oag} presents the object
allocation graph of the example program.
Object allocation graph connects objects with their allocator objects.
For example, the two objects {\tt
F1} and {\tt F2} have edges toward the objects {\tt A1, B1, C1,} and
{\tt C2} because the four objects are allocated in the method {\tt
  foo} with receiver objects {\tt F1} and {\tt F2}.
OAG has been considered as a good program representation to design heuristics for object-sensitivity that previous graph-based heuristics investigate the graph and figure out
objects that should be analyzed precisely~\cite{Tan2017} or methods that need to be analyzed coarsely~\cite{Li2018b}.
In Fig~\ref{fig:oag}, the nodes {\tt C1} and {\tt C2} belong to our
feature because they have a successor (e.g., {\tt O}) which has two incoming
edges and have a predecessor (e.g., {\tt F1} or {\tt F2}) that has
four outgoing edges. Other nodes do not belong to the feature because
they do not have such successor or predecessor.
The heuristic, then, build an analysis strategy for the main analysis:
methods called under receiver objects {\tt C1} and {\tt C2} are analyzed with 1-object-sensitivity while others are analyzed context insensitively.
In such context abstraction, {\tt get} and {\tt set} methods are separated into two different contexts (e.g., {\tt get[C1]}, {\tt get[C2]}, {\tt set[C1]}, and {\tt set[C2]}); thereby,
{\tt get[C1]} and {\tt get[C2]} return {\tt A1} and {\tt B1} at line 18 and 19,respectively.
As we mentioned above, it is the desirable context abstraction for the example program that proves all the queries.
It separates all the precision critical method calls while merges method calls which not related to the queries.
\end{comment}
To generate an object-sensitivity strategy for the main analysis, 
our learned heuristic takes an object allocation graph obtained from a cheap pre-analysis (e.g., context-insensitive analysis).
The object allocation graph connects the objects with their allocator objects, 
and Fig~\ref{fig:oag} is OAG of the example program.
%The heuristic takes an OAG, which is obtained from a cheap
%pre-analysis (e.g., context-insensitive analysis), and builds an object-sensitivity strategy for the main analysis.
%Object allocation graph connects objects with their allocator objects.
For example, the two objects {\tt F1} and {\tt F2} have edges toward the objects {\tt A1, B1, C1,} and
{\tt C2} because these four objects are allocated in the method {\tt foo} with receiver objects {\tt F1} and {\tt F2}.
OAG has been considered as a good program representation to design heuristics for object-sensitivity;
prior works on graph-based heuristics investigate this graph and figure out the objects that should be analyzed precisely~\cite{Tan2017} or the methods that need coarse analysis~\cite{Li2018b}.
In Fig~\ref{fig:oag}, the nodes {\tt C1} and {\tt C2} are the only nodes which satisfy our learned feature 
because they have a successor (e.g., {\tt O}) with two incoming edges and a predecessor (e.g., {\tt F1} or {\tt F2}) with
four outgoing edges. 
After collecting the nodes belonging to the feature, the heuristic starts to generate an analysis strategy for the main analysis;
the methods called under receiver objects {\tt C1} and {\tt C2} are analyzed with 1-object-sensitivity while others are analyzed context insensitively.
In such context abstraction, {\tt get} and {\tt set} methods are distinguished as two different contexts (e.g., {\tt get[C1]}, {\tt get[C2]}, {\tt set[C1]}, and {\tt set[C2]});
{\tt get[C1]} and {\tt get[C2]} return {\tt A1} and {\tt B1} at line 18 and 19,respectively.
This is a desirably cost-effective context abstraction to prove all queries in the example program.
It distinguishes all precision critical method calls while merging the method calls that are not related to the queries.

\begin{comment}
The performance of a learned heuristic is heavily depends on features it contains.
For example, let's consider a heuristic contains the following feature which takes
off a node from the feature in Fig~\ref{fig:oag_heuristic}:
\begin{center}
\resizebox{0.4\columnwidth}{!}{
	\begin{tikzpicture}
	\node [block3,fill = gray] (n2) {{\tt\Large [0,$\infty$],[0,$\infty$]}};
	\node [block3,right = 0.8cm of n2] (n3) {{\tt\Large [2,$\infty$],[0,$\infty$]}};
	\path [line] (n2) -> (n3);
	\end{tikzpicture}
	}.
\end{center}
From the OAG in Fig~\ref{fig:oag},
Unlike the feature in Fig~\ref{fig:oag_heuristic}, the above feature selects four nodes {\tt C1, C2, F1,} and {\tt F2} from OAG.
Although all the precision critical nodes (e.g., {\tt C1} and {\tt C2}) belong to the feature, it additionally includes {\tt F1} and {\tt F2} which are not related to the queries.
It proves all the queries but pays additional analysis cost than the analysis with the heuristic Fig~\ref{fig:oag_heuristic}
because the method {\tt foo} is separated into two different contexts {\tt foo[F1]} and {\tt foo[F2]}.
The goal of our algorithm is to generate qualified features
that include precision critical nodes while exclude others as much as possible.
\end{comment}
The performance of the main analysis heavily depends on the features in learned heuristics.
For example, let's assume that a heuristic contains the following feature which takes
off a predecessor of a target node from the feature in Fig~\ref{fig:oag_heuristic}:
\begin{center}
\resizebox{0.4\columnwidth}{!}{
	\begin{tikzpicture}
	\node [block3,fill = gray] (n2) {{\tt\Large [0,$\infty$],[0,$\infty$]}};
	\node [block3,right = 0.8cm of n2] (n3) {{\tt\Large [2,$\infty$],[0,$\infty$]}};
	\path [line] (n2) -> (n3);
	\end{tikzpicture}
	}.
\end{center}
Unlike the feature in Fig~\ref{fig:oag_heuristic} which only two nodes ,{\tt C1} and {\tt C2}, satisfy, 
four nodes {\tt C1, C2, F1,} and {\tt F2} hold the above feature.
Because a set of nodes collected still include precision critical nodes, {\tt C1} and {\tt C2}, the above feature is able to prove the queries; 
however, it pays additional analysis costs as it also collects {\tt F1} and {\tt F2} which are not related to the queries.
As such, inappropriately learned heuristics can degrade the performance in costs and even the precision 
of the analysis.
The goal of our algorithm is to generate qualified heuristics that are able to maintain as many precision critical nodes as possible while excluding others that are not.




%%% Local Variables:
%%% mode: latex
%%% TeX-master: "paper"
%%% End:


