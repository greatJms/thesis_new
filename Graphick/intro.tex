% !TEX root = ./paper.tex

\section{Introduction}
\iffalse
Pointer analysis is a fundamental program analysis technique that serves as a
key component of various software engineering tools.  The goal of
pointer analysis is to statically and conservatively estimate heap objects that
pointer variables may refer to at runtime.  The pointer
information is essential for virtually all kinds of program analysis tools, including bug
detectors~\cite{Naik2006,NaikPSG09,Blackshear2015,Sui2014,Livshits2003},
security analyzers~\cite{Avots2005,Arzt2014,Tripp2009,Yan2017,Grech17},
program verifiers~\cite{Fink2008}, symbolic executors~\cite{Kapus2019}, and program
repair tools~\cite{memfix,Gao2015,vfix2019,saver2020}. The success of these tools depends eventually on the precision and scalability of the underlying pointer analysis algorithm.
\fi

Developing a fast and precise pointer analysis requires coming up with good analysis heuristics.
For example, context sensitivity is critical for accurately analyzing object-oriented programs as it distinguishes method's local variables and objects in different calling contexts~\cite{Smaragdakis2015}. In reality, however, it is too expensive to apply deep context sensitivity (e.g. 2-object-sensitivity) to all methods in a nontrivial program.
Therefore practical pointer analysis applies context sensitivity selectively using a context abstraction heuristic that determines the  amount of context sensitivity that each method should receive~\cite{Smaragdakis2014,JeJeChOh17,Li2018a,Lu2019}.
Similarly, the performance of pointer analysis depends heavily on how heap objects are represented~\cite{DBLP:journals/csur/KanvarK16}.
Pointer analysis usually employs allocation-site-based heap abstraction, which models heap objects with their allocation sites.
However, because uniformly applying it to all heap objects is costly, a heap abstraction heuristic can be used to apply it selectively and otherwise use a less precise scheme such as type-based abstraction~\cite{Tan2017}.


\myparagraph{Trend: Graph-based Heuristics}

A recent trend in state-of-the-art pointer analyses is use of graph-based analysis heuristics~\cite{Li2018a,Li2018b,Tan2017,TanLX16,Lu2019,JeOh22}.
These graph-based heuristics commonly work in the following two steps:
(1) they first use a cheap pre-analysis to construct a graph representation of the input program and
(2) they reason about the graph structure to produce a program-specific policy for the main analysis.

For example,
\cite{TanLX16} presented \Bean, which first runs a context-insensitive
pre-analysis to generate the object allocation graph (OAG)
and infers from it a policy  for improving the precision of $k$-object-sensitive analysis.
\cite{Li2018b} proposed \Scaler, which also uses a context-insensitive pre-analysis to derive the object allocation graph and analyzes its structure to identify method calls that are likely to blow up the analysis cost during the 2-object-sensitive analysis.
%With this information, \Scaler~applies 2-object sensitivity selectively in the main analysis.
\cite{Li2018a} presented \Zipper, another graph-based heuristic for context-sensitive analysis.
\Zipper~uses a pre-analysis to generate a so-called precision flow graph (PFG)
and identifies precision-critical method calls that may lose precision significantly if context insensitivity is used.
 \cite{Lu2019} presented
a graph-based heuristic, called \Eagle,~that uses a CFL-reachability-based pre-analysis to find out variables and objects that need
context sensitivity in the main analysis.
\cite{Tan2017} developed \Mahjong, a graph-based heap
abstraction heuristic that first runs a cheap pre-analysis to derive a field points-to graph (FPG) and decides when to merge and differentiate heap objects based on the structure of the points-to graph.

%All of the
%graph-based heuristics above are manually designed with a variety of
%insights by pointer analysis experts.
%
%
%, which reveals the
%contexts to be produced in
%$k$-object-sensitivity~\cite{Milanova2002,Milanova2005,Smaragdakis2011}.
%Using the derived graph, it produces analysis strategies that
%determine the appropriate heap contexts for each object, which
%improves the precision of the main analysis (e.g.,
%2-object-sensitivity).  In like manner, \cite{Li2018b} proposes a
%context-sensitivity heuristic \Scaler~which employs the same
%pre-analysis and the type of graph as mentioned above.  From an
%object-allocation-graph, it figures out the method calls which would
%likely blow up the analysis costs when applying
%$2$-object-sensitivity; with this information, it applies cheap
%alternative contexts for them in the main analysis.  \cite{Li2018a}
%introduces another graph-based context-sensitivity heuristic \Zipper.
%It uses a precision-flow-graph, an extended version of an
%object-flow-graph~\cite{Tonella05}, which can predict the precision
%loss occurred when we apply coarse context-sensitivity.  To obtain
%this graph,~\cite{Li2018a} performs context insensitive analysis as a
%pre-analysis and then figures out precision-critical method calls
%based on the information in the graph.  \cite{Tan2017} designs heap
%abstraction heuristic \Mahjong~from field-points-to-graphs, where the
%nodes are the heap objects.  Field-points-to-graphs represent each
%object's field as a graph of which nodes are connected when a field of
%an object points-to the other objects.  \Mahjong~runs a cheap
%pre-analysis to obtain this graph and merges objects when their
%successor objects have the same type.  \cite{Lu2019} presents
%lightweight pre-analysis corresponding to CFL-reachability formulation
%of $k$-object-sensitivity to find out variables and objects that need
%to be analyzed context-sensitively in the main analysis.  All of the
%graph-based heuristics above are manually designed with a variety of
%insights by pointer analysis experts.

\myparagraph{This Work}

we aim to advance this line of research by automating the process of creating graph-based analysis heuristics for pointer analysis.
While all of the existing graph-based heuristics have been designed manually by analysis experts,
our technique generates such heuristics automatically from a given graph without any human effort, significantly increasing  applicability and accessibility of the emerging and promising approach in pointer analysis.

We achieve this goal by developing (1) a feature language for describing graph structures and (2) an algorithm for learning analysis heuristics in terms of the sentences of the language.
We first present a language for describing structural features of nodes in a graph.
This feature description language is simple and general, allowing it to be reused for various analysis instances (e.g. object sensitivity and heap abstraction).
Second, we present a learning
algorithm that takes training programs (and their graph representations) and produces graph-based heuristics by
automatically discovering features appropriate for the given analysis task.
Compared to prior data-driven static analysis techniques~\cite{JeJeChOh17,JeJeOh18,Singh2018cav,He20pldi},
a salient characteristic of our technique is that it does not require pre-designed, application-specific features; instead, it uses a feature language to generate a proper set of features during the learning process. By contrast, existing learning-based techniques for static analysis need a different set of hand-tuned features for each analysis task.

% The algorithm first figures out
%performance-critical nodes in the graph
%produced by the pre-analysis of the training programs, and it
%formulates the nodes as features using the feature language.  The
%generated features describe the performance-critical nodes of graph,
%which constitute analysis heuristics for the main analysis.


The evaluation results show that our technique is effective and general;
it can automatically produce competitive heuristics for two different analysis instances.
We implemented our approach on top of the \Doop~pointer
analysis framework for Java~\cite{BravenboerS09}.
We used our approach to produce
a object-sensitivity heuristic from the object allocation graph on
which the state-of-the-art object-sensitivity heuristic
\Scaler~\cite{Li2018b} was developed.
Additionally, we learned a heap
abstraction heuristic from the field points-to graph, which is
 used in the state-of-the-art heap abstraction heuristic
\Mahjong~\cite{Tan2017}.  For both instances, our approach
successfully generated high-quality heuristics that are as competitive as
\Scaler~and \Mahjong~in terms of the precision and scalability of the main analysis.
In particular, the generated heuristic by our framework successfully
analyzes large programs which the state-of-the-art heap abstraction
heuristic, \Mahjong, cannot handle within a time budget.

%perform as cost-effectively as the state-of-the-art techniques do.



\myparagraph{Contributions}
In summary, this paper makes the following contributions:
\begin{itemize}
\item We present~\ourtool, a new technique for automatically learning graph-based heuristics for pointer analysis.
Key technical contributions include a feature description language and a learning algorithm, which allow our approach to be generally used for different analysis instances without manually designing application-specific features.
%design a new feature language for describing properties of
%  nodes in graphs. The feature language is simple and general; it can
%  be applied to any graphs.
%\item We develop a learning algorithm that automatically generates
%  features as analysis heuristics. It includes an effective algorithm
%  to find out performance-critical program components that the
%  algorithm is an advance one from~\cite{Liang2011learning}.
\item We demonstrate the effectiveness and generality of our technique in comparison with state-of-the-art heuristics for object sensitivity and heap abstraction.

%
%evaluate our approach for two important analysis tasks, namely context sensitivity and heap abstraction, using real-world pointer analysis algorithms and benchmarks.
%heuristic instances and graphs.
%We show that the learned heuristics are cost-effective.
\end{itemize}



%%% Local Variables:
%%% mode: latex
%%% TeX-master: "paper"
%%% End:






























