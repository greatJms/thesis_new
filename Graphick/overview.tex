\section{Overview}
In this section, we illustrate how our approach learns a context sensitivity heuristic from codebases; the learned heuristic generate analysis strategy for unknown programs. 

\subsection{Our learning framework}
\myparagraph{Learning analysis heuristic from codebases}
Below graph depicts how our learning framework:
\begin{center}
\resizebox{0.7\columnwidth}{!}{
	\centering
	\begin{tikzpicture}
	\node [onlyText] (codebases) {\text{codebases $\vec{P}$}};
	\node [onlyText, right =
	 4.4cm of codebases]
	(minimalabs) {\text{minimal abstractions}};
	\node [onlyText, below = 1.4cm of minimalabs,align=center] (graphs) {\text{set of graphs}};
	\node [onlyText, below right = 0.2 and -1.0cm of minimalabs] (mytext) {\text{transforms to} \\ \text{set of graphs}};		
	\node [onlyText, left = 4.4cm of graphs] (heuristic) {\text{heuristic $\heuristic$}};	

	\path [line] (codebases) edge node[above] {$\text{learn minimal abstractions}$} (minimalabs);
	\path [line] (minimalabs) edge node[right] {} (graphs);	
	\path [line] (graphs) edge node[below] {$\text{learn heuristic graphs}$} (heuristic);
	\end{tikzpicture}
}
\end{center}
Suppose program in Fig~\ref{fig:codebase} is a given codebase. The program has an identity method {\tt h}.
Method {\tt f} calls {\tt h} twice with its parameter {\tt a} and a value from users. A method {\tt g} calls method {\tt f} with parameter 8. Method {\tt m} invokes {\tt f} with 8 and {\tt g} twice.
A query in method {\tt f} asks if the value of x is larger than 0. In real execution, it always holds because the value of x is 4 or 8. 
%To prove the query, we need 2-call-site-sensitivity. 
Our framework first learns minimal abstractions of the codebases. 
An abstraction maps each method in the codebase to an abstraction degree (e.g., 0, 1, or 2).
Minimal abstraction maps each method to an as small as possible one while the proves the query. 
In our example, the minimal abstraction is Fig~\ref{fig:minimal}; if we give lower abstraction degree to any of the method, the analysis fails to prove the query.



After it obtains minimal abstractions, our framework transforms each program component to a graph which can be obtained by a cheap analysis.
In our example, we use context-insensitive analysis to get call-graph (e.g., Fig~\ref{fig:callgraph}) and transforms each method in minimal abstraction to corresponding graph like Fig~\ref{fig:setOfGraph}.


From the sets of graphs in minimal abstractions, we transforms each graph to a small abstract graph.
The learned sets of abstract graphs are returned as a call-site-sensitivity heuristic heuristic;
it produces context sensitivity strategy for unknown programs.



%
%\begin{enumerate}
%	\item From the codebases, our framework obtain minimal abstraction for each program.
%	\item It transforms the minimal abstraction to a set of graphs.
%	\item It learns suitable abstract graphs from the extracted ones.
%	\item The learned abstract graphs becomes the heuristic; it generates an abstraction for unknown programs.
%\end{enumerate}

\begin{figure}
	\centering
	\begin{multicols}{2}
	\begin{lstlisting}
	class A{} class B{}
	class C{
	main(){
	  Container c1 = new Container();//C1
	  Container c2 = new Container();//C2
	
	  c1.add(new A());//A1
	  c1.add(new A());//A2
	  
	  c2.add(new B());//B1
	  c2.add(new B());//B2
	
	  Iterator it1 = c1.iterator();
	  Iterator it1 = c2.iterator();
	  A a = (A)it1.next();//query
	  B b = (B)it2.next();//query
	  }
	}
	class Container{
	  Object[] data = new Object[2];//O
	  int size = 0;
	  void add(Object e){
	    data[size++] = e;
	  }
	  Iterator iterator(){
	    return new Iterator();//It
	  } 
	}
	class Iterator{
	  Container container; cursor = 0;
	  Iterator(Container c){
	    container = c;}
	  Object next(){
	    return c.data[cursor++];}
	}
	\end{lstlisting}
	\end{multicols}
	\caption{Example to illustrate heap abstraction and context abstraction}
\end{figure}

\begin{figure}
	\begin{multicols}{2}
	\begin{subfigure}[t]{0.7\columnwidth}
		\begin{center}
			\resizebox{1.\columnwidth}{!}{
				\begin{tikzpicture}
				\node [shape=circle,draw=black] (it) {{\tt It}};

				\node [shape=circle,draw=black, below right = 0.cm and 0.4cm of it] (c1) {{\tt C1}};				
				\node [shape=circle,draw=black, above right = 0.cm and 0.4cm of it] (c2) {{\tt C2}};

				\node [shape=circle,draw=black, right = 1.2cm of it] (o) {{\tt O}};				


				\node [shape=circle,draw=black, below right = 0.0cm and 0.5cm of o] (a1) {{\tt A1}};
				\node [shape=circle,draw=black, below = 0.1cm of a1] (a2) {{\tt A2}};

				\node [shape=circle,draw=black, above right = 0.0cm and 0.5cm of o] (b1) {{\tt B1}};
				\node [shape=circle,draw=black, above = 0.1cm of b1] (b2) {{\tt B2}};




				\path [line] (it) -- (c1);
				\path [line] (it) -- (c2);

				\path [line] (c1) -- (o);
				\path [line] (c2) -- (o);
				
				\path [line] (o) -- (a1);
				\path [line] (o) -- (a2);
				\path [line] (o) -- (b1);
				\path [line] (o) -- (b2);
				\end{tikzpicture}
			}
		\end{center}
		\caption{Points-to graph}
	\end{subfigure}
	
	
	\begin{subfigure}[t]{0.7\columnwidth}
		
		\begin{center}
			\resizebox{1.\columnwidth}{!}{
				\begin{tikzpicture}
				\node [shape=circle,draw=black] (root) {{\tt root}};
						
				\node [shape=circle,draw=black, right =  0.4cm of root] (c2) {{\tt C2}};
				\node [shape=circle,draw=black, above= 0.4cm of c2] (c1) {{\tt C1}};		
				
				\node [shape=circle,draw=black, right = 0.5cm of c2] (o) {{\tt O}};				
				\node [shape=circle,draw=black, right = 0.4cm of c1] (it) {{\tt It}};				


				\node [shape=circle,draw=black, below left = 0.8cm and 0.4cm of root] (a1) {{\tt A1}};		
				\node [shape=circle,draw=black, right = 0.4cm of a1] (a2) {{\tt A2}};
				\node [shape=circle,draw=black, right = 0.4cm of a2] (b1) {{\tt B1}};
				\node [shape=circle,draw=black, right = 0.4cm of b1] (b2) {{\tt B2}};



				\path [line] (root) -- (c1);
				\path [line] (root) -- (c2);

				\path [line] (root) -- (a1);
				\path [line] (root) -- (a2);
				\path [line] (root) -- (b1);
				\path [line] (root) -- (b2);

				\path [line] (c1) -- (it);
				\path [line] (c1) -- (o);
				\path [line] (c2) -- (it);
				\path [line] (c2) -- (o);

				\end{tikzpicture}
			}
		\end{center}
		\caption{Object allocation graph}
	\end{subfigure}	
	\end{multicols}
	\caption{Points-to graph and object allocation graph produced by a cheap analysis (e.g., context-insensitive)}
\end{figure}






\begin{figure}[t]	
	\begin{multicols}{3}
		\begin{subfigure}[t]{1.0\columnwidth}
			\begin{center}
				\begin{lstlisting}[escapeinside={(*}{*)}]
int h(n) {ret n;}
void f(a){
  x = h(a);
  assert(x==4||x==8);//query
  y=h(input());
}
void g(){f(8);}
void m(){
  f(4);
  g();
  g();
}
\end{lstlisting}
\caption{Codebase}
\label{fig:codebase}
\end{center}
\end{subfigure}~\linebreak
		
		%\scalefont{0.9}
		\begin{subfigure}[b]{1.0\columnwidth}
			\begin{center}
				\resizebox{0.4\columnwidth}{!}{
					\begin{tikzpicture}
					\node [onlyText] (main) {\Large \text{Apply 2-call: \{h\}}\\\text{Apply 1-call: \{f\}}\\\text{Apply 0-call: \{g,m\}}};
					\end{tikzpicture}
				}
			\end{center}
			\caption{Minimal abstraction}
			\label{fig:minimal}
		\end{subfigure}
		\vspace{8pt}~\\

					
		\qquad\linebreak%\linebreak
		\begin{subfigure}[b]{1.0\columnwidth}
			\begin{center}
				\resizebox{0.5\columnwidth}{!}{
					\begin{tikzpicture}
					\node [shape=circle,draw=black] (m) {{\tt m}};
					\node [shape=circle,draw=black, right = 0.4cm of m] (f) {{\tt f}};
					\node [shape=circle,draw=black, below = 0.4cm of m] (g) {{\tt g}};
					\node [shape=circle,draw=black, below = 0.4cm of f] (h) {{\tt h}};
					
					\path [line] (m) -- (f);
					\path [line] (f) edge [bend left=20] (h);
					\path [line] (f) edge [bend right=20] (h);
					\path [line] (g) -- (f);
					\path [line] (m) edge [bend left=20] (g);
					\path [line] (m) edge [bend right=20] (g);
					
					\end{tikzpicture}
				}
			\end{center}
			\caption{Call-graph of ci}
			\label{fig:callgraph}
		\end{subfigure}
	
	
	\columnbreak%\linebreak
		\begin{subfigure}[b]{1.0\columnwidth}
			\begin{center}
				\resizebox{0.5\columnwidth}{!}{
					\begin{tikzpicture}
					\node [shape=circle,draw=black] (m) {{\tt 0}};
					\node [shape=circle,draw=black, right = 0.4cm of m] (f) {{\tt 1}};
					\node [shape=circle,draw=black, below = 0.4cm of m] (g) {{\tt 0}};
					\node [shape=circle,draw=black, below = 0.4cm of f] (h) {{\tt 2}};
					
					\path [line] (m) -- (f);
					\path [line] (f) edge [bend left=20] (h);
					\path [line] (f) edge [bend right=20] (h);
					\path [line] (g) -- (f);
					\path [line] (m) edge [bend left=20] (g);
					\path [line] (m) edge [bend right=20] (g);
					
					\end{tikzpicture}
				}
			\end{center}
			\caption{Graph with a minimal abstraction}
			\label{fig:graphmin}
		\end{subfigure}
%			
%	\begin{subfigure}[b]{1.0\columnwidth}
%		\begin{center}
%			\resizebox{0.7\columnwidth}{!}{
%				\begin{tikzpicture}
%
%				\node [shape=circle,draw=black] (m) {};
%				\node [onlyText, below left = -0.02cm and
%				1.5cm of f] (depth2) {\Large \text{Apply 2-call:\{$\qquad\quad$\}}};
%				\node [shape=circle,draw=black, right = 0.4cm of m] (f) {};
%				\node [shape=circle,draw=black, below = 0.4cm of m] (g) {};
%				\node [shape=circle,draw=black, below = 0.4cm of f,fill=gray] (h) {};
%				
%
%
%
%
%				\path [line] (m) -- (f);
%				\path [line] (f) edge [bend left=20] (h);
%				\path [line] (f) edge [bend right=20] (h);
%				\path [line] (g) -- (f);
%				\path [line] (m) edge [bend left=20] (g);
%				\path [line] (m) edge [bend right=20] (g);
%				
%				
%				\node [shape=circle,draw=black,below = 1.cm of m] (m2) {};
%				\node [shape=circle,draw=black, right = 0.4cm of m2,fill=gray] (f2) {};
%				\node [onlyText, below left = 0.05cm and
%				1.3cm of f2] (depth1) {\Large \text{Apply 1-call:\{$\qquad\quad$\}}};
%				\node [shape=circle,draw=black, below = 0.4cm of m2] (g2) {};
%				\node [shape=circle,draw=black, below = 0.4cm of f2] (h2) {};
%				
%				\path [line] (m2) -- (f2);
%				\path [line] (f2) edge [bend left=20] (h2);
%				\path [line] (f2) edge [bend right=20] (h2);
%				\path [line] (g2) -- (f2);
%				\path [line] (m2) edge [bend left=20] (g2);
%				\path [line] (m2) edge [bend right=20] (g2);
%				
%							
%				\end{tikzpicture}
%			}
%		\end{center}
%		\caption{Graphs}
%		\label{fig:setOfGraph}
%	\end{subfigure}~\\
	\vspace{8pt}~\\
	\vspace{8pt}~\\
	

	\begin{subfigure}[b]{1.0\columnwidth}
		\begin{center}
			\resizebox{1\columnwidth}{!}{
				\begin{tikzpicture}

				\node [] (f) {};
				\node [shape=ellipse,draw=black, right = 0.4 of f,fill=gray] (h) {((2,2),(0,0))};
				\node [onlyText, left = -0.1cm of f] (depth2) {\Large \text{Apply 2-call:\{$\qquad\qquad\qquad\quad$\}}};

				\node [shape=ellipse,draw=black, below = .5cm of f,fill=gray] (f2) {((2,2),(2,2))};
				\node [shape=ellipse,draw=black, right = 0.4 of f2] (h2) {((2,2),(0,0))};
				\node [onlyText, left = 0.5cm of f2] (depth1) {\Large \text{Apply 1-call:\{$\qquad\qquad\qquad\qquad\qquad\qquad\quad$\}}};
				\path [line] (f2) edge [] (h2);
%				\node [shape=ellipse, above left = 0.1 and 0.3 of f2] (blank1) {};
%				\node [shape=ellipse, below left = 0.1 and 0.3 of f2] (blank2){};				



%				\path [line] (f) edge [bend left=20] (h);
%				\path [line] (f) edge [bend right=20] (h);


%				\path [line] (f2) edge [bend left=20] (h2);
%				\path [line] (f2) edge [bend right=20] (h2);

%				\path [line] (blank1) edge  (f2);
%				\path [line] (blank2) edge  (f2);
				
				
				
				\end{tikzpicture}
			}
		\end{center}
		\caption{Heuristic graphs}
	\end{subfigure}
	
	\end{multicols}
	\vspace{-1em}
	\caption{How our learning framework obtains a heuristic}
\end{figure}




\myparagraph{How learned heuristic works}
We illustrate how learned heuristic generate analysis strategy with Fig~\ref{fig:heuristicWork}.
Suppose Fig~\ref{fig:unknown} is given unknown program and Fig~\ref{fig:learnedHeuristic} is learned heuristic from codebases.
To obtain analysis strategy, we run context insensitive analysis and obtain call-graph in Fig~\ref{fig:unknwon:callgraph}.
The learned heuristic assign 2-call-site-sensitivity to the method {\tt h} as
the node {\tt h} in Fig~\ref{fig:unknwon:callgraph} has no out going edges and has 2 incoming edges from a node like Apply 2-call in Fig~\ref{fig:learnedHeuristic}.
The heuristic assign 1-call-site-sensitivity to the method {\tt f} as
the node {\tt f} in in Fig~\ref{fig:unknwon:callgraph} has two out going edges to a node and has two incoming edges like Apply 1-call in Fig~\ref{fig:learnedHeuristic}.
Other methods belong to 0-call as node {\tt m} and {\tt d} has different form from abstract graphs in Fig~\ref{fig:learnedHeuristic}.
It generates call-site-sensitivity strategy in Fig~\ref{fig:strategy}. It produces context abstraction in Fig~\ref{fig:unknown:callGraph} and it can prove the queries.




\begin{figure}[t]	
	\begin{multicols}{3}
		\begin{subfigure}[t]{1.0\columnwidth}
			\begin{center}
				\begin{lstlisting}[escapeinside={(*}{*)}]
d(){}
h(v){
  return 2*v;}
f(v){
  r = h(v);
  return h(r);}
main(){
  q1 = f(1);
  q2 = f(2);
  assert(q1 == 4);//query1
  assert(q2 == 8);//query2
  d();
  d();
  d();
}
				\end{lstlisting}
				\caption{Unknown program}
				\label{fig:unknown}
			\end{center}
		\end{subfigure}
			
			\columnbreak%\linebreak
		
		%\scalefont{0.9}
%		\begin{subfigure}[b]{1.0\columnwidth}
%			\begin{center}
%				\resizebox{0.4\columnwidth}{!}{
%					\begin{tikzpicture}
%					\node [onlyText] (main) {\Large \text{Apply 2-call: \{h\}}\\\text{Apply 1-call: \{f\}}\\\text{Apply 0-call: \{g,m\}}};
%					\end{tikzpicture}
%				}
%			\end{center}
%			\caption{Minimal abstraction}
%		\end{subfigure}
%		\vspace{8pt}~\\
%		\vspace{8pt}~\\
		\begin{subfigure}[b]{0.8\columnwidth}
			\begin{center}
				\resizebox{1.2\columnwidth}{!}{
					\begin{tikzpicture}
					
					\node [] (f) {};
					\node [shape=ellipse,draw=black, right = 0.4 of f,fill=gray] (h) {((2,2),(0,0))};
					\node [onlyText, left = -0.1cm of f] (depth2) {\Large \text{Apply 2-call:\{$\qquad\qquad\qquad\quad$\}}};
					
					\node [shape=ellipse,draw=black, below = .5cm of f,fill=gray] (f2) {((2,2),(2,2))};
					\node [shape=ellipse,draw=black, right = 0.4 of f2] (h2) {((2,2),(0,0))};
					\node [onlyText, left = 0.5cm of f2] (depth1) {\Large \text{Apply 1-call:\{$\qquad\qquad\qquad\qquad\qquad\qquad\quad$\}}};
					\path [line] (f2) edge [] (h2);
					\end{tikzpicture}
				}
			\end{center}
			\caption{Learned heuristic graphs}
			\label{fig:learnedHeuristic}
		\end{subfigure}		
		\vspace{8pt}~\\
		\vspace{8pt}~\\
				
			
		\qquad\linebreak%\linebreak
		\begin{subfigure}[b]{.7\columnwidth}
			\begin{center}
				\resizebox{.8\columnwidth}{!}{
					\begin{tikzpicture}
					\node [shape=circle,draw=black] (m) {{\tt m}};
					\node [shape=circle,draw=black, right = 0.4cm of m] (f) {{\tt f}};
					\node [shape=circle,draw=black, below = 0.4cm of m] (d) {{\tt d}};
					\node [shape=circle,draw=black, below = 0.4cm of f] (h) {{\tt h}};
					
					\path [line] (m) edge [bend left=20] (f);
					\path [line] (m) edge [bend right=20] (f);
					\path [line] (f) edge [bend left=20] (h);
					\path [line] (f) edge [bend right=20] (h);
					\path [line] (m) edge [bend left=20] (d);
					\path [line] (m) edge [bend right=20] (d);
					\path [line] (m) edge (d);
										
					\end{tikzpicture}
				}
			\end{center}
			\caption{Call-graph of ci}
			\label{fig:unknwon:callgraph}
		\end{subfigure}~\\




		\columnbreak%\linebreak

		\begin{subfigure}[b]{1\columnwidth}
			\begin{center}
				\resizebox{0.5\columnwidth}{!}{
					\begin{tikzpicture}
					\node [onlyText] (main) {\Large \text{Apply 2-call: \{h\}}\\\text{Apply 1-call: \{f\}}\\\text{Apply 0-call: \{d,m\}}};
					\end{tikzpicture}
				}
			\end{center}
			\caption{Generated abstraction}
			\label{fig:strategy}
		\end{subfigure}~\\
		\vspace{8pt}~\\
		

		\begin{subfigure}[b]{1\columnwidth}
			\begin{center}
				\resizebox{.8\columnwidth}{!}{
					\begin{tikzpicture}
					\node [shape=circle,draw=black] (m) {m$[\cdot]$};
					\node [shape=circle,draw=black, right = 0.4cm of m] (f1) {{\tt f$[8]$}};
					\node [shape=circle,draw=black, below = 0.4cm of f1] (f2) {{\tt f$[10]$}};
					\node [shape=circle,draw=black, right = 0.4cm of f1] (h1) {{\tt h$[8,5]$}};
					\node [shape=circle,draw=black, above = 0.4cm of h1] (h11) {{\tt h$[8,6]$}};
					
					\node [shape=circle,draw=black, right = 0.4cm of f2] (h2) {{\tt h$[9,5]$}};
					\node [shape=circle,draw=black, below = 0.4cm of h2] (h22) {{\tt h$[9,6]$}};
					
					\node [shape=circle,draw=black, below = 0.4cm of m] (d) {{\tt d$[\cdot]$}};

					
					\path [line] (m) edge (f1);
					\path [line] (m) edge (f2);					
					\path [line] (f1) edge (h1);
					\path [line] (f2) edge (h2);
					\path [line] (f1) edge (h11);
					\path [line] (f2) edge (h22);
					\path [line] (m) edge [bend left=20] (d);
					\path [line] (m) edge [bend right=20] (d);
					\path [line] (m) edge (d);
					
					\end{tikzpicture}
				}
			\end{center}
			\caption{Call-graph}
			\label{fig:unknown:callGraph}
		\end{subfigure}~\\
		
		

		
	\end{multicols}
	\vspace{-1em}
	\caption{How learned heuristic generates analysis strategy}
	\label{fig:heuristicWork}
\end{figure}


