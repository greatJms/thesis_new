\clearpage
\section{Overview}
In this section, we illustrate overview of our approach with an example. 

\subsection{Selective Heap Abstraction \& Object-sensitivity}
Fig~\ref{fig:example} shows an example code to illustrate heap
abstraction and object-sensitivity heuristics in points-to analysis.
The example program has {\tt Container} and {\tt Iterator} class which
keeps and iterates data, respectively. 
{\tt main} method allocates two containers to {\tt c1} and {\tt
  c2}. {\tt c1} contains two heaps wich are allocated at {\tt A1} and
{\tt A2} while {\tt c2} has two heaps from {\tt B1} and
{\tt B2}.
{\tt it1} and {\tt it2} correspond to iterator of {\tt c1} and {\tt c2}, respectively.
There are two queries {\tt qry1} and {\tt qry2} which are asking the
safety of down castings; the type of a element in {\tt c1} and {\tt
  c2} is {\tt A} and {\tt B}, respectively. 
In real execution, the castings are always safe because the type of
elements in {\tt c1} is {\tt A} while {\tt c2} keeps elements with the
type {\tt B}. 


\myparagraph{2obj \& allocbased}
First, we consider a precise analysis 2-object-sensitive analysis with
allocation site based heap abstraction. 
For array modeling, we use array-insensitivity which is the most
common one~\cite{Smaragdakis2015} that do not distinguish array
locations. In such analysis, we can prove all the safeties.
2-object-sensitivity separates all the method calls from different
base variables. As a result, the variables {\tt
a} and {\tt b} at line 35 and 36 point to {\tt A1, A2} and {\tt B1,
B2}, respectively as the analysis is array-insensitive. Note that if
we use context-insensitive analsysis or type-based heap abstraction,
it is unable to prove the queries. Type-based heap abstraction
would merge the containers {\tt C1} and {\tt C2} into a type {\tt C}
,and context-insensensitive analysis is unable to separate method calls
from base variable {\tt c1} and {\tt c2} or {\tt it1} and {\tt it2}.


\myparagraph{partial-2obj \& alloc-based}
If we selectively apply allocation-based heap abstraction and
2-object-sensitivity, we can enhance scalability while main precision.
Let assume we apply allocation-site based heap abstraction for only
heaps allocated from {\tt C1} and {\tt C2} while other heaps use type
based heap abstraction. 





\myparagraph{Graph and learned heuristic}



%\subsection{Our learning framework}
%\myparagraph{Learning analysis heuristic from codebases}
%Below graph depicts how our learning framework:
%\begin{figure}
%\begin{center}
%\resizebox{0.7\columnwidth}{!}{
%	\centering
%	\begin{tikzpicture}
%	\node [onlyText] (codebases) {\text{codebases $\vec{P}$}};
%	\node [onlyText, right =
%	 4.4cm of codebases]
%	(minimalabs) {\text{minimal abstractions}};
%	\node [onlyText, below = 1.4cm of minimalabs,align=center] (graphs) {\text{set of graphs}};
%	\node [onlyText, below right = 0.2 and -1.0cm of minimalabs] (mytext) {\text{transforms to} \\ \text{set of graphs}};		
%	\node [onlyText, left = 4.4cm of graphs] (heuristic) {\text{heuristic $\heuristic$}};	
%
%	\path [line] (codebases) edge node[above] {$\text{learn minimal abstractions}$} (minimalabs);
%	\path [line] (minimalabs) edge node[right] {} (graphs);	
%	\path [line] (graphs) edge node[below] {$\text{learn heuristic graphs}$} (heuristic);
%	\end{tikzpicture}
%}
%\end{center}
%\caption{Overview}
%\label{fig:overview}
%\end{figure}











\begin{figure}
	\centering

	\begin{multicols}{2}
\begin{subfigure}[t]{1.5\columnwidth}
	\begin{multicols}{2}
	\begin{lstlisting}[basicstyle=\small\ttfamily]
class A{} class B{}
class Container{
  elem = new Object[2];//O1
  size = 0;
  add(Object e){
    data[size++] = e;
  }
  iterator(){
    itr = new Itr(this);//It1
    return itr;
  } 
}
class Itr{
  Container ctner; 
  idx = 0;
  Iterator(Container c){
    ctner = c;
  }
  Object next(){
   return elem.data[idx++];
  }
}
main(){
 c1 = new Container();//C1
 c2 = new Container();//C2

 c1.add(new A());//A1
 c1.add(new A());//A2
  
 c2.add(new B());//B1
 c2.add(new B());//B2

 Itr it1 = c1.iterator();
 Itr it2 = c2.iterator();
 A a = (A)it1.next();//qry1
 B b = (B)it2.next();//qry2
 }
\end{lstlisting}
\end{multicols}
\caption{Example code}
\label{fig:example}
\end{subfigure}
\begin{flushright}
\begin{subfigure}[t]{0.5\columnwidth}
	\begin{center}
		\resizebox{1.\columnwidth}{!}{
			\begin{tikzpicture}
			\node [shape=circle,draw=black] (it) {{\tt It1}};
			\node [shape=circle,draw=black, below right = 0.cm and 0.4cm of it] (c1) {{\tt C1}};				
			\node [shape=circle,draw=black, above right = 0.cm and 0.4cm of it] (c2) {{\tt C2}};
			\node [shape=circle,draw=black, right = 1.2cm of it] (o) {{\tt O1}};				


			\node [shape=circle,draw=black, below right = 0.0cm and 0.5cm of o] (a1) {{\tt A1}};
			\node [shape=circle,draw=black, below = 0.1cm of a1] (a2) {{\tt A2}};

			\node [shape=circle,draw=black, above right = 0.0cm and 0.5cm of o] (b1) {{\tt B1}};
			\node [shape=circle,draw=black, above = 0.1cm of b1] (b2) {{\tt B2}};


			\path [line] (it) -- (c1);
			\path [line] (it) -- (c2);

			\path [line] (c1) -- (o);
			\path [line] (c2) -- (o);
				
			\path [line] (o) -- (a1);
			\path [line] (o) -- (a2);
			\path [line] (o) -- (b1);
			\path [line] (o) -- (b2);
			\end{tikzpicture}
		}
	\end{center}
	\caption{\centering Points-to graph}
\end{subfigure}~\qquad\linebreak\linebreak\linebreak
\begin{subfigure}[t]{0.6\columnwidth}	
	\begin{center}
		\resizebox{1.\columnwidth}{!}{
			\begin{tikzpicture}
			\node [shape=circle,draw=black] (root) {{\tt root}};
					
			\node [shape=circle,draw=black, right =  0.4cm of root] (c2) {{\tt C2}};
			\node [shape=circle,draw=black, above= 0.4cm of c2] (c1) {{\tt C1}};		
			
			\node [shape=circle,draw=black, right = 0.5cm of c2] (o) {{\tt O}};				
			\node [shape=circle,draw=black, right = 0.4cm of c1] (it) {{\tt It}};				

			\node [shape=circle,draw=black, below left = 0.8cm and 0.4cm of root] (a1) {{\tt A1}};		
			\node [shape=circle,draw=black, right = 0.4cm of a1] (a2) {{\tt A2}};
			\node [shape=circle,draw=black, right = 0.4cm of a2] (b1) {{\tt B1}};
			\node [shape=circle,draw=black, right = 0.4cm of b1] (b2) {{\tt B2}};

			\path [line] (root) -- (c1);
			\path [line] (root) -- (c2);
			\path [line] (root) -- (a1);
			\path [line] (root) -- (a2);
			\path [line] (root) -- (b1);
			\path [line] (root) -- (b2);
			\path [line] (c1) -- (it);
			\path [line] (c1) -- (o);
			\path [line] (c2) -- (it);
			\path [line] (c2) -- (o);
			\end{tikzpicture}
		}
	\end{center}
	\caption{\centering Object allocation graph}
\end{subfigure}	
\end{flushright}
\end{multicols}
\caption{Running example and graphs from a cheap analysis (e.g., context-insensitive)}
\end{figure}



Learned heap abstraction heuristic:
		\begin{center}
			\resizebox{0.7\columnwidth}{!}{
				\begin{tikzpicture}
				\node [block3,fill = gray] (n1) {{\tt [0,$\infty$],[0,$\infty$]}};
				\node [block3,right = 0.8cm of n1] (n2) {{\tt [0,$\infty$],[2,$\infty$]}};
				\node [onlyText,left = -0.5cm of n1] (lbr) {{\tt\LARGE \{ }};
				\node [longText,left = -2.6cm of lbr] (alloc) {{\tt\LARGE Alloc based : }};
				\node [onlyText,right = -0.5cm of n2] (rbr) {{\tt\LARGE \} }};
				\path [line] (n1) -- (n2);

				\end{tikzpicture}
			}
		\end{center}


Learned context abstraction heuristic:
		\begin{center}
				\begin{flushleft}
			\resizebox{0.7\columnwidth}{!}{
				\begin{tikzpicture}
				\node [block3] (n1) {{\tt [1,1],[2,2]}};
				\node [block3,fill = gray,right = 0.8cm of n1] (n2) {{\tt [2,2],[0,0]}};
				\node [onlyText,left = -0.5cm of n1] (lbr) {{\tt\LARGE \{ }};
				\node [longText,left = -3.5cm of lbr] (alloc) {{\tt\LARGE 2obj : }};
				\node [onlyText,right = -0.5cm of n2] (rbr) {{\tt\LARGE \} }};
				\path [line] (n1) -- (n2);

				\node [block3,fill = gray,below = 0.4cm of n1] (n11) {{\tt [1,1],[2,2]}};
				\node [block3,right = 0.8cm of n11] (n12) {{\tt [2,2],[0,0]}};
				\node [onlyText,left = -0.5cm of n11] (lbr1) {{\tt\LARGE \{ }};
				\node [longText,left = -3.5cm of lbr1] (alloc1) {{\tt\LARGE 1obj : }};
				\node [onlyText,right = -0.5cm of n12] (rbr1) {{\tt\LARGE \} }};
				\path [line] (n11) -- (n12);

				\end{tikzpicture}
			}
		\end{flushleft}
		\end{center}





%%% Local Variables:
%%% mode: latex
%%% TeX-master: "paper"
%%% End:


