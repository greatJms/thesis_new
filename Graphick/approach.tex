% !TEX root = ./paper.tex

%\clearpage

\section{\textsc{Graphick}}\label{sec:approach}

Now, we present our approach for automatically learning graph-based analysis heuristics.
In Section~\ref{sec:setting}, we define static analyses with $k$-limited abstractions.
Section~\ref{sec:feat-language} presents a feature description language for directed graphs, which is important for the generality and effectiveness of our approach.
In Section~\ref{sec:param-heuristic}, we define a parameterized abstraction heuristic based on the feature language.
Section~\ref{sec:learning-algorithm} presents our algorithm for learning parameters of the heuristic.

\subsection{Static Analyses with $k$-limited Abstractions}\label{sec:setting}

Let us first model a static analysis with $k$-limited abstractions.
Given a program $P$ to analyze, let $\component_P$ be a finite set of program components in $P$.
For example, $\component_P$ may denote the set of methods~\cite{JeJeChOh17} or the set of allocation sites~\cite{Tan2017} in $P$.
We define $\mca_P$ to be the set of abstractions over $\component_P$ as follows:
\[
\abs \in \mca_P = \component_P \to  \{0,1,\dots,k\}.
\]
An abstraction $\abs \in \mca_P$ maps program components to natural numbers between $0$ and $k$.
For example, in a partially context-sensitive analysis, it assigns one of $0,1,\dots, k$ to each method call,
indicating the amount of context sensitivity that each method is allowed to receive during the analysis.
Abstractions are partially ordered as follows:
\[
\abs_1 \sqsubseteq \abs_2 \iff \forall \comp \in \component_P.\; \abs_1(\comp) \le \abs_2(\comp).
\]
We write $\absk$ and $\absz$ for the most precise and least precise abstractions, respectively:
\[
\absk = \lambda \comp \in \component_P.\; k, \qquad
\absz = \lambda \comp \in \component_P.\; 0
\]

We assume that a set $\mbq_P$ of assertions is given together with the program $P$. For instance, $\mbq_P$ may denote the set of all type casts in the program. The goal of static analysis is to prove that assertions in $\mbq_P$ do not fail at runtime.
We model a static analyzer as a function that takes as input an abstraction and
produces a set of proved queries and, as a by-product, a directed graph over program components:
\[
F_P: \mca_P \to \power{\mbq_P}\times \mbg_P
\]
where $\mbg_P$ denotes the set of possible graphs. A graph $G = (N,\edges) \in \mbg_P$ consists of nodes $N = \component_P$ and edges $(\edges) \subseteq \component_P \times \component_P$.
For example, $G$ is the object allocation graph~\cite{Li2018b} or the field points-to graph~\cite{Tan2017} depending on the purpose of the analysis.
We use two projection functions, $\projproved$ and
$\projgraph$, which are used for obtaining the proven queries and the graph, respectively, from the analysis output.

In this chapter, we generally assume  the analysis $F_P$ is monotone with respect to the abstractions in the sense that more refined abstractions imply higher analysis precision:
\begin{equation}\label{eq:mono}
\abs \sqsubseteq \abs' \implies \projproved(F_P(\abs)) \subseteq {\projproved}(F_P(\abs')).
\end{equation}
Many static analysis problems are monotone~\cite{JeJeChOh17,Liang2011learning,Liang2011,Zhang2014,Li2018a,Tan2017} and therefore our approach is directly applicable to them. For non-monotone analyses (e.g. interval analysis with widening~\cite{cha2016learning}), our approach is still applicable in practice but it does not guarantee its theoretical property (Theorem~\ref{THM:REDUCESPACE}).

\subsection{Feature Description Language}\label{sec:feat-language}

Our approach uses a simple and general language for describing properties of nodes in a graph.

\myparagraph{Observation}


Our feature language has been inspired from existing graph-based heuristics.
Existing works~\cite{Li2018a,Li2018b,TanLX16} have demonstrated that the number of incoming and outgoing edges of nodes in
graphs play key roles in designing analysis heuristics.
For example, \cite{Li2018a} identify precision-critical method calls
by figuring out the nodes with multiple incoming edges in precision flow graph (PFG).
%whose nodes represent variables and heaps of programs.
%In the main analysis, \cite{Li2018a} analyze these identified methods context-sensitively
%while the others are analyzed context-insensitively.
Besides the PFG, the number of incoming edges in object allocation graph (OAG) also helps to design
effective analysis heuristics, which is used by both \cite{TanLX16}~and \cite{Li2018b}.
The conventional 2-object-sensitive analysis produces only one heap context for objects when they have only one incoming edge in OAG.
\cite{TanLX16}, however, design an analysis heuristic which assigns alternative multiple heap contexts to these objects for improving precision.
When an object has lots of incoming edges, multiple contexts are applied to the methods called from the object in 2-object-sensitive analysis.
These methods are critical for scalability in pointer analysis, and~\cite{Li2018b} identify these methods to apply alternatively coarse yet cheap contexts to improve the performance in scalability.
Based on these observations,
we designed a feature language that can express various combinations of the number of edges around nodes, successors, and predecessors.

\myparagraph{Formal Definition}
Let $G=(N, \edges)$ be a directed graph over program components, i.e., $N = \component_P$ and $(\edges) \subseteq \component_P \times \component_P$.
Let $\incoming_G(n)$ and $\outgoing_G(n)$ be the numbers of incoming and outgoing edges, respectively, of node $n$ in graph $G$:
\[
\incoming_{G}(n) = |\myset{ p \in N \mid p \edges n }|, \qquad \outgoing_{G}(n) = |\myset{s \in N \mid n \edges s}|.
\]

A {\em feature} in our language denotes a set of nodes. We define a feature $f$ to be a triple:
\[
\feature \in \Feature = \aNode^* \times \aNode \times \aNode^*
\]
where $\aNode$ means abstract nodes:
\[
\begin{array}{rcl}
\myhat{n}, \myhat{p}, \myhat{s} \in \aNode &=& \Itv \times \Itv \\
\Itv &=& \myset{[a,b] \mid a \in \mbn, b \in \mbn \cup \myset{\infty}}
\end{array}
\]
An abstract node $([a,b], [c,d]) \in \aNode$ is a pair of intervals and denotes a set of nodes as follows:
\[
\gamma_G(([a,b], [c,d])) = \myset{n \in N \mid a \le \incoming_G(n) \le b, c \le \outgoing_G(n) \le d}.
\]
We extend the definition to a sequence of abstract nodes ($\aNode^*$).
The empty sequence $\epsilon$ denotes the empty set of nodes.
A non-empty sequence $\seq{(\itv_0, \itv'_0), (\itv_1, \itv'_1), \dots, (\itv_m, \itv'_m)}$
of pairs of intervals denotes sequences of nodes as follows:
\[
\begin{array}{l}
\gamma_G(\seq{(\itv_0, \itv'_0),(\itv_1, \itv'_1),\cdots, (\itv_m, \itv'_m)}) = \\
\qquad\qquad \myset{\seq{n_0,n_1,\dots,n_m} \in N^* \mid n_0 \edges n_1 \edges \cdots \edges n_m, \forall i \in [0,m].\; n_i \in \gamma_G((\itv_i, \itv'_i))}.
\end{array}
\]
Finally, a feature  $(\seq{\myhat{p}_0,\myhat{p}_1,\dots,\myhat{p}_q}, \myhat{n}, \seq{\myhat{s}_0,\myhat{s}_1,\dots,\myhat{s}_r}) \in \Feature$
 denotes a set of nodes in $\gamma_G(\myhat{n})$ whose predecessors and successors are described as
$\seq{\myhat{p}_0,\myhat{p}_1,\dots,\myhat{p}_q}$ and $\seq{\myhat{s}_0,\myhat{s}_1,\dots,\myhat{s}_r}$, respectively:
\[
\begin{array}{l}
 \gamma_G(\seq{\myhat{p}_0,\myhat{p}_1,\dots,\myhat{p}_q}, \myhat{n}, \seq{\myhat{s}_0,\myhat{s}_1,\dots,\myhat{s}_r}) =  \{ n \in \gamma_G(\myhat{n}) \mid
\exists p_0,p_1,\dots,p_q, s_0,s_1,\dots,s_r \in N.\; \\
\qquad
\seq{p_0,p_1,\dots,p_q} \in \gamma_G(\seq{\myhat{p}_1,\myhat{p}_2,\dots,\myhat{p}_q}), p_q \edges n \edges s_0,
\seq{s_0,s_1,\dots,s_r} \in \gamma_G(\seq{\myhat{s}_1,\myhat{s}_2,\dots,\myhat{s}_r}).
\}
\end{array}
\]

%\[
%(\epsilon, ([0,3],[1,\infty]), \seq{([1,2],[1,1])})
%\]

For example, feature $(\epsilon,([0,3],[5,\infty]), \seq{([0,2],[0,0])} )$ describes the set of nodes that have 1) three or less incoming edges and five or more outgoing edges, and 2) a successor node with two or less incoming edges and no outgoing edges.
For another example, the following feature
\[
(\seq{([0,0],[0,5]), ([1,2], [3,\infty])}, ([0,3], [100,\infty), \seq{([1,1], [2,2])})
\]
describes a node $n$ iff 1) $n$ has three or less incoming edges and 100 or more outgoing edges, 2) $n$ has a predecessor $p$  with one or two incoming edges and three or more outgoing edges,  3) $p$ also has a predecessor with no incoming edge and five or less outgoing edges, and 4) $n$ has a successor $s$ with a single incoming edge and two outgoing edges.


\subsection{Parameterized Abstraction Heuristic}\label{sec:param-heuristic}

In our approach, abstraction heuristics work on a graph over program components.
That is, a heuristic $\heuristic$ is a function that takes a graph $G$ for program $P$ and produces an abstraction of $P$:
\[
\heuristic(G): \mbc_P \to \{0,1,\dots,k\}.
\]
The graph $G$ is typically obtained by running an imprecise but fast pre-analysis~\cite{TanLX16,Li2018b,Tan2017,Lu2019}. For example,
it can be obtained by running the analysis $F_P$ with the least precise abstraction:
\[
G = \projgraph(F_P(\absz)).
\]

We define a {\em template} of such heuristics whose behavior is controlled by $k$ parameters $\params = \myvec{\Feat_1, \Feat_2, \dots, \Feat_k}$, where each parameter $\Feat_i \subseteq {\Feature}$ is a set of features in our language.
We define the parameterized heuristic $\heuristic_\params$ as follows:
\[
\heuristic_{\params} (G) = \lambda c \in \mbc_P.\; \left\{
\begin{array}{rcl}
k & & \mbox{if}~c \in \gamma_G(\Feat_k) \\
k-1 & & \mbox{if}~c \in \gamma_G(\Feat_{k-1}) \land c \not\in \gamma_G(\Feat_k) \\
& \cdots& \\
k- i & & \mbox{if}~c \in \gamma_G({\Feat_{k-i}}) \land c \not\in \bigcup_{k \ge j > k-i}  \gamma_G({\Feat_j}) \\
& \cdots & \\
1 & & \mbox{if}~c \in \gamma_G(\Feat_1) \land c\not\in \bigcup_{k \ge j > 1}\gamma_G(\Feat_j) \\
0 & & \mbox{otherwise}
\end{array}
\right.
\]
where $\gamma_G(\Feat_i) = \bigcup_{f \in \Feat_i} \gamma_G(f)$.
Basically, the heuristic $\heuristic_{\params}$ assigns an abstraction degree $j$ to program component $c$ if $c$ is implied by the $j^{\it th}$ parameter $\Feat_j$. If $c$ is implied by multiple parameters, the heuristic assigns the highest abstraction degree among them. For example,  when $c \in \Feat_1$ and $c \in \Feat_2$, we define $\heuristic_{\params}(G)(c) = 2$.






\subsection{Learning Algorithm}\label{sec:learning-algorithm}

Now we present an algorithm for learning parameters $\params = \myvec{\Feat_1, \Feat_2, \dots, \Feat_k}$ from a set $\vec{P} = \myset{P_1, P_2, \dots, P_n}$ of training programs.

\myparagraph{Overall  Process}
Algorithm~\ref{alg:overall-algorithm} describes the overall learning process.
The algorithm takes training programs $\vec{P}$, static analyzer $F$, and the maximum abstraction degree $k$. As output, it produces $k$ parameters $\params = \myvec{\Feat_1, \Feat_2, \dots, \Feat_{k}}$.
%The process consists of computing minimal abstractions for each training program and generating features that best fit them.
Initially, parameters $\params$ are set to empty sets $\myvec{\emptyset,\emptyset,\dots,\emptyset}$ (line 2).
At line 3, the algorithm computes a map $A_\vec{P}$ from programs in $\vec{P}$ to their {\em ideal} abstractions.
For each training program $P_i \in \vec{P}$, $A_\vec{P}(P_i)$ denotes the desired abstraction for $P_i$ that we want our heuristic to produce for $P_i$.
The ideal abstraction is computed by procedure $\LearnMinimalAbstraction$, which is explained shortly.
At line 4, we run a pre-analysis (e.g. $F_P(\absz)$) to transform each training program $P$ into its graph representation:
 $G_\vec{P}$ is a map from programs in $\vec{P}$ to their graph representations.
At lines 5 and 6, the algorithm
uses procedure $\LearnSetOfFeatures$ to learn each parameter $\Feat_i~(1 \le i \le k)$, which denotes the set of nodes in the graphs in $G_\vec{P}$ that should receive the abstraction degree $i$.
Although $\Feat_i$ is obtained iteratively,
there is no dependency between loop iterations and therefore the $k$ different tasks at lines 5 and 6 can be run in parallel to reduce the learning cost.
At line 8, the learned parameters $\myvec{\Feat_1, \Feat_2,\dots, \Feat_k}$ are returned as final output.

\begin{algorithm}[t]
	\caption{Overall learning algorithm}\label{alg:overall-algorithm}
	\begin{algorithmic}[1]
		\Require Training programs $\vec{P}$, static analyzer $F$, maximum abstraction degree $k$
		\Ensure Parameters $\myvec{\Feat_1,\Feat_2,\dots,\Feat_k}$
		\Procedure{Learn}{$\vec{P}, F, k$}
		\State
		$\myvec{\Feat_1,\Feat_2,\dots,\Feat_k} \gets \myvec{\emptyset,\emptyset,\dots,\emptyset}$
		\State
		$A_\vec{P} \gets \lambda P \in \vec{P}. {\LearnMinimalAbstraction}(P,F,k)$ \Comment{minimal abstractions}
		\State
		$G_\vec{P} \gets \lambda P\in \vec{P}.{\projgraph}(F_P(\absz))$ \Comment{graphs from pre-analysis}
		\For{$i=1$ to $k$}
		\State
			$\Feat_i \gets \LearnSetOfFeatures(i,A_\vec{P},G_\vec{P})$
		\EndFor
		\State \Return $\myvec{\Feat_1,\Feat_2,\dots,\Feat_k}$
		\EndProcedure
	\end{algorithmic}
\end{algorithm}

\begin{algorithm}[t]
	\caption{Learning minimal abstraction}\label{alg:Learnminimal}
	\begin{algorithmic}[1]
		\Require Program $P$, static analyzer $F$, maximum abstraction degree $k$
		\Ensure A minimal abstraction for $P$
		\Procedure{\LearnMinimalAbstraction}{$P,F,k$}
		\State
		$C \gets \component_{P}$
		\State
		$\abs \gets \lambda c. k$
		\For{$i=k$ to 1}
		\State
		$C' \gets C$
		\While {$C' \neq \emptyset$}
			\State
			$c' \gets pick(C')$
			\State
			$C' \gets C' \setminus \{c'\}$
			\State
			$\abs' \gets \lambda c. {\it if}~c=c'~{\it then}~i-1~{\it else}~\abs(c)$
			\If {${\projproved}(F_P(\absk)) = {\projproved}(F_P(\abs'))$}
				\State
				$\abs \gets \abs'$
			\EndIf
		\EndWhile
		\State
		$C \gets C \setminus \{c \mid \abs(c) = i\}$
		\EndFor
		\State \Return $\abs$
		\EndProcedure
	\end{algorithmic}
\end{algorithm}

\myparagraph{Learning Minimal Abstraction}


The objective of learning is to find a set of parameters $\params=\myvec{\Feat_1,\Feat_2, \dots, \Feat_k}$ with which the heuristic $\heuristic_\params$ can produce ideal abstractions for training programs.
We define ideal abstractions to be minimal abstractions~\cite{Liang2011learning} and therefore the learning objective is as follows:
\[
\mbox{Find $\params = \myvec{\Feat_1, \Feat_2,\dots,\Feat_k}$
 such that $\forall P_i \in \vec{P}. \heuristic_\params(G_i)$ is a minimal abstraction for $P_i$.}
\]
where $G_i$ is a graph obtained by running a pre-analysis on $P_i$ (e.g. $G_i = \projgraph(F_{P_i}(\absz))$).
The definition of minimal abstractions is as follows:
\begin{definition}[Minimal Abstraction~\cite{Liang2011learning} ]\label{def:minimalAbs}
An abstraction $\abs$ is a minimal abstraction for program $P$ if
\begin{enumerate}
\item $\abs$ is precise: ${\projproved}(F_P(\abs)) = {
    \projproved}(F_P(\absk))$, and
\item $\abs$ is minimal: $(\abs'\sqsubseteq \abs \land {\projproved}(F_P(\abs')) = {
    \projproved}(F_P(\abs))) \implies \abs' = \abs$.
\end{enumerate}
\end{definition}




Algorithm~\ref{alg:Learnminimal} presents our algorithm for efficiently computing a minimal abstraction for program $P$.
Our algorithm is similar to the \textsc{ScanCoarsen} algorithm by~\cite{Liang2011learning}, but ours is more efficient than the prior algorithm as we exploit the high-level structure of $k$-limited abstractions to reduce the search space.
The algorithm by \cite{Liang2011learning} first transforms $k$-limited abstractions into binary abstractions (where $k$ is 1), losing the opportunity to leverage the properties of the search space induced by monotone $k$-limited analyses.
As a result, the size of search space is $(k+1)^{|\mbc_P|}$ for the existing algorithm~\cite{Liang2011learning}.
%We safely reduce the space to $k\cdot |\mbc_P|$.
We safely reduce the space to $k\cdot 2^{|\mbc_P|}$.


At line 2, we set $C$ to all program components $\component_P$.
%Searching minimal abstractions over the codebases $\vec{P}$ poses a huge search space
%%$(k+1)^C$ where $C$ presents all program components in the codebases.
%$(k+1)^C$ where $C$ presents all program components in $\vec{P}$.
%%Our algorithm reduces the search space into $k*C$ while guarantees to find minimal abstractions.
%Our algorithm reduces the search space into $k*C$ but guarantees to find minimal abstractions.
%At line 2, $C_i$ gets all the program components.
The algorithm begins with the most precise abstraction (line 3).
%$A_\vec{P}$ starts from the most precise one; it always assigns $k$.
At lines 4--15, it considers each of the abstraction degrees $1, 2, \dots, k$ in reverse.
Iterating the abstraction degrees in reverse (from $k$ to $1$) is important to reduce the search space safely.
%it searches for the components that requires $i$ degree,
%where $i$ is equal to $k$ at a start point and decreases as the loop iterates.
%$i$ start from the biggest one $k$ and it decreases over the iterations.
%In the first iteration, it finds program component that need $k$ abstraction degree.
At lines 6--13, it iteratively picks a program component (line 7) and assigns the lower abstraction degree $i-1$ to it (line 9).
%At line 10, the algorithm checks if the refined abstraction still preserves the precision.
%If it is, the algorithm finds that the program component needs a lower abstraction degree.
At line 10, the algorithm checks if the refined abstraction still preserves the precision;
if so, the lower abstraction degree is sufficient for that program component.
%If it loses precision, it finds that the component need $k$.
Otherwise, the program component needs the degree $i$ to preserve the precision.
At the end of the iteration (line 14), we exclude from $C$ the program components that are determined to require the current degree $i$ (i.e. $\myset{c \mid \abs(c) = i}$).
%it subtracts the components from $C_i$.
%The algorithm ends after it learns program components that need $1$.
%It assigns $0$ to the remaining ones.
In the worst case (when the minimal abstraction is $\lambda c.0$), our algorithm iterates
$k\cdot |C|$ times where the search space for each degree $i$ is
$2^C$ and we have $k$ different degrees.
Although the algorithm considers a significantly smaller search space than the original one, it still
guarantees to find a minimal abstraction:
\begin{theorem}\label{THM:REDUCESPACE}
%The abstraction $A_\vec{P}$ returned from
Algorithm~\ref{alg:Learnminimal} returns a minimal abstraction for the input program $P$.
\end{theorem}
\begin{proof}
See Appendix~\ref{sec:proof}.
\end{proof}

%Our algorithm can be considered as a generalized version of \textsc{ScanCoarsen} by~\cite{Liang2011learning}; the prior algorithm corresponds to a special case of our algorithm when $k$ is 1.

%Our algorithm is a generalized one from~\cite{Liang2011learning}.
%Our algorithm is a generalized version of~\cite{Liang2011learning}.
%The prior algorithm corresponds to a special case of our algorithm that
%$i$ is $1$ because it considers binary abstractions which assigns either
%0 or 1 to each program component.
%%Our one, however, is generalized to
%Ours, however, is generalized to find a minimal abstraction where an abstraction assign $0~k$ to each
%program component.

%\newpage
\begin{algorithm}[t]
	\caption{Learning a set of features}\label{alg:learnFeatures}%\small
	\begin{algorithmic}[1]
		\Require Abstraction level $i$, minimal abstractions $A_\vec{P}$, graphs $G_\vec{P}$
		\Ensure A set $\Feat$ of features
		\Procedure{\LearnSetOfFeatures}{$i,A_\vec{P},G_\vec{P}$}
		\State
		$C \gets \{c \mid P \in \vec{P}, c \in \component_P, A_\vec{P}(P)(c) = i\}$
		\State
		$\Feat \gets \emptyset$
                \While {$C \not = \emptyset $}
                \State $\feature \gets {\LearnFeature}(C,G_\vec{P})$ %\Comment Learn abstract graph
                \State $\Feat \gets \Feat \cup \myset{\feature}$
                \State $C \gets C \setminus  \{c \mid P\in\vec{P}, c\in \component_P, c \in \gamma_{G_\vec{P}(P)}(f) \}$
                \EndWhile
                \State \Return $\Feat$
		\EndProcedure
	\end{algorithmic}
\end{algorithm}

%
%
%\begin{algorithm}
%	\caption{Learning a feature}\label{alg:learnFeature}%\small
%	\begin{algorithmic}[1]
%		\Require Program components $C$, graphs $G_\vec{P}$
%		\Ensure A feature $\feature$
%
%				\Procedure{\LearnFeature}{$C,G_\vec{P}$}
%		\State
%		$\feature \gets
%				(\epsilon,([0,\infty],[0,\infty]),\epsilon)$
%
%		\While{$ (\Score(\feature,C) < \gamma$}
%		\State $\feature' \gets \Refine(\feature,G_\vec{P},C)$
%		\If {$ (\Score(\feature',C) > \Score(\feature,C)$)}
%		\State	$\feature \gets \feature'$
%		\Else
%		\State {\bf break}
%		\EndIf
%%		\If {$\Score(\feature',C) \ge \gamma$}
%%		\State
%%			\Return $\feature'$
%%		\EndIf
%%		\State
%%		$\feature' \gets \Refine(\feature,G_\vec{P}, C)$
%%                % \If {$( \Score(\feature',C_i) \ge \gamma$}
%%                % \State
%%                % \Return $\feature'$
%%                %\EndIf
%		\EndWhile
%		\State \Return $\feature$
%		\EndProcedure
%
%	\end{algorithmic}
%\end{algorithm}

\algdef{SE}[DOWHILE]{Do}{doWhile}{\algorithmicdo}[1]{\algorithmicwhile\ #1}%
%Ver. 2
\begin{algorithm}[t]
	\caption{Learning a feature}\label{alg:learnFeature}\small
	\begin{algorithmic}[1]
		\Require Program components $C$, graphs $G_\vec{P}$
		\Ensure A feature $\feature$
		\Procedure{LearnFeature}{$C,G_\vec{P}$}
		\State
		$\feature \gets
				(\epsilon,([0,\infty],[0,\infty]),\epsilon)$
		\State
		$\feature' \gets
				(\epsilon,([0,\infty],[0,\infty]),\epsilon)$
		\Do
		\State
		$\feature \gets \feature'$
		\State
		$\feature' \gets \Refine(\feature, C)$
		\If{$\Score(\feature',C) \ge \hyper$}
			\State
			\Return $\feature'$
		\EndIf
		\doWhile{$ \Score(\feature',C) > \Score(\feature,C)$}
		
		\State \Return $\feature$
		\EndProcedure
	\end{algorithmic}
\end{algorithm}



\myparagraph{Learning a Set of Features}
Algorithm~\ref{alg:learnFeatures} describes the algorithm for learning a set of features.
It takes the abstraction level $i$, minimal abstractions $A_\vec{P}$, and graphs $G_\vec{P}$ as input.
It returns as output a set of features $\Feat$ that best describe the nodes assigned the abstraction level $i$ according to minimal abstractions in $A_{\vec{P}}$.
%Thereby, the returned features can cover the components that are assigned the degree $i$:\TODO{Revise}
%\[ \forall P\in\vec{P}. \{c \mid P \in \vec{P}, c \in \component_P, A_\vec{P}(P)(c) = i\}
%\subseteq \{ c \mid P\in \vec{P}, c\in \component_P, c \in \hyper_{G_\vec{P}(P)}(\Feat) \}.
%\]
At line 2, it collects all program components $C$ (e.g. nodes) whose abstraction degrees are $i$ according to minimal abstractions.
At line 3, it initializes $\Feat$  to be the empty set.
At lines 4--8, the algorithm iteratively calls ${\LearnFeature}$ to generate a feature.
The algorithm adds the generated features to $\Feat$ until the
features cover all program components, and returns $\Feat$ as learned features when $\Feat$ does so.




\myparagraph{Learning a Feature}\label{sec:gamma}
Algorithm~\ref{alg:learnFeature} presents how each feature $\feature$ in $\Feat$ is learned.
%${\textsc LearnFeature}$ takes oracle components $C$ and graphs $G_\vec{P}$.
%It tries to generates a feature $\feature$ that maximizes the following function:
${\LearnFeature}$ takes as input components $C$ and graphs $G_\vec{P}$,
and aims to generate a feature $\feature$ that maximizes the following score function:
\[
\Score (\feature,C) = {\sum_{P\in\vec{P}} |C \cap \gamma_{G_\vec{P}(P)}({\feature})  | \over
  \sum_{P\in\vec{P}} |\gamma_{G_\vec{P}(P)}(\feature) |}
\]
where the score is a real number between 0 and 1.
%Intuitively, it is a score function of a feature $\feature$.
%If a feature $\feature$ accurately selects component in oracle that $\sum_{P\in\vec{P}}\sem{\feature}_{G_\vec{P} (P)}$ belongs to $C$,
%the score become the highest one 1.
Intuitively, the score describes how accurately a feature describes the program components in $C$.
For example, the score becomes the highest value 1
when
$\forall{P \in \vec{P}}.\; \gamma_{G_{\vec{P}}(P)}(f) \subseteq C$.
%Collecting qualified feature make the learned parameter
%$\params = \myvec{\Feat_1, \Feat_2,\dots,\Feat_k}$ satisfy the learning objective.
%The score decreases when the feature selects components which are not in $C$.
The score decreases as the feature selects components not in $C$.

The algorithm starts from the most general feature, i.e., $(\epsilon,([0,\infty],[0,\infty]),\epsilon)$, and
iteratively refines it until the feature becomes sufficiently informative, meaning that the score of the refined feature becomes higher than the hyper parameter $\hyper$.
The value of $\hyper$ has great impacts on the performance of learned heuristics, and
we discuss how we determine the value of $\hyper$ in Section~\ref{sec:hyper_param}.
%as all nodes in any graphs belong to this feature.
%$(\epsilon,([0,\infty],[0,\infty]),\epsilon)$ is the most general feature as all the nodes in any graph belong to it.
At lines 4--10, the algorithm iteratively calls $\Refine$ to make the current feature $\feature$ more specific.
When no more improvement is possible (i.e. $\Score(f', C) \le \Score(f,C)$), the loop terminates and the algorithm returns the current feature $f$.
We define the refinement function $\Refine$ as follows:
\[
\Refine(\feature,C) =
\argmax_{f' \in \Append(f) \cup \Replace(f)}
\Score(\feature',C)
\]
where $\Append(\feature)$ and $\Replace(\feature)$ produce new features that are more specific than $\feature$.
From the set of new features, $\Refine$ chooses the one with the highest score.
$\Append(\feature)$ denotes the features that are obtained by appending an abstract node to $\feature$:
\begin{multline*}
	\Append((\seq{\myhat{p}_0,\dots,\myhat{p}_q}, \myhat{n}, \seq{\myhat{s}_0,\dots,\myhat{s}_r})) =\\
	\myset{(\seq{\myhat{a}', \myhat{p}_0,\dots,\myhat{p}_q}, \myhat{n}, \seq{\myhat{s}_0,\dots,\myhat{s}_r}), (\seq{\myhat{p}_0,\dots,\myhat{p}_q}, \myhat{n}, \seq{\myhat{s}_0,\dots,\myhat{s}_r,\myhat{a}'}) \mid\\ \myhat{a}' \in \Specify(([0,\infty],[0,\infty])) }
\end{multline*}
where the function $\Specify$ denotes a strategy for making a feature more specific.
In experiments, we used the following strategy:
\[
\begin{array}{l}
\Specify(([a,b],[c,d])) =
\myset{
([\frac{a+b}{2},b],[c,d]),
([a,\frac{a+b}{2}],[c,d]),
([a,b],[\frac{c+d}{2},d]),
([a,b],[c,\frac{c+d}{2}])
}
\end{array}
\]
where, if $b$ (resp., $d$) equals to $\infty$, it is replaced by the maximum number of incoming (resp., outgoing) edges in the training graphs ($G_\vec{P}$).

The definition of $\Specify$ is a design choice.
For example, we can consider the following definition for $\Specify$:
\begin{multline*}
\Specify(([a,b],[c,d])) = \\
\myset{
([\mbox{{\it a+$\frac{b-a}{3}$}},b ],[c,d]),
([a,\mbox{{\it b-$\frac{b-a}{3}$}}],[c,d]),
([a,b],[\mbox{{\it c+$\frac{d-c}{3}$}},d ]),
([a,b],[c,\mbox{{\it d-$\frac{d-c}{3}$}} ])
}.
\end{multline*}
With the above definition, we can still find desirable features for the training set.
It, however, takes more iterations to obtain such features because it specifies the interval value of features more carefully than the one we used.
%[{{\it $a$+$\frac{b-a}{3}$}},$b$ ] and [$a$,{{\it $b$-$\frac{b-a}{3}$}}]
%It is a more careful than the one we used because [{{\it $a$+$\frac{b-a}{3}$}},$b$ ] and [$a$,{{\it $b$-$\frac{b-a}{3}$}}] are more general %intervals than $[\frac{a+b}{2},b]$ and $[a,\frac{a+b}{2}]$, respectively.
%Using this $\Specify$, we can also find qualified features for the training set, but it needs more iterations to produce such features.
%Thus, we choose $\Specify$ with the intervals $[\frac{a+b}{2},b]$ and $[a,\frac{a+b}{2}]$ to minimize the learning cost.

On the other hand, $\Replace(\feature)$ denotes the features that are obtained by replacing one of the abstract nodes in $\feature$ by more specific ones:
\[
\begin{array}{l}
\Replace((\seq{\myhat{p}_0,\dots,\myhat{p}_q}, \myhat{n}, \seq{\myhat{s}_0,\dots,\myhat{s}_r})) =\\
\qquad
\begin{array}{l@{~}l}
& \myset{(\seq{ \myhat{p}_0,\dots,\myhat{p}_q}, \myhat{n}', \seq{\myhat{s}_0,\dots,\myhat{s}_r}) \mid \myhat{n}' \in \Specify(\myhat{n}) } \\
 \cup&\myset{(\seq{ \myhat{p}'_0,\dots,\myhat{p}'_q}, \myhat{n}, \seq{\myhat{s}_0,\dots,\myhat{s}_r}) \mid
j \in [0,q], \myhat{p}'_j \in \Specify(\myhat{p}_j), \forall i \not= j.\; \myhat{p}'_j = \myhat{p}_j} \\
\cup & \myset{(\seq{ \myhat{p}_0,\dots,\myhat{p}_q}, \myhat{n}, \seq{\myhat{s}'_0,\dots,\myhat{s}'_r}) \mid
j \in [0,r], \myhat{s}'_j \in \Specify(\myhat{s}_j), \forall i \not= j.\; \myhat{s}'_j = \myhat{s}_j}
\end{array}
\end{array}
\]

\myparagraph{Example}
With an example, we explain how an actual feature used in our evaluation (Section~\ref{sec:eval_heap}) is generated where $\hyper$ is 0.5.
\begin{enumerate}
\item Our algorithm starts from the most general feature $f$:
\begin{center}
\resizebox{0.18\columnwidth}{!}{
	\begin{tikzpicture}
	\node [block3,fill = gray] (n2) {{\tt\Large [0,$\infty$],[0,$\infty$]}};
	\end{tikzpicture}
	}.
\end{center}
\item It enumerates 12 cases of refined features from $f = (\epsilon,([0,\infty],[0,\infty]),\epsilon)$ (e.g., 8 cases of \Append($f$) and 4 cases of \Replace($f$)).
It chooses the following feature produced from \Replace($f$):
\begin{center}
\resizebox{0.18\columnwidth}{!}{
	\begin{tikzpicture}
	\node [block3,fill = gray] (n2) {{\tt\Large [0,97],[0,$\infty$]}};
	\end{tikzpicture}}
\end{center}
 which has the highest score 0.06 among the 12 cases of features.
\item Because the score is less than $0.5$,
it refines the feature again to the following specific one, which comes from $\Append(\epsilon,([0,97],[0,\infty]),\epsilon),$ with the same manner:
\begin{center}
\resizebox{0.4\columnwidth}{!}{
	\begin{tikzpicture}
	\node [block3,fill = gray] (n2) {{\tt\Large [0,97],[0,$\infty$]}};
	\node [block3,right = 0.8cm of n2] (n3) {{\tt\Large [97,$\infty$],[0,$\infty$]}};
	\path [line] (n2) -> (n3);
	\end{tikzpicture}
	}
\end{center}
where the feature has score 0.23.
\item To find a better one, it refines the feature further; it enumerates 16 cases of refined features (e.g., 8 cases for replacing and 8 cases for appending a node). The following feature is selected:
\begin{center}
\resizebox{0.4\columnwidth}{!}{
	\begin{tikzpicture}
	\node [block3,fill = gray] (n2) {{\tt\Large [0,97],[0,$\infty$]}};
	\node [block4,right = 0.8cm of n2] (n3) {{\tt\Large [97,$\infty$],[140,$\infty$]}};
	\path [line] (n2) -> (n3);
	\end{tikzpicture}
	}
\end{center}
where the score is 0.37.
\item In the next iteration, it finally finds an informative feature which has a score 0.55:
\begin{center}
\resizebox{0.4\columnwidth}{!}{
	\begin{tikzpicture}
	\node [block3,fill = gray] (n2) {{\tt\Large [0,48],[0,$\infty$]}};
	\node [block4,right = 0.8cm of n2] (n3) {{\tt\Large [97,$\infty$],[140,$\infty$]}};
	\path [line] (n2) -> (n3);
	\end{tikzpicture}
	}.
\end{center}
\end{enumerate}




















