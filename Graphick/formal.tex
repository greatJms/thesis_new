% !TEX root = ./paper.tex


\begin{figure}[t]
	\[
	  \begin{array}{c}
		\infer[]
		{
		\begin{array}{c}
		  \ctx' \in \Reachable ({m}) \quad
		  ({\heap}, {\hctx}) \in  \VarPointsTo ({m_{\this}}, {\ctx'}) \\
		  \VarPointsTo({\it arg}, \ctx) \subseteq \VarPointsTo(m_{\it param}, \ctx')
		  \quad
		  \VarPointsTo(m_{\it return}, \ctx') \subseteq \VarPointsTo(x, \ctx)\\
		  (\invo, \ctx, m, \ctx') \in \CallGraph
		\end{array}
		}{
		\begin{array}{c}
		  {\it \invo : x = y.\sig({\it arg})} \quad
		  \ctx \in    \Reachable ({\methof(\invo)}) \\
		  ({\heap}, {\hctx})\in  \VarPointsTo ({\it y},\ctx) \quad
		  t = {\typeof}(\heap)  \quad m = {\lookup}(t, \sig) \\
  (e, {\pctx}, p) = \parent(\heap,\hctx, \invo, \ctx) \quad\\
  \ctx' = \left\{
	\begin{array}{l@{\quad}l}
	\truncateCtx{\appendCtx{{ \hctx}}{heap}}{\CallDepth}
	& \mbox{if}~\ctxAbstraction(heap) = \CallDepth \\
	\truncateCtx{\appendCtx{{ \hctx}}{heap}}{\CallDepth-1}     & \mbox{if}~\ctxAbstraction(heap) = \CallDepth-1\\
	\dots\\
	\truncateCtx{\appendCtx{{ \hctx}}{heap}}{0}
	& \mbox{if}~\ctxAbstraction(heap) = 0 \\
	\end{array}
	\right.
		\end{array}
															}
	  \end{array}
	\]
	\caption{Parametric context sensitivity}
  \label{fig:param-obj}
  \end{figure}


\section{Setup: Parameterized Pointer Analysis}

In this section, we illustrate how to parameterize the analysis rule in Section~\ref{pre:baseline-rules} with context sensitivity and heap abstraction.




\myparagraph{Parametric Context Sensitivity}
The analysis in Figure~\ref{pre:baseline-rules} uses the same $\CallDepth$ value for every method call.
The parametric object-sensitive analysis generalizes it to be able to assign different call depths for different method calls.
To do so, the parameterized analysis uses the rule
in Figure~\ref{fig:param-obj}
instead of the last rule in Figure~\ref{pre:baseline-rules}.
In Figure~\ref{fig:param-obj}, we use the function $\ctxAbstraction: \mbh \to [0,\CallDepth]$, which assigns a context depth between 0 and $\CallDepth$ to each method call.
When a method is called on a receiver object  $\heap$, $\ctxAbstraction$
produces an appropriate context depth for it.
In Section~\ref{sec:approach}, we present a technique for automatically learning a heuristic that produces the $\ctxAbstraction$ information for a given program.


\begin{figure}[t]
		\[
		\begin{array}{c}
		
		%% alloc %%
		\infer[]{
			(\heap', \hctx) \in \VarPointsTo({\var}, {\ctx})
		}{ {(\var,\heap,\inmeth)\in Alloc} \qquad \qquad \ctx \in \MethodCtx(\inmeth) \qquad \qquad \qquad \qquad\\
		{ \hctx = \truncateCtx{\ctx}{\HeapDepth}}  \quad
		\heap' = \left\{
		\begin{array}{l@{\quad}l}
		\heap
		& \mbox{if}~ \heapAbstraction(heap) = \mbox{`\alloc'} \\
		\typeof(\heap)     & \mbox{if}~\heapAbstraction(heap) = \mbox{`\type'}
		\end{array}
		\right.		}
		\end{array}
	\]
	\caption{Parametric heap abstraction}
	\label{fig:param-heap}
	\end{figure}
	
\myparagraph{Parametric Heap Abstraction}\label{sec:param-heap}

The analysis in Figure~\ref{pre:baseline-rules} uses allocation-based heap abstraction for every heap object.
We can generalize it to support selective use of allocation-site- and type-based heap abstractions. We first need to generalize the analysis results as follows:
\begin{itemize}
	\item $\VarPointsTo:\mbv \times \mbc \to \power{(\mbh + \mbt) \times \mbh\mbc}$
	\item $\FldPointsTo:(\mbh + \mbt) \times \mbh\mbc \times \mbf \to \power{(\mbh + \mbt) \times \mbh\mbc}$
%	\item $\ArrPointsTo:(\mbh + \mbt) \times \mbh\mbc \to \power{(\mbh + \mbt) \times \mbh\mbc}$	
%\item $\MethodCtx: \mbm \to \power{\mbc}$
\item $\Reachable: \mbm \to \power{\Context}$
\item $\CallGraph: \power{\mbi \times \Context \times \mbm \times \mbc}$
\end{itemize}
$\VarPointsTo$ maps each variable with a context to abstract heaps, where an abstract heap is now either allocation site ($\mbh$) or class type ($\mbt$) with their heap context. $\FldPointsTo$ is also extended in a similar way to support type-based abstraction.
We replace the first rule in Figure~\ref{pre:baseline-rules} by the parameterized rule in Figure~\ref{fig:param-heap}.
The new rule uses the function $\heapAbstraction$ that takes an allocation site ($\heap$) and determines whether we use allocation-site-based abstraction (`{\it alloc}') or type-based abstraction (`{\it type}'). When $\heapAbstraction$ returns `{\it type}', we use the class type of the allocated object (i.e. $\typeof(\heap)$) instead of the allocation site. Otherwise, the analysis performs the conventional allocation-site-based heap abstraction.
Our technique in Section~\ref{sec:approach} can be also used for learning a heuristic that produces appropriate $\heapAbstraction$ for each input program.






%\subsection{Graph Representation}\label{sec:graphs}

\myparagraph{Graph Representations}
Before we present \ourtool, we introduce two popular graphs, object allocation graph (OAG) and field points-to graph (FPG), that existing graph-based heuristics~\cite{TanLX16,Tan2017,Li2018b} use. 
The object allocation graph is a directed graph, $\GraphOAG = (\NodeOAG,\EdgeOAG)$, where nodes are
allocation sites in the program (i.e. $\NodeOAG = \mbh$) and edges $(\EdgeOAG) \subseteq \mbh \times \mbh$ describe the object allocation relation defined as follows:
\[
h\; \EdgeOAG \; h' \iff \exists m \in \mbm.\; (h, \_) \in \VarPointsTo({\it this}_m, \_)~\mbox{and}~\methof(h') = m
.
\]
In words, we have $h \; \EdgeOAG \; h'$ if $h$ is a receiver object of method $m$, i.e. $(h,\_) \in \VarPointsTo({\it this}_m, \_)$, and $m$ allocates $h'$, i.e. $methof(h') = m$.
Intuitively, object allocation graph is the ``call graph'' in object sensitivity, which
provides information about how each context is constructed in $k$-object-sensitive analysis~\cite{Li2018b}.
The field points-to graph $\GraphPtsTo = (\NodePtsTo,\EdgePtsTo)$
 is simply a context-insensitive representation of the $\FldPointsTo$ relation.
We define $\NodePtsTo = \mbh$ and
$h\; \EdgePtsTo\; h' \iff (h',\_) \in \FldPointsTo(h,\_,\_)$.
In our evaluation, we give the two graphs to \ourtool~for learning graph-based heuristics.


%%% Local Variables:
%%% mode: latex
%%% TeX-master: "paper"
%%% End:


