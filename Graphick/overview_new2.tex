% !TEX root = ./paper.tex

\begin{figure}
	\centering

	\begin{multicols}{2}
\begin{subfigure}[t]{1.5\columnwidth}
	\begin{multicols}{2}
		\begin{lstlisting}[basicstyle=\small\ttfamily]
class A{} class B{}
class C{
  Object data;
  void C(){
    data = new Object;//O
  }
  void set(Object e){
    data = e;
  }
  Object get(){
    return data;
  }
}
class F{
  void foo(){
    C c1 = new C();//C1
    C c2 = new C();//C2

    c1.set(new A());//A1
    c2.set(new B());//B1
    A a = (A)c1.get();//query1
    B b = (B)c2.get();//query2
    }
  }
}

main(){
  F f1 = new F(); //F1
  f1.foo();
  F f2 = new F(); //F2
  f2.foo();
}

\end{lstlisting}
\end{multicols}
\caption{Example code}
\label{fig:example}
\end{subfigure}
\begin{flushright}
\begin{subfigure}[t]{0.6\columnwidth}	
	\begin{center}
		\resizebox{0.8\columnwidth}{!}{
			\begin{tikzpicture}
			\node [shape=circle,draw=black] (f1) {{\tt
                            F1}};
			\node [shape=circle,draw=black,right = 0.5cm
                        of f1] (f2) {{\tt F2}};

			 \node [shape=circle,draw=black, below = 0.4cm of f1] (b1) {{\tt B1}};
			 \node [shape=circle,draw=black, left =
                         0.4cm of b1] (a1) {{\tt A1}};

			 \node [shape=circle,draw=black, below =
                         0.4cm of f2] (c1) {{\tt C1}};				
			 \node [shape=circle,draw=black, right =
                         0.4cm of c1] (c2) {{\tt C2}};

			\node [shape=circle,draw=black, below right =
                        0.4cm and   0.1cm of c1] (o) {{\tt O}};			

			\path [line] (f1) -- (c1);
			\path [line] (f1) -- (c2);

			\path [line] (f2) -- (c1);
			\path [line] (f2) -- (c2);

			\path [line] (c1) -- (o);
			\path [line] (c2) -- (o);
				
			\path [line] (f1) -- (a1);
			\path [line] (f1) -- (b1);

			\path [line] (f2) -- (a1);
			\path [line] (f2) -- (b1);
			\end{tikzpicture}
		}
	\end{center}
	\caption{\centering Object allocation graph}
        \label{fig:oag}
\end{subfigure}
~\\~\\~\\
\begin{subfigure}[t]{0.6\columnwidth}	
	\begin{center}
		\resizebox{0.6\columnwidth}{!}{
			\begin{tikzpicture}
				\node [block3] (n1) {{\tt\Large [0,$\infty$],[2,$\infty$]}};
				\node [block3,fill = gray,below = 0.8cm of n1] (n2) {{\tt\Large [0,$\infty$],[0,$\infty$]}};
				\node [block3,below = 0.8cm of n2] (n3) {{\tt\Large [2,$\infty$],[0,$\infty$]}};
				% \node [onlyText,left =0.5cm of n2]
                                % (lbr) {{\tt \Huge 1obj : \{ }};
				% \node [onlyText,right = -0.7cm of n2] (rbr) {{\tt\Huge \} }};
				\path [line] (n1) -> (n2);
				\path [line] (n2) -> (n3);
			\end{tikzpicture}
		}
	\end{center}
	\caption{\centering Object-sensitivity heuristic}
        \label{fig:oag_heuristic}
\end{subfigure}
% \begin{subfigure}[t]{0.6\columnwidth}	
% 	\begin{center}
% 		\resizebox{0.8\columnwidth}{!}{
% 			\begin{tikzpicture}
% 			\node [shape=circle,draw=black] (f1) {{\tt
%                             F1}};
% 			\node [shape=circle,draw=black,below = 0.4cm
%                         of f1] (f2) {{\tt F2}};
% 			\node [shape=circle,draw=black, right =
%                         0.4cm of f1] (c1) {{\tt C1}};				
% 			\node [shape=circle,draw=black, right =
%                         0.4cm of f2] (c2) {{\tt C2}};
% 			\node [shape=circle,draw=black, below right =
%                         0.1cm and   0.4cm of c1] (o) {{\tt O1}};			
% 			\node [shape=circle,draw=black, left =
%                         0.4cm of f1] (a1) {{\tt A1}};
% 			\node [shape=circle,draw=black, left = 0.4cm of f2] (b1) {{\tt B1}};
% 			\path [line] (f1) -- (c1);
% 			\path [line] (f1) -- (c2);
% 			\path [line] (f2) -- (c1);
% 			\path [line] (f2) -- (c2);
% 			\path [line] (c1) -- (o);
% 			\path [line] (c2) -- (o);
% 			\path [line] (f1) -- (a1);
% 			\path [line] (f1) -- (b1);
% 			\path [line] (f2) -- (a1);
% 			\path [line] (f2) -- (b1);
% 			\end{tikzpicture}
% 		}
% 	\end{center}
% 	\caption{\centering Object allocation graph}
%         \label{fig:oag}
% \end{subfigure}	
\end{flushright}
\end{multicols}
\caption{Example to illustrate our graph-based object-sensitivity heuristic}
\end{figure}


\section{Overview}
We illustrate how our graph-based heuristic looks like and works with an example. %graph-based heuristic works with an example program.
% Figure~\ref{fig:oag_heuristic} presents an example of our graph based
% object-sensitivity heuristic for object allocation graph.


%\subsection{Example program}
\myparagraph{Example Program}
%\begin{comment}
%Figure~\ref{fig:example} shows an example program to illustrates a
%graph-based object-sensitivity heuristics which can be obtained from our framework.
%For this example, we use allocation-site based heap abstraction that
%present each objects with their allocation-site.
%This example has a class {\tt C} which provides getter and setter method
%to manipulate its field {\tt data}.
%A class {\tt F} has a method {\tt foo} which allocates two
%objects {\tt C1} and {\tt C2} to variable {\tt c1} and {\tt c2}, respectively.
%The variables {\tt c1} and {\tt c2} call {\tt set} method with newly
%allocated objects {\tt A1} and {\tt B1}, respectively.
%There are two queries {\tt query1} and {\tt query2} asking the
%safety of the down castings at line 18 and 19 are safe.
%Obviously, the queries are safe in the real execution because {\tt get} method returns objects
%{\tt A1} and {\tt B1} at line 18 and 19, respectively.
%\end{comment}



Figure~\ref{fig:example} is an example program with two queries checking the down-casting safety.
This example has a main method that calls the method {\tt foo} with two different receiver objects {\tt F1} and {\tt F2}.
Class {\tt C} provides getter and setter methods to manipulate its field {\tt data}.
Class {\tt F} has a method {\tt foo} which allocates two
objects {\tt C1} and {\tt C2} to variables {\tt c1} and {\tt c2}, respectively.
These variables call the {\tt set} method with newly allocated objects {\tt A1} and {\tt B1}.
There are two queries {\tt query1} and {\tt query2} asking the down-casting safety at lines 21 and 22.
The safety holds because the {\tt get} method returns objects {\tt A1} and {\tt B1} at lines 21 and 22, respectively.



\myparagraph{Goal: Selective Object Sensitivity}
Our goal is to analyze the program cost-effectively by applying context sensitivity only when it is necessary.
%selectively to a small subset of method calls.
To prove the queries, we need an object-sensitive analysis that differentiates the methods called under receiver objects {\tt C1} and {\tt C2}.
Without object sensitivity, the analysis merges the methods {\tt get} and {\tt set} called on receiver objects {\tt C1} and {\tt C2};
eventually, the analysis misjudges that the {\tt get} method can return both {\tt A} and {\tt B} at lines 21 and 22, and fails to prove the down-casting safety.
The context-sensitive analysis, however, is not necessary for the method {\tt foo} called from other objects {\tt F1} and {\tt F2}
because {\tt foo} is not related to the queries.
If we apply context sensitivity to this method, it only increases the analysis cost without any precision gain.
Thus, our heuristic aims to infer the following policy for the main analysis:
\[
\mbox{Apply object sensitivity only to method calls whose receiver objects are {\tt C1} or {\tt C2}.}
\]


%\subsection{How Our Heuristic Works}

\myparagraph{Graph-based Heuristics}
To generate such a policy, graph-based heuristics first run a cheap pre-analysis (e.g. context-insensitive analysis) to
obtain a graph representation of the program.
For object-sensitivity heuristics, the object allocation graph (OAG) has been considered as a good program
representation~\cite{Li2018b,TanLX16}.
%
%prior works on graph-based heuristics investigate this graph and figure out the objects that should be analyzed precisely~\cite{Tan2017} or the methods that need coarse analysis~\cite{Li2018b}.
Nodes in an OAG are heap objects (represented by allocation sites) and edges represent the connections between objects and their allocators. Figure~\ref{fig:oag} shows the OAG of the example program.
%is a set of graph which connects the objects with their allocators,
%and Fig~\ref{fig:oag} is that of the example program.
In Figure~\ref{fig:oag}, for instance, two objects {\tt F1} and {\tt F2} have edges toward the objects {\tt A1}, {\tt B1}, {\tt C1}, and
{\tt C2} because these four objects are allocated inside the method {\tt foo} that is called on the receiver objects {\tt F1} and {\tt F2}.
Given the OAG, the goal of graph-based heuristics is to choose a set of nodes in the graph.
Ideally, a good heuristic would accurately identify the set $\myset{{\tt C1}, {\tt C2}}$ that needs object sensitivity during the main analysis.


\myparagraph{How Our Heuristic Works}

Our heuristic consists of a set of {\em features}, where a feature describes a set of nodes in the given graph.
A feature is of the form $({\it prev}, ([a,b], [c,d]),{\it succ})$, where $[a,b]$ and $[c,d]$ are intervals, and ${\it prev}$ and ${\it succ}$ are sequences of pairs of intervals. A node $n$ in a graph is described by the feature iff the number of incoming edges of $n$ is between $a$ and $b$, the number of outgoing edges is between $c$ and $d$, and the node has a sequence of predecessors satisfying {\it prev}, and the node has a sequence of successors satisfying {\it succ}.

For example, Figure~\ref{fig:oag_heuristic} shows a heuristic comprising of a single feature $({\it prev} , ([0,\infty],[0,\infty]), {\it succ})$, where ${\it prev}$ and ${\it succ}$ are single pair of intervals (i.e. ${\it prev} = ([0,\infty],[2,\infty])$ and ${\it succ} = ([2,\infty], [0,\infty])$). It describes nodes that have at least 0 incoming and 0 outgoing edges, have a predecessor with at least 0 incoming and 2 outgoing edges, and have a successor with at least 2 incoming and 0 outgoing edges.
In Figure~\ref{fig:oag}, {\tt C1} and {\tt C2} are the only nodes that satisfy these conditions because they have a successor (i.e. {\tt O}) with two incoming edges and a predecessor ({\tt F1} or {\tt F2}) with four outgoing edges.
From a set of training programs, our learning algorithm in Section~\ref{sec:learning-algorithm} can generate such features automatically.



%%graph-based object-sensitivity heuristic that our learning algorithm can generate automatically from data.
%%example for object-allocation-graph (OAG) which our framework can learn.
%%It is a singleton set of a feature of which elements are a target node (e.g., gray colored node) and its neighbor nodes.
%Note that a heuristic can contain a number of features; our learned object-sensitivity heuristic used in our evaluation contains 195 features.
%In each node, the former interval presents the number of incoming edges while the other corresponds to the number of outgoing edges.
%Fig~\ref{fig:oag_heuristic} presents nodes, which have a predecessor with at least 2 outgoing edges and a successor
%with at least 2 incoming edges, in OAG.
%
%
%In Figure~\ref{fig:oag}, the nodes {\tt C1} and {\tt C2} are the only nodes which satisfy our example heuristic
%because they are the nodes having a successor (e.g., {\tt O}) with two incoming edges and a predecessor (e.g., {\tt F1} or {\tt F2}) with four outgoing edges.
%


%Our framework can automatically learn cost-effective heuristics, and

Given a graph and a set of features, %in Figure~\ref{fig:oag} and the feature in Figure~\ref{fig:oag_heuristic},
our heuristic finds out all the nodes that satisfy one of the features. This information is used by the main analysis to perform a selective object-sensitive analysis;
the methods called under receiver objects {\tt C1} and {\tt C2} are analyzed with 1-object-sensitivity
while the others are analyzed context insensitively.


%After specifying the nodes which satisfy the heuristic,
%an analysis strategy for the main analysis is generated;
%the methods called under receiver objects {\tt C1} and {\tt C2} are analyzed with 1-object-sensitivity
%while the others are analyzed context insensitively.
%In this context abstraction, {\tt get} and {\tt set} methods are distinguished as two different contexts (e.g., {\tt get[C1]}, {\tt get[C2]}, {\tt set[C1]}, and {\tt set[C2]});
%{\tt get[C1]} and {\tt get[C2]} return {\tt A1} and {\tt B1} at line 18 and 19, respectively.
%This is a desired context abstraction to prove all queries in the example program.
%It distinguishes all precision critical method calls while merging the others that are not related to the queries.



Note that the performance of the main analysis heavily depends on the features in learned heuristics.
For example, assume that a heuristic contains the following feature which takes
off a predecessor of a target node from the feature in Figure~\ref{fig:oag_heuristic}:
\begin{center}
\resizebox{0.4\columnwidth}{!}{
	\begin{tikzpicture}
	\node [block3,fill = gray] (n2) {{\tt\Large [0,$\infty$],[0,$\infty$]}};
	\node [block3,right = 0.8cm of n2] (n3) {{\tt\Large [2,$\infty$],[0,$\infty$]}};
	\path [line] (n2) -> (n3);
	\end{tikzpicture}
	}.
\end{center}
Unlike the feature in Figure~\ref{fig:oag_heuristic}, which only {\tt C1} and {\tt C2} satisfy,
four nodes {\tt C1}, {\tt C2}, {\tt F1}, and {\tt F2} are implied by the above feature.
Because the feature still includes precision-critical nodes, {\tt C1} and {\tt C2}, the heuristic is able to prove the queries;
however, it pays additional analysis costs as the set of nodes include {\tt F1} and {\tt F2} which are not related to the queries.
As such, inappropriately learned heuristics can degrade the performance in costs and even the precision
of the main analysis.
Therefore, the goal of our learning algorithm is to find out qualified heuristics that are able to maintain as many precision-critical nodes as possible while excluding others that are not.




%%% Local Variables:
%%% mode: latex
%%% TeX-master: "paper"
%%% End:


