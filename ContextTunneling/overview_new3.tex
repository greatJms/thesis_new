%\newpage

\section{Motivating Examples}
\label{sec:overview}

In this section, we illustrate context tunneling with examples.
We describe our technique with $k$-call-site-sensitivity, but it is applicable to other flavors of
context-sensitivity too. The general description
is presented in Section~\ref{sec:tunneling}.

% \sehun{To
% show existence of aforementioned $k$-limiting context-sensitivity's
% limitation regardless of choice of context flavor, our examples are designed
% around two major principles, $k$-limiting call-site-sensitivity and
% $k$-object sensitivity. But it is also applicable to other flavors of
% context-sensitivity too, such as type-sensitivity.}

% \myparagraph{Caveat}
In this section, we restrict our discussion to the
traditional $k$-limited context-sensitive analysis. % Although the example
% programs in this section could be analyzed precisely with
Other approaches
such as value-based context-sensitivity~\cite{Khedker2008,PadhyeK13}
may not suffer from the issue described in this section.
However, their applicability to real-world pointer analysis remains to
be seen, which is beyond the scope of this paper. On the other hand,
the $k$-limited method is more widely used in practical settings~\cite{Bravenboer2009,Smaragdakis2011,KastrinisS13a,Smaragdakis2014,JeJeChOh17}.



%applying them to real-world programs remains to be seen and is beyond the scope of this paper.
% On the other hand, the $k$-limited method is popular in practical settings~\cite{Bravenboer2009,Smaragdakis2011,kastrinis2013hybrid,Smaragdakis2014,JeJeChOh17}.


% The goal and scope of this paper are to identify and fix the
% major weakness of $k$-limited context-sensitive analysis that is
% widely used in modern pointer
% analyses~\cite{Bravenboer2009,Smaragdakis2011,kastrinis2013hybrid,Smaragdakis2014,JeJeChOh17}.
% Applying other methods such as value
% contexts~\cite{Khedker2008,PadhyeK13} to avoid the problem is beyond the scope of
% this paper.

\subsection{Call-Site-Sensitivity}
\myparagraph{Example Code}

Suppose that we analyze the example Java code in
Fig.~\ref{fig:overview:code} using a $k$-call-site-sensitive pointer
analysis~\cite{Smaragdakis2015}.  The example code has three class
definitions, namely \texttt{A}, \texttt{B}, and \texttt{C}.  Classes
\texttt{A} and \texttt{B} are empty and class \texttt{C} has two
methods: \texttt{id} and \texttt{main}.  Method \texttt{id} is
essentially the identity function on the first argument, which takes an
object \texttt{v} and an integer \texttt{i}, and returns the same
object after making recursive calls until \texttt{i} becomes 0. In
\texttt{main}, \texttt{id} is invoked twice with different
objects. Both method invocations share the same argument \texttt{i}
that comes from the external environment (i.e. user input).  The goal
of the pointer analysis is to prove the safety of two type casts at
lines 8 and 9.

\begin{figure}
\hspace*{0.1cm}
\centering
\begin{subfigure}[b]{.49\columnwidth}
\begin{lstlisting}
class A {} class B {}
class C {
  static Object id (Object v, int i){
    return i >= 0 ? id(v, i-1) : v;
 }
 public static void main (){
   int i = input();
   A a = (A) id(new A(), i);//Query 1
   B b = (B) id(new B(), i);//Query 2
 }
}
\end{lstlisting}
\caption{Example code}
\label{fig:overview:code}
\end{subfigure}
\quad
\hspace*{-1.cm}
\begin{subfigure}[b]{.45\columnwidth}

%%% k-CFA
\begin{center}
\resizebox{\columnwidth}{!}{
\begin{tikzpicture}
    \node [block] (main) {{\tt main} \\ $[\cdot]$};
    \node [block, above right=0cm and 0.4cm of main] (id1) {{\tt id}\\$[8]$};
    \node [block, below right=0cm and 0.4cm of main] (id2) {{\tt id}\\$[9]$};
    \node [block, right=0.4cm of id1] (id3) {{\tt id} \\$[8,4]$};
    \node [block, right=0.4cm of id2] (id4) {{\tt id} \\$[9,4]$};
    \node [block2, right=0.4cm of id3] (id5) {{\tt id}\\$[\underbrace{8,4,\cdots,4}_k]$};
    \node [block2, right=0.4cm of id4] (id6) {{\tt id}\\$[\underbrace{9,4,\cdots,4}_k]$};
    \node [block2, below right=-0.5cm and 0.4cm of id5] (id7) {{\tt id} \\$[\underbrace{4,\cdots,4}_{k}]$};

    \path [line] (main) -- (id1);
    \path [line] (main) -- (id2);
    \path [line] (id1) -- (id3);
    \path [line] (id2) -- (id4);
    \path [line, dashed] (id3) -- (id5);
    \path [line, dashed] (id4) -- (id6);
    \path [line] (id5) -- (id7);
    \path [line] (id6) -- (id7);
    \path [line] (id7) edge[loop above] ();
\end{tikzpicture}
}
\end{center}


\caption{Call-graph by $k$-CFA}
\label{fig:overview:kCFA}
\vspace{2ex}

\begin{center}
\resizebox{0.45\columnwidth}{!}{
\begin{tikzpicture}
    \node [block] (main) {{\tt main}\\$[\cdot]$};
    \node [block, above right=0cm and 0.5cm of main] (id1) {{\tt id}\\$[8]$};
    \node [block, below right=0cm and 0.5cm of main] (id2) {{\tt id}\\ $[9]$};

    \path [line] (main) -- (id1);
    \path [line] (main) -- (id2);

    \path [line] (id1) edge[loop right] ();
    \path [line] (id2) edge[loop right] ();
\end{tikzpicture}
}
\end{center}
\caption{Call-graph by 1-CFA with tunneling}
\label{fig:overview:tunneling}
\end{subfigure}\qquad
\caption{Example and call-graphs constructed by the conventional
  $k$-call-site-sensitivity ($k$-CFA) and our 1-call-site-sensitivity
  with context tunneling}
\label{fig:overview}
\end{figure}




\myparagraph{Conventional Call-Site-Sensitivity}

The traditional $k$-call-site-sensitive analysis fails to prove the queries no matter what $k$ value is used.
Because the value of \texttt{i} can be any integer, the following set $K_{\infty}$ of infinite number of call-strings can be generated for method \texttt{id} at runtime:
\[
K_{\infty} = \myset{[8], [9], [8,4], [9,4], [8,4,4], [9,4,4], [8,4,4,4], [9,4,4,4], \dots}
\]
where, for example, $[8,4]$ denotes the context that method \texttt{id} was called along the sequence of
call-sites 8 and 4.
 The $k$-call-site-sensitive analysis approximates the call-strings in $K_{\infty}$ by their suffixes of length up to $k$. For example, when $k=2$, the analysis uses the following set $K_2$ of abstract contexts:
\[
K_2 = \myset{[8], [9], [8,4], [9,4], [4,4]}
\]
where an infinite number of contexts, $\myset{[8,4,4], [9,4,4], [8,4,4,4], [9,4,4,4],\dots}$, in $K_{\infty}$ are approximated by their common suffix $[4,4]$. The analysis analyzes the method \texttt{id} separately for each context in $K_2$. % producing the call-graph in Fig.~\ref{fig:overview}(b).
Although this approximation ensures termination, the analysis is now unable to differentiate the two separate calls to \texttt{id} at lines 8 and 9, causing the argument {\tt v} of {\tt id} to point to both objects {\tt A} and {\tt B} at the same time when the context becomes $[4,4]$. Thus, both objects get returned to both queries simultaneously, making the analysis fail to prove their cast safety.
Note that the analysis fails to prove the queries for any $k$ values, because all the context strings longer than $k$ are eventually
merged into a single context $[\underbrace{4,\dots,4}_{k}]$ (see
Fig.~\ref{fig:overview:kCFA}).


% the $k$-limited method is
% widely used in modern pointer
% analysis frameworks such as Doop~\cite{Bravenboer2009}.\minseok{redundant??}


% \minseok{
% 	Be aware that other kinds of context sensitivity like value context
% 	can prove the queries but	our goal is to overcome limitation of k-limiting context-sensitivity
% 	which is widely used.}
	
\myparagraph{Context Tunneling}

With context tunneling, however, even $1$-call-site sensitive analysis
becomes to prove the queries.
% Our idea to improve the analysis is to identify and maintain important
% context elements only.
The main weakness of the conventional
$k$-context-sensitive analysis is that it blindly updates the context
of a method at every call-site, thereby allowing important context
elements for the method to be easily overwritten by less important
ones.  In the example, distinguishing the two call-sites
at lines 8 and 9 is essential to proving the queries. However, this
information is eventually lost by repeatedly appending the less
important call-site 4 to the context of {\tt id}
(Fig.~\ref{fig:overview:kCFA}).
Our technique aims to overcome this weakness by maintaining important
context elements only during analysis. For example, we exclude the
call-site 4 from the context strings generated for method {\tt id} and
thus approximate the set $K_{\infty}$ to the set $K_T$ of contexts:
\[
K_T = \myset{[8], [9]}.
\]
Here abstract contexts $[8]$ and $[9]$ in $K_{T}$ denote the sets of concrete contexts
$\myset{[8], [8,4], [8,4,4], \dots}$ and
$\myset{[9], [9,4], [9,4,4], \dots}$ in $K_{\infty}$, respectively. The resulting
context-sensitive analysis with $K_{T}$ is able to prove the queries as it differentiates the two calls
to {\tt id} at lines 8 and 9 completely, producing the call-graph in
Fig.~\ref{fig:overview:tunneling}.

% With context tunneling, however, even $1$-call-site sensitive analysis
% is able to prove the queries.

\myparagraph{Challenge}

The example shows that the idea of context tunneling is potentially
powerful; even $1$-context-sensitivity with
tunneling can be as precise as
$\infty$-context-sensitivity. The goal of this paper is to maximize
this tremendous, yet untapped, potential of the existing $k$-context-sensitive analysis.
However, achieving effective context tunneling in practice is
challenging. The effectiveness of the technique is very sensitive
to the choice of important context elements.  For example,
choosing the call-site 4, rather than call-sites 8 and 9, does not
work for the example program.  In the worst case, the analysis
precision degenerates into the context-insensitive case (e.g. when
selecting no context elements at all), becoming even inferior to the
ordinary
$k$-context-sensitive analysis.  Coming up with a good
heuristic rule for choosing important context elements is nontrivial;
hand-crafting such a rule not only requires a huge amount of
engineering effort and domain knowledge but also likely fails to
maximize the potential of the technique.
% adaptable as a fixed heuristic rule tuned for a particular situation
% may not be effective for other programs, analyses, or queries.


\myparagraph{Data-Driven Context Tunneling}


We address this challenge with a data-driven approach that
automatically learns a good heuristic rule for context tunneling from
a training set of programs.
Our aim is to generate a heuristic $\heuristic$,
% To do so, we first define a parameterized heuristic $\heuristic_\Pi$
% (where $\Pi$ is a parameter). $\heuristic_\Pi$
which takes a program and returns a relation $\TunnelingRelation$ on methods.
The relation contains a method pair $(m_1, m_2)$ if the calling context of
$m_1$ is more important than that of $m_2$. Thus, if $(m_1, m_2) \in
\TunnelingRelation$, the analysis applies context tunneling when $m_2$ is
invoked; that is, $m_2$ inherits the context of $m_1$ ignoring the current
context element. For the $k$-call-site-sensitivity example, $\heuristic_\Pi$
produces the relation $\TunnelingRelation = \myset{({\tt id}, {\tt id})}$,
which means that tunneling is applied when {\tt id} is called from {\tt id}.
As a result, the call-site 4, on which {\tt id} is called from {\tt id}, will
not be used as important context elements during analysis.  When {\tt main}
calls {\tt id}, however, the analysis updates the contexts as the ordinary
analysis since $({\tt main}, {\tt id}) \not\in \TunnelingRelation$.  In our
approach, the heuristic is parameterized and the parameter determines the
contents of the relation $\TunnelingRelation$.  The goal of our data-driven
technique is to characterize the two classes of methods: 1) methods that
increase the analysis precision by passing their contexts to others, and 2)
methods that increase the analysis precision by inheriting contexts from
others. Section~\ref{sec:learning} defines the learning problem and presents
an efficient algorithm to solve the problem.


%\subsection{Object-Sensitivity}
%
%The use of context tunneling is not limited to call-site-sensitivity.
%% Context tunneling is also effective for other flavors of
%% context-sensitivity.
%Below, we describe a typical situation where object-sensitivity
%benefits from context
%tunneling.
%
%%\lstset{
  language=Java,
  aboveskip=3mm,
  belowskip=3mm,
  showstringspaces=false,
  columns=flexible,
  basicstyle={\scriptsize\ttfamily},
  numbers=left,
  numbersep=5pt,
  numberstyle=\small\color{gray},
  keywordstyle=\color{blue},
  commentstyle=\color{dkgreen},
  stringstyle=\color{mauve},
  breaklines=true,
  breakatwhitespace=true,
  tabsize=3
}
\begin{figure}
\begin{subfigure}[b]{.9\columnwidth}
\begin{multicols}{2}
\centering
\begin{lstlisting}
class A {} class B {}
class C {
  public static void main (){
    ArrayList al1 = new ArrayList();//AL1
    ArrayList al2 = new ArrayList();//AL2

    al1.add(new A());
    al2.add(new B());

    Itr it1 = al1.iterator();
    Itr it2 = al2.iterator();

    A a = (A)it1.next();  //Query 1
    B b = (B)it2.next();  //Query 2
  }
}

class ArrayList{
  Object[] elementData = new Object[10];
  int size = 0;
  void add(Object e){
    elementData[size++] = e;
  }
  Itr iterator(){
    return new Itr(); //IT
  }
  class Itr{
    int cursor = 0;
    Object next(){
      return elementData[cursor++];
    }
  }
}
\end{lstlisting}
\end{multicols}
\caption{Example code}
\label{fig:obj:code}
\end{subfigure}
\begin{multicols}{2}
\begin{subfigure}[b]{.9\columnwidth}
\begin{center}
\resizebox{\columnwidth}{!}{
\begin{tikzpicture}
    \node [block] (main) {{\tt main}\\$[*]$};
    \node [block, below left=1cm and 2.7cm of main] (add1) {{\tt add}\\$[AL2]$};
    \node [block, right=0.4cm of add1] (add2) {{\tt add}\\$[AL2]$};
    \node [block2, right=0.4cm of add2] (iterator1) {{\tt iterator}\\ $[AL1]$};
    \node [block2, right=0.4cm of iterator1] (iterator2) {{\tt iterator}\\$[AL2]$};
    \node [block, right=0.4cm of iterator2] (next) {{\tt next}\\$[IT]$};

    \path [line] (main) -- (add1);
    \path [line] (main) -- (add2);
    \path [line] (main) -- (iterator1);
    \path [line] (main) -- (iterator2);
    \path [line] (main) -- (next);

\end{tikzpicture}
}
\end{center}
\caption{Call-graph by $1$-object-sensitive analysis}
\label{fig:obj:callgraph}
\end{subfigure}
\vfill\null
\columnbreak
\begin{subfigure}[b]{.9\columnwidth}
\begin{center}
\resizebox{\columnwidth}{!}{
\begin{tikzpicture}
    \node [block] (main) {\tt main\\$[*]$};
    \node [block, below left=1cm and 2.7cm of main] (add1) {{\tt add}\\$[AL2]$};
    \node [block, right=0.4cm of add1] (add2) {\tt add\\$[AL2]$};
    \node [block2, right=0.4cm of add2] (iterator1) {\tt iterator\\$[AL1]$};
    \node [block2, right=0.4cm of iterator1] (iterator2) {\tt iterator\\$[AL2]$};
    \node [block, right=0.4cm of iterator2] (next1) {\tt next\\$[AL1]$};
    \node [block, above=0.4cm of next1] (next2) {\tt next\\$[AL2]$};

    \path [line] (main) -- (add1);
    \path [line] (main) -- (add2);
    \path [line] (main) -- (iterator1);
    \path [line] (main) -- (iterator2);
    \path [line] (main) -- (next1);
    \path [line] (main) -- (next2);

\end{tikzpicture}
}

\end{center}
\caption{Call-graph by $1$-object-sensitive analysis with tunneling}
\label{fig:obj:tunneling}
\end{subfigure}
\end{multicols}
\setlength{\abovecaptionskip}{-5pt}
\caption{Example code and callgraphs}
\label{fig:overviewObj}
\end{figure}

\myparagraph{Example Code for Object Sensitivity}
Example code in Fig.~\ref{fig:obj:code} present common pattern that 1-object-sensitive+heap
analysis lose precision. The main method use two data structure Arraylist, and it add
different data to each Arraylist. Main method makes iterator for each Arraylist and 
get data with iterators. The queries ask if two type casts at line 13 and 14 are
safe. 1-object-sensitive+heap analysis fails to prove the queries because
\texttt{next} method invoked in 13 and 14 are merged as
the method invocations have same context \texttt{IT}, 
constructing call-graph in Fig.~\ref{fig:obj:callgraph}. 

\myparagraph{Context Tunneling for Object sensitivity} 
With context tunneling, 1-object-sensitive+heap analysis can prove the queries. Note that Object sensitivity
has different flavor from call-site sensitivity; it updates context from heap context. Base
variable \texttt{it1} and \texttt{it2} point to same heap allocation but have different heap context: 
\texttt{AL1} and \texttt{AL2}. Applying context tunneling for \texttt{next} method make method call
at line 13 have context \texttt{AL1} and 14 have context \texttt{AL2},
resulting context sensitive analysis prove the queries with call-graph 
Fig.~\ref{fig:obj:tunneling}. 


%%% Local Variables:
%%% mode: latex
%%% TeX-master: "paper"
%%% End:

%\lstset{
%  language=Java,
%  aboveskip=3mm,
%  belowskip=3mm,
%  showstringspaces=false,
%  columns=flexible,
%  basicstyle={\scriptsize\ttfamily},
%  numbers=left,
%  numbersep=5pt,
%  numberstyle=\small\color{gray},
%  keywordstyle=\color{blue},
%  commentstyle=\color{dkgreen},
%  stringstyle=\color{mauve},
%  breaklines=true,
%  breakatwhitespace=true,
%  tabsize=3
%}
%\begin{figure}
%\begin{subfigure}[b]{0.9\columnwidth}
%\begin{multicols}{2}
%\centering
%\begin{lstlisting}
%class A {} class B {}
%class C {
%  public static void main (){
%    ArrayList al1 = new ArrayList();//AL1
%    ArrayList al2 = new ArrayList();//AL2
%
%    al1.add(new A());
%    al2.add(new B());
%
%    ArrayList.ListItr it1 = al1.iterator();
%    ArrayList.ListItr it2 = al2.iterator();
%
%    A a = (A)it1.next();  //Query 1
%    B b = (B)it2.next();  //Query 2
%  }
%}
%
%class ArrayList{
%  Object[] elementData = new Object[10];
%  int size = 0;
%  void add(Object e){
%    elementData[size++] = e;
%  }
%  ListItr iterator(){
%    return new ListItr(); //IT
%  }
%  class ListItr{
%    int cursor = 0;
%    Object next(){
%      return elementData[cursor++];
%    }
%  }
%}
%\end{lstlisting}
%\end{multicols}
%\caption{Example code}
%\label{fig:obj:code}
%\end{subfigure}
%\begin{multicols}{2}
%\begin{subfigure}[b]{.9\columnwidth}
%\begin{center}
%\resizebox{\columnwidth}{!}{
%\begin{tikzpicture}
%    \node [block] (main) {{\tt main}\\$[\cdot]$};
%    \node [block, below left=1cm and 2.7cm of main] (add1) {{\tt add}\\$[AL1]$};
%    \node [block, right=0.4cm of add1] (add2) {{\tt add}\\$[AL2]$};
%    \node [block2, right=0.4cm of add2] (iterator1) {{\tt iterator}\\ $[AL1]$};
%    \node [block2, right=0.4cm of iterator1] (iterator2) {{\tt iterator}\\$[AL2]$};
%    \node [block, right=0.4cm of iterator2] (next) {{\tt next}\\$[IT]$};
%
%    \path [line] (main) -- (add1);
%    \path [line] (main) -- (add2);
%    \path [line] (main) -- (iterator1);
%    \path [line] (main) -- (iterator2);
%    \path [line] (main) -- (next);
%
%\end{tikzpicture}
%}
%\end{center}
%\caption{Call-graph by $1$-object-sensitive analysis}
%\label{fig:obj:callgraph}
%\end{subfigure}
%\vfill\null
%\columnbreak
%\begin{subfigure}[b]{.9\columnwidth}
%\begin{center}
%\resizebox{\columnwidth}{!}{
%\begin{tikzpicture}
%    \node [block] (main) {\tt main\\$[\cdot]$};
%    \node [block, below left=1cm and 2.7cm of main] (add1) {{\tt add}\\$[AL1]$};
%    \node [block, right=0.4cm of add1] (add2) {\tt add\\$[AL2]$};
%    \node [block2, right=0.4cm of add2] (iterator1) {\tt iterator\\$[AL1]$};
%    \node [block2, right=0.4cm of iterator1] (iterator2) {\tt iterator\\$[AL2]$};
%    \node [block, right=0.4cm of iterator2] (next1) {\tt next\\$[AL1]$};
%    \node [block, above=0.4cm of next1] (next2) {\tt next\\$[AL2]$};
%
%    \path [line] (main) -- (add1);
%    \path [line] (main) -- (add2);
%    \path [line] (main) -- (iterator1);
%    \path [line] (main) -- (iterator2);
%    \path [line] (main) -- (next1);
%    \path [line] (main) -- (next2);
%
%\end{tikzpicture}
%}
%
%\end{center}
%\caption{Call-graph by $1$-object-sensitive analysis with tunneling}
%\label{fig:obj:tunneling}
%\end{subfigure}
%\end{multicols}
%\setlength{\abovecaptionskip}{-5pt}
%\caption{Example and call-graphs constructed by the conventional 1-object-sensitivity and our
%1-object-sensitivity with context tunneling}
%\label{fig:overviewObj}
%\end{figure}
%
%\myparagraph{Example Code}
%
%Fig.~\ref{fig:obj:code} shows
%a common code pattern that $1$-object-sensitivity with a context-sensitive heap
%loses precision, which is found frequently in data
%structure implementations such as List and Map.
%In the example code, we have four
%top-level classes, \texttt{A}, \texttt{B}, \texttt{C}, and
%\texttt{ArrayList}, and an inner class \texttt{ListIter} inside of the
%\texttt{ArrayList} class. The \texttt{ArrayList} class % , as the name suggests,
%% models on \texttt{java.util.ArrayList}, which is
%implements a resizable array.
%For simplicity, it maintains data in the % (fixed-size)
%\texttt{elementData} array and has the \texttt{add} method to append new data at
%the end of \texttt{elementData}. It also has the \texttt{iterator} method that
%returns an object of type \texttt{ListIter}, providing access to \texttt{elementData}.
%The \texttt{main} method in class \texttt{C} performs usual
%allocate-add-retrieve
%tasks using \texttt{ListIter}. The goal of the pointer
%analysis is to prove the safety of two type casts at lines 13
%and 14.
%
%\myparagraph{Conventional Object-Sensitivity}
%
%Conventional object-sensitivity analyzes different method calls
%separately for their base objects (and their heap contexts, if needed)~\cite{Smaragdakis2011}.  For
%example, when we analyze the \texttt{next} method at line 13,
%2-object-sensitivity creates the context $[\texttt{AL1},
%\texttt{IT}]$, where \texttt{IT} is the allocation-site of the base
%object (\texttt{it1}) and \texttt{AL1} is its heap context.
%Likewise, the
%method invocation at line 14 creates the context $[\texttt{AL2}, \texttt{IT}]$, enabling
%the analysis to distinguish the first and second calls to the same method. Using base objects
%as contexts makes object-sensitivity more favorable than
%call-site-sensitivity for object-oriented languages. However, it still
%updates contexts unconditionally at every method call and may lose important
%context elements when $k$ is not sufficiently large. For example, when we
%analyze the code with 1-object-sensitivity, the two calls to
%\texttt{next} are analyzed under the same context $[\texttt{IT}]$
%(Fig.~\ref{fig:obj:callgraph}).
%As a result, both objects \texttt{A} and \texttt{B} get returned to both
%queries simultaneously, making the analysis fail to prove the safety of type
%casting.
%
%
%
%\myparagraph{Object-Sensitivity with Context Tunneling}
%With context tunneling, however, even 1-object-sensitivity can prove the queries. Note that
%object-sensitivity has a different flavor from call-site-sensitivity; it
%updates the context with the heap context of the base object~\cite{Milanova2005, Milanova2002}. Base
%variables \texttt{it1} and \texttt{it2} point to the same heap allocation with different heap contexts: \texttt{AL1} and \texttt{AL2}. Applying context
%tunneling for the two method calls to \texttt{next} corresponds to
%propagating the heap contexts \texttt{AL1} and \texttt{AL2} at lines
%13 and 14, respectively, without modification.
%% makes the method call at line 13 to have
%% the context \texttt{AL1} and the method call at 14 to have the context \texttt{AL2}.
%The resulting context-sensitive analysis proves the queries with a call-graph
%shown in Fig.~\ref{fig:obj:tunneling}.


%%% Local Variables:
%%% mode: latex
%%% TeX-master: "paper"
%%% End:
