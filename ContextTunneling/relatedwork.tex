\section{Related Work}\label{sec:related}

Points-to analysis has a large body of past
literature\cite{Lhotak2006,Smaragdakis2015,hind2001pointer,Lhotak2008,Might2010,
 liang1999efficient,wilson1995efficient, liang2005evaluating, chatterjee1999relevant, Whaley2004, Shivers1988,Sharir1981, Milanova2005,Milanova2002, Smaragdakis2011, kastrinis2013hybrid}. Below, we discuss prior works that are closely related to
ours.

% been widely studied in past literature\cite{Lhotak2006,Smaragdakis2015,hind2001pointer,Lhotak2008,Might2010,
% liang1999efficient,wilson1995efficient, liang2005evaluating, chatterjee1999relevant, Whaley2004, Shivers1988,Sharir1981, Milanova2005,Milanova2002, Smaragdakis2011, kastrinis2013hybrid}.
% In this section, we discuss how this work contributes to context sensitive points-to anlaysis.

\subsection{Context-Sensitive Analysis} To our knowledge, all
existing techniques for context-sensitive analyses
(e.g.~\cite{Sharir1981,Milanova2002,Shivers1988,
  Whaley2004,Milanova2005,Smaragdakis2011, Khedker2008,
  Karkare2007,kastrinis2013hybrid,WeiR15,Smaragdakis2014,JeJeChOh17,Oh2014,TanLX16,PadhyeK13})
suffer from the problem we tackle in this paper. All of those
techniques update the calling contexts at every call-site and
therefore may lose important context elements when $k$-limiting is used.  For example,
techniques for varying context depths
adaptively~\cite{WeiR15,JeJeChOh17,Oh2014, Smaragdakis2014} have proven
their effectiveness for tuning the performance of context-sensitivity.  However,
the primary goal of these techniques is to selectively analyze methods
with appropriate $k$ values, rather than to be selective in context
construction.

% , instead of blindly applying $k$-context depth for all program
% parts and waste resources, making good prediction models that assign
% promising context depth for each method
% invocation. \citet{Smaragdakis2014} uses a manually-designed
% heuristic to identify ``break-even points'' of context sensitivity
% with pre-analysis, and then the main analysis selectively applies
% context-sensitivity with the prediction. \citet{JeJeChOh17}
% introduced a disjunctive model and a learning algorithm to infer
% such heuristics automatically from codebase. However, as far as
% those techniques rely on conventional $k$-limited
% context-sensitivity at their core, they suffer from the same problem
% that our paper concerns.

The technique, called \Bean, by~\citet{TanLX16} is similar to ours in
that it aims to improve precision of $k$-object-sensitive points-to
analysis without increasing $k$. The idea is to identify ``redundant''
context elements by running a pre-analysis and building a so-called
object allocation graph. An object allocation graph is analogous to a
call-graph in call-site-sensitivity and captures how context strings
are generated during an object-sensitive analysis. The technique
identifies redundant nodes in the graph, which can be removed without
reducing the number of distinct contexts; in fact, this technique
guarantees to result same or more distinct contexts than conventional
analysis with same $k$. The redundant nodes are excluded when
selecting heap contexts for an object. The main focus of \Bean~is in
choosing good heap contexts and therefore it still suffers from the
core problem we tackle in this paper; that is, \Bean~always append the
last context element to the parent context on method calls (i.e. it
uses the last rule in Figure~\ref{fig:baseline-rules}, not
Figure~\ref{fig:tunneling-rules}). As the result, when $k=1$ for
example, \Bean~analyzes every method under the same calling context as
the ordinary $1$-object-sensitive analysis. The goal of context
tunneling is to mitigate this problem.

Merging equivalent contexts~\cite{Xu2008} or abstract heaps~\cite{Tan2017} also have been studied to
increase scalability of context-sensitive points-to analysis. \citet{Xu2008} first defines a fine-grained
notion of context equivalence that guarantees to increase scalability without losing original analysis's
precision, which is $\infty$-context-sensitivity. Based on the equivalence, the paper introduces its abstracted
version to extend scalability even further at the cost of precision. \citet{Tan2017} merges two
abstract heaps if their fields-points-to-graphs indicate that any field access sequence (i.e.,
$o.f.g.\cdots .h$) ends up with same typed objects for two graphs. This approach demonstrated that it brings
little or no negative impact on precision but scales deep context
sensitive analysis.

% \citet{WeiR15} suggest a technique that choose a context from multiple kinds of context for each method to
% achieve precision. Such paramter share same goal of ours, but our technique uses a single kind of
% context to improve precision.


% Full context string approaches~\cite{Khedker2008, Karkare2007} has been used for a limited classes of data
% flow analysis problems for instance bit-vector problems. Naturally, various heuristics such as replacing
% consecutive context elements with the representative one~\cite{Khedker2008} or limiting occurrences of same
% context element~\cite{Karkare2007} have been proposed in order to mitigate possibly infinite length of call
% string thus poor scalability.


\subsection{Parametric and Data-Driven Program Analysis}

Our work presents a new instance of parametric program analysis.
Previously, parametric program analyses have been used for
context-sensitivity~\cite{Tan2017,Oh2014, JeJeChOh17, Liang2011learning, Zhang2014,Liang2011}, flow-sensitivity~\cite{Oh2015}, variable clustering for
relational analysis~\cite{Heo2016learning}, and widening
thresholds~\cite{cha2016learning}.
Typically, the goal of these analyses is to find a parameter which
is as scalable as possible while sacrificing as little precision as possible.
For example,
\citet{Oh2015,Oh2014,Smaragdakis2014, JeJeChOh17, Liang2011learning}
sacrifice the precision of full flow- or context-sensitivity to obtain
tractable scalability.
\citet{Heo2016learning} compromises the full precision of relational
analysis and cluster variables whose relationships should be kept
during analysis.
\citet{Hassanshahi2017} and \citet{Tan2017} aim to find appropriate heap
abstraction that scales well without losing precision too much.
In this paper, we propose a new knob, called context tunneling, which
is able to improve both precision and scalability of the conventional
context-sensitive analysis.

% Context tunneling is a unique knob to control both scalability
% and precision in context sensitive points to analysis. In parametric program analysis, analyzers are
% equipped with parameters and achieve cost effective analysis with suitable parameters.
% Most of the existing parameters in program
% analysis are designed to achieve scalability
% by sacrificing precision\cite{Oh2014,JeJeChOh17,Oh2015,Heo2016learning,cha2016learning}.
% \citet{Oh2015,Oh2014,Smaragdakis2014, Liang2011learning} use parameter that choose methods that should be analyzed
% context sensitively, \citet{JeJeChOh17} use parameter that assign context depth (0$\sim$k)for each method and
% parameters in \citet{Heo2016learning} cluster variables that analyzer should track the relations.
% \citet{Hassanshahi2017} use depth of heap contexts as a knob and the goal is to find appropriate heap
% context depth to each heap alloation site.
%  In such parameters, the goal of parameteric anlaysis is to find a parameter which
% are as scalable as possible while sacrificing precision as little as possible. Parameter in context
% tunneling, however, has different property that it is designed to achieve precision while not sacrificing
% scalability.

% \citet{Liang2011learning} defined the notion of minimal abstraction and suggested an algorithm to find the minimal
% abstractions for context-sensitivity in points-to analysis. However,
% the algorithm cannot be used for learning a program analysis
% heuristic.
% \citet{Zhang2014} proposed a CEGAR-based
% algorithm that learns a parameter for context-sensitive points-to analysis for Java. Both
% \citet{Liang2011learning} and \citet{Zhang2014} assume that abstractions are ordered with respect to
% precision like \citet{JeJeChOh17} while abstractions in context tunneling are not.


From the perspective of data-driven program analysis, our work
presents a new learning algorithm for disjunctive model that is
applicable even when the
underlying analysis is not monotone with respect to the parameter
space.
Recently, data-driven approaches to program analysis have been
popular to address the challenge of generating program-analysis
heuristics~\cite{JeJeChOh17,Oh2015,ChOhHeYa17,Heo2016learning,heo2016unsound,WeiR15}.
% Prior data-driven approaches in parametric program analysis are
% unable to learn suitable parameters in context tunneling. Data-driven program analysis aims to generate good
% program analysis heuristics automatically from
% codebase~\cite{JeJeChOh17,Oh2015,ChOhHeYa17,Heo2016learning,heo2016unsound,WeiR15}, and our work lies in
% this line of research.
In particular, \citet{JeJeChOh17} recently proposed an algorithm with
disjunctive model based on boolean formulas and used the model to
learn a heuristic to determine appropriate context depths for each
method. This algorithm drastically reduces the search space by
exploiting the monotonicity of the analysis (i.e. applying deeper context-sensitivity will never decrease precision). Our approach is based on the
disjunctive model proposed by~\cite{JeJeChOh17} but uses a
different, non-greedy algorithm for learning context tunneling
heuristics.
The learning algorithm by~\citet{Oh2015} is applicable to non-monotone
analyses, but its applicability is limited to analysis heuristics
based on the linear combination of input features.
\citet{ChOhHeYa17} solves an orthogonal problem, automatic
generation features for learning heuristics, where feature programs
are generated by running a program reducer and get abstracted to
features represented by abstract data-flow graphs. However, it remains
to be seen whether the technique is applicable beyond intraprocedural
settings.


% proposed an algorithm based on the simple linear
% combination of features and Bayesian optimization.
% learning strategy that use Bayesian optimization technique to
% learn parameters that use linear model.
% However, learning algorithms which learn linear combination do not
% work in our setting as we use different model\cite{Oh2015,Heo2016learning,heo2016unsound,cha2016learning}.
% \citet{JeJeChOh17} propose algorithms to learn parameter that assign appropriate context depth for each
% method. The algorithm drastically reduce search space by exploiting a property that applying deeper context
% sensitivity to methods do not lose precision. Although \citet{JeJeChOh17} use same model with ours, the
% algorithm is inappropriate to
% learn parameters in context tunneling as applying tunneling to methods may lose or gain precision.
% \citet{Liang2011learning} defines minimal abstraction and suggested an algorithm to finds the minimal
% abstractions of context-sensitivity in points-to analysis. \citet{Zhang2014} proposed a CEGAR-based
% algorithm that learns a parameter for context-sensitive points-to analysis for Java. Both
% \citet{Liang2011learning} and \citet{Zhang2014} assume that abstractions are ordered with respect to
% precision like \citet{JeJeChOh17} while abstractions in context tunneling are not.



%\myparagraph{Parametric program analysis}






% \myparagraph{Demand-driven Program Analysis}
% As we aim to analyze all queries in target programs, our approach differs from existing work on
% demand-driven program analysis~\cite{Guyer2003,
% Sridharan2006, Sridharan2005}. \citet{Yan2011} proposed demand-driven
% approach that improve scalability in CFL-reachability analysis. It achieves scalability by summarizing procedural reachability for frequently used methods
% during analysis and apply the summaries when CFL-reachability needs to enter the methods.
% % \citet{Allen2015} combine type analysis with points-to analysis for library codes to approximate behaviors of appli
%cations that use the libraries.
% \citet{Xu2009} improve performance of CFL-reachability analysis by using pre-analysis that compute
% must-not-alias information. Main analysis use pre-calculated information to prune out infeasible paths.


\begin{comment}
\begin{itemize}
\item{\citet{Hassanshahi2017} use selective object-sensitivity as a knob and the goal is to find appropriate heap context depth to each heap alloation site, and \citet{Hassanshahi2017} employ pre-analysis to identify program parts. -O}
\item{\citet{Yan2011} proposed demand-driven approach that improve scalability in CFL-reachability analysis.
\cite{Yan2011} achieve analysis scalability by summarizing procedural reachability for heavily used methods
during analysis and apply the summary when CFL-reachability needs to enter the methods. O}
\item{\citet{Xu2009} improve preformance of CFL-reachability analysis by using pre-analysis that compute
must-not-alias information. Main analysis use pre-calculated information to prune out infeasible paths. -O}
\item{\citet{Hassanshahi2017} use selective object-sensitivity as a knob and the goal is to find appropriate heap context depth to each heap alloation site, and \citet{Hassanshahi2017} employ pre-analysis to identify program parts. -O}
\item{\cite{Allen2015} combine type analysis with points to analysis for library codes to approximate behaviors of applications that use the libraries. O}

\item{To improve precision, \citet{Khedker2012} combined points-to anlaysis with liveness analysis resulting
precision improvement. X}
\item{\cite{Avots2005} states that pointer analysis can help bug detector for C by reducing overhead.X}
\item{\citet{Lam2005} propose hybrid pointer alias analysis that apply precise analysis for some parts
while  the other parts are being analyzed imprecisely. \cite{Lam2005} also introduce an unsound assumption
to reduce false positives and achieve scalability. X}
\item{\cite{Livshits2003}X}
\item{\cite{bodden2008object}}
\end{itemize}
\end{comment}
%\cite{Yan2011, Xu2009, Lam2005, Hassanshahi2017, Dietrich2015, Allen2015, Khedker2012, Avots2005, Livshits2003, bodden2008object}

%%%% Local Variables:
%%%% mode: latex
%%%% TeX-master: "paper"
%%%% End:
