\lstset{
  language=Java,
  aboveskip=3mm,
  belowskip=3mm,
  showstringspaces=false,
  columns=flexible,
  basicstyle={\scriptsize\ttfamily},
  numbers=left,
  numbersep=5pt,
  numberstyle=\small\color{gray},
  keywordstyle=\color{blue},
  commentstyle=\color{dkgreen},
  stringstyle=\color{mauve},
  breaklines=true,
  breakatwhitespace=true,
  tabsize=3
}
\begin{figure}
\begin{subfigure}[b]{.9\columnwidth}
\begin{multicols}{2}
\centering
\begin{lstlisting}
class A {} class B {}
class C {
  public static void main (){
    ArrayList al1 = new ArrayList();//AL1
    ArrayList al2 = new ArrayList();//AL2

    al1.add(new A());
    al2.add(new B());

    Itr it1 = al1.iterator();
    Itr it2 = al2.iterator();

    A a = (A)it1.next();  //Query 1
    B b = (B)it2.next();  //Query 2
  }
}

class ArrayList{
  Object[] elementData = new Object[10];
  int size = 0;
  void add(Object e){
    elementData[size++] = e;
  }
  Itr iterator(){
    return new Itr(); //IT
  }
  class Itr{
    int cursor = 0;
    Object next(){
      return elementData[cursor++];
    }
  }
}
\end{lstlisting}
\end{multicols}
\caption{Example code}
\label{fig:obj:code}
\end{subfigure}
\begin{multicols}{2}
\begin{subfigure}[b]{.9\columnwidth}
\begin{center}
\resizebox{\columnwidth}{!}{
\begin{tikzpicture}
    \node [block] (main) {{\tt main}\\$[*]$};
    \node [block, below left=1cm and 2.7cm of main] (add1) {{\tt add}\\$[AL2]$};
    \node [block, right=0.4cm of add1] (add2) {{\tt add}\\$[AL2]$};
    \node [block2, right=0.4cm of add2] (iterator1) {{\tt iterator}\\ $[AL1]$};
    \node [block2, right=0.4cm of iterator1] (iterator2) {{\tt iterator}\\$[AL2]$};
    \node [block, right=0.4cm of iterator2] (next) {{\tt next}\\$[IT]$};

    \path [line] (main) -- (add1);
    \path [line] (main) -- (add2);
    \path [line] (main) -- (iterator1);
    \path [line] (main) -- (iterator2);
    \path [line] (main) -- (next);

\end{tikzpicture}
}
\end{center}
\caption{Call-graph by $1$-object-sensitive analysis}
\label{fig:obj:callgraph}
\end{subfigure}
\vfill\null
\columnbreak
\begin{subfigure}[b]{.9\columnwidth}
\begin{center}
\resizebox{\columnwidth}{!}{
\begin{tikzpicture}
    \node [block] (main) {\tt main\\$[*]$};
    \node [block, below left=1cm and 2.7cm of main] (add1) {{\tt add}\\$[AL2]$};
    \node [block, right=0.4cm of add1] (add2) {\tt add\\$[AL2]$};
    \node [block2, right=0.4cm of add2] (iterator1) {\tt iterator\\$[AL1]$};
    \node [block2, right=0.4cm of iterator1] (iterator2) {\tt iterator\\$[AL2]$};
    \node [block, right=0.4cm of iterator2] (next1) {\tt next\\$[AL1]$};
    \node [block, above=0.4cm of next1] (next2) {\tt next\\$[AL2]$};

    \path [line] (main) -- (add1);
    \path [line] (main) -- (add2);
    \path [line] (main) -- (iterator1);
    \path [line] (main) -- (iterator2);
    \path [line] (main) -- (next1);
    \path [line] (main) -- (next2);

\end{tikzpicture}
}

\end{center}
\caption{Call-graph by $1$-object-sensitive analysis with tunneling}
\label{fig:obj:tunneling}
\end{subfigure}
\end{multicols}
\setlength{\abovecaptionskip}{-5pt}
\caption{Example code and callgraphs}
\label{fig:overviewObj}
\end{figure}

\myparagraph{Example Code for Object Sensitivity}
Example code in Fig.~\ref{fig:obj:code} present common pattern that 1-object-sensitive+heap
analysis lose precision. The main method use two data structure Arraylist, and it add
different data to each Arraylist. Main method makes iterator for each Arraylist and 
get data with iterators. The queries ask if two type casts at line 13 and 14 are
safe. 1-object-sensitive+heap analysis fails to prove the queries because
\texttt{next} method invoked in 13 and 14 are merged as
the method invocations have same context \texttt{IT}, 
constructing call-graph in Fig.~\ref{fig:obj:callgraph}. 

\myparagraph{Context Tunneling for Object sensitivity} 
With context tunneling, 1-object-sensitive+heap analysis can prove the queries. Note that Object sensitivity
has different flavor from call-site sensitivity; it updates context from heap context. Base
variable \texttt{it1} and \texttt{it2} point to same heap allocation but have different heap context: 
\texttt{AL1} and \texttt{AL2}. Applying context tunneling for \texttt{next} method make method call
at line 13 have context \texttt{AL1} and 14 have context \texttt{AL2},
resulting context sensitive analysis prove the queries with call-graph 
Fig.~\ref{fig:obj:tunneling}. 


%%% Local Variables:
%%% mode: latex
%%% TeX-master: "paper"
%%% End:
