%\newpage

\section{Pointer Analysis with Context Tunneling} \label{sec:tunneling}
Now, we formally describe the idea of context tunneling on the top the analysis rule in Figure~\ref{pre:baseline-rules}.
%\subsection{Analysis with Context Tunneling}\label{sec:tunneling-analysis}
%Now we describe context tunneling on top of the generic context-sensitive analysis.
We assume that a binary relation $\TunnelingRelation$ on methods,
called tunneling relation, is given prior to the analysis:
% The idea of tunneling is simple: compare the
% parent and child contexts and use the more important one as the
% calling context of the called method.  We do so by using a binary
% relation
% $\TunnelingRelation$ on methods, called tunneling relation:
\[
  \TunnelingRelation \subseteq \Methods \times \Methods.
\]
Intuitively, $\TunnelingRelation$ relates the parent and child methods
that require context tunneling. Suppose that a method
$m_2$ is called during analysis where its parent method is $m_1$. If
$m_1$ and $m_2$ are related by $\TunnelingRelation$, i.e., $(m_1,m_2) \in \TunnelingRelation$, we apply context tunneling by
reusing the context of the parent method ($m_1$) in analyzing the
child method ($m_2$) without creating a new context for $m_2$.  Given
a tunneling relation $\TunnelingRelation$, the analysis rule for
performing context tunneling is defined in
Figure~\ref{fig:tunneling-rules}. All the existing rules except for
method calls are carried over to the new analysis without any modifications.
The rule for method call in Figure~\ref{pre:baseline-rules} is
replaced by the rule in Figure~\ref{fig:tunneling-rules}, where the
only difference is in the way of constructing the child context
$\ctx'$. If $(p,m) \not\in \TunnelingRelation$ (i.e. tunneling
disabled), we update the calling context as the conventional $k$-context-sensitive
analysis does. Otherwise, when $(p,m)\in \TunnelingRelation$
(i.e. tunneling enabled), we just pass the parent context $\pctx$ to
the child method (i.e. $\ctx' = \pctx$) without appending the current
context element to $\pctx$.
% Note that the ordinary $k$-context-sensitive analysis and
% context-insensitive analysis are special cases with
% $\TunnelingRelation = \emptyset$ and
% $\TunnelingRelation = \Methods \times \Methods$, respectively.


Although the idea of context tunneling is simple, realizing effective
tunneling in practice is challenging. We found that the performance of
context tunneling is very sensitive to the choice of the tunneling
relation; with a good relation, context tunneling is able to improve the
baseline analysis significantly but, with an inappropriate choice, the analysis
becomes even inferior to the context-insensitive analysis.  Manually finding a general heuristic rule to
generate an optimal tunneling relation for a given program was
difficult for real-world Java programs.
The goal of Section~\ref{tunneling:learning} is to address this challenge
with a data-driven technique that automatically learns a good
tunneling heuristic from data.

\begin{figure}[t]
  \[
    \begin{array}{c}
      \infer[]
      {
      \begin{array}{c}
        \ctx' \in \Reachable ({m}) \quad
        ({\heap}, {\hctx}) \in  \VarPointsTo ({m_{\this}}, {\ctx'}) \\
        \VarPointsTo({\it arg}, \ctx) \subseteq \VarPointsTo(m_{\it param}, \ctx')
        \quad
        \VarPointsTo(m_{\it return}, \ctx') \subseteq \VarPointsTo(x, \ctx)\\
        (\invo, \ctx, m, \ctx') \in \CallGraph
      \end{array}
      }{
      \begin{array}{c}
        {\it \invo : x = y.\sig({\it arg})} \quad
        \ctx \in    \Reachable ({\methof(\invo)}) \\
        ({\heap}, {\hctx})\in  \VarPointsTo ({\it y},\ctx) \quad
        t = {\typeof}(\heap)  \quad m = {\lookup}(t, \sig) \\
(e, {\pctx}, p) = \parent(\heap,\hctx, \invo, \ctx) \quad
        \ctx' = \left\{
        \begin{array}{l@{\quad}l}
          \truncateCtx{\appendCtx{{ \pctx}}{{e}}}{\CallDepth}
          & \mbox{if}~(p,m)\not\in \TunnelingRelation \\
          \truncateCtx{\pctx}{\CallDepth}     & \mbox{if}~(p,m)
                                                \in \TunnelingRelation
        \end{array}
                                                          \right.
      \end{array}
                                                          }
    \end{array}
  \]
  \caption{Context-sensitive pointer analysis with tunneling}
\label{fig:tunneling-rules}
\end{figure}

%%% Local Variables:
%%% mode: latex
%%% TeX-master: "paper"
%%% End:
