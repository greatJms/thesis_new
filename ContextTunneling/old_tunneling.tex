

\commentout
{
\newpage
\section{Context Tunneling} \label{sec:tunneling}

\subsection{Baseline Context-Sensitive Points-to Analysis}

% To make this paper self-contained, 
We first describe the baseline points-to analysis
for Java~\cite{Smaragdakis2015}, which our work builds on. 
It is a flow-insensitive, context-sensitive, and
field-sensitive subset-based points-to analysis with on-the-fly
call-graph construction. 
The analysis is specified in Datalog, for which we use the same notations and
rules introduced by~\citet{kastrinis2013hybrid}. 

\paragraph{Program Representation}
Given a Java program, we define the following sets of program elements: 
\begin{itemize}
\item $V$: the set of program variables. 
\item $H$: the set of allocation-sites (i.e. abstract heaps).
\item $M$: the set of method names.
\item $S$: the set of method signatures (method name and type).
\item $F$: the set of field names.
\item $I$: the set of instructions.
\item $T$: the set of class types.
% \item $\mbn$: the set of natural numbers.
% \item $C$ is a set of calling contexts.
% \item $HC$ is a set of heap contexts.
\end{itemize}
We assume auxiliary functions $\classof: H \to T$ and $\methof: (H+I) \to M$, which map heaps or instructions to their
containing classes (types) and methods, respectively.  A Java program is represented
by a set of the following input relations:
\begin{itemize}
\item $\Alloc ({\it var}: V, {\it heap}: H, {\it inMeth}: M)$
\item $\Move ({\it to}: V, {\it from}: V)$
\item $\Load ({\it to}: V, {\it base}: V, {\it fld}: F)$
\item $\Store ({\it base}: V, {\it fld}: F, {\it from}: V)$
\item
  $\VCall ({\it base}: V, {\it sig}: S, {\it invo}: I, {\it inMeth}:
  M)$
\item $\FormalArg ({\it meth}: M, {\it i}: \mbn, {\it arg}: V)$
\item $\ActualArg ({\it invo}: I, {\it i}: \mbn, {\it arg}: V)$
\item $\ThisVar ({\it meth}: M, {\it this}: V)$
\item $\HeapType ({\it heap}: H, {\it type}: T)$
\item $\LookUp ({\it type}: T, {\it sig}: S, {\it meth}: M)$
\end{itemize}
$\Alloc ({\it var}, {\it heap}, {\it inMeth})$ models object
allocation. An allocated object is represented by the allocation-site
{\it heap} and pointed to by the variable {\it var}. The method {\it
  inMeth} is what encloses the allocation statement.
$\Move ({\it to}, {\it from})$ assigns the value of {\it from}
to the variable {\it to}.
$\Load ({\it to}, {\it base}, {\it fld})$ represents a load
instruction, i.e., ${\it to} = {\it base}.{\it fld}$.
$\Store ({\it base}, {\it fld}, {\it from})$ represents a
store instruction, i.e., ${\it base}.{\it fld} = {\it from}$.
$\VCall ({\it base}, {\it sig}, {\it invo}, {\it inMeth})$
represents a virtual call, i.e., ${\it base}.{\it sig}(...)$, where
{\it invo} denotes the call instruction and {\it inMeth} is the method
containing the instruction.
$\FormalArg ({\it meth}, {\it i}, {\it arg})$ and
$\ActualArg ({\it invo}, {\it i}, {\it arg})$ indicate
that the variable {\it arg} is the ${\it i}^{\it th}$ formal and
actual argument of the method {\it meth}, respectively.
$\ThisVar ({\it meth}, {\it this})$ maps methods to their {\it
  this} variables.  $\HeapType ({\it heap}, {\it type})$ maps
allocation-sites to the types of allocated objects.
$\LookUp ({\it type}, {\it sig}, {\it meth})$ indicates that
the method {\it meth} has the signature {\it sig} inside the class {\it type}.

% \[
%   \begin{array}{l}
% \Alloc ({\it var}: V, {\it heap}: H, {\it inMeth}: M) \\ 
% \Move ({\it to}: V, {\it from}: V) \\
% \Load ({\it to}: V, {\it base}: V, {\it fld}: F) \\
% \Store ({\it base}: V, {\it fld}: F, {\it from}: V) \\
% \VCall ({\it base}: V, {\it sig}: S, {\it invo}: I, {\it inMeth}: M) \\
% % \SCall ({\it meth}: M, {\it invo}: I, {\it inMeth}: M) \\
% \FormalArg ({\it meth}: M, {\it i}: \mbn, {\it arg}: V) \\
% \ActualArg ({\it invo}: I, {\it i}: \mbn, {\it arg}: V) \\
% \FormalReturn ({\it meth}: M, {\it ret}: V) \\
% \ActualReturn ({\it invo}: I, {\it var}: V) \\
% \ThisVar ({\it meth}: M, {\it this}: V) \\
% \HeapType ({\it heap}: H, {\it type}: T)\\
% \LookUp ({\it type}: T, {\it sig}: S, {\it meth}: M)
% \end{array}
% \]

\paragraph{Analysis Output}
Given a set of input relations, 
the analysis produces output relations defined as follows: 
\begin{itemize}
\item $\VarPointsTo({\it var}: V, {\it ctx}: C, {\it heap}: H, {\it hctx}: HC)$
\item
  $\CallGraph({\it invo}: I, {\it callerCtx}: C, {\it meth}: M, {\it
    calleeCtx}: C)$
\item
  $\FldPointsTo({\it baseH}: H, {\it baseHCtx}: HC, {\it fld}: F, {\it
    heap}: H, {\it hctx}: HC)$
\item
  $\InterProcAssign({\it to}:V, {\it toCtx}: C, {\it from}: V, {\it
    fromCtx}: C)$
\item $\Reachable({\it meth}: M,{\it ctx}: C)$
\end{itemize}
Here we assume the sets $C$ and $HC$ of calling and heap contexts
that will be defined later for each context-sensitivity flavor.  The
primary outcomes of the analysis are the $\VarPointsTo$ and
$\CallGraph$ relations.
$\VarPointsTo({\it var}, {\it ctx}, {\it heap}, {\it hctx})$ means
that the local variable {\it var} under the calling context {\it ctx}
may point to the object {\it heap} of the heap context {\it hctx}.
The $\CallGraph$ relation asserts that the method {\it meth} whose
context is {\it calleeCtx} can be called from the invocation site {\it
  invo} under the calling context {\it callerCtx}.  Others are
intermediate relations necessary for computing $\VarPointsTo$ and
$\CallGraph$.
$\FldPointsTo({\it baseH}, {\it baseHCtx}, {\it fld}, {\it heap}, {\it
  hctx})$ indicates that the field {\it fld} of the object {\it baseH}
of the heap context {\it baseHCtx} may point to the object {\it heap}
of the heap context {\it hctx}.  $\InterProcAssign$ is used to handle
parameter passing and return-value propagation during method calls.
$\Reachable({\it meth},{\it ctx})$ indicates that the method {\it
  meth} under the calling context {\it ctx} is reachable from the
entry of the program.

% \[
%   \begin{array}{l}
% \VarPointsTo({\it var}: V, {\it ctx}: C, {\it heap}: H, {\it hctx}: HC)\\
% \CallGraph({\it invo}: I, {\it callerCtx}: C, {\it meth}: M, {\it calleeCtx}: C)\\
% \FldPointsTo({\it baseH}: H, {\it baseHCtx}: HC, {\it fld}: F, {\it heap}: H, {\it hctx}: HC)\\
% \InterProcAssign({\it to}:V, {\it toCtx}: C, {\it from}: V, {\it fromCtx}: C)\\
% \Reachable({\it meth}: M,{\it ctx}: C)
% \end{array}
% \]

\paragraph{Analysis Rules}
Figure~\ref{fig:baseline-rules} presents the analysis rules for
computing the output relations from input relations.  The analysis is
generic in terms of context-sensitivity, where various flavors of
context-sensitivity can be obtained by defining constructor functions
$\Record$ and $\Merge$ appropriately. $\Record$ and $\Merge$ are
responsible for generating new heap and calling contexts,
respectively. Given a heap ({\it heap}) and a calling context ({\it
  ctx}), $\Record({\it heap}, {\it ctx})$ creates a heap context ({\it
  hctx}). $\Merge$ takes information available at method calls
(i.e. the base object {\it heap}, its heap context {\it hctx}, the
invocation site {\it invo}, and the calling context {\it callerCtx}
under which the method call is being made), and creates the calling
context {\it calleeCtx} for the called method. Shortly, we will
provide their definitions for a number of existing context-sensitive
analyses.

Most of the rules are intuitive. For example, the first rule is
responsible for allocation statements; if {\it heap} is allocated and
assigned to {\it var} in {\it meth} and {\it meth} is reachable under
the context {\it ctx}, the variable {\it var} under {\it ctx} points
to {\it heap} with the heap context {\it hctx}. Note that the heap
context is generated by \Record. The most complex rule is for
$\VCall$. When the current method {\it inMeth} is reachable
($\Reachable({\it inMeth}, {\it callerCtx})$), we find the heap {\it
  heap} that the variable {\it base} refers to, the type {\it heapT}
of the heap, the called method {\it toMeth}, and its {\it this}
variable. Then, we update the $\Reachable$, $\VarPointsTo$, and
$\CallGraph$ relations. The context {\it calleeCtx} of the callee
method is generated by the \Merge~constructor. The last three rules
are responsible for interprocedural assignments. 

\begin{figure}[t]
\[
  \begin{array}{l}
    \Record({\it heap}, {\it ctx}) = {\it hctx}, \\
    \VarPointsTo({\it var}, {\it ctx}, {\it heap}, {\it hctx}) \leftarrow \\
    \quad  \Reachable({\it meth}, {\it ctx}),
    \Alloc({\it var}, {\it heap}, {\it meth}). \\[1em]

  \VarPointsTo({\it to}, {\it ctx}, {\it heap}, {\it hctx}) \leftarrow \\
   \quad \Move({\it to}, {\it from}),
    \VarPointsTo({\it from}, {\it ctx},{\it heap},{\it hctx}). \\[1em]

  \VarPointsTo({\it to}, {\it ctx}, {\it heap}, {\it hctx}) \leftarrow\\
  \quad \Load({\it to}, {\it base}, {\it fld}), \VarPointsTo({\it base}, {\it ctx}, {\it baseH}, {\it baseHCtx}), \\
  \quad \FldPointsTo({\it baseH}, {\it baseHCtx}, {\it fld}, {\it heap},
   {\it hctx}). \\[1em]


  \FldPointsTo({\it baseH}, {\it baseHCtx}, {\it fld},
   {\it heap}, {\it hctx}) \leftarrow \\
  \quad \Store({\it base}, {\it fld}, {\it from}),
  \VarPointsTo({\it from}, {\it ctx}, {\it heap}, {\it hctx}), \\
    \quad \VarPointsTo({\it base}, {\it ctx}, {\it baseH}, {\it baseHCtx}).\\[1em]

  \Merge ({\it heap}, {\it hctx}, {\it invo}, {\it callerCtx}) = {\it calleeCtx}, \\
  \Reachable ({\it toMeth}, {\it calleeCtx}), \\
  \VarPointsTo ({\it this}, {\it calleeCtx}, {\it heap}, {\it hctx}), \\
  \CallGraph ({\it invo}, {\it callerCtx}, {\it toMeth}, {\it calleeCtx})
  \leftarrow \\
    \quad \VCall ({\it base}, {\it sig}, {\it invo}, {\it inMeth}), 
     \Reachable ({\it inMeth}, {\it callerCtx}), \\
   \quad \VarPointsTo ({\it base}, {\it callerCtx}, {\it heap}, {\it hctx}), 
  \HeapType({\it heap}, {\it heapT}), \\
  \quad \LookUp ({\it heapT}, {\it sig}, {\it toMeth}),
  \ThisVar ({\it toMeth}, {\it this}).\\[1em]

  \InterProcAssign({\it to}, {\it calleeCtx}, {\it from}, {\it callerCtx}) \leftarrow\\
  \quad \CallGraph({\it invo}, {\it callerCtx}, {\it meth}, {\it calleeCtx}), \\
  \quad \FormalArg({\it meth}, {\it i}, {\it to}),
  \ActualArg({\it invo}, {\it i}, {\it from}).\\[1em]
  
  \InterProcAssign({\it to}, {\it callerCtx}, {\it from}, {\it calleeCtx})
  \leftarrow\\
  \quad \CallGraph({\it invo}, {\it callerCtx}, {\it meth}, {\it calleeCtx}), \\
  \quad \FormalReturn({\it meth}, {\it from}),
  \ActualReturn({\it invo}, {\it to}). \\[1em]

    \VarPointsTo({\it to}, {\it toCtx}, {\it heap}, {\it hctx}) \leftarrow \\
    \quad \InterProcAssign({\it to}, {\it toCtx}, {\it from}, {\it fromCtx}), \\
    \quad \VarPointsTo({\it from}, {\it fromCtx}, {\it heap}, {\it hctx}). 

  \end{array}
\]
\caption{Datalog rules for context-sensitive points-to analysis}
\label{fig:baseline-rules}
\end{figure}


\paragraph{Instances}

\begin{itemize}

\item \onecall~(1-call-site-sensitive): In a 1-call-site-sensitive
  analysis, a calling context of a method consists of a single invocation-site
  ($C = I$) where the method is called and the analysis does not distinguish different heap
  contexts ($HC = \myset{\star}$). Thus, the constructor functions are
  defined as follows: 
\[
\begin{array}{l}
\Record ({\it heap}, {\it ctx}) = \star\\
\Merge ({\it heap, hctx, invo, ctx}) = {\it invo} \\
\end{array}
\]

\item {\onecallH~(1-call-site-sensitive with a context-sensitive
    heap)}. $\onecallH$ is equivalent to $\onecall$ except that 
every heap object is qualified with the calling context of the method
that allocates the object: 
\[
\begin{array}{l}
\Record ({\it heap}, {\it ctx}) = {\it ctx}\\
\Merge ({\it heap, hctx, invo, ctx}) = {\it invo} \\
\end{array}
\]

\item {\twocallH~(2-call-site-sensitive with a context-sensitive
    heap)}. A calling context of \twocallH~consists of a pair of
  invocation-sites ($C = I\times I$) while a heap context is a single
  instruction ($H = I$).  The heap context is the invocation-site where
  the method containing the heap is called. The first element of a calling
  context is the invocation-site and the second is the invocation-site
  where the caller is called, thus maintaining two most recent
  call-sites as calling contexts: 
\[
\begin{array}{l}
% C = I \times I, \qquad HC = I\\
 \Record ({\it heap}, {\it ctx}) = \first({\it ctx})\\
 \Merge ({\it heap, hctx, invo, ctx}) = \pair({\it invo}, \first({\it ctx})) \\
\end{array}
\]

\item {\oneobjH~(1-object-sensitive with a context-sensitive heap)}.
The idea of object-sensitivity is to use base objects (receiver
objects) as calling contexts~\cite{Milanova2005}. $\oneobjH$ uses the allocation-site of the
base object as the context of the called method  ($C = H$) and, for
heap contexts ($HC = H$), it uses
the calling context of the allocating method (the allocation-site of the object on which the method
containing the heap is called): 
\[
\begin{array}{l}
%C = H, \qquad HC = H \\
\Record ({\it heap}, {\it ctx}) = {\it ctx}\\
\Merge ({\it heap, hctx, invo, ctx}) = {\it heap} \\
\end{array}
\]

\item {\twoobjH~(2-object-sensitive with a context-sensitive
    heap)}. $\twoobjH$ abstracts the base object as a pair of its
  allocation-site and heap context ($C = H \times H$). For the heap
  context, it uses the allocation-site of the base object of the
  allocating method ($HC = H$).
\[
\begin{array}{l}
%C = H \times H, \qquad HC = H\\
\Record ({\it heap}, {\it ctx}) = \first({\it ctx})\\
\Merge ({\it heap, hctx, invo, ctx}) = \pair({\it heap}, {\it hctx}) \\
\end{array}
\]

\item {\onetypeH~(1-type-sensitive with a context-sensitive heap)}. 
$\onetypeH$ is an abstraction of $\oneobjH$ in that an allocation-site
is abstracted to the class containing the
allocation-site ($C =T$, $HC=T$): 
\[
\begin{array}{l}
%C = T, \qquad HC = T \\
\Record ({\it heap}, {\it ctx}) = {\it ctx}\\
\Merge ({\it heap, hctx, invo, ctx}) = \classof({\it heap}) \\
\end{array}
\]


\item {\twotypeH~(2-type-sensitive with a context-sensitive heap)}
Similarly, $\twotypeH$ is an abstraction of $\twoobjH$ ($C = T$, $HC=T$): 
\[
\begin{array}{l}
%C = T \times T, \qquad HC = T \\
\Record ({\it heap}, {\it ctx}) = \first({\it ctx})\\
\Merge ({\it heap, hctx, invo, ctx}) = \pair(\classof({\it heap}), {\it hctx}) \\
\end{array}
\]
\end{itemize}




%\newpage
\subsection{Context-Sensitive Analysis with Tunneling}
\begin{figure}[t]
\[
\begin{array}{l}

  \Merge ({\it heap}, {\it hctx}, {\it invo}, {\it callerCtx}) = {\it calleeCtx}, \\
  \ContextMadeIn ({\it heap}, {\it hctx}, {\it invo}, {\it callerCtx}) = {\it parentMeth}, \\
  \Reachable ({\it toMeth}, {\it calleeCtx}), \\
  \VarPointsTo ({\it this}, {\it calleeCtx}, {\it heap}, {\it hctx}), \\
  \CallGraph ({\it invo}, {\it callerCtx}, {\it toMeth}, {\it calleeCtx})
  \leftarrow \\
    \quad \VCall ({\it base}, {\it sig}, {\it invo}, {\it inMeth}), 
     \Reachable ({\it inMeth}, {\it callerCtx}), \\
   \quad \VarPointsTo ({\it base}, {\it callerCtx}, {\it heap}, {\it hctx}), 
  \HeapType({\it heap}, {\it heapT}), \\
  \quad \LookUp ({\it heapT}, {\it sig}, {\it toMeth}),
  \ThisVar ({\it toMeth}, {\it this}),  \\
\quad  !\Tunneling({\it parentMeth}, {\it toMeth}).\\[1em]


  \MergeTunneled ({\it heap}, {\it hctx}, {\it invo}, {\it callerCtx}) = {\it parentCtx}, \\
  \ContextMadeIn ({\it heap}, {\it hctx}, {\it invo}, {\it callerCtx}) = {\it parentMeth}, \\
  \Reachable ({\it toMeth}, {\it parentCtx}), \\
  \VarPointsTo ({\it this}, {\it parentCtx}, {\it heap}, {\it hctx}), \\
  \CallGraph ({\it invo}, {\it callerCtx}, {\it toMeth}, {\it parentCtx})
  \leftarrow \\
    \quad \VCall ({\it base}, {\it sig}, {\it invo}, {\it inMeth}), 
     \Reachable ({\it inMeth}, {\it callerCtx}), \\
   \quad \VarPointsTo ({\it base}, {\it callerCtx}, {\it heap}, {\it hctx}), 
  \HeapType({\it heap}, {\it heapT}), \\
  \quad \LookUp ({\it heapT}, {\it sig}, {\it toMeth}),
  \ThisVar ({\it toMeth}, {\it this}),  \\
\quad  \Tunneling({\it parentMeth}, {\it toMeth}).
\end{array}
\]
\caption{Modified rules for tunneling}
\label{fig:tunneling-rules}
\end{figure}

% Figure~\ref{fig:tunneling-rules} presents the analysis rules for
% tunneling. 


% \[
% \begin{array}{ccc}
% {\it fromCtx} & \Rightarrow & {\it toCtx} \\
% \vdots & & \vdots \\
% {\it fromMeth} & & {\it toMeth}
% \end{array}
% \]

Now we extend the analysis rules in Figure~\ref{fig:baseline-rules} to
enable context tunneling. 

\paragraph{Analysis Rules}
The idea of tunneling is to compare the parent and child contexts and
use the more important one for the called method.  In Datalog, we make
such a decision with the $\Tunneling$ relations of the following form:
\[
\Tunneling({\parentMeth}: M, {\childMeth}: M).
\]
Given two methods {\parentMeth} and {\childMeth},
$\Tunneling ({\parentMeth}, {\childMeth})$ means that the calling
context of {\it parentMeth} is more important than that of
{\childMeth} and therefore should be reused for {\childMeth} without
creating a new context. In other words, we apply context tunneling for
$\childMeth$ when it is related with some {\parentMeth} via the
$\Tunneling$ relation.  For the moment, we assume that a set of the relations is given
prior to the analysis.  In Section~\ref{sec:learning}, we present a
machine-learning approach, where the relations will be automatically generated by a
heuristic rule learned from training programs.


Given the $\Tunneling$ relations, the analysis rules are defined as
presented in Figure~\ref{fig:tunneling-rules}. All the existing rules
except for method calls carry over to the new analysis without any
modifications.  The rule for method calls in
Figure~\ref{fig:baseline-rules} is replaced by the two rules in
Figure~\ref{fig:tunneling-rules}.  The first rule describes the
conventional context-sensitive analysis without tunneling.  When the
called method ${\it toMeth}$ does not have the $\Tunneling$ relation
with the parent method ${\parentMeth}$ (i.e.
$!\Tunneling({\parentMeth}, {\it toMeth})$), we create the new context
${\it calleeCtx}$ for the called method ${\it toMeth}$ by using the
ordinary $\Merge$ function. Otherwise, when ${\it toMeth}$ is related
with ${\parentMeth}$ (i.e. $\Tunneling({\parentMeth}, {\it childMeth})$), the analysis uses
the second rule, where we just pass the parent context ${\it parentCtx}$
to the child method ${\it toMeth}$ instead of creating a child context. The
functions $\ContextMadeIn$ and $\MergeTunneled$ of the form
\[
  \begin{array}{l}
    \MergeTunneled ({\it heap}: H, {\it hctx}: HC, {\it invo}: I, {\it
    ctx}: C) = {\it parentCtx}: C \\
    \ContextMadeIn ({\it heap}: H, {\it hctx}: HC, {\it invo}: I, {\it
    ctx}: C) = {\parentMeth}:M
  \end{array}
\]
are responsible
for finding out the parent method and context, respectively, which
should be defined for each flavor of context-sensitivity. 

\paragraph{Instances}

The notions of the parent context and parent method are different for each
flavor of baseline analysis. Below, we present a number of
tunneled variants of the existing context-sensitive analyses, which
are obtained by
defining $\ContextMadeIn$ and $\MergeTunneled$ appropriately. 

\begin{itemize}
\item {\it \onecallHT~(Tunneled 1-call-site sensitive with
    1-context-sensitive heap)}. The tunneled variant of
  \onecallH~is obtained by defining the functions as follows: 
\[
\begin{array}{l}
\MergeTunneled ({\it heap, hctx, invo, ctx}) = {\it ctx} \\
\ContextMadeIn ({\it heap}, {\it invo}) = \methof({\it invo})
\end{array}
\]
In call-site-sensitivity, the calling context of a callee method is
created by appending (with truncation) the current invocation site to the context of the
caller method. Thus, the parent method corresponds to the caller (i.e. the method
containing the invocation site) and the parent context is the calling
context of the caller method.

\item {\it \oneobjHT (Tunneled 1-object-sensitive with
    1-context-sensitive heap)}. 
In object-sensitivity, a new context is created by appending the
allocation-site of the receiver object to the calling context of the
allocating method of the object. Thus, the parent method and context
are the method containing the allocation-site and the receiver object's heap
context, respectively. The definitions are as follows: 
\[
\begin{array}{l}
\MergeTunneled ({\it heap, hctx, invo, ctx}) = {\it hctx} \\
\ContextMadeIn ({\it heap}, {\it invo}) = \methof({\it heap})
\end{array}
\]

\item {\it \onetypeHT (Tunneled 1-type-sensitive with
    1-context-sensitive heap)}
As type-sensitivity is an abstraction of object-sensitivity, the
definitions are the same: 
\[
\begin{array}{l}
% C = T, \qquad HC = T \\
% \Merge ({\it heap, hctx, invo, ctx}) = \classof({\it heap}) \\
\MergeTunneled ({\it heap, hctx, invo, ctx}) = {\it hctx} \\
\ContextMadeIn ({\it heap}, {\it invo}) = \methof({\it heap})
\end{array}
\]

\end{itemize}

\newpage
\paragraph{Principle}

Notation:
\[
\truncateCtx{l}{k} = \mbox{the first $k$ elements of $l$ }
\]

General $k$-limiting context-sensitive analysis with
$h$-context-sensitive heap: 
\[
\begin{array}{l}
\Record ({\it heap}, {\it ctx}) =  \truncateCtx{{\it ctx}}{h} \\
\Merge ({\it heap, hctx, invo, ctx}) =
  \truncateCtx{\appendCtx{{\ctxElem({\it heap},{\it
  invo})}}{{ \parentCtx({\it hctx},{\it ctx})}}}{k} \\
\MergeTunneled ({\it heap, hctx, invo, ctx}) = {\it ctx} \\
\ContextMadeIn ({\it heap}, {\it invo}) = \methof({\it invo})
\end{array}
\]

\begin{itemize}

\item $k$-call-site-sensitive:
\[
\begin{array}{l}
C = I^k, \qquad HC = I^{k}\\
\Record ({\it heap}, {\it ctx}) =  {\it ctx}\\
\Merge ({\it heap, hctx, invo, ctx}) = \truncateCtx{\appendCtx{{\it invo}}{{\it ctx}}}{k} \\
\end{array}
\]

\item $k$-call-site-sensitive with tunneling:
\[
\begin{array}{l}
C = I^k, \qquad HC = I^{k}\\
\Record ({\it heap}, {\it ctx}) =  {\it ctx}\\
\Merge ({\it heap, hctx, invo, ctx}) = \truncateCtx{\appendCtx{{\it invo}}{{\it ctx}}}{k} \\
\MergeTunneled ({\it heap, hctx, invo, ctx}) = {\it ctx} \\
\ContextMadeIn ({\it heap}, {\it invo}) = \methof({\it invo})
\end{array}
\]


\item $k$-object-sensitive:
\[
\begin{array}{l}
C = H^k, \qquad HC = H^{k}\\
\Record ({\it heap}, {\it ctx}) =  {\it ctx}\\
\Merge ({\it heap, hctx, invo, ctx}) = \truncateCtx{\appendCtx{{\it heap}}{{\it hctx}}}{k} \\
\end{array}
\]
\item $k$-object-sensitive with tunneling:
\[
\begin{array}{l}
C = H^k, \qquad HC = H^{k}\\
\Record ({\it heap}, {\it ctx}) =  {{\it ctx}}\\
\Merge ({\it heap, hctx, invo, ctx}) = \truncateCtx{\appendCtx{{\it heap}}{{\it hctx}}}{k} \\
\MergeTunneled ({\it heap, hctx, invo, ctx}) = {\it hctx} \\
\ContextMadeIn ({\it heap}, {\it invo}) = \methof({\it heap})
\end{array}
\]

\item $k$-type-sensitive:
\[
\begin{array}{l}
C = T^k, \qquad HC = T^k \\
\Record ({\it heap}, {\it ctx}) = {\it ctx}\\
\Merge ({\it heap, hctx, invo, ctx}) = \truncateCtx{\appendCtx{\classof({\it heap})}{{\it hctx}}}{k} \\
\end{array}
\]

\item $k$-type-sensitive with tunneling:
\[
\begin{array}{l}
C = T^k, \qquad HC = T^k \\
\Record ({\it heap}, {\it ctx}) = {\it ctx}\\
\Merge ({\it heap, hctx, invo, ctx}) =
  \truncateCtx{\appendCtx{\classof({\it heap})}{{\it hctx}}}{k} \\
\MergeTunneled ({\it heap, hctx, invo, ctx}) = {\it hctx} \\
\ContextMadeIn ({\it heap}, {\it invo}) = \methof({\it heap})
\end{array}
\]


\end{itemize}
}


