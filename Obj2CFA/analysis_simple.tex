% !TEX root = ./paper.tex
\begin{figure}[t]
  \[
    \begin{array}{c}
      \infer[]
      {
      \begin{array}{c}
        \ctx' \in \Reachable ({m}) \quad
        ({\heap}, {\hctx}) \in  \VarPointsTo ({m_{\this}}, {\ctx'}) \\
        \VarPointsTo({\it arg}, \ctx) \subseteq \VarPointsTo(m_{\it param}, \ctx')
        \quad
        \VarPointsTo(m_{\it return}, \ctx') \subseteq \VarPointsTo(x, \ctx)\\
        (\invo, \ctx, m, \ctx') \in \CallGraph
      \end{array}
      }{
      \begin{array}{c}
        {\it \invo : x = y.\sig({\it arg})} \quad
        \ctx \in    \Reachable ({\methof(\invo)}) \\
        ({\heap}, {\hctx})\in  \VarPointsTo ({\it y},\ctx) \quad
        t = {\typeof}(\heap)  \quad m = {\lookup}(t, \sig) \\
(e, {\pctx}, p) = \parent(\heap,\hctx, \invo, \ctx) \quad
        \ctx' = \left\{
        \begin{array}{l@{\quad}l}
          \truncateCtx{\appendCtx{{ \pctx}}{{e}}}{\CallDepth}
          & \mbox{if}~\invo\not\in T \\
          \truncateCtx{\pctx}{\CallDepth}     & \mbox{if}~\invo                       \in T
        \end{array}
                                                          \right.
      \end{array}
                                                          }
    \end{array}
  \]
  \caption{Context-sensitive pointer analysis using sets of invocation sites as tunneling abstractions}
\label{obj2cfa:fig:tunneling-rules}
\end{figure}

\section{Setup: Pointer Analysis with Context Tunneling}\label{sec:setting}

Unlike Section~{\ref{sec:tunneling}} that uses relations between methods as tunneling a abstraction ($\TunnelingRelation \subseteq \Methods \times \Methods$) for describing when to apply context tunneling, in this chapter we use a set of invocation sites as a tunenling abstraction $T$ as follows:
\[
  T \subseteq \mbi
\]
Then, the rule for method call in Figure~\ref{pre:baseline-rules} is replaced by the rule defined in Figure~\ref{obj2cfa:fig:tunneling-rules}.







\myparagraph{Instance Analyses}
Given  a program $P$ and 
its tunneling abstraction $T \subseteq \mbi_P$, 
we write $\call_k(P, T)$ and
$\obj_k(P, T)$ for the $k$-call-site- and
$k$-object-sensitive analyses, respectively. In the rest of this chapter, we fix $k$ and omit the subscript $k$ from $\call_k$ and $\obj_k$. 
%We assume that $\call(P, T)$ and $\obj(P, T)$ report as output some
%precision metric (e.g. the number of may-fail casts, where lower means better precision). 
%
%
%%\subsection{Context-Tunneling Policy}\label{sec:tunnelingpolicy}
%
These instance analyses are used with a {\em context-tunneling policy}. 
A context-tunneling policy $\heuristic$ is a function that maps a program $P$ into
a tunneling abstraction for $P$: 
\[
  \heuristic(P) \subseteq \mbi_P.
\]
With a policy $\heuristic$, 
we  perform the analysis for a program $P$ as follows: $\call(P, \heuristic(P))$ or $\obj(P, \heuristic(P))$. 


 % which first uses $\heuristic$ to produce a tunneling abstraction $\heuristic(P)$ and analyzes the program with the abstraction. 

%%% Local Variables:
%%% mode: latex
%%% TeX-master: "paper"
%%% End:
