% !TEX root = ../thesis.tex


\subsection{Simulation}\label{sec:simulation}

%A key insight behind our technique is that call-site sensitivity with
%context tunneling can simulate object sensitivity.  That is, 

The first technical contribution of this chapter is the simulation procedure. 
Given a program $P$ and its tunneling abstraction $\Tobj \subseteq \mbi_P$, where $\Tobj$ is given by $\objheuristic$, i.e., 
$\Tobj = \objheuristic(P)$,
the goal of simulation is to infer a tunneling abstraction
$\Tcall \subseteq \mbi_P$ such that $\call(P, \Tcall)$ becomes more
precise than ${\sf obj} (P, \Tobj)$.


% !TEX root = ./paper.tex


%basicstyle=\linespread{1.2}\small\ttfamily
\begin{figure}[t]
%\setlength{\columnsep}{0.07cm}
\begin{lstlisting}[multicols=2, escapeinside={(*}{*)},xleftmargin=4.0ex ,basicstyle={\linespread{1.2}\fontsize{9.5}{10}\selectfont\ttfamily}]
class D {
 Object id(v) {
  return v;}
 Object id1(v) {
  return this.id(v);}
 void m() {
  A a = (A)this.id(new A());//A1, qry1
  B b = (B)this.id(new B());//B1, qry2
 }
}
void main() {
 D d1 = new D();//D1
 D d2 = new D();//D2
 D d3 = new D();//D3 

 A a3 = (A)d1.id1(new A());//A2, qry3 
 B b3 = (B)d2.id1(new B());//B2, qry4
 d3.m();
}
\end{lstlisting}
\vspace{-15pt}
\caption{Running example}
\vspace{-15pt}
\label{simulation:example}
\end{figure}


%%% Local Variables:
%%% mode: latex
%%% TeX-master: "paper"
%%% End:


\myparagraph{Running Example}
\label{sec:simulation-overview}
We illustrate the simulation procedure with the
example program in Figure~\ref{simulation:example}.  The code
contains class {\tt D} that has three methods \texttt{id},
\texttt{id1}, and \texttt{m}.  Methods \texttt{id} and \texttt{id1}
are identity functions.
The method \texttt{m} contains two method invocations at lines 7 and
8, which call \texttt{id} with new \texttt{A} and \texttt{B} objects.
The \texttt{main} method creates three objects at the
allocation-sites \texttt{D1}, \texttt{D2}, and \texttt{D3}, and stores
them in variables \texttt{d1}, \texttt{d2} and \texttt{d3},
respectively.  At line 16, \texttt{main} calls \texttt{id1} with
a new object of type {\tt A} and the base variable \texttt{d1}.  At line 17,
\texttt{id1} is called with a new object with type {\tt B} and base variable
\texttt{d2}.  At line 18, \texttt{main} also calls {\tt m} with base
variable \texttt{d3}.  We assume that the code has four queries, which ask the
safety of casting operations at lines 7, 8, 16, and 17. Note that all
of these are safe since {\tt id} and {\tt id1} are identity
functions.








In this example, for simplicity, we assume an 1-object-sensitive analysis
without context tunneling (i.e. $\Tobj = \emptyset$) is given but our technique is applicable
to object sensitivity with arbitrary $k$ and tunneling abstraction $\Tobj \subseteq \mbi_P$. 
Figure~\ref{fig:obj:callgraph} shows the call-graph produced by the
baseline 1-object-senstivie analysis, 
where a call-graph edge is represented by invocation-site, caller method, caller context, 
callee method, and callee context. For example, the edge
$\texttt{id1[D1]} \stackrel{5}{\to} \texttt{id[D1]}$ indicates that
method {\tt id} is called from {\tt id1} at invocation-site $5$,
where the callee and caller contexts are {\tt D1}.  Note that
this object-sensitive analysis is not precise enough to prove all queries.
Although it can prove queries at lines 16 and
17 as it distinguishes the two different contexts of {\tt id1}, it
fails to prove queries at lines 7 and 8 because it uses the same
context {\tt [D3]} for {\tt id} at both call-sites.





%\begin{comment}
%\textcolor{blue}{
Figure~\ref{fig:CFA:callgraph} shows the call-graph obtained by the
ordinary 1-call-site-sensitive analysis without context tunneling. 
Note that the precision of the analysis is incomparable to
that of the baseline 1-object sensitivity. 
Because the analysis uses the
call-site as the calling context, it is able to prove
the queries at
lines 7 and 8 by separately analyzing the two calls to \texttt{id}. However,
it fails to prove the queries at lines
16 and 17 as the variable {\tt v} in \texttt{id} under context
\texttt{[5]} points-to both heap objects \texttt{A2} and \texttt{B2}
that in turn
propagates back to the variables {\tt a} and {\tt b} in the
\texttt{main} method.
%}
%\end{comment}


% !TEX root = ../thesis.tex



\begin{figure*}[t]
	\begin{multicols}{4}
		\begin{subfigure}[b]{1.\columnwidth}
			\begin{center}
				\resizebox{1.\columnwidth}{!}{
					\tikzstyle{every node}=[font=\LARGE]
					\begin{tikzpicture}
					\node [block] (main) {{\tt main}\\$[\cdot]$};
					\node [block, right = 0.7cm of main] (id11) {\tt {id1}\\$[$D2$]$};
					\node [block, above = 0.4cm of id11] (id12) {\tt {id1}\\$[$D1$]$};
					
					\node [block, right=0.6cm of id11] (oid1) {\tt {id}\\$[$D2$]$};
					\node [block, right=0.6cm of id12] (oid2) {\tt {id}\\$[$D1$]$};
					
					\node [block, below = 0.6cm of id11] (m) {\tt {m}\\$[$D3$]$};
					
					\node [block, right=0.6cm of m] (nid) {\tt {id}\\$[$D3$]$};
					
					\path [line] (main) edge node[above] {17} (id11);
					\path [line] (main) edge node[above left] {16} (id12);
					\path [line] (id11) edge node[above] {5} (oid1);
					\path [line] (id12) edge node[above] {5} (oid2);
					\path [line] (main) edge node[below left] {18} (m);
					\path [line] (m) edge [bend left=20] node[above] {7} (nid);
					\path [line] (m) edge [bend right=20] node[below] {8} (nid);
					\end{tikzpicture}
				}
			\end{center}
			\vspace{7pt}
			~\\
			\caption{\small $1$-object sensitivity with $\Tobj = \emptyset$}
			\label{fig:obj:callgraph}
		\end{subfigure}
		\vfill\null
		\columnbreak
		\begin{subfigure}[b]{1\columnwidth}
			\begin{center}
				\resizebox{1\columnwidth}{!}{
					\begin{tikzpicture}
					\tikzstyle{every node}=[font=\LARGE]				
					\node [block] (main) {\tt {main}\\$[\cdot]$};
					\node [block, right = 0.7cm of main] (id11) {\tt {id1}\\$[17]$};
					\node [block, above = 0.4cm of id11] (id12) {\tt {id1}\\$[16]$};
					
					%\node [block, right = 0.4cm of id11] (oid1) {{\tt id}\\$[17]$};
					\node [block, right = 0.4cm of id12] (oid2) {\tt {id}\\$[5]$};
					
					\node [block, below = 0.4cm of id11] (m) {\tt {m}\\$[18]$};
					
					\node [block, right=0.4cm of m] (nid1) {\tt {id}\\$[7]$};
					\node [block, below=0.4cm of nid1] (nid2) {\tt {id}\\$[8]$};
					
					
					\path [line] (main) edge node[above] {$17$} (id11);
					\path [line] (main) edge node[above left] {$16$} (id12);
					\path [line] (id11) edge node[below right] {$5$} (oid2);
					\path [line] (id12) edge node[above] {$5$} (oid2);
					\path [line] (main) edge node[below left] {$18$} (m);
					\path [line] (m) edge node[above] {$7$} (nid1);
					\path [line] (m) edge node[below left] {$8$} (nid2);				
					\end{tikzpicture}
				}
			\end{center}
			\vspace{5pt}
			\caption{\small $1$-CFA with $\Tcall = \emptyset$}
			\label{fig:CFA:callgraph}
		\end{subfigure}
		\vfill\null
		\columnbreak
		\begin{subfigure}[b]{1\columnwidth}
			\begin{center}
				\resizebox{1\columnwidth}{!}{
					\begin{tikzpicture}
					\tikzstyle{every node}=[font=\LARGE]
					\node [block] (main) {\tt {main}\\$[\cdot]$};
					\node [block, right = 0.7cm of main] (id11) {\tt {id1}\\$[17]$};
					\node [block, above = 0.4cm of id11] (id12) {\tt {id1}\\$[16]$};
					
					\node [block, right=0.6cm of id11] (oid1) {\tt {id}\\$[17]$};
					\node [block, right=0.6cm of id12] (oid2) {\tt {id}\\$[16]$};
					
					\node [block, below = 0.4cm of id11] (m) {\tt {m}\\$[18]$};
					
					\node [block, right=0.6cm of m] (nid) {\tt {id}\\$[18]$};
					
					
					\path [line] (main) edge node[above] {$17$} (id11);
					\path [line] (main) edge node[above left] {$16$} (id12);
					\path [line] (id11) edge node[above] {$5$} (oid1);
					\path [line] (id12) edge node[above] {$5$} (oid2);
					\path [line] (main) edge node[below left] {$18$} (m);
					\path [line] (m) edge [bend left=20]  node[above] {$7$} (nid);
					\path [line] (m) edge [bend right=20]  node[below] {$8$} (nid);
					
					\end{tikzpicture}
				}			
			\end{center}
			%\vspace{-10pt}
			~\\
			\vspace{9pt}
			\caption{\small $1$-CFA with $\Tcall = \myset{5,7,8}$}
			\label{fig:call:callgraph}
		\end{subfigure}
		\vfill\null
		\columnbreak
		\begin{subfigure}[b]{1\columnwidth}
			\begin{center}
				\resizebox{1\columnwidth}{!}{
					\begin{tikzpicture}
					\tikzstyle{every node}=[font=\LARGE]
					\node [block] (main) {\tt {main}\\$[\cdot]$};
					\node [block, right = 0.7cm of main] (id11) {\tt {id1}\\$[17]$};
					\node [block, above = 0.4cm of id11] (id12) {\tt {id1}\\$[16]$};
					
					\node [block, right=0.4cm of id11] (oid1) {\tt {id}\\$[17]$};
					\node [block, right=0.4cm of id12] (oid2) {\tt {id}\\$[16]$};
					
					\node [block, below = 0.4cm of id11] (m) {\tt {m}\\$[18]$};
					
					\node [block, right=0.4cm of m] (nid1) {\tt {id}\\$[7]$};
					\node [block, below=0.4cm of nid1] (nid2) {\tt {id}\\$[8]$};
					
					
					\path [line] (main) edge node[above] {$17$} (id11);
					\path [line] (main) edge node[above left] {$16$} (id12);
					\path [line] (id11) edge node[above] {$5$} (oid1);
					\path [line] (id12) edge node[above] {$5$} (oid2);
					\path [line] (main) edge node[below left] {$18$} (m);
					\path [line] (m) edge node[above] {$7$} (nid1);
					\path [line] (m) edge node[below left] {$8$} (nid2);				
					\end{tikzpicture}
				}
			\end{center}
			\vspace{5pt}
			\caption{$1$-CFA with $\Tcall = \myset{5}$}
			\label{fig:opt:callgraph}
		\end{subfigure}
		
	\end{multicols}
	\vspace{-25pt}
	\caption{Call-graphs of running example produced by object sensitivity
		and call-site sensitivity. % From left to right, our context-simulation gradually 
		% refines call-site-sensitivity's call-graphs so the outcome is more precise than the 
		% object-sensitivity's one
              }
	\label{Fig:Simulation}
			\vspace{-12pt}
\end{figure*}


%%% Local Variables:
%%% mode: latex
%%% TeX-master: "../thesis.tex"
%%% End:


\myparagraph{Inferring Tunneling Abstraction}
Simulation is a two-step process. 
It first runs the baseline object-sensitive analysis 
%\begin{comment}
(i.e. ${\sf obj}(P, \Tobj)$) 
%\end{comment}
to obtain its
call-graph $\CallGraph \subseteq \mbi \times \mbc \times \mbm
\times \mbc$.
Next, it analyzes the structure of $\CallGraph$ and infers a tunneling abstraction
$\Tcall$ that makes call-site sensitivity to simulate $\CallGraph$. 
At a high-level, we infer three kinds of invocation sites and define
\[
\Tcall = (I_1 \cup I_2) \setminus I_3
\]
where $I_1$, $I_2$, and $I_3$ are invocations sites in $P$.
Intuitively, $I_1$ and $I_2$ denote the invocation sites
that require context tunneling in order for call-site sensitivity to simulate
object sensitivity. On the other hand, $I_3$ is the invocation sites
where context tunneling must be avoided to preserve the original
precision of call-site sensitivity. % Thus, we can increase the
% precision of call-site sensitivity even more than object sensitivity
% by including $I_1\cup I_2$ and excluding $I_3$. 


Our key idea to infer $I_1$ and $I_2$ is to assume that $\CallGraph$ was produced by a context-tunneled
call-site-sensitive analysis and infer
backward its tunneling abstraction.
To this end, we identify and exploit two fundamental properties of context-tunneled call-site sensitivity.


The first property is that the callee method's context becomes
equivalent to the caller's context when context tunneling is applied during call-site sensitivity.
This is because, in call-site sensitivity, applying
context tunneling at an invocation site always makes the called
method inherit the caller's context.
Thus, we scan each call-graph edge $(l, c, m, c')$ of $\CallGraph$ and
identify those that have this property ($c=c'$). 
We define $I_1$ to be the set of invocation sites of all such edges:
\vspace{-3pt}
\[
I_1 = \myset{l \in \mbi_P \mid (l,c,m,c') \in \CallGraph, c = c'}.
\vspace{-3pt}\]
For example, $I_1$ is $\myset{5, 7, 8}$ for the call-graph in
Figure~\ref{fig:obj:callgraph}, where the invocation-site 5 comes from
the call-graph edges $(5,{\tt D1},\texttt{id}, {\tt D1})$ and
$(5,{\tt D2},\texttt{id}, {\tt D2})$,
7 comes from $(7,{\tt D3},\texttt{id},{\tt D3})$, and 8 comes from $(8,{\tt
  D3},\texttt{id},{\tt D3})$.

  %$I_1$ is sound but incomplete property of call-site
  %sensitivity with tunneling. 
%\textcolor{red}{If context tunneling is not applied to the invocation $l$, the callee method's context $c'$ becomes different from the caller method's context $c$ as the new context $c'$ will include the context element $l$ which are not in $c$.}
% Figure~\ref{fig:call:callgraph} shows the call-graph produced by the
% 1-call-site-sensitive analysis with the tunneling abstraction $T_{\it call} =
% \myset{5,7,9}$. Note that call-site-sensitivity with $T_{\it call} = I_1$ already
% simulates
% object-sensitivity, producing the call-graph identical to
% that of the baseline object-sensitive analysis in Figure~\ref{fig:obj:callgraph}.


% Intuition behind $I_1$ came from object-sensitivity's property of carrying
% caller method's context over callee method if two methods share same base
% object. The property is crucial in practice because it gives
% object-sensitivity
%\begin{comment}
In practice, applying context tunneling at $I_1$ gives call-site
sensitivity immunity against nested call chains that are popular in
object-oriented programs. For example, Figure~\ref{fig:obj:callgraph} shows that
object sensitivity precisely distinguishes two invocations of {\tt id}
according to their base objects $D1$ and $D2$.  In contrast,
conventional call-site sensitivity must use larger $k$ to precisely
analyze those nested call chains. 
%\end{comment}





%$I_2$ includes invocation-sites where different caller contexts
% imply different callee contexts:
%\[
%I_2 =\{\invo \in Invo \mid \forall (\invo, c_1, m, c_1'), (\invo, c_2, m, c_2') \in
%G.\; c_1 \neq c_2 \implies c_1' \neq c_2'\}
%\]
%where we assume $(l,c_1,m,c_1')$ and $(l,c_2,m,c_2')$ are different
%call-edges, i.e., $c_1 \not= c_2 \vee c_1' \neq c_2'$.


%\textcolor{red}{
The second property of context-tunneled call-site sensitivity is that
different caller contexts imply different callee contexts. Suppose
two different call-graph edges $(l, c_1, m, c_1')$ and
$(l, c_2, m, c_2')$, where the last (i.e., $k$th) context element of $c_1$ is different from that of $c_2$, are generated in call-site sensitivity. 
%$(l, c_2, m, c_2')$, where $c_1 \not= c_2$, are generated in call-site sensitivity. 
If context tunneling was applied at $l$, then the last context elements of $c_1'$ and
$c_2'$ are certainly different because the callee should inherit the
caller's contexts (i.e. $c_1 = c_1'$ and $c_2 = c_2'$).
We collect invocation sites in $\CallGraph$ with this property: %i.e., $I_2$ is%~\minseok{TODO}
%\[
% \{\invo \in S \mid \forall (\invo, c_1, m, c_1'), (\invo, c_2, m, c_2') \in
%  \CallGraph.  c_1 \neq c_2 \implies  c_1' \neq c_2'\}
%\]
\begin{multline*}
	I_2 =  \{\invo \in S \mid \forall (\invo, c_1, m, c_1'), (\invo, c_2, m, c_2') \in
	\CallGraph.  \kth(c_1) \neq \kth(c_2) \implies \\ \kth(c_1') \neq \kth(c_2')\}
\end{multline*}
where $S$ denotes the invocation sites where a method is called under two different contexts:
\[
S = \{l \in \mbi_P \mid \exists (l,c1,m,c1'),(l,c2,m,c2')\in \CallGraph,  \kth(c1) \not = \kth(c2)\},
\]


%\[
%S = \{l \in \Invo_P \mid \exists (l,c1,m,c1'),(l,c2,m,c2')\in \CallGraph,  c1 \not = c2\}.
%\]
and the function \kth~takes a context and returns its last context element, i.e., $\kth(a_1a_2\dots$ $ a_k) = a_k$.  
 $I_2$ denotes a
sound and complete property of context-tunneled call-site sensitivity. 
That is, if context tunneling is not applied to $l$, the callee methods'
contexts inevitably share the last context element $l$.
%}
In Figure~\ref{fig:obj:callgraph}, $I_2$ is $\myset{5}$ because the invocation-site 5
has two outgoing edges $(5,{\tt D1},\texttt{id},{\tt D1})$ and $(5,{\tt
D2},\texttt{id},{\tt D2})$, where ${\tt D1} \neq {\tt
D2} \implies {\tt D1} \neq {\tt D2}$ holds. 
%\textcolor{blue}{
The invocation-sites 7 and 8 do not
belong to $I_2$ because they have only one call-graph edge.
%}
In Figure~\ref{fig:obj:callgraph}, $I_1$ includes $I_2$, but in
general, $I_2$
may be distinct from $I_1$ as shown in the following call-graph:

{
	%\renewcommand{\baselinestretch}{1.5} 
	\setstretch{1.0}
	\begin{center}
	\resizebox{0.3\columnwidth}{!}{
                \tikzstyle{every node}=[font=\LARGE]
		\begin{tikzpicture}
		\node [block] (main) {{\tt main$_{}$}\\$[\cdot]$};
		\node [block, above right = -0.4cm and
		0.7cm of main] (Dm1) {\tt {D.m}\\$[$D1$]$};
		\node [block, below = 0.4cm of Dm1] (Dm2) {\tt{D.m}\\$[$D2$]$};
		\node [block, right = 0.7cm of Dm1] (Cm1) {\tt{C.m}\\$[$C1$]$};
		\node [block, right = 0.7cm of Dm2] (Cm2) {\tt{C.m}\\$[$C2$]$};
		
		\path [line] (main) edge node[above] {$4$} (Dm1);
		\path [line] (main) edge node[above] {$5$} (Dm2);
		\path [line] (Dm1) edge node[above] {$9$} (Cm1);
		\path [line] (Dm2) edge node[above] {$9$} (Cm2);
		\end{tikzpicture}
    }
    \end{center}
}
where $I_1 = \emptyset$ and $I_2 = \myset{9}$. 
% A detail example is described in Section 1.2 of our supplementary material~\footnote{\url{https://zenodo.org/record/5560499}}.
      %Appendix~\ref{sec:unwrapped}.
% Without context tunneling, $1$-call-site-sensitivity merges
% two {\tt C.m} method calls under context of \texttt{[9]}.
% By applying context-tunneling to invocations in $I_2$, however, 
% the method calls inherit contexts from their callers,
%  \texttt{[4]} and \texttt{[5}] respectively, thus becomes 
% as precise as $1$-object-sensitivity.

%\textcolor{red}{
%Intuitively, the invocation-sites in a tunneling abstraction ($\Tcall$) are considered as bad context elements for call-site sensitivity; the invocation-sites will not be used as context elements. The tunneling abstraction $I_1\cup I_2 = \myset{5,7,8}$ considers the invocation-sites 5,7,8 as bad elements because they make call-site sensitivity produce a different context abstraction from the given object sensitivity.
%}


Note that applying context tunneling at $I_1\cup I_2 = \myset{5,7,8}$
makes call-site sensitivity simulate the baseline object
sensitivity (Figure~\ref{fig:call:callgraph} and Figure~\ref{fig:obj:callgraph} are equivalent). However, it loses the precision benefit of the original call-site
sensitivity (compare Figure~\ref{fig:call:callgraph}
vs. Figure~\ref{fig:CFA:callgraph}).
The purpose of $I_3$ is to avoid this precision loss. 
We define $I_3$ to be the set of invocation-sites where the called
  method has a single context:
\[
I_3 = \myset{l \in \Invo_P \mid \forall (l, c_1, m, c_1'), (l, c_2,
	m, c_2') \in G.\; c_1' = c_2'}.
\]
% where the edges $(l, c_1, m_1, c_1')$ and $(l, c_2,
%   m_2, c_2')$ can be identical.
In Figure~\ref{fig:obj:callgraph}, $I_3$ equals to $\myset{7, 8, 16,
  17, 18}$. For example, $I_3$ includes 7 because only one call-graph
edge exists out of that invocation-site and $I_3$ does not include 5
because it has two outgoing edges to the same method ({\tt id}) under
different contexts ({\tt [D1]} and {\tt [D2]}).
%$I_3$ guarantees that call-site sensitivity analyzes the called method
%more precisely than (or at least equal to) object sensitivity.\minseok{Check} This is because
Intuitively, 
$I_3$ represents the set of invocation-sites that make conventional call-site sensitivity %, which always updates contexts, 
 at least as precise as the given object sensitivity; avoiding context tunneling at 
$I_3$ would make call-site sensitivity analyze the called method
more precisely than (or at least equal to) object sensitivity.
 It is because updating context with the invocation sites ensures that the method invocation is not conflated with another invocation from different invocation-sites which is not the case for object sensitivity.
%It is because conventional call-site sensitivity separates methods called from an invocation-site from the others which is not the case for object sensitivity.
%Updating contexts at $I_3$ ensures that the method invocation is not conflated with another invocation from different invocation-sites.
%As object sensitivity failed to separate any of the methods invoked in $I_3$.
%The intuition behind $I_3$ is leveraging a property of conventional call-site sensitivity that methods called from an invocation-site is guaranteed to be separated from the ones called from the other invocation-sites which is not the case for object sensitivity. 
%Updating contexts at $I_3$ would ensures that the method invocation is not conflated with another invocation from different invocation-sites, which is not the case for
%object sensitivity.
%This is because
%it ensures that the method invocation is not conflated with another
%invocation from different invocation-sites, which is not the case for
%object sensitivity. 
%Figure~\ref{fig:opt:callgraph} shows the resulting call-graph
In summary, we infer $\Tcall = \myset{5}$ from
Figure~\ref{fig:obj:callgraph}: 
 $\Tcall = (I_1 \cup I_2)  \setminus I_3 = (\myset{5,7,8} \cup \myset{5})
 \setminus \myset{7,8,16,17,18}
= \myset{5}$. 

%\textcolor{red}{
Additionally, we apply context tunneling to the invocation-sites if all their parameters are passed from those of the caller methods. 
For example, $\myset{5}$ in our example
belongs to this case because all the parameters (i.e., \texttt{this} and \texttt{v}) come from the caller. 
Such invocations need context tunneling because caller methods' contexts determine the value of the parameters.
For the same reason, we avoid context tunneling if all the parameters
are allocated just before the invocations (e.g., $\myset{16, 17}$ in
our example). The invocation-sites are useful because using them as context elements determines the value of the invocations' parameters.

% !TEX root = ./paper.tex



\begin{figure*}
\begin{center}
%\begin{subfigure}[b]{1\columnwidth}
%\begin{multicols}{2}
\centering
\begin{lstlisting}[multicols=2, escapeinside={(*}{*)},xleftmargin=4.0ex ,basicstyle={\linespread{1.2}\fontsize{9.5}{10}\selectfont\ttfamily}]
class Container {
  Object elem;
  void add(Object el) {
    this.elem = el;}
  Itr itr() {
    Object e = this.elem;
    Itr itr = new Itr(e);//It
    return itr;}
}
class Itr {
  Object next;
  Itr(Object obj) {
    this.next = obj;}
  Object next() {
    return this.next;}
}
void main() {
  Container c1 = new Container();//C1
  c1.add(new A());//A
  Itr i1 = c1.itr();
  object o1 = i1.next();
	
  Container c2 = new Container();//C2
  c2.add(new B());//B
  Itr i2 = c2.itr();
  object o2 = i2.next();
}
\end{lstlisting}
%\end{multicols}
%\end{subfigure}
%\begin{multicols}{2}
%	\begin{subfigure}[b]{\columnwidth}
%	\centering
%		\begin{center}
%			\resizebox{\columnwidth}{!}{
%				\begin{tikzpicture}
%                \tikzstyle{every node}=[font=\Large]
%				\node [block] (main) {\tt {main}\\$[\cdot]$};
%				
%				\node [block, left = 1.28cm of main] (add1) {\tt {add}\\$[$C1$]$};
%				\node [block, below = 0.6cm of add1] (add2) {\tt {add}\\$[$C2$]$};
%				
%				\node [block, right = 0.4cm of add2] (next1) {\tt {next}\\$[$C1$]$};
%				\node [block, right = 0.4cm of next1] (next2) {\tt {next}\\$[$C2$]$};
%				
%				\node [block, right=0.4cm of next2] (itr1) {\tt {itr}\\$[$C1$]$};
%				\node [block, above=0.6cm of itr1] (itr2) {\tt {itr}\\$[$C2$]$};
%				
%				\node [block, right=0.4cm of itr1] (Itr1) {\tt {Itr}\\$[$C1$]$};
%				\node [block, right=0.4cm of itr2] (Itr2) {\tt {Itr}\\$[$C2$]$};
%				
%				
%				\path [line] (main) edge node[above] {$23$} (add1);
%				\path [line] (main) edge node[above] {$28$} (add2);
%				
%				\path [line] (main) edge node[left] {$25$} (next1);
%				\path [line] (main) edge node[right] {$30$} (next2);
%				
%				\path [line] (main) edge node[above] {$24$} (itr1);
%				\path [line] (main) edge node[above] {$29$} (itr2);
%				
%				\path [line] (itr1) edge node[above] {$8$} (Itr1);
%				\path [line] (itr2) edge node[above] {$8$} (Itr2);
%				
%				
%				\end{tikzpicture}
%			}
%		\end{center}
%		\caption{$1$-object sensitivity with $T_{\it obj}=\myset{8,25,30}$}
%		\label{fig:wrappedflow:1objT}
%	\end{subfigure}
%	\vfill\null
%	\columnbreak
%	\begin{subfigure}[b]{\columnwidth}
%		\begin{center}
%			\resizebox{\columnwidth}{!}{
%				\begin{tikzpicture}
%                \tikzstyle{every node}=[font=\Large]
%				\node [block] (main) {\tt {main}\\$[\cdot]$};
%				
%				\node [block, left = 1.28cm of main] (add1) {\tt {add}\\$[23]$};
%				\node [block, below = 0.6cm of add1] (add2) {\tt {add}\\$[28]$};
%				
%				\node [block, right = 0.4cm of add2] (next1) {\tt {next}\\$[25]$};
%				\node [block, right = 0.4cm of next1] (next2) {\tt {next}\\$[30]$};
%				
%				\node [block, right=0.4cm of next2] (itr1) {\tt {itr}\\$[24]$};
%				\node [block, above=0.6cm of itr1] (itr2) {\tt {itr}\\$[29]$};
%				
%				\node [block, right=0.4cm of itr1] (Itr1) {\tt {Itr}\\$[24]$};
%				\node [block, right=0.4cm of itr2] (Itr2) {\tt {Itr}\\$[29]$};
%				
%				
%				\path [line] (main) edge node[above] {$23$} (add1);
%				\path [line] (main) edge node[above] {$28$} (add2);
%				
%				\path [line] (main) edge node[left] {$25$} (next1);
%				\path [line] (main) edge node[right] {$30$} (next2);
%				
%				\path [line] (main) edge node[above] {$24$} (itr1);
%				\path [line] (main) edge node[above] {$29$} (itr2);
%				
%				\path [line] (itr1) edge node[above] {$8$} (Itr1);
%				\path [line] (itr2) edge node[above] {$8$} (Itr2);
%				
%				
%				\end{tikzpicture}
%			}
%			
%		\end{center}
%		\caption{$1$-call-site sensitivity with $T_{\it call}=\myset{8}$}
%		\label{fig:wrappedflow:1callT}
%	\end{subfigure}
%\end{multicols}
%\vspace{-20pt}
%\caption{Wrapped flow pattern~\cite{Li2018a} and call-graphs produced by
%1-object sensitivity with context-tunneling and 1-call-site
%sensitivity with context tunneling}
%\label{fig:wrappedflow}
\end{center}
\vspace{-12pt}
\caption{Example code}
\vspace{-12pt}
\label{fig:wrappedflow:example}
\end{figure*}

\myparagraph{Effectiveness in the Real-World}\label{sec:simulation:real-world}
To show the effectiveness of simulation more clearly, we provide a representative real-world example.
%In this example, we aim to show that call-site sensitivity with tunneling can effectively analyze the representative example, and call-site sensitivity 
Also, this example shows that call-site sensitivity can simulate object sensitivity with a nontrivial tunneling abstraction ($\Tobj \neq \emptyset$).



%\textcolor{red}
{
Containers and iterators have been popular to exemplify the strength
of object sensitivity (e.g.~\cite{Milanova2005,JeJeOh18,TanLX16}), as
they are prevalent in object-oriented
programs.
Figure~\ref{fig:wrappedflow:example} shows a typical example.
%, called {\em wrapped flow}~\cite{Li2018a}.
In Figure~\ref{fig:wrappedflow:example}, the {\tt Container} class has
its iterator of which the field wraps the container's
data, and the data is obtained by the calling iterator's method.
}

%\textcolor{red}
{
In this case, 1-object sensitivity needs context tunneling to distinguish all the method calls.
The conventional 1-object-sensitive analysis is not precise as
the two iterators share the same heap allocation-site {\tt It}. 
Consider a tunneling abstraction
$\Tobj =\{7, 21, 26\}$.
With this tunneling abstraction, 1-object sensitivity produces precise
results with the
call-graph shown in Figure~\ref{fig:wrappedflow:1objTGraph}.
}


\begin{figure*}
\begin{center}
%\begin{subfigure}[b]{1\columnwidth}
%\begin{multicols}{2}

%\end{multicols}
%\end{subfigure}
\begin{multicols}{2}
	\begin{subfigure}[b]{\columnwidth}
	\centering
		\begin{center}
			\resizebox{.9\columnwidth}{!}{
				\begin{tikzpicture}
                \tikzstyle{every node}=[font=\Large]
				\node [block] (main) {\tt {main}\\$[\cdot]$};
				
				\node [block, left = 1.28cm of main] (add1) {\tt {add}\\$[$C1$]$};
				\node [block, below = 0.6cm of add1] (add2) {\tt {add}\\$[$C2$]$};
				
				\node [block, right = 0.4cm of add2] (next1) {\tt {next}\\$[$C1$]$};
				\node [block, right = 0.4cm of next1] (next2) {\tt {next}\\$[$C2$]$};
				
				\node [block, right=0.4cm of next2] (itr1) {\tt {itr}\\$[$C1$]$};
				\node [block, above=0.6cm of itr1] (itr2) {\tt {itr}\\$[$C2$]$};
				
				\node [block, right=0.4cm of itr1] (Itr1) {\tt {Itr}\\$[$C1$]$};
				\node [block, right=0.4cm of itr2] (Itr2) {\tt {Itr}\\$[$C2$]$};
				
				
				\path [line] (main) edge node[above] {$19$} (add1);
				\path [line] (main) edge node[above] {$24$} (add2);
				
				\path [line] (main) edge node[left] {$21$} (next1);
				\path [line] (main) edge node[right] {$26$} (next2);
				
				\path [line] (main) edge node[above] {$20$} (itr1);
				\path [line] (main) edge node[above] {$25$} (itr2);
				
				\path [line] (itr1) edge node[above] {$7$} (Itr1);
				\path [line] (itr2) edge node[above] {$7$} (Itr2);
				
				
				\end{tikzpicture}
			}
		\end{center}
		\caption{$1$-object sensitivity with $T_{\it obj}=\myset{7,21,26}$}
		\label{fig:wrappedflow:1objTGraph}
	\end{subfigure}
	\vfill\null
	\columnbreak
	\begin{subfigure}[b]{\columnwidth}
		\begin{center}
			\resizebox{.9\columnwidth}{!}{
				\begin{tikzpicture}
                \tikzstyle{every node}=[font=\Large]
				\node [block] (main) {\tt {main}\\$[\cdot]$};
				
				\node [block, left = 1.28cm of main] (add1) {\tt {add}\\$[19]$};
				\node [block, below = 0.6cm of add1] (add2) {\tt {add}\\$[24]$};
				
				\node [block, right = 0.4cm of add2] (next1) {\tt {next}\\$[21]$};
				\node [block, right = 0.4cm of next1] (next2) {\tt {next}\\$[26]$};
				
				\node [block, right=0.4cm of next2] (itr1) {\tt {itr}\\$[20]$};
				\node [block, above=0.6cm of itr1] (itr2) {\tt {itr}\\$[25]$};
				
				\node [block, right=0.4cm of itr1] (Itr1) {\tt {Itr}\\$[20]$};
				\node [block, right=0.4cm of itr2] (Itr2) {\tt {Itr}\\$[25]$};
				
				
				\path [line] (main) edge node[above] {$19$} (add1);
				\path [line] (main) edge node[above] {$24$} (add2);
				
				\path [line] (main) edge node[left] {$21$} (next1);
				\path [line] (main) edge node[right] {$26$} (next2);
				
				\path [line] (main) edge node[above] {$20$} (itr1);
				\path [line] (main) edge node[above] {$25$} (itr2);
				
				\path [line] (itr1) edge node[above] {$7$} (Itr1);
				\path [line] (itr2) edge node[above] {$7$} (Itr2);
				
				
				\end{tikzpicture}
			}
			
		\end{center}
		\caption{$1$-call-site sensitivity with $T_{\it call}=\myset{7}$}
		\label{fig:wrappedflow:1callTGraph}
	\end{subfigure}
\end{multicols}
\vspace{-30pt}
\caption{call-graphs produced by
1-object sensitivity with context-tunneling and 1-call-site
sensitivity}
\vspace{-12pt}
\label{fig:wrappedflow}
\end{center}
\end{figure*}

%\textcolor{red}
{
With our technique, call-site-sensitivity can simulate the call-graph
produced by the context-tunneled object-sensitive analysis. 
From the call-graph in Figure~\ref{fig:wrappedflow:1objTGraph}, our
technique finds out
\[
I_1 = \myset{7}, \quad  I_2 = \{7\},\quad I_3 =
\{19,20,21,24,25,26\}.
\]
and produces a tunneling abstraction $\Tcall = \{7\}$.
Figure~\ref{fig:wrappedflow:1callTGraph} shows the call-graph of
1-call-site sensitivity with the tunneling abstraction, which is
exactly the same as that of the baseline object-sensitive analysis
in Figure~\ref{fig:wrappedflow:1objTGraph}.
Note that 1-object sensitivity and 1-call-site sensitivity use
different tunneling abstractions, i.e., $\Tcall = \{7\}$ vs.
$\Tobj = \{7, 21, 26\}$. 
}





%\textcolor{red}{
%\myparagraph{Effectiveness in the Real-World}
%The simulation technique described above is principled (i.e., it exploits fundamental properties of context-tunneled call-site sensitivity) and yet powerful
%enough to make call-site sensitivity to enjoy most of the precision
%benefits of object sensitivity in practice. For example, our technique
%enables 1-call-site sensitivity to effectively handle all
%patterns of {\em precision-critical} methods
%in~\cite{Li2018a}.  
%\citet{Li2018a} identified fairly representative and
%exhaustive patterns that require object sensitivity in real-world Java
%programs;
%applying 2-object sensitivity only to the methods with those patterns
%is sufficient to achieve 98.8\% of the full precision~\cite{Li2018a}. 
%Conventional call-site-sensitive analysis is
%ineffective to handle such patterns as they require the
%analysis to maintain deep calling contexts. With context tunneling,
%however, our technique enables call-site
%sensitivity to simulate object sensitivity in all cases. We provide
%detailed description of the patterns and how our technique works in
%Appendix~\ref{appendix:effectiveness}. 



\myparagraph{Simulated Policy}
Now we define the simulated policy $\simheuristic$. 
Let $\simulate$ be the simulation process described above. 
As input, $\simulate$ takes a program $P$ and a tunneling abstraction
$\Tobj \subseteq \Invo_P$ of baseline object sensitivity.  
Running the simulation procedure, denoted $\simulate(P, \Tobj)$, produces a tunneling abstraction $\Tcall \subseteq
\Invo_P$ for call-site sensitivity.  
With $\simulate$, we can transform $\objheuristic$ into
$\simheuristic$ as follows:
\begin{equation}\label{eq:simheuristic}
  \simheuristic = \lambda P.\; \simulate(P,
  \objheuristic(P)).
\end{equation}
Although the simulated policy $\simheuristic$ makes call-site
sensitivity more precise than the baseline object sensitivity, using
$\simheuristic$  is
impractical because the simulation procedure incurs the overhead of running the baseline object-sensitive analysis.






%%% Local Variables:
%%% mode: latex
%%% TeX-master: "../thesis"
%%% End:
