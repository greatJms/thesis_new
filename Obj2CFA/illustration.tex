% !TEX root = ../thesis.tex
%\vspace{-1pt}
\section{Our Claim}\label{sec:insight}
%\noindent
In this section, we illustrate the main message of this chapter with examples. 



\subsection{The Previously Known Superiority}\label{sec:call-vs-obj}
%\noindent

First of all, we note that traditional call-site sensitivity and
object sensitivity can complement each other~\cite{liang2005evaluating}.

%For any $k$, we can always find a program
%for which $k$-call-site sensitivity is more precise than
%$k$-object sensitivity, and vice versa.
% !TEX root = ./paper.tex
\begin{figure}[t]	
\begin{multicols}{2}
\vfill\null
\begin{subfigure}[t]{1.3\columnwidth}
\begin{center}
\begin{lstlisting}[escapeinside={(*}{*)},xleftmargin=0.6cm]
class D {
 Object id (v) { 
  return v; }
}
main() {
 D d = new D();//D
 A a = (A)d.id(new A()); //A, query1
 B b = (B)d.id(new B()); //B, query2
 C c = (C)d.id(new C()); //C, query3
}
\end{lstlisting}
\caption{Example code}
\label{background:example2}
\end{center}
\end{subfigure}
\columnbreak
%\scalefont{0.9}



\qquad
\begin{subfigure}[t]{0.80\columnwidth}
	\begin{center}
		\resizebox{0.4\columnwidth}{!}{
			\begin{tikzpicture}
			\node [block] (main) {\LARGE{\tt main$_{}$}\\$[\cdot]$};
			\node [block, above right=0.5cm and 0.5cm of main] (id1) {\LARGE{\tt id$_{}$}\\$[7]$};
			\node [block, below = 0.5cm of id1] (id2) {\LARGE{\tt id$_{}$}\\ $[8]$};
			\node [block, below = 0.5cm of id2] (id3) {\LARGE{\tt id$_{}$}\\ $[9]$};
			
			\path [line] (main) -- (id1);
			\path [line] (main) -- (id2);
			\path [line] (main) -- (id3);
					
			\end{tikzpicture}
		}
	\end{center}
	\caption{Call-graph by 1-call-site sensitivity}
	\label{back:cfa:callgraph2}
\end{subfigure}
\qquad\linebreak\linebreak

\qquad
\begin{subfigure}[t]{0.9\columnwidth}
	\begin{center}
		\resizebox{0.4\columnwidth}{!}{
			\begin{tikzpicture}
			\node [block] (main) {\LARGE{\tt main$_{}$}\\$[\cdot]$};
			\node [block, right = 0.5cm of main] (id) {\LARGE{\tt id$_{}$}\\${\tt[D]}$};
			
			\path [line] (main) -- (id);
			
			\end{tikzpicture}
		}
	\end{center}
	\caption{Call-graph by $k$-object sensitivity (for any $k$)}
	\label{back:obj:callgraph2}
\end{subfigure}
\end{multicols}
\vspace{-1em}
\caption{Typical situation that benefits from call-site sensitivity}
\label{back:b:Fig}
\vspace{-11pt}
\end{figure}




%%% Local Variables:
%%% mode: latex
%%% TeX-master: "paper"
%%% End:


\myparagraph{Benefit of Call-Site Sensitivity}
Figure~\ref{back:b:Fig} describes a typical situation where
call-site sensitivity has better precision than object sensitivity.
The example program has class {\tt D} that includes the identity
function {\tt id}. The {\tt main} method allocates an object of class
{\tt D} at line 6
and calls method {\tt id} on it in three places
at lines 7, 8, and 9 with different objects of type {\tt A}, {\tt B},
and {\tt C}, respectively.  Suppose pointer analysis aims to prove
that the three type-casting operations at lines 7, 8, and 9 are safe.
%(i.e. no down-casting violations occur).
Figure~\ref{back:cfa:callgraph2} shows the call-graph from
1-call-site-sensitive analysis. Note that the analysis analyzes the
method {\tt id} separately for the different call-sites at lines 
7, 8, and 9, and therefore is able to prove the safety of the queries. 


By contrast, $k$-object sensitivity is unable to prove any of the
queries in the program no matter what $k$ value is
used. Object sensitivity uses allocation-sites of receiver objects as
calling contexts. In this example, because the three method calls
share the same receiver object (i.e. the object pointed to by variable
{\tt d}), object sensitivity analyzes the method {\tt id}
with a single context element, namely the allocation-site {\tt D},
merging the three method calls (Figure~\ref{back:obj:callgraph2}).


% !TEX root = ./paper.tex

\begin{figure*}[t]	
\vspace{20pt}
\begin{multicols}{2}
\vfill\null
\begin{center}
\begin{subfigure}[b]{1.2\columnwidth}
\begin{lstlisting}[escapeinside={(*}{*)}, xleftmargin=12pt]
class C {
 Object id(*$_0$*)(v) {
  return v; }
 Object id(*$_1$*)(v) {
  return this.id(*$_{0}$*)(v); }
  ...
 Object id(*$_k$*)(v) {
  return this.id(*$_{k-1}$*)(v); }
}
main() {
 C c1 = new C();//C1
 C c2 = new C();//C2
 A a = (A)c1.id(*$_k$*)(new A());//A, query1
 B b = (B)c2.id(*$_k$*)(new B());//B, query2
}
\end{lstlisting}
\caption{Example code}
\label{background:example1}
\end{subfigure}
\end{center}
\columnbreak~\\~\\~\\
\vspace{-30pt}

%\scalefont{0.9}
%\qquad\qquad\qquad\qquad
%\begin{multicols}{3}
\begin{subfigure}[b]{0.9\columnwidth}
\begin{center}
				\resizebox{\columnwidth}{!}{
					\begin{tikzpicture}
					\node [block] (main) {\LARGE{\tt main$_{}$}\\$[\cdot]$};
					\node [block, above right =
                                        -0.4cm and 0.4cm of main]
                                        (idk1) {\LARGE{ {\tt id}$_k$}\\$[13]$};
					\node [block, below = 0.4cm of idk1] (idk2) {\LARGE{\tt {id}$_k$}\\$[14]$};
					
					\node [onlyText, right=0.4cm of idk1] (dots1) {...};
					\node [onlyText, right=0.4cm of idk2] (dots2) {...};
					
					\node [block2, right=0.4cm of
                                        dots1] (id11) {\LARGE{\tt {id}$_1$}\\$[13,8,...]$};
					\node [block2, right=0.4cm of
                                        dots2] (id12) {\LARGE{\tt {id}$_1$}\\$[14,8,...]$};
					
					\node [block2, right = 7.2cm of
                                        main] (id0) {\LARGE{\tt id$_0$}\\$\underbrace{[8,...,5]}_k$};
					
					\path [line] (main) -- (idk1);
					\path [line] (main) -- (idk2);
					\path [line] (idk1) -- (dots1);
					\path [line] (idk2) -- (dots2);
					\path [line] (dots1) -- (id11);
					\path [line] (dots2) -- (id12);
					
					\path [line] (id11) -- (id0);
					\path [line] (id12) -- (id0);
					
					\end{tikzpicture}
				}
			\end{center}
			\vspace{-4pt}
			\caption{Call-graph by $k$-call-site sensitivity}
			\label{back:cfa:callgraph}
\end{subfigure} \vspace{8pt}
%	\vfill\null

%\qquad\qquad\qquad\qquad
\begin{subfigure}[b]{0.8\columnwidth}
\begin{center}
				\resizebox{\columnwidth}{!}{
					\begin{tikzpicture}
					\node [block] (main) {\LARGE{\tt main$_{}$}\\$[\cdot]$};
					\node [block, above right =
                                        -0.4cm and 0.4cm of main]
                                        (idk1) {\LARGE{\tt ${ id}_k$}\\${\tt [C1]}$};
					\node [block, below = 0.4cm of
                                        idk1] (idk2) {\LARGE{\tt ${ id}_k$}\\${\tt [C2]}$};
					
					\node [onlyText, right=0.4cm of idk1] (dots1) {...};
					\node [onlyText, right=0.4cm of idk2] (dots2) {...};
					
					\node [block, right=0.4cm of
                                        dots1] (id11) {\LARGE{\tt id$_1$}\\${\tt [C1]}$};
					\node [block, right=0.4cm of
                                        dots2] (id12) {\LARGE{\tt id$_1$}\\${\tt [C2]}$};
					
					\node [block, right=0.4cm of
                                        id11] (id01) {\LARGE{\tt id$_0$}\\${\tt [C1]}$};
					\node [block, right=0.4cm of
                                        id12] (id02) {\LARGE{\tt id$_0$}\\${\tt [C2]}$};
					
					\path [line] (main) -- (idk1);
					\path [line] (main) -- (idk2);
					\path [line] (idk1) -- (dots1);
					\path [line] (idk2) -- (dots2);
					\path [line] (dots1) -- (id11);
					\path [line] (dots2) -- (id12);
					
					\path [line] (id11) -- (id01);
					\path [line] (id12) -- (id02);
					
					\end{tikzpicture}
				}
			\end{center}
			\vspace{-4pt}
			\caption[]{Call-graph by 1-object sensitivity}
			\label{back:obj:callgraph}
\end{subfigure}
\vspace{8pt}
%	\vfill\null

%\qquad\qquad\qquad\qquad

\begin{subfigure}[b]{1.0\columnwidth}
	\begin{center}
				\resizebox{0.8\columnwidth}{!}{
					\begin{tikzpicture}
					\node [block] (main) {\LARGE{\tt main$_{}$}\\$[\cdot]$};
					\node [block, above right =
                                        -0.4cm and 0.4cm of main]
                                        (idk1) {\LARGE{\tt ${ id}_k$}\\${\tt [13]}$};
					\node [block, below = 0.4cm of
                                        idk1] (idk2) {\LARGE{\tt ${ id}_k$}\\${\tt [14]}$};
					
					\node [onlyText, right=0.4cm of idk1] (dots1) {...};
					\node [onlyText, right=0.4cm of idk2] (dots2) {...};
					
					\node [block, right=0.4cm of
                                        dots1] (id11) {\LARGE{\tt id$_1$}\\${\tt [13]}$};
					\node [block, right=0.4cm of
                                        dots2] (id12) {\LARGE{\tt id$_1$}\\${\tt [14]}$};
					
					\node [block, right=0.4cm of
                                        id11] (id01) {\LARGE{\tt id$_0$}\\${\tt [13]}$};
					\node [block, right=0.4cm of
                                        id12] (id02) {\LARGE{\tt id$_0$}\\${\tt [14]}$};
					
					\path [line] (main) -- (idk1);
					\path [line] (main) -- (idk2);
					\path [line] (idk1) -- (dots1);
					\path [line] (idk2) -- (dots2);
					\path [line] (dots1) -- (id11);
					\path [line] (dots2) -- (id12);
					
					\path [line] (id11) -- (id01);
					\path [line] (id12) -- (id02);
					
					\end{tikzpicture}
				}
			\end{center}
			\vspace{-4pt}
			\caption[]{1-call-site sensitivity with context tunneling}
			\label{back:callT:callgraph}
\end{subfigure}





\end{multicols}
%	\vfill\null
	%\columnbreak
% \begin{subfigure}[b]{0.5\columnwidth}
% \begin{center}
% 				\resizebox{\columnwidth}{!}{
% 					\begin{tikzpicture}
% 					\node [block] (main) {\Large{$\tt main_{}$}\\$[\cdot]$};
% 					\node [block, above right =
%                                         -0.4cm and 0.4cm of main]
%                                         (idk1) {\Large{\tt ${\tt id}_k$}\\$[13]$};
% 					\node [block, below = 0.4cm of idk1] (idk2) {{\tt $id_k$}\\$[14]$};
					
% 					\node [onlyText, right=0.4cm of idk1] (dots1) {...};
% 					\node [onlyText, right=0.4cm of idk2] (dots2) {...};
					
% 					\node [block, right=0.4cm of
%                                         dots1] (id11) {\Large{\tt ${\tt id}_1$}\\$[13]$};
% 					\node [block, right=0.4cm of
%                                         dots2] (id12) {\Large{\tt ${\tt id}_1$}\\$[14]$};
					
% 					\node [block, right=0.4cm of
%                                         id11] (id01) {\Large{\tt ${\tt id}_0$}\\$[13]$};
% 					\node [block, right=0.4cm of
%                                         id12] (id02) {\Large{\tt ${\tt id}_0$}\\$[14]$};
					
% 					\path [line] (main) -- (idk1);
% 					\path [line] (main) -- (idk2);
% 					\path [line] (idk1) -- (dots1);
% 					\path [line] (idk2) -- (dots2);
% 					\path [line] (dots1) -- (id11);
% 					\path [line] (dots2) -- (id12);
					
% 					\path [line] (id11) -- (id01);
% 					\path [line] (id12) -- (id02);
					
% 					\end{tikzpicture}
% 				}
% 			\end{center}
% 			\caption{Call-graph by 1-call-site-sensitivity
%                           with context tunneling}
% 			\label{back:callT:callgraph}
%  \end{subfigure}
%\vspace{-1em}
\vspace{-14pt}
	\caption{Typical situation that benefits from object sensitivity}
	\label{back:a:Fig}
\vspace{-10pt}
\end{figure*}


%%% Local Variables:
%%% mode: latex
%%% TeX-master: "paper"
%%% End:



\myparagraph{Benefit of Object Sensitivity}
Figure~\ref{back:a:Fig} %describes a major weakness of call-site sensitivity. 
describes a representative scenario where
object sensitivity is more precise than call-site sensitivity.
The example code in Figure~\ref{background:example1} has class {\tt C}
that contains $k+1$ methods
(${\tt id}_0, {\tt id}_1, \dots {\tt id}_k$), where each method
${\tt id}_i$ is semantically equivalent to the identity function
because ${\tt id}_0$ is the identity function and ${\tt id}_i$
%$(0< i \le k+1)$ 
{$(0< i \le k)$} calls ${\tt id}_{i-1}$ without modifying the formal
parameter {\tt v}.  The {\tt main} method has four heap allocation-sites:
namely, \texttt{C1}, \texttt{C2}, \texttt{A}, and \texttt{B}.  At line
13, {\tt main} calls ${\tt id}_k$ with the base variable {\tt c1} and
parameter \texttt{new A()}. At line {14}, ${\tt id}_k$ is called with
the base variable {\tt c2} and argument \texttt{new B()}.  Again, the goal
of pointer analysis is to prove the safety of the casting
operations at lines 13 and 14. For this program, a
$k$-call-site-sensitive analysis produces the call-graph in
Figure~\ref{back:cfa:callgraph}. Note that the method ${\tt id}_0$ is analyzed
under the single context ${[8,\dots,5]}$, where the critical information where
${\tt id}_k$ was originally called from is lost due to the truncation of
context strings to keep their last $k$ elements. 

Object sensitivity nicely addresses this shortcoming of call-site sensitivity. It uses the
allocation-sites, {\tt C1} and {\tt C2}, to represent the contexts of the
method calls to ${\tt id}_k$ at lines 13 and {14}, respectively. Note that the
receiver object remains the same in the subsequent calls to ${\tt id}_{k-1},
\dots {\tt id}_0$, propagating the initial contexts down to ${\tt id}_0$ and
producing the call-graph in Figure~\ref{back:obj:callgraph}. The
analysis is able to distinguish the two call chains and therefore proves the
queries.

%\textcolor{red}{Note that the shortcoming in handling nested method calls is the only weakness of call-site sensitivity; if there is no nested method call in a program, 1-call-site sensitivity becomes the most precise context abstraction. Suppose a program consists of methods that do not have any method call in their body except the main method. When analyzing the program, 1-call-site sensitivity will have the same precision with $\infty$-call-site sensitivity which is concrete context (the most precise context abstraction).}




\myparagraph{Known Superiority} 
 Though call-site sensitivity and object sensitivity have their own
 strengths and weaknesses, object
  sensitivity is widely known to be superior to call-site sensitivity 
  because real-world programs involve code patterns such as one in
  Figure~\ref{back:a:Fig} more often. 
  Note that the existing superiority holds empirically, rather than theoretically, based on the experimental results of prior work~(e.g., \cite{Lhotak2008,BravenboerS09}).  
  

%\textcolor{red}{
%Though call-site sensitivity and object sensitivity have their own
%strengths and weaknesses, object sensitivity has been known to be
%superior to call-site sensitivity based on prior works' empirical evaluation
%result~\cite{Lhotak2008,BravenboerS09}.  
%Object sensitivity has shown a better performance than call-site
%sensitivity because real-world object-oriented programs involve code
%patterns such as one in Figure~\ref{back:a:Fig} more often. 
%}


%\vspace{-20pt}
\subsection{Revisiting the Superiority in Generalized $k$-Limited Setting}\label{sec:overview-result}
%\noindent
Our new claim is that the known superiority of object sensitivity over call-site sensitivity no longer holds when the notion of $k$-limiting is generalized with context tunneling. 

%In this paper, we show that 
%the existing 
%superiority of object sensitivity over call-site sensitivity holds only in the traditional $k$-limited setting, where the
%analysis is enforced to keep the most recent $k$ context elements, but no longer holds in a more general setting with context tunneling. 

\myparagraph{Context Tunneling}
Context tunneling~\cite{JeJeOh18} allows an analysis to maintain an arbitrary $k$-length subsequence of context strings.
For example, when $s = [C_1, C_2, C_3, C_4, C_5]$ is a sequence of
context elements that may appear in an unbounded ($k=\infty$)
context-sensitive analysis, the traditional $3$-limited analysis
abstracts the context string into its suffix $[C_3,C_4,C_5]$. 
With context tunneling, however, the analysis is free to use any
subsequence of $s$ such as $[C_1, C_3, C_5]$ and $[C_2, C_4, C_5]$, as
a $k$-limited abstraction of the original context string. 
Note that the traditional $k$-limited approach is a special case of the
generalized approach with context tunneling. 


\myparagraph{Key Insight}
Our key insight is summarized as follows: 
%that call-site sensitivity is superior to
%object sensitivity when they are generalized
%with context tunneling: 
\begin{itemize}[leftmargin=1.3em]
\item The major weakness of call-site sensitivity
in the traditional setting is no longer a weakness in the generalized
$k$-limited setting with context tunneling.%~\minseok{conflict. Need revision??} 
\item By contrast, object sensitivity still suffers from its
limitation even with the generalization. 
\end{itemize}

%For example, 
With context tunneling, call-site sensitivity does not
suffer from its shortcoming and can now prove the queries in
Figure~\ref{background:example1}. Suppose that we use a
context-tunneling policy that chooses the first $k$ elements of a
context string rather than the last $k$ ones.
%\textcolor{red}{Context tunneling achieves this by selectively updating contexts. 
%When {\tt id$_i$} (i<k) is called, the context-tunneling policy makes the callee method inherit the context of the caller method instead of updating context with the invocation site.} 
Then, the resulting
1-call-site-sensitive analysis produces the call-graph in Figure~\ref{back:callT:callgraph},
which is exactly the same as the
call-graph of the 1-object-sensitive analysis in
Figure~\ref{back:obj:callgraph}. Because the call-graphs are
equivalent, the call-site-sensitive analysis is as precise as the
object-sensitive analysis, successfully proving the queries. 



On the other hand, object sensitivity fails to simulate call-site
sensitivity even with context tunneling.  Consider the program in
Figure~\ref{back:b:Fig} where call-site sensitivity is
typically more precise than object sensitivity. For this program,
object sensitivity cannot prove all of the queries no matter what
context-tunneling policy and $k$ value are used. 
%\minseok{$k$-object-sensitivity with context tunneling can prove only one query among them.}
In
Figure~\ref{back:b:Fig}, object sensitivity can use the
allocation-site {\tt D} as context elements. Thus, only the two
context subsequences, i.e., \texttt{[D]} and \texttt{[$\cdot$]}, are
possible for the method {\tt id} with context tunneling, all of which
fail to analyze {\tt id} separately for the three different
call-sites.  % In this sense, object sensitivity is 
% inevitably less precise than call-site sensitivity in the generalized $k$-limited setting. 

%\myparagraph{Incompleteness}

We clarify that, like the previously known superiority, our claim in this chapter is empirical. 
On the theoretical side, we do not know yet whether or not call-site sensitivity can always simulate object sensitivity in the general setting with context tunneling, which we leave as an open question for future work. We discuss this in more detail in Appendix~\ref{sec:appendix}.
%not theoretical.  
%We can construct a counter-example such that
%context-tunneled call-site sensitivity is unable to simulate object
%sensitivity (Section~\ref{sec:counter_example}).  
%However,  such a counter-example is uncommon in practice, allowing the simulation to be almost complete for real-world programs. 
%(Sections \RNum{1} and \RNum{2} of the supplementary
%material and Section~\ref{sec:eval_learning}).


\myparagraph{\ourtechnique}
Based on the insight, 
%To support the claim, 
we develop \ourtechnique, a practical technique for transforming a given $k$-object-sensitive analysis 
into a more precise, context-tunneled $k$-call-site-sensitive analysis (Section~\ref{sec:technique}).  The
resulting analysis is (empirically) more precise than the baseline
object sensitivity, as it enjoys the benefits of both object
sensitivity and call-site sensitivity. For example, it
produces precise results for both cases in Figure~\ref{back:b:Fig} and~\ref{back:a:Fig}.


%%% Local Variables:
%%% mode: latex
%%% TeX-master: "thesis"
%%% End:
