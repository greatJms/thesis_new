% !TEX root = ./paper.tex

\section{Conclusion and Future Work}

Unfortunately, the program analysis community for object-oriented programs has dismissed call-site sensitivity for a long time. 
In this paper, 
showed that 
call-site sensitivity has vast untapped potential, even more than object sensitivity, when the notion of $k$-limiting is generalized. We provided an insight that call-site sensitivity with context tunneling can simulate object sensitivity and experimentally proved that the observation holds in practice by developing a technique to transform a baseline object-sensitive analysis into more precise, context-tunneled call-site sensitivity. 
Based on our results, we hope that the community reconsiders call-site sensitivity from now on.  


%We demonstrated that our technique enables call-site sensitivity to outperform modern object-sensitive analyses for real-world Java programs in both precision and cost.

%when designing context-sensitive analyses.


Many problems remain as future work. We already discussed a theoretical issue in Section~\ref{sec:counter_example}. 
Other problems include the following. 


\begin{itemize}

\item {\em Can we learn better tunneling strategies than just simulating object sensitivity?} 
Our goal in this paper was to show that call-site sensitivity can be superior to object sensitivity, and simulating object sensitivity was an effective means of achieving this goal. 
However, simulating object sensitivity would be a suboptimal strategy for call-site sensitivity; 
we believe an optimal tunneling strategy would enable call-site sensitivity to show far better precision than ours (\ours).
Thus, an interesting direction for future work is to develop a powerful learning algorithm to find such strategies, where 
the main challenge is how to efficiently explore the huge and non-monotone tunneling space~\cite{JeJeOh18}. Using reinforcement learning, for example, could be a promising approach to address this challenge. 

%in the huge and non-monotone tunneling space~\cite{JeJeOh18}. 
%For example, reinforcement learning could be a promising approach to address this challenge.

\item {\em Can our approach be adapted for other flavors of context-sensitive analyses?}
Our current simulation technique relies on properties specific to $k$-CFA (e.g., $I_2$ in Section~\ref{sec:simulation} leverages a unique property of context-tunneled $k$-CFA). However, the high-level idea (i.e., simulating object sensitivity) would be applicable to other analyses. 
For example, it would be interesting if our idea could be adapted for $m$-CFA~\cite{Might10}. 
% by leveraging properties of context-tunneled $m$-CFA. 



% instead of k-CFA with your technique would produce better, similar, or worse results.


% High-level idea of our approach (simulating object sensitivity) is applicable to other contexts. 
%Directly applying our technique may not effective as
%it relies on specific properties of k-CFA (e.g., $I_2$ leverages a unique property of context-tunneled k-CFA).
%However, the high-level idea of our approach could be adapted.
%For example, applying our approach could be adapted for m-CFA~\cite{Might10} by leveraging properties of context tunneled m-CFA.

\end{itemize}






% \myparagraph{Future Work}

% \begin{itemize}

% \item {\bf Better context tunneling for call-site-sensitivity}:

% \item {\bf General technique for simulation}:  Our simulation technique was designed to work on the
%   call-graph produced by an object-sensitive analysis. Therefore, the
%   resulting call-site-sensitive analysis is likely to be effective for
%   type-dependent clients (e.g. may-fail casts).

%\end{itemize}

%%% Local Variables:
%%% mode: latex
%%% TeX-master: "paper"
%%% End:
