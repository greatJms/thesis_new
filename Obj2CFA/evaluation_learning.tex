% !TEX root = ./paper.tex


\subsection{Impact of Simulation and Learning}\label{sec:eval_learning}


Next, we discuss the impact of our technical contributions, simulation and learning. 

\myparagraph{Simulation Accuracy}
We first note that our technique enabled call-site
sensitivity to accurately simulate object sensitivity for real-world
applications.
For all benchmark programs except for jython, we ran the
baseline object-sensitive analysis (\oneobjHT) and the corresponding
call-site sensitive analysis (\ours) and counted the number ($A$) of may-fail cast
queries that both analyses can prove and the number ($B$) of queries
that \oneobjHT~can prove. The ratio $A/B$ hints at how accurately
\ours~covers the baseline object sensitivity. The
average ratio over the 11 programs was $0.98$, implying that \ourtechnique~can simulate object sensitivity almost completely.

%\textcolor{red}{
Most of the remaining 2\% of queries, which \ourtechnique~missed, were caused by imperfect simulation rather than learning; 
when we calculated the ratio using simulation only (i.e., \oursim), 
the ratio $A'/B$ was still $0.98$, where $A'$ denotes the number of may-fail casts that both \oursim~and \oneobjHT~can prove. 
%, the may-fail casts proved by \oursim~($A'$)
%still cover 98\% of the queries proved by \oneobjHT~(e.g., $A'/B = 0.98$).
We found that this was because we used a coarse tunneling space, i.e., invocation sites, and that 
using a more fine-grained tunneling space (e.g., pairs of invocation-sites and receiver objects) would improve the simulation accuracy. 
We describe an example in Appendix~\ref{sec:appendix}. 
%We also found the (coarse) tunneling space we used makes the simulation imperfect.
%In this paper, we used invocation sites as tunneling space (e.g., $T \subseteq \Invo$); a tunneling abstraction determines whether to apply context tunneling with invocation sites.
%Using a more fine-grained space (e.g., pair of invocation-sites and receiver objects), however, would improve the simulation accuracy and precision. 
%An example is described in Section~\ref{sec:counter_example}.
%}


%Without simulation, our learning algorithm was unable to find good-enough tunneling policies for call-site sensitivity. 

%
%
%The learning phase of \ourtechnique~spent 58 hours in total, and it effectively captured the behavior of the simulated policy. For all programs, the simulated analysis with $\simheuristic$ was more precise but expensive than the baseline (\oneobjHT).  For eclipse, for example, it reported 573 may-fail casts and took 172s while \oneobjHT~reported 698 may-fail casts and took 43s. With the learned policy, \ours~took 48s with 569 may-fail casts. 
%%We also found our learning algorithm in Section~\ref{sec:learning} is essential; using existing  algorithms~\cite{JeJeOh18,Pedregosa11} ends up with suboptimal results. 
%%We provide detailed results in Section C of the supplementary material.

\myparagraph{Roles of Simulation and Learning}
%This accurate simulation played a key role in improving the precision of call-site sensitivity. 
The results in Table~\ref{tbl:eval:main} show that our simulation technique plays a key role in improving the precision of call-site sensitivity. % and learning reduces the overhead of the simulation. 
%  \ours~mainly comes from our simulation technique ($\simheuristic$).
The column~\oursim~in Table~\ref{tbl:eval:main} presents the
performance of the call-site sensitive analysis obtained by simulating \oneobjHT. For all programs,
\oursim~shows a far better precision than \oneobjHT. 
%Minseok
%\textcolor{red}{For example in the program chart,~\oursim~produces 127 fewer alarms than \oneobjHT.}

%\myparagraph{Impact of Learning}
Comparing the performance of \oursim~and \ours~reveals that 
the use of learning reduced the overhead of \oursim~significantly.  
As the simulation needs to run the baseline object sensitivity (\oneobjHT), the simulated call-site-sensitive analysis (\oursim) is inherently more expensive than the baseline object sensitivity (\oneobjHT). For example, \oursim~is unable to analyze the program jython because the baseline object sensitivity
  (\oneobjHT) failed to analyze it. Thanks to our learning technique, however, our call-site sensitivity (\ours) removed the limitation. For example, \ours~successfully analyzed jython within the time budget.


We checked that the result of learning is not overfitted to the shared library. 
%; we observed reasonable precision gain comes from application code. 
%\textcolor{red}{
In eclipse, for example, \ours~proved 17\% (resp., 15\%) more may-fail casts compared to \oneobjHT~in the application (resp., library) code, which implies that learning is equally effective in both application and library code. 
We also checked \ours~is not overfitted to DaCapo benchmarks. In a non-DaCapo benchmark checkstyle, for example, \ours~proved 41\% (resp., 19\%) more may-fail casts compared to \oneobjHT~in the application (resp., library) code.
%}

%\begin{center}
%\begin{tabular}{@{}lrrrrrrr@{}}
%\toprule
%\multicolumn{1}{c}{} & \multicolumn{1}{c}{eclipse} & \multicolumn{1}{c}{xalan} & \multicolumn{1}{c}{chart} & \multicolumn{1}{c}{fop} & \multicolumn{1}{c}{bloat} & \multicolumn{1}{c}{jpc} & \multicolumn{1}{c}{checkstyle} \\ \midrule
%\#total proven-cast  & 1.17                        & 1.13                      & 1.18                      & 1.14                    & 1.12                      & 1.10                    & 1.19                           \\
%\#app proven-cast    & 1.15                        & 0.91                      & 0.99                      & 0.99                    & 1.03                      & 1.02                    & 1.41                           \\ \bottomrule
%\end{tabular}
%\end{center}


%In eclipse, \ours~produces 22 fewer alarms than \oneobjHT~in application code; about 17\% of the precision gain, a similar portion (18\%) of application code, comes from application code.


\myparagraph{Impact of Simulation-Guided Learning}
Our simulation-guided learning (Section~\ref{sec:learning}) was essential for effectively capturing the precision of the simulated policy. 
%simulated policy ($\simheuristic$ in Section~\ref{sec:simulation})
%; existing or simpler learning algorithms were not powerful enough 
%to capture its behavior. 
To demonstrate this, we conducted two
experiments. 
%Table~\ref{fig:sim-learning-needed} shows the effectiveness of our learning algorithm by comparing the precision of call-site-sensitive analyses with the policy learned from our algorithm. 
% (e.g., \OurLearn) and those produced from existing or simpler learning algorithms~(e.g.,~\Existing~\cite{JeJeOh18} and  \DecisionTree~\cite{Pedregosa11}):
We first replaced our algorithm by the existing
unguided algorithm for learning context tunneling~\cite{JeJeOh18} and trained a policy using the same set of
atomic features in Table~\ref{tbl:features} and training programs.  
Table~\ref{tbl:impact-of-learning} shows that the existing algorithm ended up with a much less precise policy.
Over the four training programs, the context-tunneled call-site sensitivity obtained using the existing learning algorithm~\cite{JeJeOh18}, denoted \Existing~in Table~\ref{tbl:impact-of-learning}, reported 2,809 may-fail casts while the
number for \OurLearn~is much smaller (1,912). This result shows that the use of simulation in our approach 
is critical.
%%Minseok
%\textcolor{red}{; the learning phase alone does not produce the desired results.}
 
Second, we replaced our algorithm by a simple supervised learning
method. We generated labeled data, which consists of feature vectors
of invocation sites and labels indicating whether selected by the
simulated policy ($\simheuristic$) or not. We used the decision tree
algorithm in~\cite{Pedregosa11} to learn a policy. 
Again, the resulting
analysis (denoted \DecisionTree~in Table~\ref{tbl:impact-of-learning}) was unsatisfactory in precision; over the training programs, it
reported 2,604 may-fail casts.  This is mainly because the labeled
data does not have enough information; although the simulated policy
labels which invocations need context tunneling, it is unable to label
which invocations are precision critical.


\begin{table}[t]
\caption{Impact of our simulation-guided learning (numbers indicate {\failcasts})}
\vspace{-5pt}
\label{tbl:impact-of-learning}
\begin{center}
\begin{tabular}{@{}c r r r r r@{}}
\toprule
%              & \multicolumn{5}{c}{\#fail-cast}                                                                                                            \\ \cmidrule{2-6}
              & \multicolumn{1}{c}{luindex} & \multicolumn{1}{c}{lusearch} & \multicolumn{1}{c}{antlr} & \multicolumn{1}{c}{pmd} & \multicolumn{1}{c}{Total} \\\midrule
\ours          & 357                         & 371                          & 477                       & 707                     & 1,912                    \\
\Existing      & 565                         & 580                          & 735                       & 929                     & 2,809                   \\ 
\DecisionTree & 519                         & 533                          & 659                       & 895                     & 2,606                    \\
%\onecallHT      & 784                         & 843                          & 945                       & 1,200                   & 3,772                   \\ 
\bottomrule
\end{tabular}
\end{center}
\vspace{-10pt}
\end{table}


\myparagraph{Impact of New Features}
Another important factor for effective learning was the use of the features in Table~\ref{tbl:features} that are specifically designed for call-site sensitivity (recall that we crafted those features guided by the simulated policy $\simheuristic$). 
For example, the \Existing~analysis described above, which differs from \onecallHT~\cite{JeJeOh18} only in the use of the new set of features, is much more precise than \onecallHT: \Existing~produces 970 fewer alarms than \onecallHT~that is learned using the same algorithm but with the different features designed by~\cite{JeJeOh18}. 

%The features in Table~\ref{tbl:features} are qualified features for learning call-site sensitivity tunneling heuristics. \Existing~in the above, learned with the existing algorithm~\cite{JeJeOh18} and the new features in Table~\ref{tbl:features}, is a lot more precise than the state-of-the-art call-site sensitivity \onecallHT~learned with the same algorithm but the old features~\cite{JeJeOh18}. \Existing~produces 970 fewer alarms than \onecallHT~over the training programs. Although they used the same learning algorithm, \Existing~shows a far better precision thanks to our newly designed qualified features.


We note that our features in Table~\ref{tbl:features} are not appropriate for learning tunneling heuristics for object sensitivity. %In any application of machine learning, the quality of features has a great impact on learning. 
This is because our features are designed to reproduce the results of the simulated call-site sensitivity ($\simheuristic$) rather than object sensitivity. 
To clarify the impact of using a different set of features, we used our features and the existing algorithm~\cite{JeJeOh18} to learn a tunneling heuristic for object sensitivity.
The resulting analysis, denoted \oneobjHTnew, was overall less precise than \oneobjHT. 
%%Minseok
%\textcolor{red}{For example, \oneobjHTnew~reported 630 may-fail casts while \oneobjHT~reported 462 for program luindex. }

% \textcolor{red}{Note that \oneobjHTnew~is also a good 1-object sensitivity analysis compared to the conventional 1-object sensitive analysis that reported 796 alarms for the program luindex.} 

%he learned object-sensitivity is less precise than the existing state-of-the-art object sensitivity \oneobjHT.
%The algorithm failed to learn a better tunneling heuristic for object sensitivity because the features in Table~\ref{tbl:features} are tuned with call-site sensitivity in mind. 
%Note that the existing algorithm successfully learned a far more precise call-site sensitivity (\Existing) than the \onecallHT~with the newly designed features.
%The features are carefully crafted for call-site sensitivity (e.g., we crafted features with~\oursim) instead of object sensitivity. 


%For example, the key feature $C1$, a key feature for call-site sensitivity, we describe above is a bad feature for object sensitivity.
%Unlike call-site-sensitivity that applying context tunneling to $C1$ (e.g., $T_{\tt C1}$) make 1-call-site sensitivity more precise than the conventional 2-call-site sensitivity, 
%applying the tunneling abstraction $T_{\tt C1}$ make 1-object-sensitivity even less precise than the conventional 1-object-sensitivity. 

\myparagraph{Learning Cost}
The learning phase of \ourtechnique~spent 58 hours in total, which is slightly faster than the prior algorithm~\cite{JeJeOh18}. 
Although our algorithm is expensive, we believe it is acceptable because learning is done off-line and saves otherwise more expensive human effort. 
%
%\subsection{Observations}\label{sec:eval:learnedInsight}
%\myparagraph{A key feautre for learning call-site sensitivity tunneling heuristics}
%\textcolor{red}{
%We investigated the learned formula $\heuristic_f$, used in \ours\footnote{The detail formula is presented in Appendix~\ref{appendix:learnedFormula}}~and~\Existing, and found that the feature $C1$ (method invocations uses \texttt{this} variable (i.e. \texttt{this.$m$(...)})) in Table~\ref{tbl:features} plays an important role for improving precision of call-site sensitivity.
%For example, applying context tunneling to the invocations belong to $C1$
%(e.g., $T_{f=C1}$) is a simple way to improve the precision of call-site sensitivity.
%For the four training programs, 1-call-site sensitive analysis with the tunneling abstraction $T_{C1}$ (apply context tunneling when the base variable is {\tt this}) is even more precise than the conventional 2-call-site sensitive analysis (\twocallH); \onecallThis~produces 114 fewer alarms than \twocallH~over the training programs.
%}
%
%
%
%
%\textcolor{red}{
%{\tt this} variables is directly related to $I_1$ (caller and callee method shares the same context in object sensitivity) in our simulation technique $\simheuristic$~where $I_1$ need to be tunneled for call-site sensitivity to simulate object sensitivity.
%Although the feature $C1$ is a syntactic feature, it presents a semantic property that the caller and callee methods share the same context in object sensitivity.
%In java programs, caller and callee methods share the same receiver object if the invocation sites use {\tt this} variable;
%caller and callee methods inevitably share the same context (receiver object) in conventional object sensitivity.
%As a result, all the invocation sites using {\tt this} variable always belongs to $I_1$ and need to be tunneled to make call-site sensitivity simulate object sensitivity. 
%}
%
%
%\myparagraph{Sensitivity to Features}
%\textcolor{red}{
%The features in Table~\ref{tbl:features} are qualified features for learning call-site sensitivity tunneling heuristics. \Existing~in the above, learned with the existing algorithm~\cite{JeJeOh18} and the new features in Table~\ref{tbl:features}, is a lot more precise than the state-of-the-art call-site sensitivity \onecallHT~learned with the same algorithm but the old features~\cite{JeJeOh18}. \Existing~produces 970 fewer alarms than \onecallHT~over the training programs. Although they used the same learning algorithm, \Existing~shows a far better precision thanks to our newly designed qualified features.
%}
%
%
%\textcolor{red}{
%The features in Table~\ref{tbl:features}, however, are inadequate for learning tunneling heuristics for object sensitivity. %In any application of machine learning, the quality of features has a great impact on learning. 
%To clarify the impact of using a different set of features, we gave the newly designed features and used the existing algorithm~\cite{JeJeOh18} for learning a tunneling heuristic for object sensitivity.
%The learning, however, ended up with a suboptimal one; the learned object-sensitivity is less precise than the existing state-of-the-art object sensitivity \oneobjHT.
%The algorithm failed to learn a better tunneling heuristic for object sensitivity mainly because the features in Table~\ref{tbl:features} are highly fitted to learn tunneling heuristics for call-site sensitivity. 
%%Note that the existing algorithm successfully learned a far more precise call-site sensitivity (\Existing) than the \onecallHT~with the newly designed features.
%The features are carefully crafted for call-site sensitivity (e.g., we crafted features with~\oursim) instead of object sensitivity. 
%For example, the key feature $C1$, a key feature for call-site sensitivity, we describe above is a bad feature for object sensitivity.
%Unlike call-site-sensitivity that applying context tunneling to $C1$ (e.g., $T_{\tt C1}$) make 1-call-site sensitivity more precise than the conventional 2-call-site sensitivity, 
%applying the tunneling abstraction $T_{\tt C1}$ make 1-object-sensitivity even less precise than the conventional 1-object-sensitivity. 
%}
%
%
%\myparagraph{Unexchangeable of learned tunneling heuristics}
%\textcolor{red}{
% We checked tunneling heuristics of each context are not exchangeable.
%The tunneling heuristic used in \ours~is inadequate to object sensitivity;
%1-object sensitivity with the tunneling heuristic is even less precise than the one without tunneling (conventional 1-object sensitive analysis).
%The tunneling heuristic used in \oneobjHT~is also a bad tunneling heuristic for call-site sensitivity.}



%\input{table_zipper}

\subsection{Comparison with Selective Object Sensitivity}\label{sec:SelectiveCtx}

We also checked how \ours~works in comparison with \Zipper~\cite{Li2018a,ZipperJournal20}, a state-of-the-art technique that performs selective 2-object sensitivity. 
Unlike the analyses considered in Section~\ref{sec:performance}, which perform uniform $k$-context sensitivity,  \Zipper~performs a selective context-sensitive analysis and applies $2$-object sensitivity only when it is necessary. In particular, compared to other selective approaches~\cite{Li2018b,JeJeChOh17,Smaragdakis2014,Graphick20}, \Zipper~is precision-focused and 
improves the scalability of uniform 2-object sensitivity while preserving most (98.8\%) of its precision. 
%
%We compare \ours~with object-sensitivity equipped with a context sensitivity heuristic \Zipper~\cite{Li2018a}. 
%Recently, various selective context sensitivity heuristics have been proposed to develop cost-effective object-sensitive analysis~\cite{Li2018a,Li2018b,Lu:2019:PYF,JeJeChOh17,Smaragdakis2014,Graphick20}. 
%As it is usually too expensive to apply object sensitivity (e.g., 2-object sensitivity) to all method calls,
%context sensitivity heuristics apply object-sensitivity only when doing so would increase precision~\cite{Li2018a,Lu:2019:PYF}.
%Otherwise, the heuristics use alternative cheap contexts (e.g., type sensitivity or context insensitivity) to improve scalability~\cite{Smaragdakis2014,Li2018b}.
%\Zipper~is a precision-focused context sensitivity heuristic that preserves almost (98.8\%) full precision of the conventional 2-object-sensitive analysis.

%\myparagraph{Setup}
For direct comparison, we implemented \ours~and \oneobjHT~on top of the artifact provided by~\cite{Li2018a}.
We used 14 programs (batik, fop, sunflow, bloat, xalan, chart, findbugs, eclipse, jpc, checkstyle, h2, xalan09, avrora09, sunflow09) used in~\cite{Li2018a}.
Five programs (fop, xalan, bloat, chart, eclipse) are DaCapo 2006 benchmarks~\cite{Blackburn2006}, four programs (xalan09, sunflow09, avrora09, and h2) are from DaCapo 2009 benchmarks, and the remaining five programs (sunflow, findbugs, jpc, checkstle, batik) are real-world Java applications.
Note that the artifact of~\cite{Li2018a} uses a quite different reflection analysis from the one we used in Table~\ref{tbl:eval:main}.
The number of alarms and analysis time varies significantly depending on how reflection is supported. 
The baseline analysis on top of \Zipper~is implemented supports reflection less conservatively than our implementation in Section~\ref{sec:performance} and therefore Table~\ref{tbl:eval:vsZipper} reports fewer alarms than Table~\ref{tbl:eval:main}. In Table~\ref{tbl:eval:vsZipper}, our analysis uses the same reflection analysis as \Zipper. 


%\myparagraph{Results}
Table~\ref{tbl:eval:vsZipper} compares the performance of \ours~and \Zipper~(denoted \twoobjZipper~in Table~\ref{tbl:eval:vsZipper}). 
For all programs, \ours~shows a better precision than \Zipper. Especially for bloat, \ours~is more precise and scalable than  than \twoobjZipper; \twoobjZipper~took 1,942 seconds and reports 1,224 alarms but \ours~analyzed it within 456 seconds with 1,136 may-fail cast alarms. However, for batik, \ours~was slower than \twoobjZipper. 

%\oneobjHT~shows comparable precision with \twoobjZipper. It shows better precision than \twoobjZipper~for 8 programs (eclipse, bloat, chart, checkstyle, avrora09, xalan09, findbugs, xalan) while \oneobjHT~is less precise than \twoobjZipper~for the others.
%%\ours~also shows good scalability. For 8 programs (luindex, lusearch, antlr, pmd, chart, xalan, jpc, and checkstyle), \ours~is faster than the others.

% \begin{table}[]
% \label{tbl:oursZipper}
% \caption{\TODO{Why not report all programs? Include the result for batik}Performance of our context tunneled 1-call-site sensitivity with \Zipper~(\oursZipper).}
% \begin{tabular}{@{}lccccccc@{}}
% \toprule
% Program         & \multicolumn{1}{c}{h2}  & \multicolumn{1}{c}{fop}   & \multicolumn{1}{c}{batik}      & \multicolumn{1}{c}{sunflow}   & \multicolumn{1}{c}{bloat}   & \multicolumn{1}{c}{findbugs} & \multicolumn{1}{c}{chart}   \\ \midrule
% \#may-fail cast & 1,225                   & 1,397                     & 1,782                          & 1,818                         & 1,168                       & 1,360                        & 890                         \\
% Time (s)        & 2,375                   & 385                       & 1,812                          & 377                           & 424                         & 207                        & 102                          \\\midrule
% Program         & \multicolumn{1}{c}{jpc} & \multicolumn{1}{c}{xalan} & \multicolumn{1}{c}{checkstyle} & \multicolumn{1}{c}{sunflow09} & \multicolumn{1}{c}{xalan09} & \multicolumn{1}{c}{avrora09} & \multicolumn{1}{c}{eclipse} \\\midrule
% \#may-fail cast & 1,368                   & 482                       & 535                            & 1,146                              &   1,004                          &      964                        & 482                         \\
% Time (s)        & 258                    & 113                        & 135                             &   135                            &    224                         &  163                            & 86                          \\ \bottomrule
% \end{tabular}
% \end{table}
%
%\begin{table}[]
%\caption{Performance of our context tunneled 1-call-site sensitivity with \Zipper~(\oursZipper).}
%\label{tbl:oursZipper}
%\begin{tabular}{@{}lccccccc@{}}
%\toprule
%Program         & \multicolumn{1}{c}{h2}  & \multicolumn{1}{c}{fop}   & \multicolumn{1}{c}{batik}      & \multicolumn{1}{c}{sunflow}   & \multicolumn{1}{c}{bloat}   & \multicolumn{1}{c}{findbugs} & \multicolumn{1}{c}{chart}   \\ \midrule
%\failcasts & 1,225                   & 1,397                     & 1,782                          & 1,818                         & 1,168                       & 1,360                        & 890                         \\
%Time (s)        & 2,318                  & 279                      & 1,649                       & 279                          & 397                        & 164                         & 58                        \\\midrule
%Program         & \multicolumn{1}{c}{jpc} & \multicolumn{1}{c}{xalan} & \multicolumn{1}{c}{checkstyle} & \multicolumn{1}{c}{sunflow09} & \multicolumn{1}{c}{xalan09} & \multicolumn{1}{c}{avrora09} & \multicolumn{1}{c}{eclipse} \\\midrule
%\failcasts & 1,368                   & 482                       & 535                            & 1,146                              &   1,004                          &      964                        & 482                         \\
%Time (s)        & 176                     & 72                        & 88                             &   84                            &    166                         &  105                            & 42                          \\ \bottomrule
%\end{tabular}
%\end{table}
%
%

%\myparagraph{Combination with Zipper}

% !TEX root = ./paper.tex

%\begin{table}[]
\begin{table}[]
\setlength\extrarowheight{-1pt}
\caption{Performance comparison of \ours, \twoobjZipper~(i.e., \Zipper~\cite{Li2018a}), and \oursZipper. 
%We also evaluate the baseline state-of-the-art 1-object sensitivity with tunneling~\oneobjHT. 
%Unlike tunneling heuristics used in \ours~or \oneobjHT~that determine whether to update contexts, the heuristic Zipper determine whether to apply 2-object sensitivity or context insensitivity for each method call.
}
\label{tbl:eval:vsZipper}
\centering\scriptsize
\begin{tabular}{@{}clrr | clrr@{}}
\toprule
Program                     & \multicolumn{1}{c}{Analysis} & \multicolumn{1}{c}{\#fail-casts} & \multicolumn{1}{c |}{Time} & Program                   & \multicolumn{1}{c}{Analysis} & \multicolumn{1}{c}{\#fail-casts} & \multicolumn{1}{c}{Time} \\ \midrule
\multirow{3}{*}{eclipse}    & \ours                    & 460                              & 62                           & \multirow{3}{*}{fop}      & \ours                    & 1,359                             & 368                          \\
                            & \oursZipper                      & 482                              & 42                           &                           & \oursZipper                      & 1,397                             & 279                          \\
                            & \twoobjZipper                  & 586                              & 45                          &                           & \twoobjZipper                  & 1,471                             & 318                          \\\midrule
\multirow{3}{*}{bloat}      & \ours                    & 1,136                             & 456                          & \multirow{3}{*}{jpc}      & \ours                    & 1,279                             & 201                          \\
                            & \oursZipper                      & 1,168                             & 397                          &                           & \oursZipper                      & 1,368                             & 176                          \\
                            & \twoobjZipper                  & 1,224                             & 1,942                         &                           & \twoobjZipper                  & 1,415                             & 147                          \\\midrule
\multirow{3}{*}{chart}      & \ours                    & 810                              & 80                           & \multirow{3}{*}{xalan}    & \ours                    & 463                              & 91                           \\
                            & \oursZipper                      & 890                              & 58                           &                           & \oursZipper                      & 482                              & 72                          \\
                            & \twoobjZipper                  & 910                              & 54                           &                           & \twoobjZipper                  & 568                              & 66                          \\\midrule
\multirow{3}{*}{sunflow}    & \ours                    & 1,787                             & 438                          & \multirow{3}{*}{findbugs} & \ours                    & 1,316                             & 288                          \\
                            & \oursZipper                      & 1,818                             & 279                          &                           & \oursZipper                      & 1,360                             & 164                          \\
                            & \twoobjZipper                  & 1,869                             & 361                          &                           & \twoobjZipper                  & 1,437                             & 604                          \\\midrule
\multirow{3}{*}{checkstyle} & \ours                    & 522                             & 106                          & \multirow{3}{*}{batik}    & \ours                    & 1,602                             & 1,422                         \\
                            & \oursZipper                      & 535                              & 88                          &                           & \oursZipper                      & 1,782                             & 1,649                          \\
                            & \twoobjZipper                  & 607                              & 248                          &                           & \twoobjZipper                  & 1,614                             & 667                          \\\midrule
\multirow{3}{*}{sunflow09}  & \ours                    & 1,101                             & 126                          & \multirow{3}{*}{xalan09}  & \ours                    & 989                              & 241                          \\
                            & \oursZipper                      & 1,146                             & 84                           &                           & \oursZipper                      & 1,004                             & 166                          \\ 
                            & \twoobjZipper                  & 1,192                             &73                           &                           & \twoobjZipper                  & 1,074                             & 214                          \\\midrule
\multirow{3}{*}{avrora09}   & \ours                    & 918                              & 147                          & \multirow{3}{*}{h2} & \ours                    & 1,207                             & 3,766                          \\
                            & \oursZipper                      & 964                             & 105                          &                           & \oursZipper                      &  1,225                            & 2,318                          \\
                            & \twoobjZipper                  & 1,042                             & 115                          &                           & \twoobjZipper                  & 1,311                             & 8,216                          \\%\midrule

\bottomrule
                            
\end{tabular}
\vspace{-10pt}
\end{table}

% \begin{table}[]
% \caption{Comparison of our 1-call-site sensitivity (\ours) against Zipper~(\twoobjZipper~\cite{Li2018a}). 
% %We also evaluate the baseline state-of-the-art 1-object sensitivity with tunneling~\oneobjHT. 
% %Unlike tunneling heuristics used in \ours~or \oneobjHT~that determine whether to update contexts, the heuristic Zipper determine whether to apply 2-object sensitivity or context insensitivity for each method call.
% }
% \label{tbl:eval:vsZipper}
% \begin{tabular}{@{}clrr | clrr@{}}
% \toprule
% Program                     & \multicolumn{1}{c}{Analysis} & \multicolumn{1}{c}{\# fail-casts} & \multicolumn{1}{c |}{sec} & Program                   & \multicolumn{1}{c}{Analysis} & \multicolumn{1}{c}{\# fail-casts} & \multicolumn{1}{c}{sec} \\ \midrule
% \multirow{2}{*}{eclipse}    & \ours                    & 460                              & 62                           & \multirow{2}{*}{fop}      & \ours                    & 1,359                             & 368                          \\
%                             & \twoobjZipper                  & 586                              & 67                          &                           & \twoobjZipper                  & 1,471                             & 424                          \\\midrule
% %                            & \oneobjHT                      & 562                              & 64                           &                           & \oneobjHT                      & 1,472                             & 357                          \\\midrule
% \multirow{2}{*}{bloat}      & \ours                    & 1,136                             & 456                          & \multirow{2}{*}{jpc}      & \ours                    & 1,279                             & 201                          \\
%                             & \twoobjZipper                  & 1,224                             & 1,969                         &                           & \twoobjZipper                  & 1,415                             & 229                          \\\midrule
% %                            & \oneobjHT                      & 1,210                             & 376                          &                           & \oneobjHT                      & 1,418                             & 123                          \\\midrule
% \multirow{2}{*}{chart}      & \ours                    & 810                              & 80                           & \multirow{2}{*}{xalan}    & \ours                    & 463                              & 91                           \\
%                             & \twoobjZipper                  & 910                              & 98                           &                           & \twoobjZipper                  & 568                              & 107                          \\\midrule
% %                            & \oneobjHT                      & 892                              & 80                           &                           & \oneobjHT                      & 541                              & 247                          \\\midrule
% \multirow{2}{*}{sunflow}    & \ours                    & 1,787                             & 438                          & \multirow{2}{*}{findbugs} & \ours                    & 1,316                             & 288                          \\
%                             & \twoobjZipper                  & 1,869                             & 459                          &                           & \twoobjZipper                  & 1,437                             & 647                          \\\midrule
% %                            & \oneobjHT                      & 1,908                             & 240                          &                           & \oneobjHT                      & 1,410                             & 217                          \\\midrule
% \multirow{2}{*}{checkstyle} & \ours                    & 522                             & 106                          & \multirow{2}{*}{batik}    & \ours                    & 1,602                             & 1,422                         \\
%                             & \twoobjZipper                  & 607                              & 299                          &                           & \twoobjZipper                  & 1,614                             & 820                          \\\midrule
% %                            & \oneobjHT                      & 591                              & 199                          &                           & \oneobjHT                      & 1,632                             & 470                          \\\midrule
% \multirow{2}{*}{sunflow09}  & \ours                    & 1,101                             & 126                          & \multirow{2}{*}{xalan09}  & \ours                    & 989                              & 241                          \\
%                             & \twoobjZipper                  & 1,192                             & 124                           &                           & \twoobjZipper                  & 1,074                             & 272                          \\\midrule
% %                            & \oneobjHT                      & 1,230                             & 88                           &                           & \oneobjHT                      & 1,065                             & 362                          \\ \midrule
% \multirow{2}{*}{avrora09}   & \ours                    & 918                              & 147                          & \multirow{2}{*}{h2} & \ours                    & 1,207                             & 3,766                          \\
%                             & \twoobjZipper                  & 1,042                             & 175                          &                           & \twoobjZipper                  & -                             & >5,400                          \\%\midrule
% %                            & \oneobjHT                      & 1,037                             & 160                          &                           & \oneobjHT                      &  1,301                            & 482                          \\

% \bottomrule
% \bottomrule
                            
% \end{tabular}
% \end{table}



Context tunneling and selective context sensitivity~\cite{Oh2014,JeJeChOh17} are orthogonal techniques; we can combine 
our analysis (\ours) and \Zipper, resulting in a selective 1-call-site-sensitive analysis with context tunneling (\oursZipper). 
Intuitively, context tunneling improves the precision of the call-site sensitivity by preserving important context element while selective context sensitivity improves the scalability by selectively applying context sensitivity for the method calls.
This is possible because the idea of \Zipper~is general and applicable to various context-sensitivity flavors including call-site sensitivity~\cite{ZipperJournal20}. 
Table~\ref{tbl:eval:vsZipper} shows the performance of \oursZipper. Thanks to \Zipper, \oursZipper~becomes faster than~\ours~at the small cost of the precision.
In h2, for example, \oursZipper~took 2,318 seconds while \ours~took 3,766 seconds.
One exception is batik, where \oursZipper~is slower than \ours. This is because 
\oursZipper~in this case loses precision significantly (reporting 180 more alarms than \ours), producing spurious points-to sets that make the analysis slow. 





%\myparagraph{Discussion}

%\clearpage


%
%
%Note that context tunneling heuristics (e.g., our technique) and context sensitivity heuristics (\Zipper) are orthogonal; we can use both techniques to develop highly precise and scalable call-site sensitivity. For example, we can apply \Zipper~to improve the scalability of \ours. In \ours, all method calls are analyzed context sensitively (1-call-site sensitivity). 
%Applying context sensitivity to all the method calls, however, makes the analysis. 
%To mitigate the problem,
%we can use the context sensitivity heuristic \Zipper, which is applicable to various context flavors including call-site sensitivity~\cite{ZipperJournal20}. 
%Unfortunately, other existing context sensitivity heuristics~are usually inapplicable to \ours~because they are usually designed~\cite{Li2018b,Lu:2019:PYF} or learned~\cite{Graphick20} to optimize object sensitivity. 
%The context sensitivity heuristic \Zipper~identifies method calls that our call-site sensitivity (\ours) needs to analyzed context sensitively (1-call-site sensitivity). Otherwise, \ours~analyzes them context insensitively. 




%\begin{table}[]
%\label{tbl:oursZipper}
%\begin{tabular}{@{}lrrrrrrrrr@{}}
%\toprule
%\multicolumn{1}{c}{} & \multicolumn{1}{c}{h2} & \multicolumn{1}{c}{fop} & \multicolumn{1}{c}{\TODO{bloat}} & \multicolumn{1}{c}{sunflow} & \multicolumn{1}{c}{\TODO{bloat}} & \multicolumn{1}{c}{findbugs} & \multicolumn{1}{c}{chart} & \multicolumn{1}{c}{avrora09} & \multicolumn{1}{c}{xalan09} \\ \midrule
%\#may-fail cast      & 1,223                   & 1,384                    & 1,166                      & 1,818                        & 1,166                      & 1,352                         & 886                          &  951                     & 1,001                        \\
%Time (s)             & 2,414                   & 310                     & 421                       & 279                         & 421                       & 172                          & 71                          & 117                        & 169                         \\ \bottomrule
%\end{tabular}
%\end{table}

% Please add the following required packages to your document preamble:
% \usepackage{booktabs}



\begin{table}[t]
\setlength\extrarowheight{-1pt}
\centering
\caption{Precision and scalability comparison between 1-hybrid-object sensitivity with tunneling (\onesobjHT) against its simulated call-site sensitivity (\simonecallH) and the call-site sensitivity obtained from learning (\sobjSimLearn). 
%The columns Time(s) of \simonecallH~present sum of its baseline analysis cost(s) (e.g.,~\onesobjHT) and the simulated call-site sensitive analysis cost(s). 
%All the other notations are the same with Table~\ref{tbl:eval:main}.
}
\label{tbl:sobj}
\centering
\footnotesize

\begin{tabular}{@{} c l r r | c l r r @{}}
\toprule
 Program                  & \,  Analysis   & \multicolumn{1}{c}{\, \#fail-casts} & \multicolumn{1}{c | }{ Times(s)} & Program                     & \, Analysis   & \multicolumn{1}{c}{\, \#fail-casts} & \multicolumn{1}{c}{Time(s)} \\ \midrule
\multirow{3}{*}{luindex} & \sobjSimLearn & 366                             & 36                           & \multirow{3}{*}{lusearch}   & \sobjSimLearn & 380                              & 35                          \\
                         & \simonecallH  & 359                             & 67                           &                             & \simonecallH  & 373                              & 69                          \\
                         & \onesobjHT   & 371                             & 32                           &                             & \onesobjHT   & 380                              & 33                          \\\midrule
\multirow{3}{*}{antlr}   & \sobjSimLearn & 486                             & 54                           & \multirow{3}{*}{eclipse}    & \sobjSimLearn & 586                              & 40                          \\
                         & \simonecallH  & 479                             & 95                           &                             & \simonecallH  & 573                              & 77                          \\
                         & \onesobjHT   & 483                             & 48                           &                             & \onesobjHT   & 586                              & 36                          \\\midrule
\multirow{3}{*}{xalan}   & \sobjSimLearn & 576                             & 68                           & \multirow{3}{*}{chart}      & \sobjSimLearn & 942                              & 67                          \\
                         & \simonecallH  & 562                             & 123                          &                             & \simonecallH  & 861                              & 139                         \\
                         & \onesobjHT   & 572                             & 59                           &                             & \onesobjHT   & 876                              & 62                          \\\midrule
\multirow{3}{*}{bloat}   & \sobjSimLearn & 1,267                           & 456                          & \multirow{3}{*}{jython}     & \sobjSimLearn & 856                              & 194                         \\
                         & \simonecallH  & 1,248                           & 870                          &                             & \simonecallH  & 836                              & 8,985                      \\
                         & \onesobjHT   & 1,251                           & 375                          &                             & \onesobjHT   & 837                              & 342                         \\\midrule
\multirow{3}{*}{jpc}     & \sobjSimLearn & 1,676                           & 131                          & \multirow{3}{*}{pmd}        & \sobjSimLearn & 721                              & 59                          \\
                         & \simonecallH  & 1,644                           & 3,002                        &                             & \simonecallH  & 710                              & 50                          \\
                         & \onesobjHT   & 1,593                           & 186                          &                             & \onesobjHT   & 713                              & 52                          \\\midrule
\multirow{3}{*}{fop}     & \sobjSimLearn & 1,084                           & 123                          & \multirow{3}{*}{checkstyle} & \sobjSimLearn & 466                              & 63                          \\
                         & \simonecallH  & 1,055                           & 959                          &                             & \simonecallH  & 468                              & 135                         \\
                         & \onesobjHT   & 1,080                           & 94                           &                             & \onesobjHT   & 474                              & 69                          \\ \bottomrule
\end{tabular}
\vspace{-15pt}
\end{table}



\subsection{Applicability to Variations of Object Sensitivity}

In this chapter, we focused on the original object sensitivity~\cite{Milanova2002} as it is the most widely known and used (e.g.,~\cite{Li2018b,Li2018a,TanLX16,Smaragdakis2014,GordonKPGNR_NDSS15,Feng2014,Lu:2019:PYF,Li2018a}). 
Therefore, though we believe our claim holds for variations of object sensitivity as well (e.g., hybrid context sensitivity~\cite{KastrinisS13a}, type sensitivity~\cite{Smaragdakis2011}), we do not claim that \ourtechnique~in its present form is readily applicable to them. 
Note that we have designed \ourtechnique~by exploiting the properties of original object sensitivity (e.g., call-graph patterns and atomic features). To apply \ourtechnique~to its variations, we may need domain-specific tuning of the simulation and learning techniques (e.g., the inference rules in Section~\ref{sec:simulation} and feature engineering in Section~\ref{sec:learning}). 
It would be interesting future work to generalize our results for other variants of object sensitivity.
%Minseok
%\textcolor{red}{It would be interesting future work to generalize our results for other variants of object sensitivity. }


When we simply used \ourtechnique~to hybrid object sensitivity (in the setting of Section~\ref{sec:performance}), for example, we found that the results are encouraging yet suboptimal. 
Hybrid context sensitivity~\cite{KastrinisS13a} is a variant of object sensitivity that selectively combines call-site and object sensitivity, and the state-of-the-art is the context-tunneled version,  \onesobjHT~\cite{JeJeOh18}. 
We applied \ourtechnique~to \onesobjHT~and transformed it into a call-site-sensitive analysis. 
We compared the performance of \onesobjHT~with the simulated call-site sensitivity without learning (denoted \simonecallH), and the final analysis with learning (denoted \sobjSimLearn). 
Table~\ref{tbl:sobj} shows that, though suboptimal, our simulation technique overall makes hybrid context sensitivity more precise (\simonecallH~vs. \onesobjHT), indicating that hybrid context sensitivity can benefit from our approach. However, the learned analysis (\sobjSimLearn) shows overall worse precision than hybrid context sensitivity.  
The failure of learning comes from the lack of atomic features appropriate for hybrid object sensitivity. We leave adaptation to hybrid object sensitivity as future work. 


\section{Open Question: Is Complete Simulation Possible?}
In this thesis, we showed that it is practically possible to outperform object sensitivity via contexttunneled call-site sensitivity. However, it remains to be seen whether or not call-site sensitivity can
be fundamentally superior to object sensitivity. 
We discuss this in Appendix~\ref{sec:setting}

%\clearpage
%%% Local Variables:
%%% mode: latex
%%% TeX-master: "../thesis.tex"
%%% End:
