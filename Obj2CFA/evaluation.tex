% !TEX root = ./paper.tex

\section{Experimental Results}\label{sec:evaluation}\label{sec:result}
We experimentally prove our claim by evaluating \ourtechnique~on real-world programs. 
Main research questions are as follows: 
%We aim to
%answer the following research questions: 
\begin{itemize}[leftmargin=1.3em]
\item \textbf{Does our claim hold in the real-world?} 
Can call-site sensitivity be significantly superior to object sensitivity for real-world programs? 
How precise and scalable can the
  context-tunneled call-site-sensitive analysis be in practice? 
%  obtained from our
%  technique?
  % effective is out technique that transforms object sensitivity into
  % a context-tunneled call-site sensitivity?
%  How far does it advance the state-of-the-art
%object-sensitive analyses?

%\item \textbf{Effectiveness of simulation and learning:} How
%  accurately can our technique simulate object sensitivity?  Does
%  learning effectively capture the behavior of the simulated policy?
%  Is our learning algorithm essential? How
%  does it compare to simpler approaches?
\item \textbf{Impact of simulation and learning}: 
Is simulation necessary? 
How accurately can our technique simulate object sensitivity?  Is the simulation-guided learning necessary to capture the behavior of the simulated policy? How important are the features in learning?


\end{itemize}


%\input{table_sobj}
%\input{table_2call}

\myparagraph{Experimental Setting}
We implemented \ourtechnique~on top of Doop~\cite{BravenboerS09}, a
popular pointer analysis framework for Java~\cite{JeJeOh18, Li2018b,
  JeJeChOh17, TanLX16, Smaragdakis2014}.  
%\textcolor{red}{Note that superiority of object sensitivity over call-site sensitivity has been consistently demonstrated on Doop~\cite{BravenboerS09,TanLX16,KastrinisS13a,Smaragdakis2014}.}
%To implement our
%popular pointer analysis framework for Java~\cite{JeJeOh18, Li2018b,
%  JeJeChOh17, Tan2017, TanLX16, Smaragdakis2014}.  To implement our
%technique, 
We used the publicly-available implementation of context
tunneling given by \cite{JeJeOh18} and newly implemented our
simulation (Section~\ref{sec:simulation}) and learning
(Section~\ref{sec:learning}) techniques in Doop.  We conducted all
experiments on a machine with Intel i7 CPU and 64GB memory running the
Ubuntu 16.04 64bit operating system.


%\paragraph{Benchmarks}
We used 12 Java programs used by~\cite{JeJeOh18}, of which 10 came from the DaCapo 2006
benchmarks~\cite{Blackburn2006} ({luindex}, {lusearch}, {antlr},
{pmd}, {fop}, {eclipse}, {xalan}, {chart}, {bloat}, and {jython}), and
the remaining two ({checkstyle} and {jpc}) are real-world open-source programs.
%\textcolor{red}{Note that the DaCapo benchmarks also have consistently been used to experimentally demonstrate the superiority of object sensitivity over call-site sensitivity~\cite{Lhotak2006,BravenboerS09,TanLX16}.}
Following prior work~\cite{JeJeOh18,Smaragdakis2014}, we classified those 12
programs into 4 small (luindex, lusearch, antlr, and pmd) and 8 large
programs. 
We used the group of small programs as training data, 
from which our context-tunneling policy $\callheuristic$ is learned, 
and used large programs as test data to evaluate the
policy for unseen programs.


%Pointer analysis START
%analysis time: 123.01s
%Pointer analysis FINISH
%disk footprint (KB)                                                              959,488
%loading statistics (simple) ...
%elapsed time: 11.96s
%
%making database available at /home/minseok/Graphick/Ctx_Sensitivity/doop/results/C-1-tunneled-call-site-sensitive+heap/jre1.6/jython.jar
%making database available at last-analysis
%#var points-to            20,027,400
%#may-fail casts           1,723
%#poly calls               2,498
%#reach methods            12,011
%#call edges               107,112



%\paragraph{Analyses}

Using \ourtechnique, we transformed \oneobjHT, a context-tunneled
1-object-sensitive analysis developed by \cite{JeJeOh18}, into our
1-call-site-sensitive analysis, denoted \ours.  We chose
\oneobjHT~as baseline because it is one of the best object-sensitive analyses
available today, which boosts the conventional 1-object-sensitive
analysis using a well-tuned context-tunneling policy. For
example, \oneobjHT~is empirically more precise than conventional
2-object-sensitive analysis (\twoobjH), which is considered to be highly
precise~\cite{JeJeChOh17,Li2018a,Li2018b,Smaragdakis2014,Graphick20}, yet more
scalable than 1-object sensitivity~\cite{JeJeOh18}.  We obtained the
tunneling policy of {\oneobjHT} from the publicly available artifact
of~\cite{JeJeOh18}.
From {\oneobjHT}, we first applied our simulation
  technique (Section~\ref{sec:simulation}) to produce
  the corresponding simulated call-site sensitivity, denoted \oursim. Then, we
  used our learning algorithm (Section~\ref{sec:learning}) to obtain the final 
   call-site-sensitive analysis, \ours. 
%\[
%\text{\oneobjHT}\xrightarrow[]{\text{simulation}} \text{\oursim} \xrightarrow[]{\text{learning}}\text{\ours}.
%\]
Note that {\oursim} runs {\oneobjHT} as a pre-analysis but {\ours} does not (thanks to learning). 

Our main objective is to compare \oneobjHT~and
\ours, but we compare with some notable analyses as well to
see the advance more clearly. In summary, we compare the following
 analyses:
\begin{itemize}[leftmargin=1.3em]
\item \oneobjHT: a state-of-the-art context-tunneled 1-object-sensitive analysis~\cite{JeJeOh18}
\item \oursim: the simulated 1-call-site-sensitive analysis
    obtained from~\oneobjHT~via 
    simulation% ($\simheuristic$)
\item \ours: our final 1-call-site sensitive analysis
  (obtained from \oursim~via learning)
\item \twoobjH: 2-object-sensitive analysis without
  tunneling~\cite{Smaragdakis2011}
%\item \twocallH: 2-call-site-sensitive analysis without
%  tunneling~\cite{Smaragdakis2011}
\item \onecallHT: the existing state-of-the-art 1-call-site sensitivity with
  tunneling~\cite{JeJeOh18}
\end{itemize}
\twoobjH~is available in Doop. 
\onecallHT~is available in the artifact provided by \cite{JeJeOh18}. 
%\onecallHT~is the existing
%state-of-the-art 1-call-site-sensitive analysis with context
%tunneling, which is available in the artifact of \cite{JeJeOh18}. 
%\textcolor{red}{
%\onecallHT~also boosts the conventional 1-call-site-sensitive analysis. 
%Empirically, \onecallHT~is both more precise and scalable than the conventional 2-call-site-sensitive analysis~\cite{JeJeOh18}.}
All analyses use 1-context-sensitive heap.
For precision metric, we mainly use may-fail
casts. % (\failcasts). % because % it is used in our learning
% algorithm (Section~\ref{sec:learning}) and
%the existing analyses 
%(\oneobjHT~and \onecallHT) have been tuned for it~\cite{JeJeOh18}.
%\begin{comment}
%(\oneobjHT~and \onecallHT) have been tuned with the metric~\cite{JeJeOh18}.
%\end{comment}
%For scalability, we report analysis time in seconds.

\begin{table}
\setlength\extrarowheight{-1pt}

\caption{Precision and cost comparison of our analysis (\ours) against various
  context-sensitive analyses: \oneobjHT, \twoobjH, \onecallHT, and \oursim.
}
\label{tbl:eval:main}
\centering
\scriptsize

\begin{tabular}{@{}c | clrrrrr@{}}
\toprule
                                    & program                     & \multicolumn{1}{c}{Metric} & \multicolumn{1}{c}{\ours} & \multicolumn{1}{c}{\oursim} & \multicolumn{1}{c}{\oneobjHT} & \multicolumn{1}{c}{\twoobjH} & \multicolumn{1}{c}{\onecallHT} \\ \midrule
\multirow{12}{*}{\rotatebox[origin=c]{90}{Training programs}} & \multirow{3}{*}{luindex}    & VarPtsTo                   & 250,012                       & 245,470                      & 256,531                     & 255,545                   & 800,715                      \\
                                    &                             & \failcasts             & 357                           & 360                          & 462                         & 496                       & 784                          \\
                                    &                             & time elapsed(s)            & 40                            & 86                           & 37                          & 40                        & 82                           \\\cmidrule(){2-8}
                                    & \multirow{3}{*}{lusearch}   & VarPtsTo                   & 264,728                       & 260,204                      & 271,765                     & 270,710                   & 890,529                      \\
                                    &                             & \failcasts             & 371                           & 374                          & 469                         & 508                       & 843                          \\
                                    &                             & time elapsed(s)            & 45                            & 94                           & 39                          & 82                        & 85                           \\\cmidrule(){2-8}
                                    & \multirow{3}{*}{antlr}      & VarPtsTo                   & 302,226                       & 297,268                      & 309,671                     & 308,643                   & 965,445                      \\
                                    &                             & \failcasts             & 477                           & 477                          & 570                         & 611                       & 945                          \\
                                    &                             & time elapsed(s)            & 62                            & 123                          & 52                          & 52                        & 128                          \\\cmidrule(){2-8}
                                    & \multirow{3}{*}{pmd}        & VarPtsTo                   & 306,462                       & 300,391                      & 329,415                     & 327,295                   & 1,116,506                    \\
                                    &                             & \failcasts             & 707                           & 711                          & 812                         & 846                       & 1,200                        \\
                                    &                             & time elapsed(s)            & 65                            & 128                          & 56                          & 138                       & 138                          \\\midrule\midrule
\multirow{24}{*}{\rotatebox[origin=c]{90}{Testing programs}}  & \multirow{3}{*}{eclipse}    & VarPtsTo                   & 353,657                       & 337,496                      & 351,898                     & 345,806                   & 1,241,995                    \\
                                    &                             & \failcasts             & 569                           & 573                          & 698                         & 729                       & 1,073                        \\
                                    &                             & time elapsed(s)            & 48                            & 159                          & 47                          & 58                        & 136                          \\\cmidrule(){2-8}
                                    & \multirow{3}{*}{xalan}      & VarPtsTo                   & 410,440                       & 394,522                      & 401,556                     & 400,872                   & 1,660,901                    \\
                                    &                             & \failcasts             & 576                           & 586                          & 680                         &                           720& 1,137                        \\
                                    &                             & time elapsed(s)            & 71                            & 590                          & 377                         & 2,288                     & 208                          \\\cmidrule(){2-8}
                                    & \multirow{3}{*}{chart}      & VarPtsTo                   & 501,615                       & 496,676                      & 502,913                     & 500,357                   & 4,694,330                    \\
                                    &                             & \failcasts             & 883                           & 942                          & 1,011                       & 1,055                     & 2,376                        \\
                                    &                             & time elapsed(s)            & 96                            & 575                          & 84                          & 382                       & 805                          \\\cmidrule(){2-8}
                                    & \multirow{3}{*}{fop}        & VarPtsTo                   & 650,218                       & 637,213                      & 726,777                     & 720,031                   & 3,467,105                    \\
                                    &                             & \failcasts             & 1,072                         & 1,072                        & 1,253                       & 1,270                     & 1,977                        \\
                                    &                             & time elapsed(s)            & 206                           & 407                        & 137                         & 493                       & 500                          \\\cmidrule(){2-8}
                                    & \multirow{3}{*}{bloat}      & VarPtsTo                   & 1,136,393                     & 1,136,366                    & 1,126,688                   & 1,114,648                 & 3,454,301                    \\
                                    &                             & \failcasts             & 1,266                         & 1,285                        & 1,374                       & 1,407                     & 1,949                        \\
                                    &                             & time elapsed(s)            & 498                           & 3,306                        & 371                         & 2,463                     & 805                          \\\cmidrule(){2-8}
                                    & \multirow{3}{*}{jython}     & VarPtsTo                   & 1,067,711                     & N/A                          &-                             & -                          & 3,085,401                    \\
                                    &                             & \failcasts             & 845                           & N/A                          &  -                           &   -                        & 1,331                        \\
                                    &                             & time elapsed(s)            & 2,731                         & N/A                          & >10,800         & >10,800       & 188                          \\\cmidrule(){2-8}
                                    & \multirow{3}{*}{jpc}        & VarPtsTo                   & 1,304,810                     & 1,118,622                    & 1,142,496                   & 1,114,946                 & 6,667,910                    \\
                                    &                             & \failcasts             & 1,639                         & 1,642                        & 1,795                       & 1,814                     & 2,620                        \\
                                    &                             & time elapsed(s)            & 493                           & 699                          & 262                         & 1,737                     & 1,511                        \\\cmidrule(){2-8}
                                    & \multirow{3}{*}{checkstyle} & VarPtsTo                   & 307,378                       & 299,101                      & 327,629                     & 314,857                   & 1,141,902                    \\
                                    &                             & \failcasts             & 465                           & 472                          & 591                         & 620                       & 913                          \\
                                    &                             & time elapsed(s)            & 83                            & 220                          & 99                          & 220                       & 139                          \\  \bottomrule
\end{tabular}
\end{table}

\begin{table}
\setlength\extrarowheight{-1pt}
\caption{Precision of call-graph related clients (\callgraphedges, \reachableMethods, \polycalls) of the analyses. Again, lower is better for all metrics. 
}
\label{tbl:eval:graph}
\centering
\scriptsize
\begin{tabular}{@{}c | clrrrrr@{}}
\toprule
                                    & program                     & \multicolumn{1}{c}{Metric} & \multicolumn{1}{c}{\ours} & \multicolumn{1}{c}{\oursim} & \multicolumn{1}{c}{\oneobjHT} & \multicolumn{1}{c}{\twoobjH} & \multicolumn{1}{c}{\onecallHT} \\ \midrule
\multirow{12}{*}{\rotatebox[origin=c]{90}{Training programs}} & \multirow{3}{*}{luindex}    & \callgraphedges                   &    36,578                    & 36,426                      & 36,504                     &  36,487                  &  40,830                     \\
                                    &                             & \reachableMethods            &  7,710                          & 7,699                          & 7,702                         &  7,702                      &  7,879                         \\
                                    &                             & \polycalls            & 908                            &   900                         & 905                          &   903                      & 1,066                           \\\cmidrule(){2-8}
                                    & \multirow{3}{*}{lusearch}   & \callgraphedges                   & 39,456                       & 39,304                      & 39,381                     &  39,362                  &   44,007                    \\
                                    &                             & \reachableMethods            &  8,354                          & 8,343                          & 8,344                         &  8,344                      & 8,551                          \\
                                    &                             & \polycalls            & 1,086                            & 1,078                           & 1,078                          &  1,075                       & 1,243                           \\\cmidrule(){2-8}
                                    & \multirow{3}{*}{antlr}      & \callgraphedges                   &  55,467                      &  55,396                     &  55,474                    & 55,455                   &  59,818                     \\
                                    &                             & \reachableMethods            & 8,721                           & 8,711                          &  8,714                        & 8,714                       & 8,885                          \\
                                    &                             & \polycalls            &  1,722                           & 1,709                          & 1,709                          &  1,716                       & 1,876                          \\\cmidrule(){2-8}
                                    & \multirow{3}{*}{pmd}        & \callgraphedges                   & 42,980                       & 42,909                      & 43,015                     & 42,998                   &  47,889                   \\
                                    &                             & \reachableMethods            & 9,095                           &   9,085                        &  9,090                        &  9,090                      & 9,296                        \\
                                    &                             & \polycalls            &  951                           &   943                        &  947                         &  946                      & 1,117                          \\\midrule\midrule
\multirow{24}{*}{\rotatebox[origin=c]{90}{Testing programs}}  & \multirow{3}{*}{eclipse}    & \callgraphedges                   & 44,947                       &  44,842                     &  44,926                    &  44,824                  &     51,724                \\
                                    &                             & \reachableMethods            &  9,204                          &   9,194                        &    9,197                      &   9,188                     &  9,444                       \\
                                    &                             & \polycalls            & 1,184                            &    1,175                       &  1,181                         &  1,179                       & 1,399                          \\\cmidrule(){2-8}
                                    & \multirow{3}{*}{xalan}      & \callgraphedges                   & 50,061                       &   49,985                    &    50,065                  &  50,051                  & 55,644                    \\
                                    &                             & \reachableMethods            & 10,338                           &  10,331                         &  10,336                        &     10,336                      & 10,539                        \\
                                    &                             & \polycalls            & 1,637                            &    1,630                       &  1,633                        &  1,628                    &  1,858                         \\\cmidrule(){2-8}
                                    & \multirow{3}{*}{chart}      & \callgraphedges                   &  58,933                      &   58,912                    &  58,993                    &  59,035                  &   80,500                  \\
                                    &                             & \reachableMethods            & 12,500                           &  12,495                         & 12,510                       &   12,510                   &  16,020                       \\
                                    &                             & \polycalls            &  1,609                           & 1,605                          &  1,616                         &  1,614                      &  2,698                         \\\cmidrule(){2-8}
                                    & \multirow{3}{*}{fop}        & \callgraphedges                   & 59,663                       &  59,440                     &   61,975                   &   61,923                 &   71,741                  \\
                                    &                             & \reachableMethods            &  13,777                        &  13,763                       &      14,376                  &  14,373                    &   15,108                      \\
                                    &                             & \polycalls            &  1,962                          &   2,063                      &         2,063                 &    2,047                    &   2,522                        \\\cmidrule(){2-8}
                                    & \multirow{3}{*}{bloat}      & \callgraphedges                   & 61,249                     &  60,990                   & 60,638                   &  60,601                &  68,674                   \\
                                    &                             & \reachableMethods            &  9,947                        &  9,928                       & 9,914                       &       9,914               &        10,113                 \\
                                    &                             & \polycalls            & 1,679                           &   1,667                      &  1,652                        &   1,650                   &        1,925                   \\\cmidrule(){2-8}
                                    & \multirow{3}{*}{jython}     & \callgraphedges                   & 52,644                     & N/A                          &N/A                             & N/A                          &  59,932                   \\
                                    &                             & \reachableMethods            & 10,625                           & N/A                          &  N/A                           &   N/A                        & 10,987                        \\
                                    &                             & \polycalls            &  14,084                        & N/A                          & N/A         & N/A       &  1,565                         \\\cmidrule(){2-8}
                                    & \multirow{3}{*}{jpc}        & \callgraphedges                   & 95,837                     &  95,098                   &    95,371                &  95,209                &   110,493                  \\
                                    &                             & \reachableMethods            & 18,634                         &  18,581                       & 18,655                       &  18,631                    &  19,854                       \\
                                    &                             & \polycalls            &  5,053                          &  4,989                         &       4,999                   &   4,963                   &   5,646                      \\\cmidrule(){2-8}
                                    & \multirow{3}{*}{checkstyle} & \callgraphedges                   & 42,410                       &   42,333                    &  42,204                    & 42,174                   &  49,346                   \\
                                    &                             & \reachableMethods            & 8,435                           & 8,424                          &  8,428                        & 8,428                       & 8,672                          \\
                                    &                             & \polycalls            &  1,096                           &  1,088                         &  1,090                         & 1,088                       &  1,304                         \\  \bottomrule
\end{tabular}
\end{table}







\subsection{Performance of \ours}\label{sec:performance}


Table~\ref{tbl:eval:main} %compares the precision and cost of the four
%analyses, which 
shows that our analysis (\ours) significantly outperforms
other analyses in both precision and cost, confirming our claim that call-site sensitivity can be superior to object sensitivity for real-world programs. 
In particular, it beats by far the baseline object sensitivity
(\oneobjHT) in precision for all programs.  For example,
\oneobjHT~reports 1,253 may-fail casts for fop but \ours~reduces the
number to 1,072. 
Also, \ours~is more scalable than \oneobjHT. For example, \ours~takes 2,731 seconds to analyze jython while \oneobjHT~times out. 



%the precision gain comes not only from the shared library but application code.
%For example, \ours~finishes the analysis of jython in
%2,731s while \oneobjHT~fails to complete.




We note that  our 1-call-site-sensitive analysis is even more precise
than the traditional 3-object-sensitive analysis with
2-context-sensitive heap (\threeobjH):

{
\setstretch{1.0}
\begin{center}
  %\small
  \begin{tabular}{ c r r r r }
    \toprule
    % after \\: \hline or \cline{col1-col2} \cline{col3-col4} ...
    \multirow{2}{*}{Program} & \multicolumn{2}{c }{\ours} & \multicolumn{2}{c }{\threeobjH} \\
    \cmidrule(lr){2-3}\cmidrule(lr){4-5}
                             & \#fail-casts & Time(s) & \#fail-cast & Time(s) \\
    \midrule
    luindex     & 357 & 40 & 435 & 564 \\
    antlr       & 477 & 62 & 543 & 561 \\
    pmd         & 707 & 65 & 782 & 584 \\
    \bottomrule
  \end{tabular}
\end{center}
}

\noindent
where we compare the results only for the three small 
programs because \threeobjH~does not scale for other programs.  Note
that \threeobjH~is the most precise object-sensitive analysis evaluated in
the literature~\cite{Lu:2019:PYF,Tan2017} and \ours~substantially
%the literature~\cite{Tan2017,Lu:2019:PYF} and \ours~substantially
improves its precision with much smaller costs.


The performance of \ours~is completely beyond the reach of
existing call-site-sensitive analyses. 
\onecallHT~is the state-of-the-art call-site sensitivity,
%\begin{comment}
which is more precise and faster than ordinary 2-call-site-sensitive analysis~\cite{JeJeOh18}.
%\end{comment}
However, \ours~reduced about 50\% of may-fail casts of \onecallHT~for all programs.

%\textcolor{red}{
Table~\ref{tbl:eval:graph} compares the precision of the analyses for three other call-graph construction related clients  used in previous works~\cite{Li2018a,Li2018b,Tan2017}. \callgraphedges~presents the number of call-graph edges without contexts, \reachableMethods~presents the number of reachable methods, and \polycalls~presents the number of call-sites that cannot be determined as monomorphic calls. The results show that our simulated call-site sensitivity \oursim~overall shows better precision than the baseline object sensitivity \onecallHT. 
%Except for two programs bloat and checkstyle, \oursim~is equal or more precise than \oneobjHT~for all the three metrics. In checkstyle, \oursim~is more precise than \oneobjHT~for two metrics. Our learned one~\ours, however, shows a competitive performance compared to \oneobjHT~(e.g., chart, fop).}
This difference between \oursim~and \ours~ comes from the learning objective. \ours~was not trained to optimize these metrics from~\oursim~(the current implementation of our algorithm uses \failcasts~as the learning objective).
%; it produces competitive results w.r.t. the metrics in Table~\ref{tbl:eval:graph}. 
The learning objective, however, can be easily adapted for other clients. 
%If we use the call-graph-related clients as objective, the learned call-site sensitivity would become more precise than  \oneobjHT~for the clients as the simulated call-site sensitivity \oursim~does.
%}
%\callgraphedges, \reachableMethods, \polycalls





%\textcolor{red}{
%In our evaluation, \ours~consistently produces fewer \#may-fail cast than \oneobjHT. It implies that \ours~ consistently analyzed target client-relevant variables more precisely than \oneobjHT. \ours~sometimes produces higher \#VarPtsTo because it analyzed client-irrelevant variables coarsely than \oneobjHT.
%}

%%% Local Variables:
%%% mode: latex
%%% TeX-master: "paper"
%%% End:
