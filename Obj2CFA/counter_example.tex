% !TEX root = ./paper.tex
\newcolumntype{Y}{>{\centering\arraybackslash}X}
%\setcopyright{none}
%\usepackage {tikz}
\usetikzlibrary {shapes,positioning}
%\usepackage {bm}
\tikzstyle{block} = [rectangle, draw, fill=white, text width=3.8em,
, text
centered, rounded corners, minimum height=4em]
\tikzstyle{block2} =
[rectangle, draw, fill=white, text width=6em, text centered, rounded
corners, minimum height=4em, minimum width = 7em] \tikzstyle{line} = [draw, -latex']
\tikzstyle{onlyText} = [text width =2em, text centered]
%\setcopyright{rightsretained}

\tikzstyle{block3} =
[rectangle, draw, fill=white, text width=7em, text centered, rounded
corners, minimum height=3em] \tikzstyle{line} = [draw, -latex']

%\setcopyright{rightsretained}


\tikzstyle{blocks} = [rectangle, draw, fill=white, text width=0.7cm, text height = 4em, text
centered,  text width=4em, rounded corners, minimum height=0.6cm]

% \chapter{Open Question: Is Complete Simulation Possible?}\label{sec:counter_example}

% In this paper, we showed that it is practically possible to outperform object sensitivity via context-tunneled call-site sensitivity. 
% %, which supports our claim that call-site sensitivity is empirically superior to object sensitivity in a more general $k$-limited setting. 
% However, it remains to be seen whether or not call-site sensitivity can be fundamentally superior to object sensitivity. 

\myparagraph{Fundamental superiority}Let us define the notion of `superiority' as follows: 
\begin{definition}[Superiority of Call-Site Sensitivity]
Let $\mbp$ be a set of target programs. Let $\mbs$ be a context-tunneling space for the target programs. 
We say call-site sensitivity is superior to object sensitivity with respect to $\mbs$ if 
is always possible to simulate object sensitivity via call-site sensitivity: 
\begin{equation}\label{eq:sup-condition}
\forall P \in \mbp. \forall T_{\it obj} \in \mbs.  \exists T_{\it call} \in \mbs.\forall k \in [0,\infty]. 
\; 
\fix F^{T_{\it call}, U_{\it call}}_{P, k} \moreprecise~\mbox{(more precise than)}~
\fix F^{T_{\it obj}, U_{\it obj}}_{P, k}
%\call_k(P, T_{\it call})~\mbox{is more precise than}~\obj_k(P, T_{\it obj}).
\end{equation}
where $U_{\it call}$ and $U_{\it obj}$ are context-update functions (Eq.~(\ref{eq:objsens}) and (\ref{eq:cfa})) that are naturally given together with the tunneling space $\mbs$, and the precision order $\moreprecise$ is defined in terms of the context-insensitive points-to sets, as follows: 
 \[
   \fix F_{P,k}^{T, \updatectx} \moreprecise \fix
   F_{P,k}^{T', \updatectx'} \iff \forall x\in \Var_P.\; \project(\fix
   F_{P,k}^{T, \updatectx})(x) \subseteq \project(\fix
   F_{P,k}^{T', \updatectx'}) (x)
 \]
where $\project(X, Y, R,\callgraph) = \lambda x.\; \bigsqcup_{\ctx \in \Ctx}
 \myset{h \mid (h, \hctx) \in X(x, \ctx)}$.
\end{definition}
Then, the question is restated as follows: Is there a context-tunneling space $\mbs$ that makes the condition (\ref{eq:sup-condition}) true? 

%Formally answering the question would be challenging, although we conjecture that the answer might be true. Let us explain why with examples.




\myparagraph{Object Sensitivity is Not Fundamentally Superior}
We first show that object sensitivity is not fundamentally superior to call-site sensitivity. 
That is, it is not possible to find a tunneling
abstraction space $\mbs$ that makes $k$-object sensitivity with
tunneling always equal or more precise than $k$-call-site
sensitivity with tunneling. The example code in
Figure~\ref{background:example2} is a universal counter-example for all 
tunneling spaces $\mbs$. By definition of context tunneling
(i.e., selective update of contexts), the method
{\tt id} in the example (Figure~\ref{background:example2}) can have
 \texttt{[D]} (updating context) or \texttt{[$\cdot$]}
(inheriting context) as a context no matter what tunneling abstraction space is
used. Thus, $k$-object sensitivity with tunneling is unable to separate the
three method calls (invoked at the three different invocation sites)
with the two 
available context choices. In summary, we conclude that object sensitivity cannot simulate call-site sensitivity even in the generalized setting with context tunneling. 

%k-call-site sensitivity, however, can separate the
%three method calls with a tunneling abstraction $\emptyset \subseteq
%S$ (always update contexts) for all tunneling abstraction space
%$S$ of context tunneling. 



\myparagraph{The Case for Call-Site Sensitivity}
On the other hand, the situation for call-site sensitivity is nontrivial and it remains to be seen whether or not call-site sensitivity can simulate object sensitivity completely. 
%we conjecture that call-site sensitivity can be fundamentally superior to object sensitivity with a well-designed context tunneling space. 
Let us explain in more detail with examples.



\begin{figure}[t]
\begin{multicols}{2}
%\vfill\null
~\\\\\\\\
\begin{subfigure}[b]{1.2\columnwidth}
\begin{lstlisting}[xleftmargin=13pt,multicols=2,basicstyle={\fontsize{8.5}{10}\selectfont\ttfamily}]
class C {
 C f;
 void m(C v) {
  this.set(v);
  v.set(this);
 }
 void set(C v) {
  this.f = v;
 }
}
void main() {
 C c1 = new C();//C1
 C c2 = new C();//C2

 c1.m(c2);
 c2.m(c1);
 assert(c1.f != c2.f); 
}
\end{lstlisting}
\caption{Example code}
\label{fig:counterexample:code}
\end{subfigure}
\columnbreak~\\


\quad\begin{subfigure}[b]{1.0\columnwidth}
\begin{center}
	\resizebox{0.55\columnwidth}{!}{
		\begin{tikzpicture}
		\tikzstyle{every node}=[font=\LARGE]
		\node [block] (main) {{\tt main} \\ $[\cdot]$};
		\node [block, above right = -0.4cm and
		0.7cm of main] (Dm1) {\tt {m} \\$[$C1$]$};
		\node [block, below = 0.4cm of Dm1] (Dm2) {\tt{m} \\$[$C2$]$};
		\node [block, right = 0.9cm of Dm1] (Cm1) {\tt {set} \\$[$C1$]$};
		\node [block, right = 0.9cm of Dm2] (Cm2) {\tt {set} \\$[$C2$]$};
		
		\path [line] (main) edge node[above] {$15$} (Dm1);
		\path [line] (main) edge node[above] {$16$} (Dm2);
		\path [line] (Dm1) edge node[above] {$4$} (Cm1);
		\path [line] (Dm2) edge node[below] {$4$} (Cm2);
		\path [line] (Dm2) edge node[above] {$5$} (Cm1);
		\path [line] (Dm1) edge node[below] {$5$} (Cm2);
		
		\end{tikzpicture}
	}
\end{center}
\caption{Call-graph by 1-object sensitivity}
\label{fig:counterexample:obj}
\end{subfigure}~\\



\quad\begin{subfigure}[b]{1.0\columnwidth}

\begin{center}
	\resizebox{0.55\columnwidth}{!}{
		\begin{tikzpicture}
		\tikzstyle{every node}=[font=\LARGE]   
		\node [block] (main) {{\tt main} \\ $[\cdot]$};
		\node [block, above right = -0.4cm and
		0.7cm of main] (Dm1) {{\tt m} \\$[15]$};
		\node [block, below = 0.4cm of Dm1] (Dm2) {{\tt m} \\$[16]$};
		\node [block, right = 0.9cm of Dm1] (Cm1) {{\tt set} \\$[4]$};
		\node [block, right = 0.9cm of Dm2] (Cm2) {{\tt set} \\$[5]$};
		
		\path [line] (main) edge node[above] {$15$} (Dm1);
		\path [line] (main) edge node[above] {$16$} (Dm2);
		\path [line] (Dm1) edge node[above] {$4$} (Cm1);
		\path [line] (Dm2) edge node[below] {$5$} (Cm2);
		\path [line] (Dm2) edge node[above left= 0.15cm and 0.02cm] {$5$} (Cm1);
		\path [line] (Dm1) edge node[below left= 0.15cm and 0.02cm] {$4$} (Cm2);	
		\end{tikzpicture}
	}
\end{center}
\caption{Call-graph by 1-call-site sensitivity}
\label{fig:counterexample:call}
\end{subfigure}

\end{multicols}
\vspace{-15pt}
\caption{Example such that call-site sensitivity cannot
  simulate object sensitivity w.r.t. tunneling space $\mbs = \Invo$.}
\vspace{-10pt}
\label{fig:counterexample}
\end{figure}

First, we point out that that the condition (\ref{eq:sup-condition}) does not hold w.r.t. the tunneling space considered in this paper. 
In Section~\ref{sec:setting}, we defined the tunneling space $\mbs$ to be the set of invocation-sites, i.e., $\mbs = \Invo$. 
With this space, there exists a tricky counter-example program that context-tunneled
call-site sensitivity is unable to simulate object sensitivity.
Consider the program in Figure~\ref{fig:counterexample}. 
In the example code, class {\tt C} contains two methods, {\tt set} and {\tt m}.
Method {\tt set} is a setter that stores the parameter value in the
the field of the base object,
and method {\tt m} calls the {\tt set} method twice
with the receiver object and parameter being swapped.
The {\tt main} method creates two objects, {\tt C1} and {\tt C2}, and
stores them in {\tt c1} and {\tt c2}. Method {\tt m} is called on {\tt
  c1} with parameter {\tt c2} at line 15, and it is called again on
{\tt c2} with parameter {\tt c1} at line 16.
At line 17, an assertion asks if
the fields of {\tt c1} and {\tt c2} are not aliased.
In the real execution, the query holds because {\tt c1.f} points to {\tt C2} and
{\tt c2.f} points to {\tt C1}.

The conventional 1-object-sensitive analysis can prove the query but
1-call-site sensitivity cannot do so no matter what tunneling abstraction from the space $\mbs = \Invo$ is
chosen.
Figure~\ref{fig:counterexample:obj} presents
the call-graph of 1-object sensitivity.
The object-sensitive analysis is effectively producing the call-graph above as
each context of {\tt set} determines the value of each parameter.
When the context is {\tt [C1]}, the receiver object is {\tt C1} and the value of the parameter is {\tt C2}.
Otherwise, when the context is {\tt [C2]}, the receiver object and the
parameter value are  {\tt C2} and {\tt C1}, respectively.
On the other hand, conventional 1-call-site sensitivity constructs a similar but
different call-graph in Figure~\ref{fig:counterexample:call}.
Note that the two edges labeled 4 go
to the same method but they are heading to different methods in the call-graph of object sensitivity. 
Thus, it is not possible to simulate object sensitivity with the invocation-site-based context tunneling. 
%However,
%doing so still fails to simulate object sensitivity.


%The code pattern, however, is uncommon in practice.
%To make call-site sensitivity unable to simulate object sensitivity,
%the program should have at least two invocation-sites where a single method
%is called with different parameters and receiver objects.
%The parameters should come from the parameters of the caller methods, and the
%caller method should be called from at least two different invocations with
%different parameters. Satisfying all of these constraints at the same time
%would happen rarely in practice.


%\textcolor{red}{
%\footnote{
%We can define the tunneling abstraction in various ways.
%\citet{JeJeOh18} originally used (pairs of) methods as tunneling abstractions.
%In this paper, we consider more fine-grained program elements, i.e., invocation-sites. 
%} 

\begin{figure}[t]
	\begin{center}
			\resizebox{0.8\columnwidth}{!}{
					\begin{tikzpicture}
					\tikzstyle{every node}=[font=\LARGE]
					\node [block] (main) {{\tt main} \\ $[\cdot]$};
					\node [block, right =
					2.0cm of main] (Dm1) {{\tt m} \\$[15]$};
					\node [block, left = 1.9cm of main] (Dm2) {{\tt m} \\$[16]$};
					\node [block, below right = -0.3cm and 1.9cm of Dm1] (Cm1) {{\tt set} \\$[5]$};
					\node [block, below left = -0.3cm and 1.9cm of Dm2] (Cm2) {{\tt set} \\$[4]$};
	
					\node [block, above = 0.4cm of Cm1] (Cm3) {{\tt set} \\$[15]$};
					\node [block, above = 0.4cm of Cm2] (Cm4) {{\tt set} \\$[16]$};
	
	
					\path [line] (main) edge node[above= 0.15cm] {\large$(C1,15)$} (Dm1);
					\path [line] (main) edge node[above= 0.15cm] {\large$(C2,16)$} (Dm2);
					\path [line] (Dm1) edge node[above left= 0.15cm and -0.7cm] {\large $(C1,4)$} (Cm3);
					\path [line] (Dm2) edge node[above right= 0.15cm and -0.7cm] {\large $(C1,5)$} (Cm4);
					\path [line] (Dm2) edge node[below right = 0.15cm and -0.7cm] {\large $(C2,4)$} (Cm2);
					\path [line] (Dm1) edge node[below left = 0.15cm and -0.7cm] {\large $(C2,5)$} (Cm1);
					\end{tikzpicture}
			}
	\end{center}
	\caption{Call-graph with a fine-grained tunneling space}
	\label{fig:cg-finegrained}
	\vspace{-10pt}
	\end{figure}
However, this counter-example does not mean that it is fundamentally impossible to simulate object sensitivity via call-site sensitivity, as the counter-example becomes no longer valid if we use a more fine-grained tunneling abstraction. 
For example, suppose we define the tunneling space to be pairs of receiver objects and invocation-sites, i.e., $\mbs = \Heap \times \Invo$ (recall that choosing a tunneling abstraction does not affect the analysis soundness). 
With this tunneling space, a context-tunneled 1-call-site-sensitive analysis can now prove the query. 
Suppose we use a tunneling abstraction $T = \myset{(C1,4), (C1,5)}$, which means that we apply context tunneling only when the receiver object is $C1$ and the invocation-site is either $4$ or $5$. 1-call-site sensitivity with $T$ produces the call-graph in Figure~\ref{fig:cg-finegrained}. 
In the call-graph, $\texttt{m[15]} \stackrel{(C1,4)}{\to}
\texttt{set[15]}$ indicates that the caller ({\tt m}) and callee ({\tt
  set}) have the same context 15 where the callee method is called at
invocation-site 4 and its receiver object is $C1$. With this call-graph, we can prove the query 
in Figure~\ref{fig:counterexample:code} as it is strictly more precise than the call-graph produced by object sensitivity
(Figure~\ref{fig:counterexample:obj}). 



%Note that the above counter-example is not valid that 1-call-site
%sensitivity with tunneling can prove the query if we use a generalized
%abstraction space of context tunneling instead of invocation sites. if
%we use pairs of receiver objects and invocation sites as tunneling
%abstraction, 1-call-site-sensitive analysis with tunneling can prove
%the query in the example
%(Figure~\ref{fig:counterexample:call}). Suppose we use a
%tunneling abstraction $T = \{(C1,4),(C1,5)\}$ where it applied context
%tunneling when the receiver object and invocation site are $C1$ and
%$4$, respectively, or the object and the invocation site are $C2$
%and $5$, respectively. It produces
%the following context abstraction:



This way, we conjecture that it would be always possible to find a suitable context-tunneling space $\mbs$ that satisfies the condition (\ref{eq:sup-condition}) w.r.t. the given set of programs ($\mbp$). We leave an in-depth theoretical analysis as future work. 

%}




%%% Local Variables:
%%% mode: latex
%%% TeX-master: "paper"
%%% End:
