% !TEX root = ./paper.tex
\begin{abstract}

In this paper, we challenge the commonly-accepted wisdom in static analysis 
that object sensitivity is superior to call-site sensitivity for object-oriented programs. 
In static analysis of object-oriented programs, object sensitivity
has been established as the dominant flavor of context sensitivity
thanks to its outstanding precision. On the other hand, call-site
sensitivity has been regarded as unsuitable and its use in practice has been 
constantly discouraged for object-oriented programs. 
In this paper, however, we claim that call-site sensitivity is generally 
a superior context abstraction because it is practically possible to transform 
object sensitivity into more precise call-site sensitivity. 
Our key
insight is that the previously known superiority of object sensitivity holds only
in the traditional $k$-limited setting, where the analysis is enforced
to keep the most recent $k$ context elements. However, it no longer holds
in a recently-proposed, more general setting with context tunneling.
With context tunneling, where the analysis is free to choose an
arbitrary $k$-length subsequence of context strings, we show that
call-site sensitivity can simulate object sensitivity almost
completely, but not vice versa. To support the claim, we present a
technique, called \ourtechnique, for transforming 
arbitrary context-tunneled object sensitivity into
more precise, context-tunneled call-site-sensitivity.
We implemented \ourtechnique~in Doop and used it to derive a
new call-site-sensitive analysis from a state-of-the-art object-sensitive pointer analysis. 
Experimental results
 confirm that the resulting call-site sensitivity outperforms 
 object sensitivity in precision and scalability for real-world Java programs. 
Remarkably, our results show that even 1-call-site sensitivity can be more precise than the
conventional 3-object-sensitive analysis.
\end{abstract}
%\minseok{state-of-the-art 1-object sensitivity $\to$ state-of-the-art object sensitivity}


%%% Local Variables:
%%% mode: latex
%%% TeX-master: "paper"
%%% End:
