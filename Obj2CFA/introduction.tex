% !TEX root = ./paper.tex

\section{Introduction}\label{sec:introduction}

\iffalse
\begin{quote}
  {\it ``Since its introduction, object sensitivity has emerged as the
  dominant flavor of context sensitivity for object-oriented
  languages.''} 
\begin{flushright}
---\cite{Smaragdakis2015}
  \end{flushright}
\end{quote}
\noindent
%\textcolor{red}{
%Pointer analysis is a foundation of many static analyses, which safely 
%estimates the objects that each variable points-to in real executions.
%Such information plays key roles in various software engineering techniques like 
%bugfinders~\cite{Sui2014},
%%bugfinders~\cite{Naik2006,NaikPSG09,Blackshear2015,Sui2014,Livshits2003},
%symbolic execution~\cite{Kapus2019}, and program repair tools~\cite{memfix,Gao2015,vfix2019}. The success of the above software engineering tools heavily depends on the performance of the underlying points-to analysis.}
Context sensitivity is critically important for static program analysis of
object-oriented programs.  A context-sensitive analysis associates
local variables and heap objects with context information of method
calls, computing analysis results separately for different
contexts. This way, context sensitivity prevents analysis information
from being merged along different call chains. For object-oriented and
higher-order languages, it is well-known that context sensitivity is
the primary means for increasing analysis precision without blowing up
analysis cost~\cite{Smaragdakis2015,Thiessen2017,Lhotak2006,
  KastrinisS13a, Smaragdakis2014, Li2018a, JeJeChOh17, %liang2005evaluating,
  Sridharan2006}.
\fi

There have been two major flavors of context sensitivity, namely {\em
  call-site sensitivity}~\cite{Sharir1981,Shivers1988} and {\em object
  sensitivity}~\cite{Milanova2002,Milanova2005}, which differ in the
choice of context information. The traditional $k$-call-site-sensitive
analysis~\cite{Sharir1981} uses a sequence of $k$ call-sites as the
context of a method. By contrast, object sensitivity uses
allocation-sites as context elements: in a virtual call, e.g., {\tt
  a.foo()}, an object-sensitive analysis uses the allocation-site of
the receiver object ({\tt a}) as the context of {\tt foo}. The standard
$k$-object-sensitive
analysis~\cite{Milanova2002,Milanova2005,Smaragdakis2011} maintains a
sequence of $k$ allocation-sites, comprising the allocation-site of
the receiver object, the allocation-site of the receiver's allocator,
and so on.


\myparagraph{The Status Quo}
Since its inception, object sensitivity has been established as the
dominant context abstraction for object-oriented
languages~\cite{Smaragdakis2015}.  Ever since \cite{Milanova2002,
  Milanova2005} proposed object sensitivity, its superiority over
other flavors of context sensitivity has been reinforced by a large
amount of 
research~\cite{Lhotak2008,BravenboerS09,Smaragdakis2011,TanLX16,
  JeJeChOh17,Lu:2019:PYF}.  Among others, \cite{Lhotak2006} and
\cite{BravenboerS09} conducted extensive experiments to conclude
that object sensitivity significantly outperforms other alternatives
including call-site sensitivity.  As a result, object sensitivity has
become an indispensable component of program analysis tools for
object-oriented languages~\cite{Fink2008,Zhang2014,NaikAW06,GordonKPGNR_NDSS15,Feng2014,vfix2019}.


In contrast, the use of call-site sensitivity has been constantly
discouraged for object-oriented
programs~\cite{TanLX16,JeJeChOh17,Li2018a,Lhotak2006,
  Smaragdakis2011, Smaragdakis2014, Milanova2002, Milanova2005}.  For
example, \cite{Milanova2002, Milanova2005} judged call-site
sensitivity as ``ill-suited" for object-oriented programs,
\cite{KastrinisS13a} claimed that call-site sensitivity should be
avoided because it is both imprecise and expensive, and
\cite{Smaragdakis2014} asserted call-site sensitivity is never
cost-effective. %as it is unscalable beyond the smallest context depth ($k=1$).  
As a result, call-site sensitivity has become obsolete in practice and
virtually not used anymore in program analysis tools for
object-oriented programs: 
\textit{``... object-sensitive analyses have
  almost completely supplanted traditional call-site-sensitive
  analyses for object-oriented languages''}~\cite{Smaragdakis2011}.

%\textcolor{red}{
%The status quo has been established on the empirical evaluation results showing superiority of object sensitivity compared to call-site sensitivity.
%For a majority of programs, even 1-object sensitivity 
%has been more precise and scalable than 2-call-site sensitivity.
%When analyzing a program (lusearch) in our benchmark, for example, 1-object sensitivity produces 27\% more precise abstraction (points-to set) 
%within 82\% fewer analysis time(s) than 2-call-site sensitivity.
%}
%%\minseok{Detailed data}

\myparagraph{This Work}
We challenge this commonly-accepted wisdom by showing that call-site
sensitivity is generally superior to object sensitivity %context abstraction
even for object-oriented programs.  Our key insight
is that the previously established superiority of object sensitivity over
call-site sensitivity is valid only when we impose a particular restriction that
the analysis should keep the {\em most recent} $k$ context elements,
but it no longer holds in a more general setting where the restriction
is eliminated. Notably, the relative superiority of object sensitivity
and call-site sensitivity gets inverted when they are generalized with
context tunneling~\cite{JeJeOh18}, where the analysis is free to use
an arbitrary $k$-length subsequence of context strings.  In this
generalized setting, we show that call-site sensitivity is able to {\em
  simulate} object sensitivity, but object sensitivity is not powerful
enough to simulate call-site sensitivity.
We note that our aim is not to debunk the previously known result. Instead, we claim that what is currently known only persists in a limited circumstance and the converse holds when the assumption is generalized. 


%\TODO{Say that the key challenge here is how to find the appropriate tunneling policy that makes CFA more precise than Obj and we address the challenge using an obj-guided approach}
To support the claim, %Exploiting this insight, 
we present \ourtechnique, a practical technique for transforming
%state-of-the-art 
object sensitivity into more precise,
context-tunneled call-site sensitivity.  Our technique takes as input
an arbitrary object-sensitive analysis with context tunneling and produces as output a
context-tunneling policy that enables call-site sensitivity to exceed
the precision limit of the baseline object sensitivity without
increasing $k$. Our key technical contributions to achieve this goal are the simulation and
simulation-guided learning procedures.  By the simulation procedure, we infer a context-tunneling policy
with which call-site sensitivity can simulate the baseline object
sensitivity.  The resulting call-site sensitivity, however,  is 
impractical since it requires running the baseline object-sensitive
analysis as a pre-analysis.  The learning phase aims to remove
this burden by capturing the behavior of the simulated policy using
a dataset of programs.


%We experimentally prove our claim for pointer analysis. 
%Experimental results show that our technique is practical and advances
%the state-of-the-art significantly.  
We implemented our technique in
Doop~\cite{BravenboerS09}, a popular pointer analysis framework for
Java.
We transformed a state-of-the-art object-sensitive pointer analysis
into the matching call-site-sensitive analysis. 
Evaluation with
real-world Java applications shows that the resulting call-site-sensitive analysis 
 significantly improves the original object-sensitive analysis in terms of both precision and scalability. 
Remarkably, our context-tunneled 1-call-site-sensitive
analysis is even more precise than the traditional 3-object-sensitive
analysis with much smaller costs,  which confirms our claim that call-site sensitivity can be superior to object sensitivity in the generalized setting. 

%\textcolor{red}{
%Note that we do not intend to say that previous results are wrong. Rather, we would like to say that call-site sensitivity should be reconsidered from now on because, though object sensitivity is superior to call-site sensitivity in the traditional k-limited setting, call-site sensitivity can be better than object sensitivity in a more general setting with context tunneling. We think that this message contrasts with what has been known in the program-analysis community.
%}

%which is far beyond the reach of
%modern object-sensitive analyses in practice.


\myparagraph{Contributions} We summarize our contributions below.
\begin{itemize}[leftmargin=1.3em]
\item %{\bf Claim} (Section~\ref{sec:insight}): 

We make a novel claim that call-site sensitivity is generally superior to object sensitivity; when the notion of $k$-limiting is
  generalized with context tunneling, call-site sensitivity can simulate object sensitivity almost completely, but not vice versa.

\item %{\bf Technique} (Section~\ref{sec:technique}):  
	We present \ourtechnique, a new technique for transforming a
context-tunneled $k$-object-sensitive analysis into a more precise, context-tunneled
  $k$-call-site-sensitive analysis. % via simulation and learning.
  Specifically, we make two technical contributions: the simulation (Section~\ref{sec:simulation}) and simulation-guided learning (Section~\ref{sec:learning}) procedures, both of which are vital to achieving the goal. 

\item %{\bf Experimental Proof} (Section~\ref{sec:result}): 
We experimentally prove our claim by applying \ourtechnique~to a
  state-of-the-art object-sensitive pointer analysis for Java. 
Our implementation and data are publicly available~\footnote{\url{https://github.com/kupl/OBJ2CFA}}.
\end{itemize}	





%%% Local Variables:
%%% mode: latex
%%% TeX-master: "paper"
%%% End:
