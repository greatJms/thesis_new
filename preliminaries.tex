


\chapter{Preliminaries~\footnote{The contents of Chapter~\ref{sec:Preliminaries} and~\ref{sec:Tunneling} are based on our previous work presented at \emph{OOPSLA 2018: ACM Conference on Object-Oriented Programming, Systems, Languages, and Applications}~\cite{JeJeOh18}}}\label{sec:Preliminaries}
In this chapter, we illustrate conventional pointer analysis~\cite{Smaragdakis2015,KastrinisS13a} on which our data-driven techniques are developed on.



\section{Conventional Pointer Analysis}\label{pre:conventional-analysis}
% \myparagraph{Generic Analysis} 
The baseline pointer analysis uses $k$-limited context sensitivity~\cite{KastrinisS13a,Smaragdakis2011} and allocation-site-based heap abstraction~\cite{Smaragdakis2015,Tan2017}.
%Our analysis is $k$-context-sensitive pointer analysis using allocation-site-based heap abstraction.
The analysis is a flow-insensitive, field-sensitive, and subset-based
analysis with on-the-fly call-graph construction. 
Our analysis supports various context flavors and depths.
For example, the analysis is able to express
call-site-sensitivity~\cite{Shivers1988},
object-sensitivity~\cite{Milanova2005}, hybrid
context-sensitivity~\cite{KastrinisS13a}, and
type-sensitivity~\cite{Smaragdakis2011} with arbitrary depths for
calling contexts and heap contexts. In our formalism, such variations
are presented by the following triple:
\[
  \langle \Flavor, \CallDepth, \HeapDepth \rangle \quad (\CallDepth \ge \HeapDepth)
\]
where $\Flavor$ is a function that presents the flavor of
context-sensitivity (e.g. call-site-sensitivity, object-sensitivity,
etc). $\CallDepth$ and $\HeapDepth$ represent the maximum context depths
of calling and heap contexts, respectively. For example,
$\langle \Flavorcall, 1, 0 \rangle$ presents the 1-call-site-sensitive
analysis with context-insensitive heap. 
% We will shortly
% define the $\Flavor$ function for call-site-sensitivity ($\Flavorcall$),
% object-sensitivity ($\Flavorobj$), and type-sensitivity
% ($\Flavortype$).

\myparagraph{Programs and Domains}

We consider the five types of instructions in Java:
\[
  \lcmd \to l: \cmd, \quad \cmd \to {\it x = new~C()} \mid {\it x = y}
  \mid {\it x = y.f} \mid {\it x.f = y} \mid {\it x = y.sig(arg)}
\]
where each instruction is associated with a unique label.  A labeled
instruction ($\lcmd$) consists of a label $(l)$ and an instruction
($\cmd$). An instruction is object allocation (${\it x = new~C()}$),
move (${\it x = y}$), load (${\it x=y.\fld}$), store
(${\it x.\fld = y}$), or method invocation (${\it
  x=y.\sig(arg)}$). For simplicity, we assume every method has a
single argument (except for the object itself).

The analysis uses the following commonly used domains:
$\mbv$ is a set of variables,
$\Methods$ is a set of methods,
$\mbs$ is a set of method names (signatures),
$\mbf$ is a set of field names,
$\mbi$ is a set of instruction labels,
$\mbh$ is a set of abstract heaps, e.g.,
  allocation-sites ($\mbh \subseteq \mbi$),
$\mbt$ is a set of class types,
$\Context$ is a set of calling contexts, and
$\HeapContext$ is a set of heap contexts.
$\Context$ and $\HeapContext$ will be defined with  respect to 
the flavor of context-sensitivity (e.g., $\langle \Flavor, \CallDepth, \HeapDepth \rangle$)


In our analysis, we additionally use four auxiliary functions: $\methof$, $\typeof$, $\classof$, and $\lookup$.
$\methof\in \mbi \to \Methods$ maps an instruction label to its belonging method.  $\typeof \in \mbh \to \mbt$ maps an
allocation-site to the allocated type.
$\classof : \mbi \to \mbt$ maps an instruction to the class that contains the instruction.
$\lookup \in \mbt \times \mbs \to \Methods$ maps a pair of a class type
and a signature to the matching the method.
Given a method $m$, we write $m_{\it this}$ for the `this' variable, $m_{\it param}$, for the formal parameter, and 
$m_{\it return}$ for return variable of the method.

\begin{figure}[t]
  \[
    \begin{array}{c}

      %% alloc %%
      \infer[]{
      (\heap_\invo, \hctx) \in \VarPointsTo({\var}, {\ctx})
      }{ {\it \invo: x = new~C()} \quad \ctx \in \Reachable({\methof(\invo)}) \quad
      { \hctx = \truncateCtx{\ctx}{\HeapDepth}}} \\[1em]


      \infer{\VarPointsTo({\fromvar}, {\ctx}) \subseteq \VarPointsTo({\tovar}, {\ctx}) }{
      \invo: {\it x = y} \quad
      \ctx \in \Reachable(\methof(\invo)) } \\[1em]

    %% load%%
      \infer{
      \FldPointsTo(\heap, \hctx,{\fld}) \subseteq \VarPointsTo({\tovar}, {\ctx})}{
      \begin{array}{c}
        \invo:  {\it x = y.\fld} \quad \ctx \in
        \Reachable(\methof(\invo)) \quad (\heap, \hctx) \in
        \VarPointsTo({{\it y}}, {\ctx})
      \end{array}
      } \\[1em]

      %% store %%
      \infer{
      \VarPointsTo({\it y}, {\ctx}) \subseteq \FldPointsTo(\heap, {\hctx}, {\fld})
      }{
      \begin{array}{c}
        {\it \invo: x.f = y} \quad \ctx \in \Reachable(\methof(\invo))
        \quad
        ({\heap}, {\hctx}) \in \VarPointsTo({\it x}, {\ctx})
      \end{array}
      } \\[1em]

      %% call %%
      \infer[]
      {
      \begin{array}{c}
        \ctx' \in \Reachable ({m}) \quad
        ({\heap}, {\hctx}) \in  \VarPointsTo ({m_{\this}}, {\ctx'}) \\
        \VarPointsTo({\it arg}, \ctx) \subseteq \VarPointsTo(m_{\it param}, \ctx')
        \quad
        \VarPointsTo(m_{\it return}, \ctx') \subseteq \VarPointsTo(x, \ctx)\\
      (\invo, \ctx, m, \ctx') \in \CallGraph
      \end{array}
      }{
      \begin{array}{c}
       {\it \invo : x = y.{\sig}}({\it arg})
        \quad
        \ctx \in    \Reachable ({\methof(\invo)}) \\
        ({\heap}, {\hctx})\in  \VarPointsTo ({\it y},\ctx) \quad
        t = {\typeof}(\heap)  \quad m = {\lookup}(t, \sig) \\
(e, {\pctx}, \_) = \parent(\heap,\hctx, \invo, \ctx) \quad
        \ctx' = \truncateCtx{\appendCtx{{ \pctx}}{{e}}}{\CallDepth}
      \end{array}
      }
    \end{array}
  \]
  \caption{Conventional context-sensitive pointer analysis}
  \label{pre:baseline-rules}
\end{figure}


\myparagraph{Analysis Ourput}
Given a program, the pointer analysis aims to compute the following information:
\begin{itemize}
\item
  $\VarPointsTo: \mbv \times \Context \to \power{\mbh \times
    \HeapContext}$
\item
  $\FldPointsTo: \mbh \times \HeapContext \times \mbf \to \power{\mbh
    \times \HeapContext}$
\item $\Reachable: \mbm \to \power{\Context}$
\item $\CallGraph: \power{\mbi \times \Context \times \mbm \times \mbc}$
\end{itemize}
The function $\VarPointsTo$ maps each variable ($x \in \mbv$) with its context ($\ctx \in \Context$) to the points-to set.
$\VarPointsTo(x,\ctx)$ denotes the set of objects and their heap
contexts that the variable $x$ with its context $\ctx$ may point to.  
%Other functions are used
% as intermediate information to compute $\VarPointsTo$: given
% $\heap \in \mbh$, $\hctx \in \HeapContext$, and $\fld
% \in \mbf$,
$\FldPointsTo(\heap,\hctx,\fld)$ represents the set of heap objects
possibly pointed by the field of the heap object.
$\Reachable(m)$ stores the set of calling contexts of $m$ under which
the method $m$ is reachable from the entry of the program.
$\CallGraph$ presents a set of context-sensitive call graph edges, 
where a call edge $(\invo, \ctx_1, m, \ctx_2)$ indicates that method $m$ is called from the invocation site $\invo$ and the contexts of caller and callee methods are $\ctx_1$ and $\ctx_2$, respectively.



\myparagraph{Analysis Rules}
Figure~\ref{pre:baseline-rules} presents the analysis rules. The goal of the analysis is to find the smallest (but sound) functions of $\VarPointsTo$, $\FldPointsTo$, $\Reachable$, and $\CallGraph$.
%  that are closed under the analysis rules, where the order is defined pointwisely.
% \[
%   \begin{array}{l}
%     (\VarPointsTo, \FldPointsTo, \Reachable) \sqsubseteq (\VarPointsTo',
%     \FldPointsTo', \Reachable')
%     \iff \\ \qquad \forall (x,\ctx) \in \mbv \times \Context.\; \VarPointsTo(x,\ctx)
%     \subseteq \VarPointsTo'(x,\ctx) ~\land \\
%     \qquad \forall (\heap, \hctx, \fld) \in \mbh \times \HeapContext \times
%     \mbf.\; \FldPointsTo (\heap, \hctx, \fld) \subseteq
%     \FldPointsTo'(\heap,\hctx,\fld) ~\land \\
%     \qquad \forall m \in \mbm.\; \Reachable(m)\subseteq \Reachable'(m).
%   \end{array}
% \]
Rules are pointwisely defined for each instruction type.  For example, the first rule corresponds to object allocation, where an abstract heap
$\heap_l \in \mbh$ is allocated from the instruction $l$. 
Its heap context $\hctx$ is obtained from the context $\ctx$ of the containing method.
$\hctx$ keeps the last $\HeapDepth$ context elements of the calling context $\ctx$, i.e.,
%It possibly truncated so as to keep the last
%$\HeapDepth$ context elements,
$\hctx = \truncateCtx{\ctx}{\HeapDepth}$, where the operator
$\truncateCtx{\cdot}{k}$ truncates contexts:
$\truncateCtx{\langle a_1, a_2, \dots, a_n \rangle }{k} = \langle
  a_{n-k+1}, \dots, a_n \rangle$.
The rules for the move, load, and store instructions are intuitive. 


The last rule for method calls needs explanation. 
Given a method call instruction
${\it l}:x = y.\sig(arg)$, where ${\it l}$ denotes the
invocation-site, $\sig$ the method signature (name and type), and
${\it arg}$ an actual parameter, we find the calling context $\ctx$
of the caller method, $(\heap,\hctx)$ the receiver object and its heap
context that {\it y}
points to, $t$ the class type of the receiver object $\heap$, and $m$ 
the callee method whose signature is $\sig$ in the class $t$. 
Then, we create a new calling context $\ctx'$ for the callee
method by appending ($\appendCtx{}{}$) a new context element $e$
to the previous context $\pctx$.
$\truncateCtx{\appendCtx{\pctx}{e}}{\CallDepth}$ makes the callee method's context keep the last $\CallDepth$ context elements
% , i.e.,
% $\truncateCtx{\appendCtx{\pctx}{e}}{\CallDepth}$, possibly truncated to keep the last $\CallDepth$ context elements. 
We call the previous context $\pctx$ the ``parent'' context and the new
one $\truncateCtx{\appendCtx{\pctx}{e}}{\CallDepth}$ the ``child''
context.

%% define C, HC
\myparagraph{Context-Sensitivity Flavors}
The notions of context elements and parent contexts depend on the
particular flavor of context-sensitivity and are defined
by $\parent$. Given a heap $\heap \in \mbh$, heap context
$\hctx \in \HeapContext$, invocation-site ${\it invo} \in \mbi$, and calling
context $\ctx \in \Context$ as input,
$\parent(\heap,\hctx,{\it invo},\ctx)$ returns the current context element ($e$) to be used, the parent context ($\pctx$), and
the parent method ($p$) where $\pctx$ is used as a calling context. Various flavors of
context-sensitivity can be obtained by defining $\parent$
appropriately. For example, call-site-sensitivity~\cite{Shivers1988},
object-sensitivity~\cite{Milanova2005}, and
type-sensitivity~\cite{Smaragdakis2011} are characterized as follows:
\begin{align*}
 & \parentcall (\heap, \hctx, {\it invo}, \ctx) = ( {\it invo}, \ctx,
  \methof({\it invo})) \qquad\quad\,\,\,\,\, \mbox{(Call-site-sensitivity)} \\
 & \parentobj (\heap, \hctx, {\it invo}, \ctx) = (\heap, \hctx,
                                               \methof(\heap)) \qquad\quad\,\,\,\,\, \mbox{(Object-sensitivity)} \\
 & \parenttype (\heap, \hctx, {\it invo}, \ctx) = (\classof(\heap),
                                               \hctx, \methof(\heap))
                                               \,\, \mbox{(Type-sensitivity)}
\end{align*}
That is, the defining characteristic of call-site-sensitivity is that
it uses invocation-sites (${\it invo}$) as
context elements, the parent context is the context ($\ctx$) of the
caller, and the parent method is the caller ($\methof({\it
  invo})$). In other words, in call-site-sensitivity, the new context
is created by appending the current context element to the calling
context of the caller method.  Object-sensitivity~\cite{Milanova2005}
differs from call-site-sensitivity in both the context element and the
parent context: it uses allocation-sites ($\heap$) as
context elements and the parent context comes from the heap context
($\hctx$) of the receiver object ($\heap$). The parent method is
$\methof(\heap)$ because the heap context ($\hctx$)
corresponds to the calling context of the method
containing the
allocation-site. Type-sensitivity~\cite{Smaragdakis2011} is similar to
object-sensitivity except that it uses the containing classes (i.e. $\classof(\heap)$), instead
of allocation-sites, as context elements.
Hybrid context-sensitivity~\cite{KastrinisS13a} combines
call-site-sensitivity ($\parentcall$) and object-sensitivity
($\parentobj$); it uses call-site-sensitivity at static calls and
object-sensitivity at virtual calls.


% Note that our formalism is general enough to cover a variety of
% existing context-sensitive analyses such as $\onecall$, $\onecallH$,
% $\oneobjH$, $\twoobjH$, $\onetypeH$, $\twotypeH$, etc~\cite{kastrinis2013hybrid}.




%%% Local Variables:
%%% mode: latex
%%% TeX-master: "thesis"
%%% End:
