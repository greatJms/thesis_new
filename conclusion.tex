\chapter{Conclusion}\label{sec:Conclusion}
This thesis explores data-driven pointer analysis, a novel technique for automatically generating qualified analysis heuristics. 
To this end, we present three key ideas: data-driven context tunneling, \ourtechnique, and \Graphick. Context tunneling shifts the context abstraction paradigm from ``the most recent $k$'' into ``the most important $k$'', and our data-driven approach successfully learned tunneling heuristics that significantly improved the performance of pointer analysis.
\ourtechnique~challenges the commonly accepted knowledge, object sensitivity is superior to call-site sensitivity for object-oriented programs. \ourtechnique~automatically transformed the state-of-the-art object sensitivity into a more precise call-site sensitivity. 
We present \Graphick~to remove the burden of designing features. \Graphick~automatically learns analysis heuristics without hand-crafting application-specific features. The evaluation results show that the automatically generated analysis heuristics significantly improved the precision and scalability of pointer analysis. 
%We believe data-driven pointer analysis




% the pointer analysis with the learned heuristics significantly improved the precision and scalability of pointer analysis.

%We present context tunneling that enable pointer analysis keep the most important $k$ context elements. We 

%\ourtechnique


% We studied {\em data-driven symbolic execution}, a brand new solution for
% the difficulty of manually crafting critical techniques of symbolic
% execution: search heuristic, input space reduction, and state-pruning.
% The key idea of our data-driven approach is to
% automatically learn how to generate a promising solution based on data (e.g.,
% test-cases) accumulated during symbolic execution. Experimental results show
% that this ``machine-tuned'' technique outperforms existing ``hand-tuned''
% techniques in both code coverage and bug detection on a wide range of
% open-source C programs. We hope that our data-driven approach can supplant
% the laborious and less rewarding task of manually generating diverse
% heuristics of symbolic execution.

% As future work, it would be possible to extend data-driven
% symbolic execution in two folds. First, we will continuously replace
% manually-tuned yet unstable techniques of symbolic execution (e.g., seed
% selection and parameter tuning) with automatically-tuned and stable ones via data-driven approach.
% Second, we will integrate four different techniques, including $\paradyse$,
% $\conmeleon$, $\contest$, and $\ourtool$, into a publicly available tool so
% that end-users can easily employ our technique.



%%% Local Variables:
%%% mode: latex
%%% TeX-master: "thesis"
%%% End:
