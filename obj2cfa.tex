\chapter{Learning a Precise Call-site Sensitivity from a Given Object Sensitivity~\footnote{The contents of this chapter are based on our previous work presented at \emph{POPL 2022: 49th ACM SIGPLAN Symposium on Principles of Programming Languages}~\cite{JeOh22}}}\label{sec:Obj2CFA}


In this chapeter, we present \ourtechnique~that 
learns a precise call-site site sensitivity from a given object sensitivity.
In other words, we challenge the commonly-accepted wisdom in static analysis 
that object sensitivity is superior to call-site sensitivity for object-oriented programs. 
In static analysis of object-oriented programs, object sensitivity
has been established as the dominant flavor of context sensitivity
thanks to its outstanding precision. On the other hand, call-site
sensitivity has been regarded as unsuitable and its use in practice has been 
constantly discouraged for object-oriented programs. 
However, we claim that call-site sensitivity is generally 
a superior context abstraction because it is practically possible to transform 
object sensitivity into more precise call-site sensitivity. 
Our key
insight is that the previously known superiority of object sensitivity holds only
in the traditional $k$-limited setting, where the analysis is enforced
to keep the most recent $k$ context elements. However, it no longer holds if context tunneling in Section~{\ref{sec:Tunneling}} is included.
With context tunneling, where the analysis is free to choose an
arbitrary $k$-length subsequence of context strings, we show that
call-site sensitivity can simulate object sensitivity almost
completely, but not vice versa. To support the claim, we present a
technique, called \ourtechnique, for transforming 
arbitrary context-tunneled object sensitivity into
more precise, context-tunneled call-site-sensitivity.
We implemented \ourtechnique~in Doop and used it to derive a
new call-site-sensitive analysis from a state-of-the-art object-sensitive pointer analysis. 
Experimental results
 confirm that the resulting call-site sensitivity outperforms 
 object sensitivity in precision and scalability for real-world Java programs. 
Remarkably, our results show that even 1-call-site sensitivity can be more precise than the
conventional 3-object-sensitive analysis.









\lstset{,language=Java, aboveskip=3mm, belowskip=3mm,
	showstringspaces=false, columns=flexible,
	basicstyle=\linespread{1.2}\small\ttfamily, numbers=left, numbersep=5pt,
	numberstyle=\small\color{gray}, keywordstyle=\color{blue},
	commentstyle=\color{dkgreen}, stringstyle=\color{mauve},
	breaklines=true, breakatwhitespace=true, tabsize=3 
}


 

\lstdefinestyle{mydefault}{frame=tb,
	language=Java,
	xleftmargin=0.5cm,
	aboveskip=-3mm, % top margin
	belowskip=0mm,  % bottom margin
	showstringspaces=false,
	columns=flexible,
	basicstyle={\small\ttfamily},
	numbers=left,
	numberstyle=\tiny\color{gray},
	keywordstyle=\color{blue},
	commentstyle=\color{dkgreen},
	stringstyle=\color{mauve},
	breaklines=true,
	breakatwhitespace=false,
	tabsize=3,
	frame=none
}


\lstdefinestyle{myCustomMatlabStyle}{
	language=Java,
	xleftmargin=0.5cm,
	aboveskip=-3mm, % top margin
	belowskip=0mm,  % bottom margin
	showstringspaces=false,
	columns=flexible,
	basicstyle={\large\ttfamily},
	numbers=left,
	numberstyle=\tiny\color{gray},
	keywordstyle=\color{blue},
	commentstyle=\color{dkgreen},
	stringstyle=\color{mauve},
	breaklines=true,
	breakatwhitespace=false,
	tabsize=3,
	frame=none
}



%\usepackage{tabularx}

\newcolumntype{Y}{>{\centering\arraybackslash}X}
%\setcopyright{none}
%\usepackage {tikz}
\usetikzlibrary {shapes,positioning}
%\usepackage {bm}
\tikzstyle{block} = [rectangle, draw, fill=white, text width=4.8em,
, text
centered, rounded corners, minimum height=4em]
\tikzstyle{block2} =
[rectangle, draw, fill=white, text width=6em, text centered, rounded
corners, minimum height=4em, minimum width = 7em] \tikzstyle{line} = [draw, -latex']
\tikzstyle{onlyText} = [text width =2em, text centered]
%\setcopyright{rightsretained}

\tikzstyle{block3} =
[rectangle, draw, fill=white, text width=7em, text centered, rounded
corners, minimum height=3em] \tikzstyle{line} = [draw, -latex']

%\setcopyright{rightsretained}


\tikzstyle{blocks} = [rectangle, draw, fill=white, text width=0.7cm, text height = 4em, text
centered,  text width=4em, rounded corners, minimum height=0.6cm]

%\usepackage[bottom]{footmisc} %putting footnotes below bottom figures



% !TEX root = ./paper.tex

\section{Introduction}\label{sec:introduction}

\iffalse
\begin{quote}
  {\it ``Since its introduction, object sensitivity has emerged as the
  dominant flavor of context sensitivity for object-oriented
  languages.''} 
\begin{flushright}
---\cite{Smaragdakis2015}
  \end{flushright}
\end{quote}
\noindent
%\textcolor{red}{
%Pointer analysis is a foundation of many static analyses, which safely 
%estimates the objects that each variable points-to in real executions.
%Such information plays key roles in various software engineering techniques like 
%bugfinders~\cite{Sui2014},
%%bugfinders~\cite{Naik2006,NaikPSG09,Blackshear2015,Sui2014,Livshits2003},
%symbolic execution~\cite{Kapus2019}, and program repair tools~\cite{memfix,Gao2015,vfix2019}. The success of the above software engineering tools heavily depends on the performance of the underlying points-to analysis.}
Context sensitivity is critically important for static program analysis of
object-oriented programs.  A context-sensitive analysis associates
local variables and heap objects with context information of method
calls, computing analysis results separately for different
contexts. This way, context sensitivity prevents analysis information
from being merged along different call chains. For object-oriented and
higher-order languages, it is well-known that context sensitivity is
the primary means for increasing analysis precision without blowing up
analysis cost~\cite{Smaragdakis2015,Thiessen2017,Lhotak2006,
  KastrinisS13a, Smaragdakis2014, Li2018a, JeJeChOh17, %liang2005evaluating,
  Sridharan2006}.
\fi

There have been two major flavors of context sensitivity, namely {\em
  call-site sensitivity}~\cite{Sharir1981,Shivers1988} and {\em object
  sensitivity}~\cite{Milanova2002,Milanova2005}, which differ in the
choice of context information. The traditional $k$-call-site-sensitive
analysis~\cite{Sharir1981} uses a sequence of $k$ call-sites as the
context of a method. By contrast, object sensitivity uses
allocation-sites as context elements: in a virtual call, e.g., {\tt
  a.foo()}, an object-sensitive analysis uses the allocation-site of
the receiver object ({\tt a}) as the context of {\tt foo}. The standard
$k$-object-sensitive
analysis~\cite{Milanova2002,Milanova2005,Smaragdakis2011} maintains a
sequence of $k$ allocation-sites, comprising the allocation-site of
the receiver object, the allocation-site of the receiver's allocator,
and so on.


\myparagraph{The Status Quo}
Since its inception, object sensitivity has been established as the
dominant context abstraction for object-oriented
languages~\cite{Smaragdakis2015}.  Ever since \cite{Milanova2002,
  Milanova2005} proposed object sensitivity, its superiority over
other flavors of context sensitivity has been reinforced by a large
amount of 
research~\cite{Lhotak2008,BravenboerS09,Smaragdakis2011,TanLX16,
  JeJeChOh17,Lu:2019:PYF}.  Among others, \cite{Lhotak2006} and
\cite{BravenboerS09} conducted extensive experiments to conclude
that object sensitivity significantly outperforms other alternatives
including call-site sensitivity.  As a result, object sensitivity has
become an indispensable component of program analysis tools for
object-oriented languages~\cite{Fink2008,Zhang2014,NaikAW06,GordonKPGNR_NDSS15,Feng2014,vfix2019}.


In contrast, the use of call-site sensitivity has been constantly
discouraged for object-oriented
programs~\cite{TanLX16,JeJeChOh17,Li2018a,Lhotak2006,
  Smaragdakis2011, Smaragdakis2014, Milanova2002, Milanova2005}.  For
example, \cite{Milanova2002, Milanova2005} judged call-site
sensitivity as ``ill-suited" for object-oriented programs,
\cite{KastrinisS13a} claimed that call-site sensitivity should be
avoided because it is both imprecise and expensive, and
\cite{Smaragdakis2014} asserted call-site sensitivity is never
cost-effective. %as it is unscalable beyond the smallest context depth ($k=1$).  
As a result, call-site sensitivity has become obsolete in practice and
virtually not used anymore in program analysis tools for
object-oriented programs: 
\textit{``... object-sensitive analyses have
  almost completely supplanted traditional call-site-sensitive
  analyses for object-oriented languages''}~\cite{Smaragdakis2011}.

%\textcolor{red}{
%The status quo has been established on the empirical evaluation results showing superiority of object sensitivity compared to call-site sensitivity.
%For a majority of programs, even 1-object sensitivity 
%has been more precise and scalable than 2-call-site sensitivity.
%When analyzing a program (lusearch) in our benchmark, for example, 1-object sensitivity produces 27\% more precise abstraction (points-to set) 
%within 82\% fewer analysis time(s) than 2-call-site sensitivity.
%}
%%\minseok{Detailed data}

\myparagraph{This Work}
We challenge this commonly-accepted wisdom by showing that call-site
sensitivity is generally superior to object sensitivity %context abstraction
even for object-oriented programs.  Our key insight
is that the previously established superiority of object sensitivity over
call-site sensitivity is valid only when we impose a particular restriction that
the analysis should keep the {\em most recent} $k$ context elements,
but it no longer holds in a more general setting where the restriction
is eliminated. Notably, the relative superiority of object sensitivity
and call-site sensitivity gets inverted when they are generalized with
context tunneling~\cite{JeJeOh18}, where the analysis is free to use
an arbitrary $k$-length subsequence of context strings.  In this
generalized setting, we show that call-site sensitivity is able to {\em
  simulate} object sensitivity, but object sensitivity is not powerful
enough to simulate call-site sensitivity.
We note that our aim is not to debunk the previously known result. Instead, we claim that what is currently known only persists in a limited circumstance and the converse holds when the assumption is generalized. 


%\TODO{Say that the key challenge here is how to find the appropriate tunneling policy that makes CFA more precise than Obj and we address the challenge using an obj-guided approach}
To support the claim, %Exploiting this insight, 
we present \ourtechnique, a practical technique for transforming
%state-of-the-art 
object sensitivity into more precise,
context-tunneled call-site sensitivity.  Our technique takes as input
an arbitrary object-sensitive analysis with context tunneling and produces as output a
context-tunneling policy that enables call-site sensitivity to exceed
the precision limit of the baseline object sensitivity without
increasing $k$. Our key technical contributions to achieve this goal are the simulation and
simulation-guided learning procedures.  By the simulation procedure, we infer a context-tunneling policy
with which call-site sensitivity can simulate the baseline object
sensitivity.  The resulting call-site sensitivity, however,  is 
impractical since it requires running the baseline object-sensitive
analysis as a pre-analysis.  The learning phase aims to remove
this burden by capturing the behavior of the simulated policy using
a dataset of programs.


%We experimentally prove our claim for pointer analysis. 
%Experimental results show that our technique is practical and advances
%the state-of-the-art significantly.  
We implemented our technique in
Doop~\cite{BravenboerS09}, a popular pointer analysis framework for
Java.
We transformed a state-of-the-art object-sensitive pointer analysis
into the matching call-site-sensitive analysis. 
Evaluation with
real-world Java applications shows that the resulting call-site-sensitive analysis 
 significantly improves the original object-sensitive analysis in terms of both precision and scalability. 
Remarkably, our context-tunneled 1-call-site-sensitive
analysis is even more precise than the traditional 3-object-sensitive
analysis with much smaller costs,  which confirms our claim that call-site sensitivity can be superior to object sensitivity in the generalized setting. 

%\textcolor{red}{
%Note that we do not intend to say that previous results are wrong. Rather, we would like to say that call-site sensitivity should be reconsidered from now on because, though object sensitivity is superior to call-site sensitivity in the traditional k-limited setting, call-site sensitivity can be better than object sensitivity in a more general setting with context tunneling. We think that this message contrasts with what has been known in the program-analysis community.
%}

%which is far beyond the reach of
%modern object-sensitive analyses in practice.


\myparagraph{Contributions} We summarize our contributions below.
\begin{itemize}[leftmargin=1.3em]
\item %{\bf Claim} (Section~\ref{sec:insight}): 

We make a novel claim that call-site sensitivity is generally superior to object sensitivity; when the notion of $k$-limiting is
  generalized with context tunneling, call-site sensitivity can simulate object sensitivity almost completely, but not vice versa.

\item %{\bf Technique} (Section~\ref{sec:technique}):  
	We present \ourtechnique, a new technique for transforming a
context-tunneled $k$-object-sensitive analysis into a more precise, context-tunneled
  $k$-call-site-sensitive analysis. % via simulation and learning.
  Specifically, we make two technical contributions: the simulation (Section~\ref{sec:simulation}) and simulation-guided learning (Section~\ref{sec:learning}) procedures, both of which are vital to achieving the goal. 

\item %{\bf Experimental Proof} (Section~\ref{sec:result}): 
We experimentally prove our claim by applying \ourtechnique~to a
  state-of-the-art object-sensitive pointer analysis for Java. 
Our implementation and data are publicly available~\footnote{\url{https://github.com/kupl/OBJ2CFA}}.
\end{itemize}	





%%% Local Variables:
%%% mode: latex
%%% TeX-master: "paper"
%%% End:

% !TEX root = ./paper.tex
%\vspace{-1pt}
\section{Our Claim}\label{sec:insight}
%\noindent
In this section, we illustrate the main message of this paper with examples. 



\subsection{The Previously Known Superiority}\label{sec:call-vs-obj}
%\noindent

First of all, we note that traditional call-site sensitivity and
object sensitivity can complement each other~\cite{liang2005evaluating}.

%For any $k$, we can always find a program
%for which $k$-call-site sensitivity is more precise than
%$k$-object sensitivity, and vice versa.
% !TEX root = ./paper.tex
\begin{figure}[t]	
\begin{multicols}{2}
\vfill\null
\begin{subfigure}[t]{1.3\columnwidth}
\begin{center}
\begin{lstlisting}[escapeinside={(*}{*)},xleftmargin=0.6cm]
class D {
 Object id (v) { 
  return v; }
}
main() {
 D d = new D();//D
 A a = (A)d.id(new A()); //A, query1
 B b = (B)d.id(new B()); //B, query2
 C c = (C)d.id(new C()); //C, query3
}
\end{lstlisting}
\caption{Example code}
\label{background:example2}
\end{center}
\end{subfigure}
\columnbreak
%\scalefont{0.9}



\qquad
\begin{subfigure}[t]{0.80\columnwidth}
	\begin{center}
		\resizebox{0.4\columnwidth}{!}{
			\begin{tikzpicture}
			\node [block] (main) {\LARGE{\tt main$_{}$}\\$[\cdot]$};
			\node [block, above right=0.5cm and 0.5cm of main] (id1) {\LARGE{\tt id$_{}$}\\$[7]$};
			\node [block, below = 0.5cm of id1] (id2) {\LARGE{\tt id$_{}$}\\ $[8]$};
			\node [block, below = 0.5cm of id2] (id3) {\LARGE{\tt id$_{}$}\\ $[9]$};
			
			\path [line] (main) -- (id1);
			\path [line] (main) -- (id2);
			\path [line] (main) -- (id3);
					
			\end{tikzpicture}
		}
	\end{center}
	\caption{Call-graph by 1-call-site sensitivity}
	\label{back:cfa:callgraph2}
\end{subfigure}
\qquad\linebreak\linebreak

\qquad
\begin{subfigure}[t]{0.9\columnwidth}
	\begin{center}
		\resizebox{0.4\columnwidth}{!}{
			\begin{tikzpicture}
			\node [block] (main) {\LARGE{\tt main$_{}$}\\$[\cdot]$};
			\node [block, right = 0.5cm of main] (id) {\LARGE{\tt id$_{}$}\\${\tt[D]}$};
			
			\path [line] (main) -- (id);
			
			\end{tikzpicture}
		}
	\end{center}
	\caption{Call-graph by $k$-object sensitivity (for any $k$)}
	\label{back:obj:callgraph2}
\end{subfigure}
\end{multicols}
\vspace{-1em}
\caption{Typical situation that benefits from call-site sensitivity}
\label{back:b:Fig}
\vspace{-11pt}
\end{figure}




%%% Local Variables:
%%% mode: latex
%%% TeX-master: "paper"
%%% End:


\myparagraph{Benefit of Call-Site Sensitivity}
Figure~\ref{back:b:Fig} describes a typical situation where
call-site sensitivity has better precision than object sensitivity.
The example program has class {\tt D} that includes the identity
function {\tt id}. The {\tt main} method allocates an object of class
{\tt D} at line 6
and calls method {\tt id} on it in three places
at lines 7, 8, and 9 with different objects of type {\tt A}, {\tt B},
and {\tt C}, respectively.  Suppose pointer analysis aims to prove
that the three type-casting operations at lines 7, 8, and 9 are safe.
%(i.e. no down-casting violations occur).
Figure~\ref{back:cfa:callgraph2} shows the call-graph from
1-call-site-sensitive analysis. Note that the analysis analyzes the
method {\tt id} separately for the different call-sites at lines 
7, 8, and 9, and therefore is able to prove the safety of the queries. 


By contrast, $k$-object sensitivity is unable to prove any of the
queries in the program no matter what $k$ value is
used. Object sensitivity uses allocation-sites of receiver objects as
calling contexts. In this example, because the three method calls
share the same receiver object (i.e. the object pointed to by variable
{\tt d}), object sensitivity analyzes the method {\tt id}
with a single context element, namely the allocation-site {\tt D},
merging the three method calls (Figure~\ref{back:obj:callgraph2}).



\myparagraph{Benefit of Object Sensitivity}
Figure~\ref{back:a:Fig} %describes a major weakness of call-site sensitivity. 
describes a representative scenario where
object sensitivity is more precise than call-site sensitivity.
The example code in Figure~\ref{background:example1} has class {\tt C}
that contains $k+1$ methods
(${\tt id}_0, {\tt id}_1, \dots {\tt id}_k$), where each method
${\tt id}_i$ is semantically equivalent to the identity function
because ${\tt id}_0$ is the identity function and ${\tt id}_i$
%$(0< i \le k+1)$ 
{$(0< i \le k)$} calls ${\tt id}_{i-1}$ without modifying the formal
parameter {\tt v}.  The {\tt main} method has four heap allocation-sites:
namely, \texttt{C1}, \texttt{C2}, \texttt{A}, and \texttt{B}.  At line
13, {\tt main} calls ${\tt id}_k$ with the base variable {\tt c1} and
parameter \texttt{new A()}. At line {14}, ${\tt id}_k$ is called with
the base variable {\tt c2} and argument \texttt{new B()}.  Again, the goal
of pointer analysis is to prove the safety of the casting
operations at lines 13 and 14. For this program, a
$k$-call-site-sensitive analysis produces the call-graph in
Figure~\ref{back:cfa:callgraph}. Note that the method ${\tt id}_0$ is analyzed
under the single context ${[8,\dots,5]}$, where the critical information where
${\tt id}_k$ was originally called from is lost due to the truncation of
context strings to keep their last $k$ elements. 

Object sensitivity nicely addresses this shortcoming of call-site sensitivity. It uses the
allocation-sites, {\tt C1} and {\tt C2}, to represent the contexts of the
method calls to ${\tt id}_k$ at lines 13 and {14}, respectively. Note that the
receiver object remains the same in the subsequent calls to ${\tt id}_{k-1},
\dots {\tt id}_0$, propagating the initial contexts down to ${\tt id}_0$ and
producing the call-graph in Figure~\ref{back:obj:callgraph}. The
analysis is able to distinguish the two call chains and therefore proves the
queries.

%\textcolor{red}{Note that the shortcoming in handling nested method calls is the only weakness of call-site sensitivity; if there is no nested method call in a program, 1-call-site sensitivity becomes the most precise context abstraction. Suppose a program consists of methods that do not have any method call in their body except the main method. When analyzing the program, 1-call-site sensitivity will have the same precision with $\infty$-call-site sensitivity which is concrete context (the most precise context abstraction).}

% !TEX root = ./paper.tex

\begin{figure*}[t]	
\vspace{20pt}
\begin{multicols}{2}
\vfill\null
\begin{center}
\begin{subfigure}[b]{1.2\columnwidth}
\begin{lstlisting}[escapeinside={(*}{*)}, xleftmargin=12pt]
class C {
 Object id(*$_0$*)(v) {
  return v; }
 Object id(*$_1$*)(v) {
  return this.id(*$_{0}$*)(v); }
  ...
 Object id(*$_k$*)(v) {
  return this.id(*$_{k-1}$*)(v); }
}
main() {
 C c1 = new C();//C1
 C c2 = new C();//C2
 A a = (A)c1.id(*$_k$*)(new A());//A, query1
 B b = (B)c2.id(*$_k$*)(new B());//B, query2
}
\end{lstlisting}
\caption{Example code}
\label{background:example1}
\end{subfigure}
\end{center}
\columnbreak~\\~\\~\\
\vspace{-30pt}

%\scalefont{0.9}
%\qquad\qquad\qquad\qquad
%\begin{multicols}{3}
\begin{subfigure}[b]{0.9\columnwidth}
\begin{center}
				\resizebox{\columnwidth}{!}{
					\begin{tikzpicture}
					\node [block] (main) {\LARGE{\tt main$_{}$}\\$[\cdot]$};
					\node [block, above right =
                                        -0.4cm and 0.4cm of main]
                                        (idk1) {\LARGE{ {\tt id}$_k$}\\$[13]$};
					\node [block, below = 0.4cm of idk1] (idk2) {\LARGE{\tt {id}$_k$}\\$[14]$};
					
					\node [onlyText, right=0.4cm of idk1] (dots1) {...};
					\node [onlyText, right=0.4cm of idk2] (dots2) {...};
					
					\node [block2, right=0.4cm of
                                        dots1] (id11) {\LARGE{\tt {id}$_1$}\\$[13,8,...]$};
					\node [block2, right=0.4cm of
                                        dots2] (id12) {\LARGE{\tt {id}$_1$}\\$[14,8,...]$};
					
					\node [block2, right = 7.2cm of
                                        main] (id0) {\LARGE{\tt id$_0$}\\$\underbrace{[8,...,5]}_k$};
					
					\path [line] (main) -- (idk1);
					\path [line] (main) -- (idk2);
					\path [line] (idk1) -- (dots1);
					\path [line] (idk2) -- (dots2);
					\path [line] (dots1) -- (id11);
					\path [line] (dots2) -- (id12);
					
					\path [line] (id11) -- (id0);
					\path [line] (id12) -- (id0);
					
					\end{tikzpicture}
				}
			\end{center}
			\vspace{-4pt}
			\caption{Call-graph by $k$-call-site sensitivity}
			\label{back:cfa:callgraph}
\end{subfigure} \vspace{8pt}
%	\vfill\null

%\qquad\qquad\qquad\qquad
\begin{subfigure}[b]{0.8\columnwidth}
\begin{center}
				\resizebox{\columnwidth}{!}{
					\begin{tikzpicture}
					\node [block] (main) {\LARGE{\tt main$_{}$}\\$[\cdot]$};
					\node [block, above right =
                                        -0.4cm and 0.4cm of main]
                                        (idk1) {\LARGE{\tt ${ id}_k$}\\${\tt [C1]}$};
					\node [block, below = 0.4cm of
                                        idk1] (idk2) {\LARGE{\tt ${ id}_k$}\\${\tt [C2]}$};
					
					\node [onlyText, right=0.4cm of idk1] (dots1) {...};
					\node [onlyText, right=0.4cm of idk2] (dots2) {...};
					
					\node [block, right=0.4cm of
                                        dots1] (id11) {\LARGE{\tt id$_1$}\\${\tt [C1]}$};
					\node [block, right=0.4cm of
                                        dots2] (id12) {\LARGE{\tt id$_1$}\\${\tt [C2]}$};
					
					\node [block, right=0.4cm of
                                        id11] (id01) {\LARGE{\tt id$_0$}\\${\tt [C1]}$};
					\node [block, right=0.4cm of
                                        id12] (id02) {\LARGE{\tt id$_0$}\\${\tt [C2]}$};
					
					\path [line] (main) -- (idk1);
					\path [line] (main) -- (idk2);
					\path [line] (idk1) -- (dots1);
					\path [line] (idk2) -- (dots2);
					\path [line] (dots1) -- (id11);
					\path [line] (dots2) -- (id12);
					
					\path [line] (id11) -- (id01);
					\path [line] (id12) -- (id02);
					
					\end{tikzpicture}
				}
			\end{center}
			\vspace{-4pt}
			\caption[]{Call-graph by 1-object sensitivity}
			\label{back:obj:callgraph}
\end{subfigure}
\vspace{8pt}
%	\vfill\null

%\qquad\qquad\qquad\qquad

\begin{subfigure}[b]{1.0\columnwidth}
	\begin{center}
				\resizebox{0.8\columnwidth}{!}{
					\begin{tikzpicture}
					\node [block] (main) {\LARGE{\tt main$_{}$}\\$[\cdot]$};
					\node [block, above right =
                                        -0.4cm and 0.4cm of main]
                                        (idk1) {\LARGE{\tt ${ id}_k$}\\${\tt [13]}$};
					\node [block, below = 0.4cm of
                                        idk1] (idk2) {\LARGE{\tt ${ id}_k$}\\${\tt [14]}$};
					
					\node [onlyText, right=0.4cm of idk1] (dots1) {...};
					\node [onlyText, right=0.4cm of idk2] (dots2) {...};
					
					\node [block, right=0.4cm of
                                        dots1] (id11) {\LARGE{\tt id$_1$}\\${\tt [13]}$};
					\node [block, right=0.4cm of
                                        dots2] (id12) {\LARGE{\tt id$_1$}\\${\tt [14]}$};
					
					\node [block, right=0.4cm of
                                        id11] (id01) {\LARGE{\tt id$_0$}\\${\tt [13]}$};
					\node [block, right=0.4cm of
                                        id12] (id02) {\LARGE{\tt id$_0$}\\${\tt [14]}$};
					
					\path [line] (main) -- (idk1);
					\path [line] (main) -- (idk2);
					\path [line] (idk1) -- (dots1);
					\path [line] (idk2) -- (dots2);
					\path [line] (dots1) -- (id11);
					\path [line] (dots2) -- (id12);
					
					\path [line] (id11) -- (id01);
					\path [line] (id12) -- (id02);
					
					\end{tikzpicture}
				}
			\end{center}
			\vspace{-4pt}
			\caption[]{1-call-site sensitivity with context tunneling}
			\label{back:callT:callgraph}
\end{subfigure}





\end{multicols}
%	\vfill\null
	%\columnbreak
% \begin{subfigure}[b]{0.5\columnwidth}
% \begin{center}
% 				\resizebox{\columnwidth}{!}{
% 					\begin{tikzpicture}
% 					\node [block] (main) {\Large{$\tt main_{}$}\\$[\cdot]$};
% 					\node [block, above right =
%                                         -0.4cm and 0.4cm of main]
%                                         (idk1) {\Large{\tt ${\tt id}_k$}\\$[13]$};
% 					\node [block, below = 0.4cm of idk1] (idk2) {{\tt $id_k$}\\$[14]$};
					
% 					\node [onlyText, right=0.4cm of idk1] (dots1) {...};
% 					\node [onlyText, right=0.4cm of idk2] (dots2) {...};
					
% 					\node [block, right=0.4cm of
%                                         dots1] (id11) {\Large{\tt ${\tt id}_1$}\\$[13]$};
% 					\node [block, right=0.4cm of
%                                         dots2] (id12) {\Large{\tt ${\tt id}_1$}\\$[14]$};
					
% 					\node [block, right=0.4cm of
%                                         id11] (id01) {\Large{\tt ${\tt id}_0$}\\$[13]$};
% 					\node [block, right=0.4cm of
%                                         id12] (id02) {\Large{\tt ${\tt id}_0$}\\$[14]$};
					
% 					\path [line] (main) -- (idk1);
% 					\path [line] (main) -- (idk2);
% 					\path [line] (idk1) -- (dots1);
% 					\path [line] (idk2) -- (dots2);
% 					\path [line] (dots1) -- (id11);
% 					\path [line] (dots2) -- (id12);
					
% 					\path [line] (id11) -- (id01);
% 					\path [line] (id12) -- (id02);
					
% 					\end{tikzpicture}
% 				}
% 			\end{center}
% 			\caption{Call-graph by 1-call-site-sensitivity
%                           with context tunneling}
% 			\label{back:callT:callgraph}
%  \end{subfigure}
%\vspace{-1em}
\vspace{-14pt}
	\caption{Typical situation that benefits from object sensitivity}
	\label{back:a:Fig}
\vspace{-10pt}
\end{figure*}


%%% Local Variables:
%%% mode: latex
%%% TeX-master: "paper"
%%% End:



\myparagraph{Known Superiority} 
 Though call-site sensitivity and object sensitivity have their own
 strengths and weaknesses, object
  sensitivity is widely known to be superior to call-site sensitivity 
  because real-world programs involve code patterns such as one in
  Figure~\ref{back:a:Fig} more often. 
  Note that the existing superiority holds empirically, rather than theoretically, based on the experimental results of prior work~(e.g., \cite{Lhotak2008,BravenboerS09}).  
  

%\textcolor{red}{
%Though call-site sensitivity and object sensitivity have their own
%strengths and weaknesses, object sensitivity has been known to be
%superior to call-site sensitivity based on prior works' empirical evaluation
%result~\cite{Lhotak2008,BravenboerS09}.  
%Object sensitivity has shown a better performance than call-site
%sensitivity because real-world object-oriented programs involve code
%patterns such as one in Figure~\ref{back:a:Fig} more often. 
%}


%\vspace{-20pt}
\subsection{Revisiting the Superiority in Generalized $k$-Limited Setting}\label{sec:overview-result}
%\noindent
Our new claim is that the known superiority of object sensitivity over call-site sensitivity no longer holds when the notion of $k$-limiting is generalized with context tunneling. 

%In this paper, we show that 
%the existing 
%superiority of object sensitivity over call-site sensitivity holds only in the traditional $k$-limited setting, where the
%analysis is enforced to keep the most recent $k$ context elements, but no longer holds in a more general setting with context tunneling. 

\myparagraph{Context Tunneling}
Context tunneling~\cite{JeJeOh18} allows an analysis to maintain an arbitrary $k$-length subsequence of context strings.
For example, when $s = [C_1, C_2, C_3, C_4, C_5]$ is a sequence of
context elements that may appear in an unbounded ($k=\infty$)
context-sensitive analysis, the traditional $3$-limited analysis
abstracts the context string into its suffix $[C_3,C_4,C_5]$. 
With context tunneling, however, the analysis is free to use any
subsequence of $s$ such as $[C_1, C_3, C_5]$ and $[C_2, C_4, C_5]$, as
a $k$-limited abstraction of the original context string. 
Note that the traditional $k$-limited approach is a special case of the
generalized approach with context tunneling. 


\myparagraph{Key Insight}
Our key insight is summarized as follows: 
%that call-site sensitivity is superior to
%object sensitivity when they are generalized
%with context tunneling: 
\begin{itemize}[leftmargin=1.3em]
\item The major weakness of call-site sensitivity
in the traditional setting is no longer a weakness in the generalized
$k$-limited setting with context tunneling.%~\minseok{conflict. Need revision??} 
\item By contrast, object sensitivity still suffers from its
limitation even with the generalization. 
\end{itemize}

%For example, 
With context tunneling, call-site sensitivity does not
suffer from its shortcoming and can now prove the queries in
Figure~\ref{background:example1}. Suppose that we use a
context-tunneling policy that chooses the first $k$ elements of a
context string rather than the last $k$ ones.
%\textcolor{red}{Context tunneling achieves this by selectively updating contexts. 
%When {\tt id$_i$} (i<k) is called, the context-tunneling policy makes the callee method inherit the context of the caller method instead of updating context with the invocation site.} 
Then, the resulting
1-call-site-sensitive analysis produces the call-graph in Figure~\ref{back:callT:callgraph},
which is exactly the same as the
call-graph of the 1-object-sensitive analysis in
Figure~\ref{back:obj:callgraph}. Because the call-graphs are
equivalent, the call-site-sensitive analysis is as precise as the
object-sensitive analysis, successfully proving the queries. 



On the other hand, object sensitivity fails to simulate call-site
sensitivity even with context tunneling.  Consider the program in
Figure~\ref{back:b:Fig} where call-site sensitivity is
typically more precise than object sensitivity. For this program,
object sensitivity cannot prove all of the queries no matter what
context-tunneling policy and $k$ value are used. 
%\minseok{$k$-object-sensitivity with context tunneling can prove only one query among them.}
In
Figure~\ref{back:b:Fig}, object sensitivity can use the
allocation-site {\tt D} as context elements. Thus, only the two
context subsequences, i.e., \texttt{[D]} and \texttt{[$\cdot$]}, are
possible for the method {\tt id} with context tunneling, all of which
fail to analyze {\tt id} separately for the three different
call-sites.  % In this sense, object sensitivity is 
% inevitably less precise than call-site sensitivity in the generalized $k$-limited setting. 

%\myparagraph{Incompleteness}

We clarify that, like the previously known superiority, our claim in this paper is empirical. 
On the theoretical side, we do not know yet whether or not call-site sensitivity can always simulate object sensitivity in the general setting with context tunneling, which we leave as an open question for future work. We discuss this issue in more detail in Section~\ref{sec:counter_example}.
%not theoretical.  
%We can construct a counter-example such that
%context-tunneled call-site sensitivity is unable to simulate object
%sensitivity (Section~\ref{sec:counter_example}).  
%However,  such a counter-example is uncommon in practice, allowing the simulation to be almost complete for real-world programs. 
%(Sections \RNum{1} and \RNum{2} of the supplementary
%material and Section~\ref{sec:eval_learning}).


\myparagraph{\ourtechnique}
Based on the insight, 
%To support the claim, 
we develop \ourtechnique, a practical technique for transforming a given $k$-object-sensitive analysis 
into a more precise, context-tunneled $k$-call-site-sensitive analysis (Section~\ref{sec:technique}).  The
resulting analysis is (empirically) more precise than the baseline
object sensitivity, as it enjoys the benefits of both object
sensitivity and call-site sensitivity. For example, it
produces precise results for both cases in Figure~\ref{back:b:Fig} and~\ref{back:a:Fig}.


%%% Local Variables:
%%% mode: latex
%%% TeX-master: "paper"
%%% End:

% !TEX root = ./paper.tex
\begin{figure}[t]
  \[
    \begin{array}{c}
      \infer[]
      {
      \begin{array}{c}
        \ctx' \in \Reachable ({m}) \quad
        ({\heap}, {\hctx}) \in  \VarPointsTo ({m_{\this}}, {\ctx'}) \\
        \VarPointsTo({\it arg}, \ctx) \subseteq \VarPointsTo(m_{\it param}, \ctx')
        \quad
        \VarPointsTo(m_{\it return}, \ctx') \subseteq \VarPointsTo(x, \ctx)\\
        (\invo, \ctx, m, \ctx') \in \CallGraph
      \end{array}
      }{
      \begin{array}{c}
        {\it \invo : x = y.\sig({\it arg})} \quad
        \ctx \in    \Reachable ({\methof(\invo)}) \\
        ({\heap}, {\hctx})\in  \VarPointsTo ({\it y},\ctx) \quad
        t = {\typeof}(\heap)  \quad m = {\lookup}(t, \sig) \\
(e, {\pctx}, p) = \parent(\heap,\hctx, \invo, \ctx) \quad
        \ctx' = \left\{
        \begin{array}{l@{\quad}l}
          \truncateCtx{\appendCtx{{ \pctx}}{{e}}}{\CallDepth}
          & \mbox{if}~\invo\not\in T \\
          \truncateCtx{\pctx}{\CallDepth}     & \mbox{if}~\invo                       \in T
        \end{array}
                                                          \right.
      \end{array}
                                                          }
    \end{array}
  \]
  \caption{Context-sensitive pointer analysis using sets of invocation sites as tunneling abstractions}
\label{obj2cfa:fig:tunneling-rules}
\end{figure}

\section{Setup: Pointer Analysis with Context Tunneling}\label{sec:setting}

Unlike Section~{\ref{sec:tunneling}} that uses relations between methods as tunneling a abstraction ($\TunnelingRelation \subseteq \Methods \times \Methods$) for describing when to apply context tunneling, in this chapter we use a set of invocation sites as a tunenling abstraction $T$ as follows:
\[
  T \subseteq \mbi
\]
Then, the rule for method call in Figure~\ref{pre:baseline-rules} is replaced by the rule defined in Figure~\ref{obj2cfa:fig:tunneling-rules}.







\myparagraph{Instance Analyses}
Given  a program $P$ and 
its tunneling abstraction $T \subseteq \mbi_P$, 
we write $\call_k(P, T)$ and
$\obj_k(P, T)$ for the $k$-call-site- and
$k$-object-sensitive analyses, respectively. In the rest of this chapter, we fix $k$ and omit the subscript $k$ from $\call_k$ and $\obj_k$. 
%We assume that $\call(P, T)$ and $\obj(P, T)$ report as output some
%precision metric (e.g. the number of may-fail casts, where lower means better precision). 
%
%
%%\subsection{Context-Tunneling Policy}\label{sec:tunnelingpolicy}
%
These instance analyses are used with a {\em context-tunneling policy}. 
A context-tunneling policy $\heuristic$ is a function that maps a program $P$ into
a tunneling abstraction for $P$: 
\[
  \heuristic(P) \subseteq \mbi_P.
\]
With a policy $\heuristic$, 
we  perform the analysis for a program $P$ as follows: $\call(P, \heuristic(P))$ or $\obj(P, \heuristic(P))$. 


 % which first uses $\heuristic$ to produce a tunneling abstraction $\heuristic(P)$ and analyzes the program with the abstraction. 

%%% Local Variables:
%%% mode: latex
%%% TeX-master: "paper"
%%% End:

% !TEX root = ./paper.tex

%\input{analysis}


\section{\ourtechnique: Transforming Object Sensitivity to Call-Site Sensitivity}\label{sec:technique}
We now present our technique, \ourtechnique. 
Given an object-sensitive analysis specified by an arbitrary
tunneling policy $\objheuristic$, our technique transforms it to another
tunneling policy $\callheuristic$
%\[
%\objheuristic \to \mbox{\ourtechnique} \to \callheuristic
%\]
such that the
call-site-sensitive analysis with $\callheuristic$ becomes more precise
than the baseline object-sensitive analysis with $\objheuristic$. 

To achieve this, \ourtechnique~works in the two steps: simulation and learning. 
It first converts $\objheuristic$ into
a {\em simulated policy} $\simheuristic$. 
%\[
%    \objheuristic \to \mbox{Simulation} \to \simheuristic
%\]
With $\simheuristic$, call-site sensitivity becomes
more precise than object sensitivity with $\objheuristic$
but running the analysis with $\simheuristic$
is expensive as it uses the baseline
object-sensitive analysis as a pre-analysis. 
The purpose of the second step is to  remove this overhead
 by learning the behavior of $\simheuristic$ from training data.  
%\[
%  \begin{array}{r}
%%    \simheuristic \to\\[0.1em]
%   \simheuristic +  \mbox{training data} \to
%  \end{array}
%  \mbox{Learning} \to \callheuristic \qquad\qquad
%\]
The learned policy $\callheuristic$ is as precise as
$\simheuristic$ but more efficient as it does not rely on the
simulation procedure. 






%%% Local Variables:
%%% mode: latex
%%% TeX-master: "paper"
%%% End:

% !TEX root = ../thesis.tex


\subsection{Simulation}\label{sec:simulation}

%A key insight behind our technique is that call-site sensitivity with
%context tunneling can simulate object sensitivity.  That is, 

The first technical contribution of this chapter is the simulation procedure. 
Given a program $P$ and its tunneling abstraction $\Tobj \subseteq \mbi_P$, where $\Tobj$ is given by $\objheuristic$, i.e., 
$\Tobj = \objheuristic(P)$,
the goal of simulation is to infer a tunneling abstraction
$\Tcall \subseteq \mbi_P$ such that $\call(P, \Tcall)$ becomes more
precise than ${\sf obj} (P, \Tobj)$.


% !TEX root = ./paper.tex


%basicstyle=\linespread{1.2}\small\ttfamily
\begin{figure}[t]
%\setlength{\columnsep}{0.07cm}
\begin{lstlisting}[multicols=2, escapeinside={(*}{*)},xleftmargin=4.0ex ,basicstyle={\linespread{1.2}\fontsize{9.5}{10}\selectfont\ttfamily}]
class D {
 Object id(v) {
  return v;}
 Object id1(v) {
  return this.id(v);}
 void m() {
  A a = (A)this.id(new A());//A1, qry1
  B b = (B)this.id(new B());//B1, qry2
 }
}
void main() {
 D d1 = new D();//D1
 D d2 = new D();//D2
 D d3 = new D();//D3 

 A a3 = (A)d1.id1(new A());//A2, qry3 
 B b3 = (B)d2.id1(new B());//B2, qry4
 d3.m();
}
\end{lstlisting}
\vspace{-15pt}
\caption{Running example}
\vspace{-15pt}
\label{simulation:example}
\end{figure}


%%% Local Variables:
%%% mode: latex
%%% TeX-master: "paper"
%%% End:


\myparagraph{Running Example}
\label{sec:simulation-overview}
We illustrate the simulation procedure with the
example program in Figure~\ref{simulation:example}.  The code
contains class {\tt D} that has three methods \texttt{id},
\texttt{id1}, and \texttt{m}.  Methods \texttt{id} and \texttt{id1}
are identity functions.
The method \texttt{m} contains two method invocations at lines 7 and
8, which call \texttt{id} with new \texttt{A} and \texttt{B} objects.
The \texttt{main} method creates three objects at the
allocation-sites \texttt{D1}, \texttt{D2}, and \texttt{D3}, and stores
them in variables \texttt{d1}, \texttt{d2} and \texttt{d3},
respectively.  At line 16, \texttt{main} calls \texttt{id1} with
a new object of type {\tt A} and the base variable \texttt{d1}.  At line 17,
\texttt{id1} is called with a new object with type {\tt B} and base variable
\texttt{d2}.  At line 18, \texttt{main} also calls {\tt m} with base
variable \texttt{d3}.  We assume that the code has four queries, which ask the
safety of casting operations at lines 7, 8, 16, and 17. Note that all
of these are safe since {\tt id} and {\tt id1} are identity
functions.








In this example, for simplicity, we assume an 1-object-sensitive analysis
without context tunneling (i.e. $\Tobj = \emptyset$) is given but our technique is applicable
to object sensitivity with arbitrary $k$ and tunneling abstraction $\Tobj \subseteq \mbi_P$. 
Figure~\ref{fig:obj:callgraph} shows the call-graph produced by the
baseline 1-object-senstivie analysis, 
where a call-graph edge is represented by invocation-site, caller method, caller context, 
callee method, and callee context. For example, the edge
$\texttt{id1[D1]} \stackrel{5}{\to} \texttt{id[D1]}$ indicates that
method {\tt id} is called from {\tt id1} at invocation-site $5$,
where the callee and caller contexts are {\tt D1}.  Note that
this object-sensitive analysis is not precise enough to prove all queries.
Although it can prove queries at lines 16 and
17 as it distinguishes the two different contexts of {\tt id1}, it
fails to prove queries at lines 7 and 8 because it uses the same
context {\tt [D3]} for {\tt id} at both call-sites.





%\begin{comment}
%\textcolor{blue}{
Figure~\ref{fig:CFA:callgraph} shows the call-graph obtained by the
ordinary 1-call-site-sensitive analysis without context tunneling. 
Note that the precision of the analysis is incomparable to
that of the baseline 1-object sensitivity. 
Because the analysis uses the
call-site as the calling context, it is able to prove
the queries at
lines 7 and 8 by separately analyzing the two calls to \texttt{id}. However,
it fails to prove the queries at lines
16 and 17 as the variable {\tt v} in \texttt{id} under context
\texttt{[5]} points-to both heap objects \texttt{A2} and \texttt{B2}
that in turn
propagates back to the variables {\tt a} and {\tt b} in the
\texttt{main} method.
%}
%\end{comment}


% !TEX root = ../thesis.tex



\begin{figure*}[t]
	\begin{multicols}{4}
		\begin{subfigure}[b]{1.\columnwidth}
			\begin{center}
				\resizebox{1.\columnwidth}{!}{
					\tikzstyle{every node}=[font=\LARGE]
					\begin{tikzpicture}
					\node [block] (main) {{\tt main}\\$[\cdot]$};
					\node [block, right = 0.7cm of main] (id11) {\tt {id1}\\$[$D2$]$};
					\node [block, above = 0.4cm of id11] (id12) {\tt {id1}\\$[$D1$]$};
					
					\node [block, right=0.6cm of id11] (oid1) {\tt {id}\\$[$D2$]$};
					\node [block, right=0.6cm of id12] (oid2) {\tt {id}\\$[$D1$]$};
					
					\node [block, below = 0.6cm of id11] (m) {\tt {m}\\$[$D3$]$};
					
					\node [block, right=0.6cm of m] (nid) {\tt {id}\\$[$D3$]$};
					
					\path [line] (main) edge node[above] {17} (id11);
					\path [line] (main) edge node[above left] {16} (id12);
					\path [line] (id11) edge node[above] {5} (oid1);
					\path [line] (id12) edge node[above] {5} (oid2);
					\path [line] (main) edge node[below left] {18} (m);
					\path [line] (m) edge [bend left=20] node[above] {7} (nid);
					\path [line] (m) edge [bend right=20] node[below] {8} (nid);
					\end{tikzpicture}
				}
			\end{center}
			\vspace{7pt}
			~\\
			\caption{\small $1$-object sensitivity with $\Tobj = \emptyset$}
			\label{fig:obj:callgraph}
		\end{subfigure}
		\vfill\null
		\columnbreak
		\begin{subfigure}[b]{1\columnwidth}
			\begin{center}
				\resizebox{1\columnwidth}{!}{
					\begin{tikzpicture}
					\tikzstyle{every node}=[font=\LARGE]				
					\node [block] (main) {\tt {main}\\$[\cdot]$};
					\node [block, right = 0.7cm of main] (id11) {\tt {id1}\\$[17]$};
					\node [block, above = 0.4cm of id11] (id12) {\tt {id1}\\$[16]$};
					
					%\node [block, right = 0.4cm of id11] (oid1) {{\tt id}\\$[17]$};
					\node [block, right = 0.4cm of id12] (oid2) {\tt {id}\\$[5]$};
					
					\node [block, below = 0.4cm of id11] (m) {\tt {m}\\$[18]$};
					
					\node [block, right=0.4cm of m] (nid1) {\tt {id}\\$[7]$};
					\node [block, below=0.4cm of nid1] (nid2) {\tt {id}\\$[8]$};
					
					
					\path [line] (main) edge node[above] {$17$} (id11);
					\path [line] (main) edge node[above left] {$16$} (id12);
					\path [line] (id11) edge node[below right] {$5$} (oid2);
					\path [line] (id12) edge node[above] {$5$} (oid2);
					\path [line] (main) edge node[below left] {$18$} (m);
					\path [line] (m) edge node[above] {$7$} (nid1);
					\path [line] (m) edge node[below left] {$8$} (nid2);				
					\end{tikzpicture}
				}
			\end{center}
			\vspace{5pt}
			\caption{\small $1$-CFA with $\Tcall = \emptyset$}
			\label{fig:CFA:callgraph}
		\end{subfigure}
		\vfill\null
		\columnbreak
		\begin{subfigure}[b]{1\columnwidth}
			\begin{center}
				\resizebox{1\columnwidth}{!}{
					\begin{tikzpicture}
					\tikzstyle{every node}=[font=\LARGE]
					\node [block] (main) {\tt {main}\\$[\cdot]$};
					\node [block, right = 0.7cm of main] (id11) {\tt {id1}\\$[17]$};
					\node [block, above = 0.4cm of id11] (id12) {\tt {id1}\\$[16]$};
					
					\node [block, right=0.6cm of id11] (oid1) {\tt {id}\\$[17]$};
					\node [block, right=0.6cm of id12] (oid2) {\tt {id}\\$[16]$};
					
					\node [block, below = 0.4cm of id11] (m) {\tt {m}\\$[18]$};
					
					\node [block, right=0.6cm of m] (nid) {\tt {id}\\$[18]$};
					
					
					\path [line] (main) edge node[above] {$17$} (id11);
					\path [line] (main) edge node[above left] {$16$} (id12);
					\path [line] (id11) edge node[above] {$5$} (oid1);
					\path [line] (id12) edge node[above] {$5$} (oid2);
					\path [line] (main) edge node[below left] {$18$} (m);
					\path [line] (m) edge [bend left=20]  node[above] {$7$} (nid);
					\path [line] (m) edge [bend right=20]  node[below] {$8$} (nid);
					
					\end{tikzpicture}
				}			
			\end{center}
			%\vspace{-10pt}
			~\\
			\vspace{9pt}
			\caption{\small $1$-CFA with $\Tcall = \myset{5,7,8}$}
			\label{fig:call:callgraph}
		\end{subfigure}
		\vfill\null
		\columnbreak
		\begin{subfigure}[b]{1\columnwidth}
			\begin{center}
				\resizebox{1\columnwidth}{!}{
					\begin{tikzpicture}
					\tikzstyle{every node}=[font=\LARGE]
					\node [block] (main) {\tt {main}\\$[\cdot]$};
					\node [block, right = 0.7cm of main] (id11) {\tt {id1}\\$[17]$};
					\node [block, above = 0.4cm of id11] (id12) {\tt {id1}\\$[16]$};
					
					\node [block, right=0.4cm of id11] (oid1) {\tt {id}\\$[17]$};
					\node [block, right=0.4cm of id12] (oid2) {\tt {id}\\$[16]$};
					
					\node [block, below = 0.4cm of id11] (m) {\tt {m}\\$[18]$};
					
					\node [block, right=0.4cm of m] (nid1) {\tt {id}\\$[7]$};
					\node [block, below=0.4cm of nid1] (nid2) {\tt {id}\\$[8]$};
					
					
					\path [line] (main) edge node[above] {$17$} (id11);
					\path [line] (main) edge node[above left] {$16$} (id12);
					\path [line] (id11) edge node[above] {$5$} (oid1);
					\path [line] (id12) edge node[above] {$5$} (oid2);
					\path [line] (main) edge node[below left] {$18$} (m);
					\path [line] (m) edge node[above] {$7$} (nid1);
					\path [line] (m) edge node[below left] {$8$} (nid2);				
					\end{tikzpicture}
				}
			\end{center}
			\vspace{5pt}
			\caption{$1$-CFA with $\Tcall = \myset{5}$}
			\label{fig:opt:callgraph}
		\end{subfigure}
		
	\end{multicols}
	\vspace{-25pt}
	\caption{Call-graphs of running example produced by object sensitivity
		and call-site sensitivity. % From left to right, our context-simulation gradually 
		% refines call-site-sensitivity's call-graphs so the outcome is more precise than the 
		% object-sensitivity's one
              }
	\label{Fig:Simulation}
			\vspace{-12pt}
\end{figure*}


%%% Local Variables:
%%% mode: latex
%%% TeX-master: "../thesis.tex"
%%% End:


\myparagraph{Inferring Tunneling Abstraction}
Simulation is a two-step process. 
It first runs the baseline object-sensitive analysis 
%\begin{comment}
(i.e. ${\sf obj}(P, \Tobj)$) 
%\end{comment}
to obtain its
call-graph $\CallGraph \subseteq \mbi \times \mbc \times \mbm
\times \mbc$.
Next, it analyzes the structure of $\CallGraph$ and infers a tunneling abstraction
$\Tcall$ that makes call-site sensitivity to simulate $\CallGraph$. 
At a high-level, we infer three kinds of invocation sites and define
\[
\Tcall = (I_1 \cup I_2) \setminus I_3
\]
where $I_1$, $I_2$, and $I_3$ are invocations sites in $P$.
Intuitively, $I_1$ and $I_2$ denote the invocation sites
that require context tunneling in order for call-site sensitivity to simulate
object sensitivity. On the other hand, $I_3$ is the invocation sites
where context tunneling must be avoided to preserve the original
precision of call-site sensitivity. % Thus, we can increase the
% precision of call-site sensitivity even more than object sensitivity
% by including $I_1\cup I_2$ and excluding $I_3$. 


Our key idea to infer $I_1$ and $I_2$ is to assume that $\CallGraph$ was produced by a context-tunneled
call-site-sensitive analysis and infer
backward its tunneling abstraction.
To this end, we identify and exploit two fundamental properties of context-tunneled call-site sensitivity.


The first property is that the callee method's context becomes
equivalent to the caller's context when context tunneling is applied during call-site sensitivity.
This is because, in call-site sensitivity, applying
context tunneling at an invocation site always makes the called
method inherit the caller's context.
Thus, we scan each call-graph edge $(l, c, m, c')$ of $\CallGraph$ and
identify those that have this property ($c=c'$). 
We define $I_1$ to be the set of invocation sites of all such edges:
\vspace{-3pt}
\[
I_1 = \myset{l \in \mbi_P \mid (l,c,m,c') \in \CallGraph, c = c'}.
\vspace{-3pt}\]
For example, $I_1$ is $\myset{5, 7, 8}$ for the call-graph in
Figure~\ref{fig:obj:callgraph}, where the invocation-site 5 comes from
the call-graph edges $(5,{\tt D1},\texttt{id}, {\tt D1})$ and
$(5,{\tt D2},\texttt{id}, {\tt D2})$,
7 comes from $(7,{\tt D3},\texttt{id},{\tt D3})$, and 8 comes from $(8,{\tt
  D3},\texttt{id},{\tt D3})$.

  %$I_1$ is sound but incomplete property of call-site
  %sensitivity with tunneling. 
%\textcolor{red}{If context tunneling is not applied to the invocation $l$, the callee method's context $c'$ becomes different from the caller method's context $c$ as the new context $c'$ will include the context element $l$ which are not in $c$.}
% Figure~\ref{fig:call:callgraph} shows the call-graph produced by the
% 1-call-site-sensitive analysis with the tunneling abstraction $T_{\it call} =
% \myset{5,7,9}$. Note that call-site-sensitivity with $T_{\it call} = I_1$ already
% simulates
% object-sensitivity, producing the call-graph identical to
% that of the baseline object-sensitive analysis in Figure~\ref{fig:obj:callgraph}.


% Intuition behind $I_1$ came from object-sensitivity's property of carrying
% caller method's context over callee method if two methods share same base
% object. The property is crucial in practice because it gives
% object-sensitivity
%\begin{comment}
In practice, applying context tunneling at $I_1$ gives call-site
sensitivity immunity against nested call chains that are popular in
object-oriented programs. For example, Figure~\ref{fig:obj:callgraph} shows that
object sensitivity precisely distinguishes two invocations of {\tt id}
according to their base objects $D1$ and $D2$.  In contrast,
conventional call-site sensitivity must use larger $k$ to precisely
analyze those nested call chains. 
%\end{comment}





%$I_2$ includes invocation-sites where different caller contexts
% imply different callee contexts:
%\[
%I_2 =\{\invo \in Invo \mid \forall (\invo, c_1, m, c_1'), (\invo, c_2, m, c_2') \in
%G.\; c_1 \neq c_2 \implies c_1' \neq c_2'\}
%\]
%where we assume $(l,c_1,m,c_1')$ and $(l,c_2,m,c_2')$ are different
%call-edges, i.e., $c_1 \not= c_2 \vee c_1' \neq c_2'$.


%\textcolor{red}{
The second property of context-tunneled call-site sensitivity is that
different caller contexts imply different callee contexts. Suppose
two different call-graph edges $(l, c_1, m, c_1')$ and
$(l, c_2, m, c_2')$, where the last (i.e., $k$th) context element of $c_1$ is different from that of $c_2$, are generated in call-site sensitivity. 
%$(l, c_2, m, c_2')$, where $c_1 \not= c_2$, are generated in call-site sensitivity. 
If context tunneling was applied at $l$, then the last context elements of $c_1'$ and
$c_2'$ are certainly different because the callee should inherit the
caller's contexts (i.e. $c_1 = c_1'$ and $c_2 = c_2'$).
We collect invocation sites in $\CallGraph$ with this property: %i.e., $I_2$ is%~\minseok{TODO}
%\[
% \{\invo \in S \mid \forall (\invo, c_1, m, c_1'), (\invo, c_2, m, c_2') \in
%  \CallGraph.  c_1 \neq c_2 \implies  c_1' \neq c_2'\}
%\]
\begin{multline*}
	I_2 =  \{\invo \in S \mid \forall (\invo, c_1, m, c_1'), (\invo, c_2, m, c_2') \in
	\CallGraph.  \kth(c_1) \neq \kth(c_2) \implies \\ \kth(c_1') \neq \kth(c_2')\}
\end{multline*}
where $S$ denotes the invocation sites where a method is called under two different contexts:
\[
S = \{l \in \mbi_P \mid \exists (l,c1,m,c1'),(l,c2,m,c2')\in \CallGraph,  \kth(c1) \not = \kth(c2)\},
\]


%\[
%S = \{l \in \Invo_P \mid \exists (l,c1,m,c1'),(l,c2,m,c2')\in \CallGraph,  c1 \not = c2\}.
%\]
and the function \kth~takes a context and returns its last context element, i.e., $\kth(a_1a_2\dots$ $ a_k) = a_k$.  
 $I_2$ denotes a
sound and complete property of context-tunneled call-site sensitivity. 
That is, if context tunneling is not applied to $l$, the callee methods'
contexts inevitably share the last context element $l$.
%}
In Figure~\ref{fig:obj:callgraph}, $I_2$ is $\myset{5}$ because the invocation-site 5
has two outgoing edges $(5,{\tt D1},\texttt{id},{\tt D1})$ and $(5,{\tt
D2},\texttt{id},{\tt D2})$, where ${\tt D1} \neq {\tt
D2} \implies {\tt D1} \neq {\tt D2}$ holds. 
%\textcolor{blue}{
The invocation-sites 7 and 8 do not
belong to $I_2$ because they have only one call-graph edge.
%}
In Figure~\ref{fig:obj:callgraph}, $I_1$ includes $I_2$, but in
general, $I_2$
may be distinct from $I_1$ as shown in the following call-graph:

{
	%\renewcommand{\baselinestretch}{1.5} 
	\setstretch{1.0}
	\begin{center}
	\resizebox{0.3\columnwidth}{!}{
                \tikzstyle{every node}=[font=\LARGE]
		\begin{tikzpicture}
		\node [block] (main) {{\tt main$_{}$}\\$[\cdot]$};
		\node [block, above right = -0.4cm and
		0.7cm of main] (Dm1) {\tt {D.m}\\$[$D1$]$};
		\node [block, below = 0.4cm of Dm1] (Dm2) {\tt{D.m}\\$[$D2$]$};
		\node [block, right = 0.7cm of Dm1] (Cm1) {\tt{C.m}\\$[$C1$]$};
		\node [block, right = 0.7cm of Dm2] (Cm2) {\tt{C.m}\\$[$C2$]$};
		
		\path [line] (main) edge node[above] {$4$} (Dm1);
		\path [line] (main) edge node[above] {$5$} (Dm2);
		\path [line] (Dm1) edge node[above] {$9$} (Cm1);
		\path [line] (Dm2) edge node[above] {$9$} (Cm2);
		\end{tikzpicture}
    }
    \end{center}
}
where $I_1 = \emptyset$ and $I_2 = \myset{9}$. 
% A detail example is described in Section 1.2 of our supplementary material~\footnote{\url{https://zenodo.org/record/5560499}}.
      %Appendix~\ref{sec:unwrapped}.
% Without context tunneling, $1$-call-site-sensitivity merges
% two {\tt C.m} method calls under context of \texttt{[9]}.
% By applying context-tunneling to invocations in $I_2$, however, 
% the method calls inherit contexts from their callers,
%  \texttt{[4]} and \texttt{[5}] respectively, thus becomes 
% as precise as $1$-object-sensitivity.

%\textcolor{red}{
%Intuitively, the invocation-sites in a tunneling abstraction ($\Tcall$) are considered as bad context elements for call-site sensitivity; the invocation-sites will not be used as context elements. The tunneling abstraction $I_1\cup I_2 = \myset{5,7,8}$ considers the invocation-sites 5,7,8 as bad elements because they make call-site sensitivity produce a different context abstraction from the given object sensitivity.
%}


Note that applying context tunneling at $I_1\cup I_2 = \myset{5,7,8}$
makes call-site sensitivity simulate the baseline object
sensitivity (Figure~\ref{fig:call:callgraph} and Figure~\ref{fig:obj:callgraph} are equivalent). However, it loses the precision benefit of the original call-site
sensitivity (compare Figure~\ref{fig:call:callgraph}
vs. Figure~\ref{fig:CFA:callgraph}).
The purpose of $I_3$ is to avoid this precision loss. 
We define $I_3$ to be the set of invocation-sites where the called
  method has a single context:
\[
I_3 = \myset{l \in \Invo_P \mid \forall (l, c_1, m, c_1'), (l, c_2,
	m, c_2') \in G.\; c_1' = c_2'}.
\]
% where the edges $(l, c_1, m_1, c_1')$ and $(l, c_2,
%   m_2, c_2')$ can be identical.
In Figure~\ref{fig:obj:callgraph}, $I_3$ equals to $\myset{7, 8, 16,
  17, 18}$. For example, $I_3$ includes 7 because only one call-graph
edge exists out of that invocation-site and $I_3$ does not include 5
because it has two outgoing edges to the same method ({\tt id}) under
different contexts ({\tt [D1]} and {\tt [D2]}).
%$I_3$ guarantees that call-site sensitivity analyzes the called method
%more precisely than (or at least equal to) object sensitivity.\minseok{Check} This is because
Intuitively, 
$I_3$ represents the set of invocation-sites that make conventional call-site sensitivity %, which always updates contexts, 
 at least as precise as the given object sensitivity; avoiding context tunneling at 
$I_3$ would make call-site sensitivity analyze the called method
more precisely than (or at least equal to) object sensitivity.
 It is because updating context with the invocation sites ensures that the method invocation is not conflated with another invocation from different invocation-sites which is not the case for object sensitivity.
%It is because conventional call-site sensitivity separates methods called from an invocation-site from the others which is not the case for object sensitivity.
%Updating contexts at $I_3$ ensures that the method invocation is not conflated with another invocation from different invocation-sites.
%As object sensitivity failed to separate any of the methods invoked in $I_3$.
%The intuition behind $I_3$ is leveraging a property of conventional call-site sensitivity that methods called from an invocation-site is guaranteed to be separated from the ones called from the other invocation-sites which is not the case for object sensitivity. 
%Updating contexts at $I_3$ would ensures that the method invocation is not conflated with another invocation from different invocation-sites, which is not the case for
%object sensitivity.
%This is because
%it ensures that the method invocation is not conflated with another
%invocation from different invocation-sites, which is not the case for
%object sensitivity. 
%Figure~\ref{fig:opt:callgraph} shows the resulting call-graph
In summary, we infer $\Tcall = \myset{5}$ from
Figure~\ref{fig:obj:callgraph}: 
 $\Tcall = (I_1 \cup I_2)  \setminus I_3 = (\myset{5,7,8} \cup \myset{5})
 \setminus \myset{7,8,16,17,18}
= \myset{5}$. 

%\textcolor{red}{
Additionally, we apply context tunneling to the invocation-sites if all their parameters are passed from those of the caller methods. 
For example, $\myset{5}$ in our example
belongs to this case because all the parameters (i.e., \texttt{this} and \texttt{v}) come from the caller. 
Such invocations need context tunneling because caller methods' contexts determine the value of the parameters.
For the same reason, we avoid context tunneling if all the parameters
are allocated just before the invocations (e.g., $\myset{16, 17}$ in
our example). The invocation-sites are useful because using them as context elements determines the value of the invocations' parameters.

% !TEX root = ./paper.tex



\begin{figure*}
\begin{center}
%\begin{subfigure}[b]{1\columnwidth}
%\begin{multicols}{2}
\centering
\begin{lstlisting}[multicols=2, escapeinside={(*}{*)},xleftmargin=4.0ex ,basicstyle={\linespread{1.2}\fontsize{9.5}{10}\selectfont\ttfamily}]
class Container {
  Object elem;
  void add(Object el) {
    this.elem = el;}
  Itr itr() {
    Object e = this.elem;
    Itr itr = new Itr(e);//It
    return itr;}
}
class Itr {
  Object next;
  Itr(Object obj) {
    this.next = obj;}
  Object next() {
    return this.next;}
}
void main() {
  Container c1 = new Container();//C1
  c1.add(new A());//A
  Itr i1 = c1.itr();
  object o1 = i1.next();
	
  Container c2 = new Container();//C2
  c2.add(new B());//B
  Itr i2 = c2.itr();
  object o2 = i2.next();
}
\end{lstlisting}
%\end{multicols}
%\end{subfigure}
%\begin{multicols}{2}
%	\begin{subfigure}[b]{\columnwidth}
%	\centering
%		\begin{center}
%			\resizebox{\columnwidth}{!}{
%				\begin{tikzpicture}
%                \tikzstyle{every node}=[font=\Large]
%				\node [block] (main) {\tt {main}\\$[\cdot]$};
%				
%				\node [block, left = 1.28cm of main] (add1) {\tt {add}\\$[$C1$]$};
%				\node [block, below = 0.6cm of add1] (add2) {\tt {add}\\$[$C2$]$};
%				
%				\node [block, right = 0.4cm of add2] (next1) {\tt {next}\\$[$C1$]$};
%				\node [block, right = 0.4cm of next1] (next2) {\tt {next}\\$[$C2$]$};
%				
%				\node [block, right=0.4cm of next2] (itr1) {\tt {itr}\\$[$C1$]$};
%				\node [block, above=0.6cm of itr1] (itr2) {\tt {itr}\\$[$C2$]$};
%				
%				\node [block, right=0.4cm of itr1] (Itr1) {\tt {Itr}\\$[$C1$]$};
%				\node [block, right=0.4cm of itr2] (Itr2) {\tt {Itr}\\$[$C2$]$};
%				
%				
%				\path [line] (main) edge node[above] {$23$} (add1);
%				\path [line] (main) edge node[above] {$28$} (add2);
%				
%				\path [line] (main) edge node[left] {$25$} (next1);
%				\path [line] (main) edge node[right] {$30$} (next2);
%				
%				\path [line] (main) edge node[above] {$24$} (itr1);
%				\path [line] (main) edge node[above] {$29$} (itr2);
%				
%				\path [line] (itr1) edge node[above] {$8$} (Itr1);
%				\path [line] (itr2) edge node[above] {$8$} (Itr2);
%				
%				
%				\end{tikzpicture}
%			}
%		\end{center}
%		\caption{$1$-object sensitivity with $T_{\it obj}=\myset{8,25,30}$}
%		\label{fig:wrappedflow:1objT}
%	\end{subfigure}
%	\vfill\null
%	\columnbreak
%	\begin{subfigure}[b]{\columnwidth}
%		\begin{center}
%			\resizebox{\columnwidth}{!}{
%				\begin{tikzpicture}
%                \tikzstyle{every node}=[font=\Large]
%				\node [block] (main) {\tt {main}\\$[\cdot]$};
%				
%				\node [block, left = 1.28cm of main] (add1) {\tt {add}\\$[23]$};
%				\node [block, below = 0.6cm of add1] (add2) {\tt {add}\\$[28]$};
%				
%				\node [block, right = 0.4cm of add2] (next1) {\tt {next}\\$[25]$};
%				\node [block, right = 0.4cm of next1] (next2) {\tt {next}\\$[30]$};
%				
%				\node [block, right=0.4cm of next2] (itr1) {\tt {itr}\\$[24]$};
%				\node [block, above=0.6cm of itr1] (itr2) {\tt {itr}\\$[29]$};
%				
%				\node [block, right=0.4cm of itr1] (Itr1) {\tt {Itr}\\$[24]$};
%				\node [block, right=0.4cm of itr2] (Itr2) {\tt {Itr}\\$[29]$};
%				
%				
%				\path [line] (main) edge node[above] {$23$} (add1);
%				\path [line] (main) edge node[above] {$28$} (add2);
%				
%				\path [line] (main) edge node[left] {$25$} (next1);
%				\path [line] (main) edge node[right] {$30$} (next2);
%				
%				\path [line] (main) edge node[above] {$24$} (itr1);
%				\path [line] (main) edge node[above] {$29$} (itr2);
%				
%				\path [line] (itr1) edge node[above] {$8$} (Itr1);
%				\path [line] (itr2) edge node[above] {$8$} (Itr2);
%				
%				
%				\end{tikzpicture}
%			}
%			
%		\end{center}
%		\caption{$1$-call-site sensitivity with $T_{\it call}=\myset{8}$}
%		\label{fig:wrappedflow:1callT}
%	\end{subfigure}
%\end{multicols}
%\vspace{-20pt}
%\caption{Wrapped flow pattern~\cite{Li2018a} and call-graphs produced by
%1-object sensitivity with context-tunneling and 1-call-site
%sensitivity with context tunneling}
%\label{fig:wrappedflow}
\end{center}
\vspace{-12pt}
\caption{Example code}
\vspace{-12pt}
\label{fig:wrappedflow:example}
\end{figure*}

\myparagraph{Effectiveness in the Real-World}\label{sec:simulation:real-world}
To show the effectiveness of simulation more clearly, we provide a representative real-world example.
%In this example, we aim to show that call-site sensitivity with tunneling can effectively analyze the representative example, and call-site sensitivity 
Also, this example shows that call-site sensitivity can simulate object sensitivity with a nontrivial tunneling abstraction ($\Tobj \neq \emptyset$).



%\textcolor{red}
{
Containers and iterators have been popular to exemplify the strength
of object sensitivity (e.g.~\cite{Milanova2005,JeJeOh18,TanLX16}), as
they are prevalent in object-oriented
programs.
Figure~\ref{fig:wrappedflow:example} shows a typical example.
%, called {\em wrapped flow}~\cite{Li2018a}.
In Figure~\ref{fig:wrappedflow:example}, the {\tt Container} class has
its iterator of which the field wraps the container's
data, and the data is obtained by the calling iterator's method.
}

%\textcolor{red}
{
In this case, 1-object sensitivity needs context tunneling to distinguish all the method calls.
The conventional 1-object-sensitive analysis is not precise as
the two iterators share the same heap allocation-site {\tt It}. 
Consider a tunneling abstraction
$\Tobj =\{7, 21, 26\}$.
With this tunneling abstraction, 1-object sensitivity produces precise
results with the
call-graph shown in Figure~\ref{fig:wrappedflow:1objTGraph}.
}


\begin{figure*}
\begin{center}
%\begin{subfigure}[b]{1\columnwidth}
%\begin{multicols}{2}

%\end{multicols}
%\end{subfigure}
\begin{multicols}{2}
	\begin{subfigure}[b]{\columnwidth}
	\centering
		\begin{center}
			\resizebox{.9\columnwidth}{!}{
				\begin{tikzpicture}
                \tikzstyle{every node}=[font=\Large]
				\node [block] (main) {\tt {main}\\$[\cdot]$};
				
				\node [block, left = 1.28cm of main] (add1) {\tt {add}\\$[$C1$]$};
				\node [block, below = 0.6cm of add1] (add2) {\tt {add}\\$[$C2$]$};
				
				\node [block, right = 0.4cm of add2] (next1) {\tt {next}\\$[$C1$]$};
				\node [block, right = 0.4cm of next1] (next2) {\tt {next}\\$[$C2$]$};
				
				\node [block, right=0.4cm of next2] (itr1) {\tt {itr}\\$[$C1$]$};
				\node [block, above=0.6cm of itr1] (itr2) {\tt {itr}\\$[$C2$]$};
				
				\node [block, right=0.4cm of itr1] (Itr1) {\tt {Itr}\\$[$C1$]$};
				\node [block, right=0.4cm of itr2] (Itr2) {\tt {Itr}\\$[$C2$]$};
				
				
				\path [line] (main) edge node[above] {$19$} (add1);
				\path [line] (main) edge node[above] {$24$} (add2);
				
				\path [line] (main) edge node[left] {$21$} (next1);
				\path [line] (main) edge node[right] {$26$} (next2);
				
				\path [line] (main) edge node[above] {$20$} (itr1);
				\path [line] (main) edge node[above] {$25$} (itr2);
				
				\path [line] (itr1) edge node[above] {$7$} (Itr1);
				\path [line] (itr2) edge node[above] {$7$} (Itr2);
				
				
				\end{tikzpicture}
			}
		\end{center}
		\caption{$1$-object sensitivity with $T_{\it obj}=\myset{7,21,26}$}
		\label{fig:wrappedflow:1objTGraph}
	\end{subfigure}
	\vfill\null
	\columnbreak
	\begin{subfigure}[b]{\columnwidth}
		\begin{center}
			\resizebox{.9\columnwidth}{!}{
				\begin{tikzpicture}
                \tikzstyle{every node}=[font=\Large]
				\node [block] (main) {\tt {main}\\$[\cdot]$};
				
				\node [block, left = 1.28cm of main] (add1) {\tt {add}\\$[19]$};
				\node [block, below = 0.6cm of add1] (add2) {\tt {add}\\$[24]$};
				
				\node [block, right = 0.4cm of add2] (next1) {\tt {next}\\$[21]$};
				\node [block, right = 0.4cm of next1] (next2) {\tt {next}\\$[26]$};
				
				\node [block, right=0.4cm of next2] (itr1) {\tt {itr}\\$[20]$};
				\node [block, above=0.6cm of itr1] (itr2) {\tt {itr}\\$[25]$};
				
				\node [block, right=0.4cm of itr1] (Itr1) {\tt {Itr}\\$[20]$};
				\node [block, right=0.4cm of itr2] (Itr2) {\tt {Itr}\\$[25]$};
				
				
				\path [line] (main) edge node[above] {$19$} (add1);
				\path [line] (main) edge node[above] {$24$} (add2);
				
				\path [line] (main) edge node[left] {$21$} (next1);
				\path [line] (main) edge node[right] {$26$} (next2);
				
				\path [line] (main) edge node[above] {$20$} (itr1);
				\path [line] (main) edge node[above] {$25$} (itr2);
				
				\path [line] (itr1) edge node[above] {$7$} (Itr1);
				\path [line] (itr2) edge node[above] {$7$} (Itr2);
				
				
				\end{tikzpicture}
			}
			
		\end{center}
		\caption{$1$-call-site sensitivity with $T_{\it call}=\myset{7}$}
		\label{fig:wrappedflow:1callTGraph}
	\end{subfigure}
\end{multicols}
\vspace{-30pt}
\caption{call-graphs produced by
1-object sensitivity with context-tunneling and 1-call-site
sensitivity}
\vspace{-12pt}
\label{fig:wrappedflow}
\end{center}
\end{figure*}

%\textcolor{red}
{
With our technique, call-site-sensitivity can simulate the call-graph
produced by the context-tunneled object-sensitive analysis. 
From the call-graph in Figure~\ref{fig:wrappedflow:1objTGraph}, our
technique finds out
\[
I_1 = \myset{7}, \quad  I_2 = \{7\},\quad I_3 =
\{19,20,21,24,25,26\}.
\]
and produces a tunneling abstraction $\Tcall = \{7\}$.
Figure~\ref{fig:wrappedflow:1callTGraph} shows the call-graph of
1-call-site sensitivity with the tunneling abstraction, which is
exactly the same as that of the baseline object-sensitive analysis
in Figure~\ref{fig:wrappedflow:1objTGraph}.
Note that 1-object sensitivity and 1-call-site sensitivity use
different tunneling abstractions, i.e., $\Tcall = \{7\}$ vs.
$\Tobj = \{7, 21, 26\}$. 
}





%\textcolor{red}{
%\myparagraph{Effectiveness in the Real-World}
%The simulation technique described above is principled (i.e., it exploits fundamental properties of context-tunneled call-site sensitivity) and yet powerful
%enough to make call-site sensitivity to enjoy most of the precision
%benefits of object sensitivity in practice. For example, our technique
%enables 1-call-site sensitivity to effectively handle all
%patterns of {\em precision-critical} methods
%in~\cite{Li2018a}.  
%\citet{Li2018a} identified fairly representative and
%exhaustive patterns that require object sensitivity in real-world Java
%programs;
%applying 2-object sensitivity only to the methods with those patterns
%is sufficient to achieve 98.8\% of the full precision~\cite{Li2018a}. 
%Conventional call-site-sensitive analysis is
%ineffective to handle such patterns as they require the
%analysis to maintain deep calling contexts. With context tunneling,
%however, our technique enables call-site
%sensitivity to simulate object sensitivity in all cases. We provide
%detailed description of the patterns and how our technique works in
%Appendix~\ref{appendix:effectiveness}. 



\myparagraph{Simulated Policy}
Now we define the simulated policy $\simheuristic$. 
Let $\simulate$ be the simulation process described above. 
As input, $\simulate$ takes a program $P$ and a tunneling abstraction
$\Tobj \subseteq \Invo_P$ of baseline object sensitivity.  
Running the simulation procedure, denoted $\simulate(P, \Tobj)$, produces a tunneling abstraction $\Tcall \subseteq
\Invo_P$ for call-site sensitivity.  
With $\simulate$, we can transform $\objheuristic$ into
$\simheuristic$ as follows:
\begin{equation}\label{eq:simheuristic}
  \simheuristic = \lambda P.\; \simulate(P,
  \objheuristic(P)).
\end{equation}
Although the simulated policy $\simheuristic$ makes call-site
sensitivity more precise than the baseline object sensitivity, using
$\simheuristic$  is
impractical because the simulation procedure incurs the overhead of running the baseline object-sensitive analysis.






%%% Local Variables:
%%% mode: latex
%%% TeX-master: "../thesis"
%%% End:

% !TEX root = ./paper.tex

\subsection{Simulation-Guided Learning}\label{sec:learning}

The second technical contribution of this paper is simulation-guided learning that aims to remove the overhead of $\simheuristic$ while maintaining its precision. 
To do so, from a dataset of programs $\vec{P} = \myset{P_1, P_2, \dots, P_n}$, 
we learn the final policy, i.e., $\callheuristic$, that captures the behavior of
$\simheuristic$ without invoking the expensive simulation procedure. 


\myparagraph{Parameterized Policy}
To learn a policy from data, we need to define a {\em parameterized
  policy}, denoted $\heuristic_f$, whose behavior is
fully controlled by the parameter $f$. The goal of learning then
is to find an appropriate parameter from data.
% such that
% $\call(P, \heuristic_f(P))$ is as precise as 
% $\call(P, \simheuristic(P))$.

For parameterization, we adapt the idea of prior work~\cite{JeJeChOh17},
where the parameter $f$ is a boolean formula over atomic features.
We assume that a set $\mba = \myset{a_1, a_2, \dots, a_m}$ of atomic
features is given (we explain them shortly).  
Formally, a feature
$a_i$ is a function from programs to predicates on invocation sites, i.e.,
$a_i(P): \Invo_P \to \myset{\true, \false}$.
That is, an invocation site $\invo$ in a program $P$ has feature
$a_i$ iff $a_i(P)(\invo)$ is $\true$.
In prior work~\cite{JeJeOh18}, atomic features are combined by a
boolean formula $f$ to express complex, in particular disjunctive, 
properties: 
\[
f \to \true \mid \false \mid a_i \in \mba \mid \neg f \mid f_1 \land
f_2 \mid f_1 \vee f_2
\]


A boolean formula $f$ denotes a set of
invocation sites. We write $\featmean{f}{P}$ for the denotation of $f$
with respect to $P$:  $\featmean{\true}{P} = \Invo_P$,
$\featmean{\false}{P} = \emptyset$,
$\featmean{a_i}{P} = \myset{l \in \Invo_P \mid a_i(P)(l)}$,
$\featmean{\neg f}{P} = \Invo_P \setminus \featmean{f}{P}$,
$ \featmean{f_1 \land f_2}{P} = \featmean{f_1}{P} \cap
\featmean{f_2}{P}$,
$\featmean{f_1 \vee f_2}{P} = \featmean{f_1}{P} \cup
                                    \featmean{f_2}{P}$.
Then, with a boolean formula $f$, we define the parameterized policy
$\heuristic_f$ as follows:
\[
\heuristic_f(P) = \featmean{f}{P}.
\]



\myparagraph{Learning Objective}
%With the parameterized policy, we can formally state the learning objective. 
The goal of learning is to find a formula $f$ that enables
$\heuristic_f$ to capture the behavior of $\simheuristic$ on the training
programs; we aim to find a formula $f$ such that 
\begin{equation}\label{eq:learning-objective}
  \sum_{P\in\vec{P}} \call(P, \heuristic_f(P)) \approx
  \sum_{P\in\vec{P}} \call(P, \simheuristic(P))
\end{equation}
where we assume $\call$ returns the number of unproved queries
(e.g., \#may-fail casts, where lower is more precise). 
%\begin{comment}
Note that our learning objective is more challenging than those considered in
prior work~\cite{JeJeChOh17,JeJeOh18}. In prior work, the objective was
typically to find a ``good-enough'' policy but, in our case, the
learned policy should capture the specific behavior of the simulated
policy.
%\end{comment}


\myparagraph{Learning Algorithm}
We present a new, simulation-guided learning algorithm that can effectively solve the problem in Eq. (\ref{eq:learning-objective}). In Section~\ref{sec:eval_learning}, we show that the existing, unguided learning algorithm~\cite{JeJeOh18} is not powerful enough to solve the problem of capturing the behavior of the simulated policy ($\simheuristic$). 

The overall structure of our algorithm is given in
Algorithm~\ref{alg:overall1}.
We invoke the algorithm by
$\textsc{Learn}(\vec{P}, \mba \cup \neg\mba, \simheuristic)$, where
$\neg\mba = \myset{\neg a \mid a \in \mba}$, so that the formula $f$
is initially set to the disjunction of all atomic features and their
negation:
$f = a_1 \vee \neg a_1 \vee a_2 \vee \neg a_2 \cdots \vee a_m \vee
\neg a_m$, where $a_1, a_2, \dots, a_m \in \mba$ (line 2).  Then, we
repeat the loop at lines 4--14.  At the beginning of each 
iteration, the formula $f$ is in disjunctive normal form; 
$f$ is of the form $c_1 \vee c_2 \vee \dots \vee c_k$, where $c_i$ is a
conjunctive clause.
A single refinement step for $f$ is done by choosing a clause $c$ in $f$ (via
$\ChooseClause$ at line 5), choosing an atomic feature (via
$\ChooseAtom$ at line 6), and replacing $c$ in $f$ by $c \land a$
(line 8).  If the quality of the refined formula $f'$ has been
improved over the original formula $f$ (line 9), we set $f$ to $f'$
and otherwise discard $f'$. This process is repeated until no more
refinement is possible. To check this termination condition, the
algorithm maintains $\Refiner$ that maps clauses to available
refiners (line 3, 10, and 12). %(line 10 and 12). 



\begin{algorithm}[t]
  \caption{Overall learning algorithm}\label{alg:overall1}\small
  \begin{algorithmic}[1]
    \Require Training programs $\vec{P}$, features $A$, and simulated
    policy $\simheuristic$ \Ensure A boolean formula $f$
    \Procedure{Learn}{$\vec{P}, A, \simheuristic$} \State
    $f \gets a_1 \vee a_2 \vee \dots \vee a_k ~(A =
    \myset{a_1,a_2,\dots,a_k})$ \State
    $\Refiner \gets \lambda c \in f. A \setminus \myset{a \mid a
%    $\Refiner \gets \lambda c \in f. \mba \setminus \myset{a \mid a
      \in c}$ \Repeat \State $c \gets \ChooseClause(f,\simheuristic,\vec{P})$ \State
    $a \gets \ChooseAtom(c, \Refiner,\simheuristic,\vec{P})$ \State $c' \gets c \land a$
    \State $f' \gets \mbox{Replace $c$ in $f$ by $c'$}$ \If
    {$\BetterThan(f', f,\simheuristic,\vec{P})$} \State
    $\Refiner \gets \Refiner[c' \mapsto \Refiner(c) \setminus
    \myset{a}]$, $f \gets f'$ \Else \State
    $\Refiner \gets \Refiner[c \mapsto \Refiner(c) \setminus
    \myset{a}]$ \EndIf \Until{no more refinement is possible
      ($\forall c.\;\Refiner(c)=\emptyset$) } \State \Return $f$
    \EndProcedure
  \end{algorithmic}
\end{algorithm}





The key feature of our algorithm is to steer the search toward the
desired solutions by receiving guidance from the
simulated policy ($\simheuristic$). 
The guidance happens in the three components, 
i.e., $\ChooseClause$, $\ChooseAtom$, and $\BetterThan$. 
Given a formula
$f = c_1 \vee c_2 \vee \dots \vee c_k$, we choose a clause whose
behavior most deviates from $\simheuristic$ over the training
programs; that is, $\ChooseClause(f, \simheuristic, \vec{P})$ chooses $c$ in $f$ that
maximizes the difference between $\sem{c}_P$ and $\simheuristic$:
\[
  \begin{array}{c}
\ChooseClause(f, \simheuristic, \vec{P}) = \argmax_{c \in f} \sum_{P \in \vec{P}} \lvert
    \featmean{c}{P} \setminus \simheuristic(P)\rvert.
    \end{array}
\]
To refine the chosen clause $c$, 
$\ChooseAtom(c, \Refiner, \simheuristic, \vec{P})$ chooses an atom $a \in \Refiner(c)$
that maximizes the following: 
\[
  \begin{array}{c}
\sum_{P \in \vec{P}}  \lvert \simheuristic(P) \cap \featmean{c}{P} \cap \featmean{a}{P}  \rvert.
  \end{array}
\]
Intuitively, it chooses the atom that most conservatively refines the clause toward
the simulation result. To this end, it selects $a$ that maximizes
$\simheuristic(P) \cap \sem{a}_{P}$ (refining toward the simulation result) and
$\sem{c}_P \cap \sem{a}_P$ (conservatively refining $c$).
%With the chosen $c$ and $a$, the formula $f$ is strengthened to $f'$
With the chosen $c$ and $a$, the formula $f$ is specified to $f'$
by replacing $c$ in $f$ by $c \land a$. To check whether $f'$ improves
over $f$, we evaluate the formulas with the following objective function:
\[
    \begin{array}{c}
\objective(f,\simheuristic,\vec{P}) = \sum_{P\in\vec{P}} \call(P, \heuristic_f(P) \cap
      \simheuristic(P)).
      \end{array}
\]
Given a formula, the objective function runs the analysis over the
training programs with the intersection of the current tunneling policy
($\heuristic_f$) and the simulated policy ($\simheuristic$). 
The condition $\BetterThan(f', f,\simheuristic,\vec{P})$ is true iff $\objective(f',\simheuristic,\vec{P}) \le \objective(f,\simheuristic,\vec{P})$. 
In the objective function, note that we evaluate the performance of
$\heuristic_{f}(P) \cap \simheuristic(P)$ instead of
$\heuristic_{f}(P)$. This is a critical step to avoid local minima.
% The abstraction space of context tunneling is not monotone and
% applying tunneling to more invocation sites does not imply better or
% worse precision. 
% Context tunneling do not have monotonicity that applying context tunneling to more do not imply a better or worse precision:
% \[
% I_1 \subset I_2 \centernot\implies \call(P, I_1) \le \call(P, I_2) \lor \call(P, I_2) \le \call(P, I_1).
% \]
% If we use $\heuristic_{f}(P)$ in the objective function, the learning may end up with local minima as
% our algorithm searches a formula by specifying the given one $f$.
For example, suppose we have formulas $f_1$ and $f_2$, where $f_2$ is
obtained by refining $f_1$, and that both $\sem{f_1}_{P}$ and
$\sem{f_2}_{P}$ are supersets of $\simheuristic(P)$
(i.e.
$\simheuristic(P) \subseteq \sem{f_2}_{P} \subseteq \sem{f_1}_{P}
$). Although $f_2$ can be refined further toward $\simheuristic$,
such refinement can be rejected as $\heuristic_{f_2}$ may have poorer
precision than $\heuristic_{f_1}$ because tunneling abstraction is not
monotone with respect to precision~\cite{JeJeOh18}. Thus, the 
learning algorithm fails to make a further progress, ending up with a
local minimum $f_1$.  With our objective function, 
however, the learning algorithm can avoid such a local minimum because
$\heuristic_{f_1}(P) \cap \simheuristic(P) =
\heuristic_{f_2}(P) \cap \simheuristic(P) = \simheuristic(P)$. 
%\textcolor{red}{
% In Section 2 of the supplementary material, we provide a simple running example that illustrates how Algorithm 1
% learns a formula $f$.
%}
%for all
%programs $P$. 

%\myparagraph{Further Optimization}
%After the algorithm terminates, we can run the algorithm once again to
%further optimize the learned policy. In this case, the algorithm is run with different
%initial formula and strategies. We set the initial formula $f$ to the
%outcome of the previous run of the algorithm.
%Because the goal of this step is simply to maximize the analysis precision, we use the
%objective function $\objective(f) = \sum_{P\in\vec{P}} \call(P, \heuristic_{f}(P))$.
%% A formula $f$ that minimizes the above objective function will satisfy
%% the learning objective with high probability.
%$\ChooseClause$ chooses a clause $c$ that maximizes $\sum_{P \in
%  \vec{P}} \lvert \featmean{c}{P} \rvert$.
%$\ChooseAtom$ chooses an atom $a$ which maximizes
%$\sum_{P \in \vec{P}}  \lvert \featmean{c \land a}{P}  \rvert$. 
%% The feature would conservatively refine the chosen clause.
%% The formula $f$ from
%% this step becomes the final parameter. 


%\end{table}

%%% Local Variables:
%%% mode: latex
%%% TeX-master: "paper"
%%% End:

% !TEX root = ./paper.tex



\begin{table}[ht]
	\renewcommand{\arraystretch}{1}
	\caption{Atomic features ($\mba$) used in our method}\label{tbl:obj2cfa:features}
	\centering
	\setstretch{1.1}
	\footnotesize
		 \begin{tabular}{c c c c c c c c c c }
			\toprule
			 A1  & ``java''    & A2  & ``lang''     & A3 & ``sun'' & A4 & ``()'' & A5  & ``void''\\ 
			 A6  & ``security'' & A7 & ``int''      & A8 & ``util'' & A9  & ``String'' &  A10 & ``init''\\ \bottomrule
		 \end{tabular}
		\begin{tabular}{ l l }
			 B1 &  Method is contained in a nested class               \\      
			 B2 & Method contains local assignments                    \\ 
			 B3 & Method contains local variables                      \\ 
			 B4 & Method is contained in a large class                      \\ 
			 B5 & Method contains a heap allocation         \\
			 B6 & Method is a static method         \\
			\midrule
			C1 & Method is called on \texttt{this} (i.e. \texttt{this.$m$(...)}) \\
			C2 & An argument is allocated in the same method      \\ 
			C3 & Method is called on object of static field					\\
			C4 & Method is called else where in the same method \\
			C5 & The base variable is passed to an initializer  \\
			C6 & The containing method's modifier is ``$\texttt{protected}$''        \\
			C7 & The containing method has exception handling        \\
			C8 & All  parameters are passed from caller method\\
			C9 & All  parameters are initialized just before the calls\qquad\,\,\\
			C10 & In ``{\tt java.util.regex}'' class\\
			C11 & Invoke constructor (i.e. \texttt{new C(...)}) \\
			C12 & All the caller method's arguments' type is integer\\
			C13 & Caller method takes more than 2 arguments \\
			C14 & In ``{\tt java.io.*}'' class\\
			C15 & In ``{\tt java.util.logging}'' class \\
			C16 & Takes at least 2 arguments\\
			C17 & Virtual method calls in application class \\
			C18 & Callee method's name is ``{\tt clone}''\\
			C19 & C16 $\land \neg$ C6 $\land \neg$ C1\\
			\bottomrule
		 \end{tabular}
	%\vspace{-10pt}
	\end{table}




\myparagraph{Feature Engineering} 
The success of learning depends also on the atomic features ($\mba$). 
We used a total of 35 atomic features in Table~\ref{tbl:obj2cfa:features},
all of which describe syntactic properties of invocation sites. 
Here, we identify invocations with called methods on them.
Features A1--A10 and B1--B6 came from~\cite{JeJeOh18}. 
% that designed features describing methods in Java programs.}
%Features A1--A10 and B1--B6 came from \citet{JeJeOh18}. 
%Jeon et al.~\cite{JeJeOh18} designed a number of syntactic features of methods. 
Features A1--A10 describe methods whose signatures 
contain the corresponding strings.
For example, when a method's signature string is ``java.lang.String: int length()'', the method has features A1, A2, A4, A7, and A9.
Features B1--B6 describe properties of method bodies. 
For example, feature B1 indicates whether a method is defined in a nested class or not.


Using existing features only was
insufficient 
and we newly designed features C1--C19 in
Table~\ref{tbl:obj2cfa:features}.  In our case, feature engineering was not very difficult as it
can be guided by the simulated policy $\simheuristic$.  To obtain those
features, we initially ran our learning algorithm on training data
with existing features (A1--A10 and B1--B6) only, which resulted in a
parameter $f$ with which the learned policy ($\heuristic_{f}$) did not
satisfy the learning objective. We investigated the reason why $\heuristic_{f}$
fails to capture the behavior of $\simheuristic$ by analyzing the
difference between $\simheuristic$ and $\heuristic_f$ 
(i.e. $\simheuristic(P)\setminus \heuristic_{f}(P)$ and
$\heuristic_{f}(P) \setminus \simheuristic(P)$). 
That is, our goal was to identify features $a_1$ and
$a_2$ that minimize
$\sum_{P\in \vec{P}} \sem{a_1}_P \xor (\simheuristic(P) \setminus
\heuristic_f(P))$ and
$\sum_{P\in \vec{P}} \sem{a_2}_P \xor (\heuristic_f(P) \setminus
\simheuristic(P))$, respectively. 
We included the new 
features $a_1$ and $a_2$ in the feature set and ran the 
algorithm  again. 
We repeated this process until the policy space was large enough to contain solutions and the learning
algorithm could find one of them.
% In our case, the loop iterated 7 times and each iteration produced a
% feature in C1--C7. 
% This feature engineering process was done with training programs only. 
% \minseok{Sunbeom: It seems too expensive, Junhee: feature engineering needs creativity.}








%%% Local Variables:
%%% mode: latex
%%% TeX-master: "paper"
%%% End:

% !TEX root = ./paper.tex

\section{Experimental Results}\label{sec:evaluation}\label{sec:result}
We experimentally prove our claim by evaluating \ourtechnique~on real-world programs. 
Main research questions are as follows: 
%We aim to
%answer the following research questions: 
\begin{itemize}[leftmargin=1.3em]
\item \textbf{Does our claim hold in the real-world?} 
Can call-site sensitivity be significantly superior to object sensitivity for real-world programs? 
How precise and scalable can the
  context-tunneled call-site-sensitive analysis be in practice? 
%  obtained from our
%  technique?
  % effective is out technique that transforms object sensitivity into
  % a context-tunneled call-site sensitivity?
%  How far does it advance the state-of-the-art
%object-sensitive analyses?

%\item \textbf{Effectiveness of simulation and learning:} How
%  accurately can our technique simulate object sensitivity?  Does
%  learning effectively capture the behavior of the simulated policy?
%  Is our learning algorithm essential? How
%  does it compare to simpler approaches?
\item \textbf{Impact of simulation and learning}: 
Is simulation necessary? 
How accurately can our technique simulate object sensitivity?  Is the simulation-guided learning necessary to capture the behavior of the simulated policy? How important are the features in learning?


\end{itemize}


%\input{table_sobj}
%\input{table_2call}

\myparagraph{Experimental Setting}
We implemented \ourtechnique~on top of Doop~\cite{BravenboerS09}, a
popular pointer analysis framework for Java~\cite{JeJeOh18, Li2018b,
  JeJeChOh17, TanLX16, Smaragdakis2014}.  
%\textcolor{red}{Note that superiority of object sensitivity over call-site sensitivity has been consistently demonstrated on Doop~\cite{BravenboerS09,TanLX16,KastrinisS13a,Smaragdakis2014}.}
%To implement our
%popular pointer analysis framework for Java~\cite{JeJeOh18, Li2018b,
%  JeJeChOh17, Tan2017, TanLX16, Smaragdakis2014}.  To implement our
%technique, 
We used the publicly-available implementation of context
tunneling given by \cite{JeJeOh18} and newly implemented our
simulation (Section~\ref{sec:simulation}) and learning
(Section~\ref{sec:learning}) techniques in Doop.  We conducted all
experiments on a machine with Intel i7 CPU and 64GB memory running the
Ubuntu 16.04 64bit operating system.


%\paragraph{Benchmarks}
We used 12 Java programs used by~\cite{JeJeOh18}, of which 10 came from the DaCapo 2006
benchmarks~\cite{Blackburn2006} ({luindex}, {lusearch}, {antlr},
{pmd}, {fop}, {eclipse}, {xalan}, {chart}, {bloat}, and {jython}), and
the remaining two ({checkstyle} and {jpc}) are real-world open-source programs.
%\textcolor{red}{Note that the DaCapo benchmarks also have consistently been used to experimentally demonstrate the superiority of object sensitivity over call-site sensitivity~\cite{Lhotak2006,BravenboerS09,TanLX16}.}
Following prior work~\cite{JeJeOh18,Smaragdakis2014}, we classified those 12
programs into 4 small (luindex, lusearch, antlr, and pmd) and 8 large
programs. 
We used the group of small programs as training data, 
from which our context-tunneling policy $\callheuristic$ is learned, 
and used large programs as test data to evaluate the
policy for unseen programs.


%Pointer analysis START
%analysis time: 123.01s
%Pointer analysis FINISH
%disk footprint (KB)                                                              959,488
%loading statistics (simple) ...
%elapsed time: 11.96s
%
%making database available at /home/minseok/Graphick/Ctx_Sensitivity/doop/results/C-1-tunneled-call-site-sensitive+heap/jre1.6/jython.jar
%making database available at last-analysis
%#var points-to            20,027,400
%#may-fail casts           1,723
%#poly calls               2,498
%#reach methods            12,011
%#call edges               107,112



%\paragraph{Analyses}

Using \ourtechnique, we transformed \oneobjHT, a context-tunneled
1-object-sensitive analysis developed by \cite{JeJeOh18}, into our
1-call-site-sensitive analysis, denoted \ours.  We chose
\oneobjHT~as baseline because it is one of the best object-sensitive analyses
available today, which boosts the conventional 1-object-sensitive
analysis using a well-tuned context-tunneling policy. For
example, \oneobjHT~is empirically more precise than conventional
2-object-sensitive analysis (\twoobjH), which is considered to be highly
precise~\cite{JeJeChOh17,Li2018a,Li2018b,Smaragdakis2014,Graphick20}, yet more
scalable than 1-object sensitivity~\cite{JeJeOh18}.  We obtained the
tunneling policy of {\oneobjHT} from the publicly available artifact
of~\cite{JeJeOh18}.
From {\oneobjHT}, we first applied our simulation
  technique (Section~\ref{sec:simulation}) to produce
  the corresponding simulated call-site sensitivity, denoted \oursim. Then, we
  used our learning algorithm (Section~\ref{sec:learning}) to obtain the final 
   call-site-sensitive analysis, \ours. 
%\[
%\text{\oneobjHT}\xrightarrow[]{\text{simulation}} \text{\oursim} \xrightarrow[]{\text{learning}}\text{\ours}.
%\]
Note that {\oursim} runs {\oneobjHT} as a pre-analysis but {\ours} does not (thanks to learning). 

Our main objective is to compare \oneobjHT~and
\ours, but we compare with some notable analyses as well to
see the advance more clearly. In summary, we compare the following
 analyses:
\begin{itemize}[leftmargin=1.3em]
\item \oneobjHT: a state-of-the-art context-tunneled 1-object-sensitive analysis~\cite{JeJeOh18}
\item \oursim: the simulated 1-call-site-sensitive analysis
    obtained from~\oneobjHT~via 
    simulation% ($\simheuristic$)
\item \ours: our final 1-call-site sensitive analysis
  (obtained from \oursim~via learning)
\item \twoobjH: 2-object-sensitive analysis without
  tunneling~\cite{Smaragdakis2011}
%\item \twocallH: 2-call-site-sensitive analysis without
%  tunneling~\cite{Smaragdakis2011}
\item \onecallHT: the existing state-of-the-art 1-call-site sensitivity with
  tunneling~\cite{JeJeOh18}
\end{itemize}
\twoobjH~is available in Doop. 
\onecallHT~is available in the artifact provided by \cite{JeJeOh18}. 
%\onecallHT~is the existing
%state-of-the-art 1-call-site-sensitive analysis with context
%tunneling, which is available in the artifact of \cite{JeJeOh18}. 
%\textcolor{red}{
%\onecallHT~also boosts the conventional 1-call-site-sensitive analysis. 
%Empirically, \onecallHT~is both more precise and scalable than the conventional 2-call-site-sensitive analysis~\cite{JeJeOh18}.}
All analyses use 1-context-sensitive heap.
For precision metric, we mainly use may-fail
casts. % (\failcasts). % because % it is used in our learning
% algorithm (Section~\ref{sec:learning}) and
%the existing analyses 
%(\oneobjHT~and \onecallHT) have been tuned for it~\cite{JeJeOh18}.
%\begin{comment}
%(\oneobjHT~and \onecallHT) have been tuned with the metric~\cite{JeJeOh18}.
%\end{comment}
%For scalability, we report analysis time in seconds.

\begin{table}
\setlength\extrarowheight{-1pt}

\caption{Precision and cost comparison of our analysis (\ours) against various
  context-sensitive analyses: \oneobjHT, \twoobjH, \onecallHT, and \oursim.
}
\label{tbl:eval:main}
\centering
\scriptsize

\begin{tabular}{@{}c | clrrrrr@{}}
\toprule
                                    & program                     & \multicolumn{1}{c}{Metric} & \multicolumn{1}{c}{\ours} & \multicolumn{1}{c}{\oursim} & \multicolumn{1}{c}{\oneobjHT} & \multicolumn{1}{c}{\twoobjH} & \multicolumn{1}{c}{\onecallHT} \\ \midrule
\multirow{12}{*}{\rotatebox[origin=c]{90}{Training programs}} & \multirow{3}{*}{luindex}    & VarPtsTo                   & 250,012                       & 245,470                      & 256,531                     & 255,545                   & 800,715                      \\
                                    &                             & \failcasts             & 357                           & 360                          & 462                         & 496                       & 784                          \\
                                    &                             & time elapsed(s)            & 40                            & 86                           & 37                          & 40                        & 82                           \\\cmidrule(){2-8}
                                    & \multirow{3}{*}{lusearch}   & VarPtsTo                   & 264,728                       & 260,204                      & 271,765                     & 270,710                   & 890,529                      \\
                                    &                             & \failcasts             & 371                           & 374                          & 469                         & 508                       & 843                          \\
                                    &                             & time elapsed(s)            & 45                            & 94                           & 39                          & 82                        & 85                           \\\cmidrule(){2-8}
                                    & \multirow{3}{*}{antlr}      & VarPtsTo                   & 302,226                       & 297,268                      & 309,671                     & 308,643                   & 965,445                      \\
                                    &                             & \failcasts             & 477                           & 477                          & 570                         & 611                       & 945                          \\
                                    &                             & time elapsed(s)            & 62                            & 123                          & 52                          & 52                        & 128                          \\\cmidrule(){2-8}
                                    & \multirow{3}{*}{pmd}        & VarPtsTo                   & 306,462                       & 300,391                      & 329,415                     & 327,295                   & 1,116,506                    \\
                                    &                             & \failcasts             & 707                           & 711                          & 812                         & 846                       & 1,200                        \\
                                    &                             & time elapsed(s)            & 65                            & 128                          & 56                          & 138                       & 138                          \\\midrule\midrule
\multirow{24}{*}{\rotatebox[origin=c]{90}{Testing programs}}  & \multirow{3}{*}{eclipse}    & VarPtsTo                   & 353,657                       & 337,496                      & 351,898                     & 345,806                   & 1,241,995                    \\
                                    &                             & \failcasts             & 569                           & 573                          & 698                         & 729                       & 1,073                        \\
                                    &                             & time elapsed(s)            & 48                            & 159                          & 47                          & 58                        & 136                          \\\cmidrule(){2-8}
                                    & \multirow{3}{*}{xalan}      & VarPtsTo                   & 410,440                       & 394,522                      & 401,556                     & 400,872                   & 1,660,901                    \\
                                    &                             & \failcasts             & 576                           & 586                          & 680                         &                           720& 1,137                        \\
                                    &                             & time elapsed(s)            & 71                            & 590                          & 377                         & 2,288                     & 208                          \\\cmidrule(){2-8}
                                    & \multirow{3}{*}{chart}      & VarPtsTo                   & 501,615                       & 496,676                      & 502,913                     & 500,357                   & 4,694,330                    \\
                                    &                             & \failcasts             & 883                           & 942                          & 1,011                       & 1,055                     & 2,376                        \\
                                    &                             & time elapsed(s)            & 96                            & 575                          & 84                          & 382                       & 805                          \\\cmidrule(){2-8}
                                    & \multirow{3}{*}{fop}        & VarPtsTo                   & 650,218                       & 637,213                      & 726,777                     & 720,031                   & 3,467,105                    \\
                                    &                             & \failcasts             & 1,072                         & 1,072                        & 1,253                       & 1,270                     & 1,977                        \\
                                    &                             & time elapsed(s)            & 206                           & 407                        & 137                         & 493                       & 500                          \\\cmidrule(){2-8}
                                    & \multirow{3}{*}{bloat}      & VarPtsTo                   & 1,136,393                     & 1,136,366                    & 1,126,688                   & 1,114,648                 & 3,454,301                    \\
                                    &                             & \failcasts             & 1,266                         & 1,285                        & 1,374                       & 1,407                     & 1,949                        \\
                                    &                             & time elapsed(s)            & 498                           & 3,306                        & 371                         & 2,463                     & 805                          \\\cmidrule(){2-8}
                                    & \multirow{3}{*}{jython}     & VarPtsTo                   & 1,067,711                     & N/A                          &-                             & -                          & 3,085,401                    \\
                                    &                             & \failcasts             & 845                           & N/A                          &  -                           &   -                        & 1,331                        \\
                                    &                             & time elapsed(s)            & 2,731                         & N/A                          & >10,800         & >10,800       & 188                          \\\cmidrule(){2-8}
                                    & \multirow{3}{*}{jpc}        & VarPtsTo                   & 1,304,810                     & 1,118,622                    & 1,142,496                   & 1,114,946                 & 6,667,910                    \\
                                    &                             & \failcasts             & 1,639                         & 1,642                        & 1,795                       & 1,814                     & 2,620                        \\
                                    &                             & time elapsed(s)            & 493                           & 699                          & 262                         & 1,737                     & 1,511                        \\\cmidrule(){2-8}
                                    & \multirow{3}{*}{checkstyle} & VarPtsTo                   & 307,378                       & 299,101                      & 327,629                     & 314,857                   & 1,141,902                    \\
                                    &                             & \failcasts             & 465                           & 472                          & 591                         & 620                       & 913                          \\
                                    &                             & time elapsed(s)            & 83                            & 220                          & 99                          & 220                       & 139                          \\  \bottomrule
\end{tabular}
\end{table}

\begin{table}
\setlength\extrarowheight{-1pt}
\caption{Precision of call-graph related clients (\callgraphedges, \reachableMethods, \polycalls) of the analyses. Again, lower is better for all metrics. 
}
\label{tbl:eval:graph}
\centering
\scriptsize
\begin{tabular}{@{}c | clrrrrr@{}}
\toprule
                                    & program                     & \multicolumn{1}{c}{Metric} & \multicolumn{1}{c}{\ours} & \multicolumn{1}{c}{\oursim} & \multicolumn{1}{c}{\oneobjHT} & \multicolumn{1}{c}{\twoobjH} & \multicolumn{1}{c}{\onecallHT} \\ \midrule
\multirow{12}{*}{\rotatebox[origin=c]{90}{Training programs}} & \multirow{3}{*}{luindex}    & \callgraphedges                   &    36,578                    & 36,426                      & 36,504                     &  36,487                  &  40,830                     \\
                                    &                             & \reachableMethods            &  7,710                          & 7,699                          & 7,702                         &  7,702                      &  7,879                         \\
                                    &                             & \polycalls            & 908                            &   900                         & 905                          &   903                      & 1,066                           \\\cmidrule(){2-8}
                                    & \multirow{3}{*}{lusearch}   & \callgraphedges                   & 39,456                       & 39,304                      & 39,381                     &  39,362                  &   44,007                    \\
                                    &                             & \reachableMethods            &  8,354                          & 8,343                          & 8,344                         &  8,344                      & 8,551                          \\
                                    &                             & \polycalls            & 1,086                            & 1,078                           & 1,078                          &  1,075                       & 1,243                           \\\cmidrule(){2-8}
                                    & \multirow{3}{*}{antlr}      & \callgraphedges                   &  55,467                      &  55,396                     &  55,474                    & 55,455                   &  59,818                     \\
                                    &                             & \reachableMethods            & 8,721                           & 8,711                          &  8,714                        & 8,714                       & 8,885                          \\
                                    &                             & \polycalls            &  1,722                           & 1,709                          & 1,709                          &  1,716                       & 1,876                          \\\cmidrule(){2-8}
                                    & \multirow{3}{*}{pmd}        & \callgraphedges                   & 42,980                       & 42,909                      & 43,015                     & 42,998                   &  47,889                   \\
                                    &                             & \reachableMethods            & 9,095                           &   9,085                        &  9,090                        &  9,090                      & 9,296                        \\
                                    &                             & \polycalls            &  951                           &   943                        &  947                         &  946                      & 1,117                          \\\midrule\midrule
\multirow{24}{*}{\rotatebox[origin=c]{90}{Testing programs}}  & \multirow{3}{*}{eclipse}    & \callgraphedges                   & 44,947                       &  44,842                     &  44,926                    &  44,824                  &     51,724                \\
                                    &                             & \reachableMethods            &  9,204                          &   9,194                        &    9,197                      &   9,188                     &  9,444                       \\
                                    &                             & \polycalls            & 1,184                            &    1,175                       &  1,181                         &  1,179                       & 1,399                          \\\cmidrule(){2-8}
                                    & \multirow{3}{*}{xalan}      & \callgraphedges                   & 50,061                       &   49,985                    &    50,065                  &  50,051                  & 55,644                    \\
                                    &                             & \reachableMethods            & 10,338                           &  10,331                         &  10,336                        &     10,336                      & 10,539                        \\
                                    &                             & \polycalls            & 1,637                            &    1,630                       &  1,633                        &  1,628                    &  1,858                         \\\cmidrule(){2-8}
                                    & \multirow{3}{*}{chart}      & \callgraphedges                   &  58,933                      &   58,912                    &  58,993                    &  59,035                  &   80,500                  \\
                                    &                             & \reachableMethods            & 12,500                           &  12,495                         & 12,510                       &   12,510                   &  16,020                       \\
                                    &                             & \polycalls            &  1,609                           & 1,605                          &  1,616                         &  1,614                      &  2,698                         \\\cmidrule(){2-8}
                                    & \multirow{3}{*}{fop}        & \callgraphedges                   & 59,663                       &  59,440                     &   61,975                   &   61,923                 &   71,741                  \\
                                    &                             & \reachableMethods            &  13,777                        &  13,763                       &      14,376                  &  14,373                    &   15,108                      \\
                                    &                             & \polycalls            &  1,962                          &   2,063                      &         2,063                 &    2,047                    &   2,522                        \\\cmidrule(){2-8}
                                    & \multirow{3}{*}{bloat}      & \callgraphedges                   & 61,249                     &  60,990                   & 60,638                   &  60,601                &  68,674                   \\
                                    &                             & \reachableMethods            &  9,947                        &  9,928                       & 9,914                       &       9,914               &        10,113                 \\
                                    &                             & \polycalls            & 1,679                           &   1,667                      &  1,652                        &   1,650                   &        1,925                   \\\cmidrule(){2-8}
                                    & \multirow{3}{*}{jython}     & \callgraphedges                   & 52,644                     & N/A                          &N/A                             & N/A                          &  59,932                   \\
                                    &                             & \reachableMethods            & 10,625                           & N/A                          &  N/A                           &   N/A                        & 10,987                        \\
                                    &                             & \polycalls            &  14,084                        & N/A                          & N/A         & N/A       &  1,565                         \\\cmidrule(){2-8}
                                    & \multirow{3}{*}{jpc}        & \callgraphedges                   & 95,837                     &  95,098                   &    95,371                &  95,209                &   110,493                  \\
                                    &                             & \reachableMethods            & 18,634                         &  18,581                       & 18,655                       &  18,631                    &  19,854                       \\
                                    &                             & \polycalls            &  5,053                          &  4,989                         &       4,999                   &   4,963                   &   5,646                      \\\cmidrule(){2-8}
                                    & \multirow{3}{*}{checkstyle} & \callgraphedges                   & 42,410                       &   42,333                    &  42,204                    & 42,174                   &  49,346                   \\
                                    &                             & \reachableMethods            & 8,435                           & 8,424                          &  8,428                        & 8,428                       & 8,672                          \\
                                    &                             & \polycalls            &  1,096                           &  1,088                         &  1,090                         & 1,088                       &  1,304                         \\  \bottomrule
\end{tabular}
\end{table}







\subsection{Performance of \ours}\label{sec:performance}


Table~\ref{tbl:eval:main} %compares the precision and cost of the four
%analyses, which 
shows that our analysis (\ours) significantly outperforms
other analyses in both precision and cost, confirming our claim that call-site sensitivity can be superior to object sensitivity for real-world programs. 
In particular, it beats by far the baseline object sensitivity
(\oneobjHT) in precision for all programs.  For example,
\oneobjHT~reports 1,253 may-fail casts for fop but \ours~reduces the
number to 1,072. 
Also, \ours~is more scalable than \oneobjHT. For example, \ours~takes 2,731 seconds to analyze jython while \oneobjHT~times out. 



%the precision gain comes not only from the shared library but application code.
%For example, \ours~finishes the analysis of jython in
%2,731s while \oneobjHT~fails to complete.




We note that  our 1-call-site-sensitive analysis is even more precise
than the traditional 3-object-sensitive analysis with
2-context-sensitive heap (\threeobjH):

{
\setstretch{1.0}
\begin{center}
  %\small
  \begin{tabular}{ c r r r r }
    \toprule
    % after \\: \hline or \cline{col1-col2} \cline{col3-col4} ...
    \multirow{2}{*}{Program} & \multicolumn{2}{c }{\ours} & \multicolumn{2}{c }{\threeobjH} \\
    \cmidrule(lr){2-3}\cmidrule(lr){4-5}
                             & \#fail-casts & Time(s) & \#fail-cast & Time(s) \\
    \midrule
    luindex     & 357 & 40 & 435 & 564 \\
    antlr       & 477 & 62 & 543 & 561 \\
    pmd         & 707 & 65 & 782 & 584 \\
    \bottomrule
  \end{tabular}
\end{center}
}

\noindent
where we compare the results only for the three small 
programs because \threeobjH~does not scale for other programs.  Note
that \threeobjH~is the most precise object-sensitive analysis evaluated in
the literature~\cite{Lu:2019:PYF,Tan2017} and \ours~substantially
%the literature~\cite{Tan2017,Lu:2019:PYF} and \ours~substantially
improves its precision with much smaller costs.


The performance of \ours~is completely beyond the reach of
existing call-site-sensitive analyses. 
\onecallHT~is the state-of-the-art call-site sensitivity,
%\begin{comment}
which is more precise and faster than ordinary 2-call-site-sensitive analysis~\cite{JeJeOh18}.
%\end{comment}
However, \ours~reduced about 50\% of may-fail casts of \onecallHT~for all programs.

%\textcolor{red}{
Table~\ref{tbl:eval:graph} compares the precision of the analyses for three other call-graph construction related clients  used in previous works~\cite{Li2018a,Li2018b,Tan2017}. \callgraphedges~presents the number of call-graph edges without contexts, \reachableMethods~presents the number of reachable methods, and \polycalls~presents the number of call-sites that cannot be determined as monomorphic calls. The results show that our simulated call-site sensitivity \oursim~overall shows better precision than the baseline object sensitivity \onecallHT. 
%Except for two programs bloat and checkstyle, \oursim~is equal or more precise than \oneobjHT~for all the three metrics. In checkstyle, \oursim~is more precise than \oneobjHT~for two metrics. Our learned one~\ours, however, shows a competitive performance compared to \oneobjHT~(e.g., chart, fop).}
This difference between \oursim~and \ours~ comes from the learning objective. \ours~was not trained to optimize these metrics from~\oursim~(the current implementation of our algorithm uses \failcasts~as the learning objective).
%; it produces competitive results w.r.t. the metrics in Table~\ref{tbl:eval:graph}. 
The learning objective, however, can be easily adapted for other clients. 
%If we use the call-graph-related clients as objective, the learned call-site sensitivity would become more precise than  \oneobjHT~for the clients as the simulated call-site sensitivity \oursim~does.
%}
%\callgraphedges, \reachableMethods, \polycalls





%\textcolor{red}{
%In our evaluation, \ours~consistently produces fewer \#may-fail cast than \oneobjHT. It implies that \ours~ consistently analyzed target client-relevant variables more precisely than \oneobjHT. \ours~sometimes produces higher \#VarPtsTo because it analyzed client-irrelevant variables coarsely than \oneobjHT.
%}

%%% Local Variables:
%%% mode: latex
%%% TeX-master: "paper"
%%% End:

% !TEX root = ./paper.tex


\subsection{Impact of Simulation and Learning}\label{sec:eval_learning}


Next, we discuss the impact of our technical contributions, simulation and learning. 

\myparagraph{Simulation Accuracy}
We first note that our technique enabled call-site
sensitivity to accurately simulate object sensitivity for real-world
applications.
For all benchmark programs except for jython, we ran the
baseline object-sensitive analysis (\oneobjHT) and the corresponding
call-site sensitive analysis (\ours) and counted the number ($A$) of may-fail cast
queries that both analyses can prove and the number ($B$) of queries
that \oneobjHT~can prove. The ratio $A/B$ hints at how accurately
\ours~covers the baseline object sensitivity. The
average ratio over the 11 programs was $0.98$, implying that \ourtechnique~can simulate object sensitivity almost completely.

%\textcolor{red}{
Most of the remaining 2\% of queries, which \ourtechnique~missed, were caused by imperfect simulation rather than learning; 
when we calculated the ratio using simulation only (i.e., \oursim), 
the ratio $A'/B$ was still $0.98$, where $A'$ denotes the number of may-fail casts that both \oursim~and \oneobjHT~can prove. 
%, the may-fail casts proved by \oursim~($A'$)
%still cover 98\% of the queries proved by \oneobjHT~(e.g., $A'/B = 0.98$).
We found that this was because we used a coarse tunneling space, i.e., invocation sites, and that 
using a more fine-grained tunneling space (e.g., pairs of invocation-sites and receiver objects) would improve the simulation accuracy. 
We describe an example in Section~\ref{sec:counter_example}. 
%We also found the (coarse) tunneling space we used makes the simulation imperfect.
%In this paper, we used invocation sites as tunneling space (e.g., $T \subseteq \Invo$); a tunneling abstraction determines whether to apply context tunneling with invocation sites.
%Using a more fine-grained space (e.g., pair of invocation-sites and receiver objects), however, would improve the simulation accuracy and precision. 
%An example is described in Section~\ref{sec:counter_example}.
%}


%Without simulation, our learning algorithm was unable to find good-enough tunneling policies for call-site sensitivity. 

%
%
%The learning phase of \ourtechnique~spent 58 hours in total, and it effectively captured the behavior of the simulated policy. For all programs, the simulated analysis with $\simheuristic$ was more precise but expensive than the baseline (\oneobjHT).  For eclipse, for example, it reported 573 may-fail casts and took 172s while \oneobjHT~reported 698 may-fail casts and took 43s. With the learned policy, \ours~took 48s with 569 may-fail casts. 
%%We also found our learning algorithm in Section~\ref{sec:learning} is essential; using existing  algorithms~\cite{JeJeOh18,Pedregosa11} ends up with suboptimal results. 
%%We provide detailed results in Section C of the supplementary material.

\myparagraph{Roles of Simulation and Learning}
%This accurate simulation played a key role in improving the precision of call-site sensitivity. 
The results in Table~\ref{tbl:eval:main} show that our simulation technique plays a key role in improving the precision of call-site sensitivity. % and learning reduces the overhead of the simulation. 
%  \ours~mainly comes from our simulation technique ($\simheuristic$).
The column~\oursim~in Table~\ref{tbl:eval:main} presents the
performance of the call-site sensitive analysis obtained by simulating \oneobjHT. For all programs,
\oursim~shows a far better precision than \oneobjHT. 
%Minseok
%\textcolor{red}{For example in the program chart,~\oursim~produces 127 fewer alarms than \oneobjHT.}

%\myparagraph{Impact of Learning}
Comparing the performance of \oursim~and \ours~reveals that 
the use of learning reduced the overhead of \oursim~significantly.  
As the simulation needs to run the baseline object sensitivity (\oneobjHT), the simulated call-site-sensitive analysis (\oursim) is inherently more expensive than the baseline object sensitivity (\oneobjHT). For example, \oursim~is unable to analyze the program jython because the baseline object sensitivity
  (\oneobjHT) failed to analyze it. Thanks to our learning technique, however, our call-site sensitivity (\ours) removed the limitation. For example, \ours~successfully analyzed jython within the time budget.


We checked that the result of learning is not overfitted to the shared library. 
%; we observed reasonable precision gain comes from application code. 
%\textcolor{red}{
In eclipse, for example, \ours~proved 17\% (resp., 15\%) more may-fail casts compared to \oneobjHT~in the application (resp., library) code, which implies that learning is equally effective in both application and library code. 
We also checked \ours~is not overfitted to DaCapo benchmarks. In a non-DaCapo benchmark checkstyle, for example, \ours~proved 41\% (resp., 19\%) more may-fail casts compared to \oneobjHT~in the application (resp., library) code.
%}

%\begin{center}
%\begin{tabular}{@{}lrrrrrrr@{}}
%\toprule
%\multicolumn{1}{c}{} & \multicolumn{1}{c}{eclipse} & \multicolumn{1}{c}{xalan} & \multicolumn{1}{c}{chart} & \multicolumn{1}{c}{fop} & \multicolumn{1}{c}{bloat} & \multicolumn{1}{c}{jpc} & \multicolumn{1}{c}{checkstyle} \\ \midrule
%\#total proven-cast  & 1.17                        & 1.13                      & 1.18                      & 1.14                    & 1.12                      & 1.10                    & 1.19                           \\
%\#app proven-cast    & 1.15                        & 0.91                      & 0.99                      & 0.99                    & 1.03                      & 1.02                    & 1.41                           \\ \bottomrule
%\end{tabular}
%\end{center}


%In eclipse, \ours~produces 22 fewer alarms than \oneobjHT~in application code; about 17\% of the precision gain, a similar portion (18\%) of application code, comes from application code.


\myparagraph{Impact of Simulation-Guided Learning}
Our simulation-guided learning (Section~\ref{sec:learning}) was essential for effectively capturing the precision of the simulated policy. 
%simulated policy ($\simheuristic$ in Section~\ref{sec:simulation})
%; existing or simpler learning algorithms were not powerful enough 
%to capture its behavior. 
To demonstrate this, we conducted two
experiments. 
%Table~\ref{fig:sim-learning-needed} shows the effectiveness of our learning algorithm by comparing the precision of call-site-sensitive analyses with the policy learned from our algorithm. 
% (e.g., \OurLearn) and those produced from existing or simpler learning algorithms~(e.g.,~\Existing~\cite{JeJeOh18} and  \DecisionTree~\cite{Pedregosa11}):
We first replaced our algorithm by the existing
unguided algorithm for learning context tunneling~\cite{JeJeOh18} and trained a policy using the same set of
atomic features in Table~\ref{tbl:features} and training programs.  
Table~\ref{tbl:impact-of-learning} shows that the existing algorithm ended up with a much less precise policy.
Over the four training programs, the context-tunneled call-site sensitivity obtained using the existing learning algorithm~\cite{JeJeOh18}, denoted \Existing~in Table~\ref{tbl:impact-of-learning}, reported 2,809 may-fail casts while the
number for \OurLearn~is much smaller (1,912). This result shows that the use of simulation in our approach 
is critical.
%%Minseok
%\textcolor{red}{; the learning phase alone does not produce the desired results.}
 
Second, we replaced our algorithm by a simple supervised learning
method. We generated labeled data, which consists of feature vectors
of invocation sites and labels indicating whether selected by the
simulated policy ($\simheuristic$) or not. We used the decision tree
algorithm in~\cite{Pedregosa11} to learn a policy. 
Again, the resulting
analysis (denoted \DecisionTree~in Table~\ref{tbl:impact-of-learning}) was unsatisfactory in precision; over the training programs, it
reported 2,604 may-fail casts.  This is mainly because the labeled
data does not have enough information; although the simulated policy
labels which invocations need context tunneling, it is unable to label
which invocations are precision critical.


\begin{table}[t]
\caption{Impact of our simulation-guided learning (numbers indicate {\failcasts})}
\vspace{-5pt}
\label{tbl:impact-of-learning}
\begin{center}
\begin{tabular}{@{}c r r r r r@{}}
\toprule
%              & \multicolumn{5}{c}{\#fail-cast}                                                                                                            \\ \cmidrule{2-6}
              & \multicolumn{1}{c}{luindex} & \multicolumn{1}{c}{lusearch} & \multicolumn{1}{c}{antlr} & \multicolumn{1}{c}{pmd} & \multicolumn{1}{c}{Total} \\\midrule
\ours          & 357                         & 371                          & 477                       & 707                     & 1,912                    \\
\Existing      & 565                         & 580                          & 735                       & 929                     & 2,809                   \\ 
\DecisionTree & 519                         & 533                          & 659                       & 895                     & 2,606                    \\
%\onecallHT      & 784                         & 843                          & 945                       & 1,200                   & 3,772                   \\ 
\bottomrule
\end{tabular}
\end{center}
\vspace{-10pt}
\end{table}


\myparagraph{Impact of New Features}
Another important factor for effective learning was the use of the features in Table~\ref{tbl:features} that are specifically designed for call-site sensitivity (recall that we crafted those features guided by the simulated policy $\simheuristic$). 
For example, the \Existing~analysis described above, which differs from \onecallHT~\cite{JeJeOh18} only in the use of the new set of features, is much more precise than \onecallHT: \Existing~produces 970 fewer alarms than \onecallHT~that is learned using the same algorithm but with the different features designed by~\cite{JeJeOh18}. 

%The features in Table~\ref{tbl:features} are qualified features for learning call-site sensitivity tunneling heuristics. \Existing~in the above, learned with the existing algorithm~\cite{JeJeOh18} and the new features in Table~\ref{tbl:features}, is a lot more precise than the state-of-the-art call-site sensitivity \onecallHT~learned with the same algorithm but the old features~\cite{JeJeOh18}. \Existing~produces 970 fewer alarms than \onecallHT~over the training programs. Although they used the same learning algorithm, \Existing~shows a far better precision thanks to our newly designed qualified features.


We note that our features in Table~\ref{tbl:features} are not appropriate for learning tunneling heuristics for object sensitivity. %In any application of machine learning, the quality of features has a great impact on learning. 
This is because our features are designed to reproduce the results of the simulated call-site sensitivity ($\simheuristic$) rather than object sensitivity. 
To clarify the impact of using a different set of features, we used our features and the existing algorithm~\cite{JeJeOh18} to learn a tunneling heuristic for object sensitivity.
The resulting analysis, denoted \oneobjHTnew, was overall less precise than \oneobjHT. 
%%Minseok
%\textcolor{red}{For example, \oneobjHTnew~reported 630 may-fail casts while \oneobjHT~reported 462 for program luindex. }

% \textcolor{red}{Note that \oneobjHTnew~is also a good 1-object sensitivity analysis compared to the conventional 1-object sensitive analysis that reported 796 alarms for the program luindex.} 

%he learned object-sensitivity is less precise than the existing state-of-the-art object sensitivity \oneobjHT.
%The algorithm failed to learn a better tunneling heuristic for object sensitivity because the features in Table~\ref{tbl:features} are tuned with call-site sensitivity in mind. 
%Note that the existing algorithm successfully learned a far more precise call-site sensitivity (\Existing) than the \onecallHT~with the newly designed features.
%The features are carefully crafted for call-site sensitivity (e.g., we crafted features with~\oursim) instead of object sensitivity. 


%For example, the key feature $C1$, a key feature for call-site sensitivity, we describe above is a bad feature for object sensitivity.
%Unlike call-site-sensitivity that applying context tunneling to $C1$ (e.g., $T_{\tt C1}$) make 1-call-site sensitivity more precise than the conventional 2-call-site sensitivity, 
%applying the tunneling abstraction $T_{\tt C1}$ make 1-object-sensitivity even less precise than the conventional 1-object-sensitivity. 

\myparagraph{Learning Cost}
The learning phase of \ourtechnique~spent 58 hours in total, which is slightly faster than the prior algorithm~\cite{JeJeOh18}. 
Although our algorithm is expensive, we believe it is acceptable because learning is done off-line and saves otherwise more expensive human effort. 
%
%\subsection{Observations}\label{sec:eval:learnedInsight}
%\myparagraph{A key feautre for learning call-site sensitivity tunneling heuristics}
%\textcolor{red}{
%We investigated the learned formula $\heuristic_f$, used in \ours\footnote{The detail formula is presented in Appendix~\ref{appendix:learnedFormula}}~and~\Existing, and found that the feature $C1$ (method invocations uses \texttt{this} variable (i.e. \texttt{this.$m$(...)})) in Table~\ref{tbl:features} plays an important role for improving precision of call-site sensitivity.
%For example, applying context tunneling to the invocations belong to $C1$
%(e.g., $T_{f=C1}$) is a simple way to improve the precision of call-site sensitivity.
%For the four training programs, 1-call-site sensitive analysis with the tunneling abstraction $T_{C1}$ (apply context tunneling when the base variable is {\tt this}) is even more precise than the conventional 2-call-site sensitive analysis (\twocallH); \onecallThis~produces 114 fewer alarms than \twocallH~over the training programs.
%}
%
%
%
%
%\textcolor{red}{
%{\tt this} variables is directly related to $I_1$ (caller and callee method shares the same context in object sensitivity) in our simulation technique $\simheuristic$~where $I_1$ need to be tunneled for call-site sensitivity to simulate object sensitivity.
%Although the feature $C1$ is a syntactic feature, it presents a semantic property that the caller and callee methods share the same context in object sensitivity.
%In java programs, caller and callee methods share the same receiver object if the invocation sites use {\tt this} variable;
%caller and callee methods inevitably share the same context (receiver object) in conventional object sensitivity.
%As a result, all the invocation sites using {\tt this} variable always belongs to $I_1$ and need to be tunneled to make call-site sensitivity simulate object sensitivity. 
%}
%
%
%\myparagraph{Sensitivity to Features}
%\textcolor{red}{
%The features in Table~\ref{tbl:features} are qualified features for learning call-site sensitivity tunneling heuristics. \Existing~in the above, learned with the existing algorithm~\cite{JeJeOh18} and the new features in Table~\ref{tbl:features}, is a lot more precise than the state-of-the-art call-site sensitivity \onecallHT~learned with the same algorithm but the old features~\cite{JeJeOh18}. \Existing~produces 970 fewer alarms than \onecallHT~over the training programs. Although they used the same learning algorithm, \Existing~shows a far better precision thanks to our newly designed qualified features.
%}
%
%
%\textcolor{red}{
%The features in Table~\ref{tbl:features}, however, are inadequate for learning tunneling heuristics for object sensitivity. %In any application of machine learning, the quality of features has a great impact on learning. 
%To clarify the impact of using a different set of features, we gave the newly designed features and used the existing algorithm~\cite{JeJeOh18} for learning a tunneling heuristic for object sensitivity.
%The learning, however, ended up with a suboptimal one; the learned object-sensitivity is less precise than the existing state-of-the-art object sensitivity \oneobjHT.
%The algorithm failed to learn a better tunneling heuristic for object sensitivity mainly because the features in Table~\ref{tbl:features} are highly fitted to learn tunneling heuristics for call-site sensitivity. 
%%Note that the existing algorithm successfully learned a far more precise call-site sensitivity (\Existing) than the \onecallHT~with the newly designed features.
%The features are carefully crafted for call-site sensitivity (e.g., we crafted features with~\oursim) instead of object sensitivity. 
%For example, the key feature $C1$, a key feature for call-site sensitivity, we describe above is a bad feature for object sensitivity.
%Unlike call-site-sensitivity that applying context tunneling to $C1$ (e.g., $T_{\tt C1}$) make 1-call-site sensitivity more precise than the conventional 2-call-site sensitivity, 
%applying the tunneling abstraction $T_{\tt C1}$ make 1-object-sensitivity even less precise than the conventional 1-object-sensitivity. 
%}
%
%
%\myparagraph{Unexchangeable of learned tunneling heuristics}
%\textcolor{red}{
% We checked tunneling heuristics of each context are not exchangeable.
%The tunneling heuristic used in \ours~is inadequate to object sensitivity;
%1-object sensitivity with the tunneling heuristic is even less precise than the one without tunneling (conventional 1-object sensitive analysis).
%The tunneling heuristic used in \oneobjHT~is also a bad tunneling heuristic for call-site sensitivity.}



%\input{table_zipper}
% !TEX root = ./paper.tex

%\begin{table}[]
\begin{table}[]
\setlength\extrarowheight{-1pt}
\caption{Performance comparison of \ours, \twoobjZipper~(i.e., \Zipper~\cite{Li2018a}), and \oursZipper. 
%We also evaluate the baseline state-of-the-art 1-object sensitivity with tunneling~\oneobjHT. 
%Unlike tunneling heuristics used in \ours~or \oneobjHT~that determine whether to update contexts, the heuristic Zipper determine whether to apply 2-object sensitivity or context insensitivity for each method call.
}
\label{tbl:eval:vsZipper}
\centering\scriptsize
\begin{tabular}{@{}clrr | clrr@{}}
\toprule
Program                     & \multicolumn{1}{c}{Analysis} & \multicolumn{1}{c}{\#fail-casts} & \multicolumn{1}{c |}{Time} & Program                   & \multicolumn{1}{c}{Analysis} & \multicolumn{1}{c}{\#fail-casts} & \multicolumn{1}{c}{Time} \\ \midrule
\multirow{3}{*}{eclipse}    & \ours                    & 460                              & 62                           & \multirow{3}{*}{fop}      & \ours                    & 1,359                             & 368                          \\
                            & \oursZipper                      & 482                              & 42                           &                           & \oursZipper                      & 1,397                             & 279                          \\
                            & \twoobjZipper                  & 586                              & 45                          &                           & \twoobjZipper                  & 1,471                             & 318                          \\\midrule
\multirow{3}{*}{bloat}      & \ours                    & 1,136                             & 456                          & \multirow{3}{*}{jpc}      & \ours                    & 1,279                             & 201                          \\
                            & \oursZipper                      & 1,168                             & 397                          &                           & \oursZipper                      & 1,368                             & 176                          \\
                            & \twoobjZipper                  & 1,224                             & 1,942                         &                           & \twoobjZipper                  & 1,415                             & 147                          \\\midrule
\multirow{3}{*}{chart}      & \ours                    & 810                              & 80                           & \multirow{3}{*}{xalan}    & \ours                    & 463                              & 91                           \\
                            & \oursZipper                      & 890                              & 58                           &                           & \oursZipper                      & 482                              & 72                          \\
                            & \twoobjZipper                  & 910                              & 54                           &                           & \twoobjZipper                  & 568                              & 66                          \\\midrule
\multirow{3}{*}{sunflow}    & \ours                    & 1,787                             & 438                          & \multirow{3}{*}{findbugs} & \ours                    & 1,316                             & 288                          \\
                            & \oursZipper                      & 1,818                             & 279                          &                           & \oursZipper                      & 1,360                             & 164                          \\
                            & \twoobjZipper                  & 1,869                             & 361                          &                           & \twoobjZipper                  & 1,437                             & 604                          \\\midrule
\multirow{3}{*}{checkstyle} & \ours                    & 522                             & 106                          & \multirow{3}{*}{batik}    & \ours                    & 1,602                             & 1,422                         \\
                            & \oursZipper                      & 535                              & 88                          &                           & \oursZipper                      & 1,782                             & 1,649                          \\
                            & \twoobjZipper                  & 607                              & 248                          &                           & \twoobjZipper                  & 1,614                             & 667                          \\\midrule
\multirow{3}{*}{sunflow09}  & \ours                    & 1,101                             & 126                          & \multirow{3}{*}{xalan09}  & \ours                    & 989                              & 241                          \\
                            & \oursZipper                      & 1,146                             & 84                           &                           & \oursZipper                      & 1,004                             & 166                          \\ 
                            & \twoobjZipper                  & 1,192                             &73                           &                           & \twoobjZipper                  & 1,074                             & 214                          \\\midrule
\multirow{3}{*}{avrora09}   & \ours                    & 918                              & 147                          & \multirow{3}{*}{h2} & \ours                    & 1,207                             & 3,766                          \\
                            & \oursZipper                      & 964                             & 105                          &                           & \oursZipper                      &  1,225                            & 2,318                          \\
                            & \twoobjZipper                  & 1,042                             & 115                          &                           & \twoobjZipper                  & 1,311                             & 8,216                          \\%\midrule

\bottomrule
                            
\end{tabular}
\vspace{-10pt}
\end{table}

% \begin{table}[]
% \caption{Comparison of our 1-call-site sensitivity (\ours) against Zipper~(\twoobjZipper~\cite{Li2018a}). 
% %We also evaluate the baseline state-of-the-art 1-object sensitivity with tunneling~\oneobjHT. 
% %Unlike tunneling heuristics used in \ours~or \oneobjHT~that determine whether to update contexts, the heuristic Zipper determine whether to apply 2-object sensitivity or context insensitivity for each method call.
% }
% \label{tbl:eval:vsZipper}
% \begin{tabular}{@{}clrr | clrr@{}}
% \toprule
% Program                     & \multicolumn{1}{c}{Analysis} & \multicolumn{1}{c}{\# fail-casts} & \multicolumn{1}{c |}{sec} & Program                   & \multicolumn{1}{c}{Analysis} & \multicolumn{1}{c}{\# fail-casts} & \multicolumn{1}{c}{sec} \\ \midrule
% \multirow{2}{*}{eclipse}    & \ours                    & 460                              & 62                           & \multirow{2}{*}{fop}      & \ours                    & 1,359                             & 368                          \\
%                             & \twoobjZipper                  & 586                              & 67                          &                           & \twoobjZipper                  & 1,471                             & 424                          \\\midrule
% %                            & \oneobjHT                      & 562                              & 64                           &                           & \oneobjHT                      & 1,472                             & 357                          \\\midrule
% \multirow{2}{*}{bloat}      & \ours                    & 1,136                             & 456                          & \multirow{2}{*}{jpc}      & \ours                    & 1,279                             & 201                          \\
%                             & \twoobjZipper                  & 1,224                             & 1,969                         &                           & \twoobjZipper                  & 1,415                             & 229                          \\\midrule
% %                            & \oneobjHT                      & 1,210                             & 376                          &                           & \oneobjHT                      & 1,418                             & 123                          \\\midrule
% \multirow{2}{*}{chart}      & \ours                    & 810                              & 80                           & \multirow{2}{*}{xalan}    & \ours                    & 463                              & 91                           \\
%                             & \twoobjZipper                  & 910                              & 98                           &                           & \twoobjZipper                  & 568                              & 107                          \\\midrule
% %                            & \oneobjHT                      & 892                              & 80                           &                           & \oneobjHT                      & 541                              & 247                          \\\midrule
% \multirow{2}{*}{sunflow}    & \ours                    & 1,787                             & 438                          & \multirow{2}{*}{findbugs} & \ours                    & 1,316                             & 288                          \\
%                             & \twoobjZipper                  & 1,869                             & 459                          &                           & \twoobjZipper                  & 1,437                             & 647                          \\\midrule
% %                            & \oneobjHT                      & 1,908                             & 240                          &                           & \oneobjHT                      & 1,410                             & 217                          \\\midrule
% \multirow{2}{*}{checkstyle} & \ours                    & 522                             & 106                          & \multirow{2}{*}{batik}    & \ours                    & 1,602                             & 1,422                         \\
%                             & \twoobjZipper                  & 607                              & 299                          &                           & \twoobjZipper                  & 1,614                             & 820                          \\\midrule
% %                            & \oneobjHT                      & 591                              & 199                          &                           & \oneobjHT                      & 1,632                             & 470                          \\\midrule
% \multirow{2}{*}{sunflow09}  & \ours                    & 1,101                             & 126                          & \multirow{2}{*}{xalan09}  & \ours                    & 989                              & 241                          \\
%                             & \twoobjZipper                  & 1,192                             & 124                           &                           & \twoobjZipper                  & 1,074                             & 272                          \\\midrule
% %                            & \oneobjHT                      & 1,230                             & 88                           &                           & \oneobjHT                      & 1,065                             & 362                          \\ \midrule
% \multirow{2}{*}{avrora09}   & \ours                    & 918                              & 147                          & \multirow{2}{*}{h2} & \ours                    & 1,207                             & 3,766                          \\
%                             & \twoobjZipper                  & 1,042                             & 175                          &                           & \twoobjZipper                  & -                             & >5,400                          \\%\midrule
% %                            & \oneobjHT                      & 1,037                             & 160                          &                           & \oneobjHT                      &  1,301                            & 482                          \\

% \bottomrule
% \bottomrule
                            
% \end{tabular}
% \end{table}


\subsection{Comparison with Selective Object Sensitivity}\label{sec:SelectiveCtx}

We also checked how \ours~works in comparison with \Zipper~\cite{Li2018a,ZipperJournal20}, a state-of-the-art technique that performs selective 2-object sensitivity. 
Unlike the analyses considered in Section~\ref{sec:performance}, which perform uniform $k$-context sensitivity,  \Zipper~performs a selective context-sensitive analysis and applies $2$-object sensitivity only when it is necessary. In particular, compared to other selective approaches~\cite{Li2018b,JeJeChOh17,Smaragdakis2014,Graphick20}, \Zipper~is precision-focused and 
improves the scalability of uniform 2-object sensitivity while preserving most (98.8\%) of its precision. 
%
%We compare \ours~with object-sensitivity equipped with a context sensitivity heuristic \Zipper~\cite{Li2018a}. 
%Recently, various selective context sensitivity heuristics have been proposed to develop cost-effective object-sensitive analysis~\cite{Li2018a,Li2018b,Lu:2019:PYF,JeJeChOh17,Smaragdakis2014,Graphick20}. 
%As it is usually too expensive to apply object sensitivity (e.g., 2-object sensitivity) to all method calls,
%context sensitivity heuristics apply object-sensitivity only when doing so would increase precision~\cite{Li2018a,Lu:2019:PYF}.
%Otherwise, the heuristics use alternative cheap contexts (e.g., type sensitivity or context insensitivity) to improve scalability~\cite{Smaragdakis2014,Li2018b}.
%\Zipper~is a precision-focused context sensitivity heuristic that preserves almost (98.8\%) full precision of the conventional 2-object-sensitive analysis.

%\myparagraph{Setup}
For direct comparison, we implemented \ours~and \oneobjHT~on top of the artifact provided by~\cite{Li2018a}.
We used 14 programs (batik, fop, sunflow, bloat, xalan, chart, findbugs, eclipse, jpc, checkstyle, h2, xalan09, avrora09, sunflow09) used in~\cite{Li2018a}.
Five programs (fop, xalan, bloat, chart, eclipse) are DaCapo 2006 benchmarks~\cite{Blackburn2006}, four programs (xalan09, sunflow09, avrora09, and h2) are from DaCapo 2009 benchmarks, and the remaining five programs (sunflow, findbugs, jpc, checkstle, batik) are real-world Java applications.
Note that the artifact of~\cite{Li2018a} uses a quite different reflection analysis from the one we used in Table~\ref{tbl:eval:main}.
The number of alarms and analysis time varies significantly depending on how reflection is supported. 
The baseline analysis on top of \Zipper~is implemented supports reflection less conservatively than our implementation in Section~\ref{sec:performance} and therefore Table~\ref{tbl:eval:vsZipper} reports fewer alarms than Table~\ref{tbl:eval:main}. In Table~\ref{tbl:eval:vsZipper}, our analysis uses the same reflection analysis as \Zipper. 


%\myparagraph{Results}
Table~\ref{tbl:eval:vsZipper} compares the performance of \ours~and \Zipper~(denoted \twoobjZipper~in Table~\ref{tbl:eval:vsZipper}). 
For all programs, \ours~shows a better precision than \Zipper. Especially for bloat, \ours~is more precise and scalable than  than \twoobjZipper; \twoobjZipper~took 1,942 seconds and reports 1,224 alarms but \ours~analyzed it within 456 seconds with 1,136 may-fail cast alarms. However, for batik, \ours~was slower than \twoobjZipper. 

%\oneobjHT~shows comparable precision with \twoobjZipper. It shows better precision than \twoobjZipper~for 8 programs (eclipse, bloat, chart, checkstyle, avrora09, xalan09, findbugs, xalan) while \oneobjHT~is less precise than \twoobjZipper~for the others.
%%\ours~also shows good scalability. For 8 programs (luindex, lusearch, antlr, pmd, chart, xalan, jpc, and checkstyle), \ours~is faster than the others.

% \begin{table}[]
% \label{tbl:oursZipper}
% \caption{\TODO{Why not report all programs? Include the result for batik}Performance of our context tunneled 1-call-site sensitivity with \Zipper~(\oursZipper).}
% \begin{tabular}{@{}lccccccc@{}}
% \toprule
% Program         & \multicolumn{1}{c}{h2}  & \multicolumn{1}{c}{fop}   & \multicolumn{1}{c}{batik}      & \multicolumn{1}{c}{sunflow}   & \multicolumn{1}{c}{bloat}   & \multicolumn{1}{c}{findbugs} & \multicolumn{1}{c}{chart}   \\ \midrule
% \#may-fail cast & 1,225                   & 1,397                     & 1,782                          & 1,818                         & 1,168                       & 1,360                        & 890                         \\
% Time (s)        & 2,375                   & 385                       & 1,812                          & 377                           & 424                         & 207                        & 102                          \\\midrule
% Program         & \multicolumn{1}{c}{jpc} & \multicolumn{1}{c}{xalan} & \multicolumn{1}{c}{checkstyle} & \multicolumn{1}{c}{sunflow09} & \multicolumn{1}{c}{xalan09} & \multicolumn{1}{c}{avrora09} & \multicolumn{1}{c}{eclipse} \\\midrule
% \#may-fail cast & 1,368                   & 482                       & 535                            & 1,146                              &   1,004                          &      964                        & 482                         \\
% Time (s)        & 258                    & 113                        & 135                             &   135                            &    224                         &  163                            & 86                          \\ \bottomrule
% \end{tabular}
% \end{table}
%
%\begin{table}[]
%\caption{Performance of our context tunneled 1-call-site sensitivity with \Zipper~(\oursZipper).}
%\label{tbl:oursZipper}
%\begin{tabular}{@{}lccccccc@{}}
%\toprule
%Program         & \multicolumn{1}{c}{h2}  & \multicolumn{1}{c}{fop}   & \multicolumn{1}{c}{batik}      & \multicolumn{1}{c}{sunflow}   & \multicolumn{1}{c}{bloat}   & \multicolumn{1}{c}{findbugs} & \multicolumn{1}{c}{chart}   \\ \midrule
%\failcasts & 1,225                   & 1,397                     & 1,782                          & 1,818                         & 1,168                       & 1,360                        & 890                         \\
%Time (s)        & 2,318                  & 279                      & 1,649                       & 279                          & 397                        & 164                         & 58                        \\\midrule
%Program         & \multicolumn{1}{c}{jpc} & \multicolumn{1}{c}{xalan} & \multicolumn{1}{c}{checkstyle} & \multicolumn{1}{c}{sunflow09} & \multicolumn{1}{c}{xalan09} & \multicolumn{1}{c}{avrora09} & \multicolumn{1}{c}{eclipse} \\\midrule
%\failcasts & 1,368                   & 482                       & 535                            & 1,146                              &   1,004                          &      964                        & 482                         \\
%Time (s)        & 176                     & 72                        & 88                             &   84                            &    166                         &  105                            & 42                          \\ \bottomrule
%\end{tabular}
%\end{table}
%
%

%\myparagraph{Combination with Zipper}
Because context tunneling and selective context sensitivity are orthogonal techniques, we can combine 
our analysis (\ours) and \Zipper, resulting in a selective 1-call-site-sensitive analysis with context tunneling (\oursZipper). 
This is possible because the idea of \Zipper~is general and applicable to various context-sensitivity flavors including call-site sensitivity~\cite{ZipperJournal20}. 
Table~\ref{tbl:eval:vsZipper} shows the performance of \oursZipper. Thanks to \Zipper, \oursZipper~becomes faster than~\ours~at the small cost of the precision.
In h2, for example, \oursZipper~took 2,318 seconds while \ours~took 3,766 seconds.
One exception is batik, where \oursZipper~is slower than \ours. This is because 
\oursZipper~in this case loses precision significantly (reporting 180 more alarms than \ours), producing spurious points-to sets that make the analysis slow. 

%
%
%Note that context tunneling heuristics (e.g., our technique) and context sensitivity heuristics (\Zipper) are orthogonal; we can use both techniques to develop highly precise and scalable call-site sensitivity. For example, we can apply \Zipper~to improve the scalability of \ours. In \ours, all method calls are analyzed context sensitively (1-call-site sensitivity). 
%Applying context sensitivity to all the method calls, however, makes the analysis. 
%To mitigate the problem,
%we can use the context sensitivity heuristic \Zipper, which is applicable to various context flavors including call-site sensitivity~\cite{ZipperJournal20}. 
%Unfortunately, other existing context sensitivity heuristics~are usually inapplicable to \ours~because they are usually designed~\cite{Li2018b,Lu:2019:PYF} or learned~\cite{Graphick20} to optimize object sensitivity. 
%The context sensitivity heuristic \Zipper~identifies method calls that our call-site sensitivity (\ours) needs to analyzed context sensitively (1-call-site sensitivity). Otherwise, \ours~analyzes them context insensitively. 




%\begin{table}[]
%\label{tbl:oursZipper}
%\begin{tabular}{@{}lrrrrrrrrr@{}}
%\toprule
%\multicolumn{1}{c}{} & \multicolumn{1}{c}{h2} & \multicolumn{1}{c}{fop} & \multicolumn{1}{c}{\TODO{bloat}} & \multicolumn{1}{c}{sunflow} & \multicolumn{1}{c}{\TODO{bloat}} & \multicolumn{1}{c}{findbugs} & \multicolumn{1}{c}{chart} & \multicolumn{1}{c}{avrora09} & \multicolumn{1}{c}{xalan09} \\ \midrule
%\#may-fail cast      & 1,223                   & 1,384                    & 1,166                      & 1,818                        & 1,166                      & 1,352                         & 886                          &  951                     & 1,001                        \\
%Time (s)             & 2,414                   & 310                     & 421                       & 279                         & 421                       & 172                          & 71                          & 117                        & 169                         \\ \bottomrule
%\end{tabular}
%\end{table}

% Please add the following required packages to your document preamble:
% \usepackage{booktabs}



\begin{table}[t]
\setlength\extrarowheight{-1pt}
\centering
\caption{Precision and scalability comparison between 1-hybrid-object sensitivity with tunneling (\onesobjHT) against its simulated call-site sensitivity (\simonecallH) and the call-site sensitivity obtained from learning (\sobjSimLearn). 
%The columns Time(s) of \simonecallH~present sum of its baseline analysis cost(s) (e.g.,~\onesobjHT) and the simulated call-site sensitive analysis cost(s). 
%All the other notations are the same with Table~\ref{tbl:eval:main}.
}
\label{tbl:sobj}
\centering
\footnotesize

\begin{tabular}{@{} c l r r | c l r r @{}}
\toprule
 Program                  & \,  Analysis   & \multicolumn{1}{c}{\, \#fail-casts} & \multicolumn{1}{c | }{ Times(s)} & Program                     & \, Analysis   & \multicolumn{1}{c}{\, \#fail-casts} & \multicolumn{1}{c}{Time(s)} \\ \midrule
\multirow{3}{*}{luindex} & \sobjSimLearn & 366                             & 36                           & \multirow{3}{*}{lusearch}   & \sobjSimLearn & 380                              & 35                          \\
                         & \simonecallH  & 359                             & 67                           &                             & \simonecallH  & 373                              & 69                          \\
                         & \onesobjHT   & 371                             & 32                           &                             & \onesobjHT   & 380                              & 33                          \\\midrule
\multirow{3}{*}{antlr}   & \sobjSimLearn & 486                             & 54                           & \multirow{3}{*}{eclipse}    & \sobjSimLearn & 586                              & 40                          \\
                         & \simonecallH  & 479                             & 95                           &                             & \simonecallH  & 573                              & 77                          \\
                         & \onesobjHT   & 483                             & 48                           &                             & \onesobjHT   & 586                              & 36                          \\\midrule
\multirow{3}{*}{xalan}   & \sobjSimLearn & 576                             & 68                           & \multirow{3}{*}{chart}      & \sobjSimLearn & 942                              & 67                          \\
                         & \simonecallH  & 562                             & 123                          &                             & \simonecallH  & 861                              & 139                         \\
                         & \onesobjHT   & 572                             & 59                           &                             & \onesobjHT   & 876                              & 62                          \\\midrule
\multirow{3}{*}{bloat}   & \sobjSimLearn & 1,267                           & 456                          & \multirow{3}{*}{jython}     & \sobjSimLearn & 856                              & 194                         \\
                         & \simonecallH  & 1,248                           & 870                          &                             & \simonecallH  & 836                              & 8,985                      \\
                         & \onesobjHT   & 1,251                           & 375                          &                             & \onesobjHT   & 837                              & 342                         \\\midrule
\multirow{3}{*}{jpc}     & \sobjSimLearn & 1,676                           & 131                          & \multirow{3}{*}{pmd}        & \sobjSimLearn & 721                              & 59                          \\
                         & \simonecallH  & 1,644                           & 3,002                        &                             & \simonecallH  & 710                              & 50                          \\
                         & \onesobjHT   & 1,593                           & 186                          &                             & \onesobjHT   & 713                              & 52                          \\\midrule
\multirow{3}{*}{fop}     & \sobjSimLearn & 1,084                           & 123                          & \multirow{3}{*}{checkstyle} & \sobjSimLearn & 466                              & 63                          \\
                         & \simonecallH  & 1,055                           & 959                          &                             & \simonecallH  & 468                              & 135                         \\
                         & \onesobjHT   & 1,080                           & 94                           &                             & \onesobjHT   & 474                              & 69                          \\ \bottomrule
\end{tabular}
\vspace{-15pt}
\end{table}



\subsection{Applicability to Variations of Object Sensitivity}

In this chapter, we focused on the original object sensitivity~\cite{Milanova2002} as it is the most widely known and used (e.g.,~\cite{Li2018b,Li2018a,TanLX16,Smaragdakis2014,GordonKPGNR_NDSS15,Feng2014,Lu:2019:PYF,Li2018a}). 
Therefore, though we believe our claim holds for variations of object sensitivity as well (e.g., hybrid context sensitivity~\cite{KastrinisS13a}, type sensitivity~\cite{Smaragdakis2011}), we do not claim that \ourtechnique~in its present form is readily applicable to them. 
Note that we have designed \ourtechnique~by exploiting the properties of original object sensitivity (e.g., call-graph patterns and atomic features). To apply \ourtechnique~to its variations, we may need domain-specific tuning of the simulation and learning techniques (e.g., the inference rules in Section~\ref{sec:simulation} and feature engineering in Section~\ref{sec:learning}). 
It would be interesting future work to generalize our results for other variants of object sensitivity.
%Minseok
%\textcolor{red}{It would be interesting future work to generalize our results for other variants of object sensitivity. }


When we simply used \ourtechnique~to hybrid object sensitivity (in the setting of Section~\ref{sec:performance}), for example, we found that the results are encouraging yet suboptimal. 
Hybrid context sensitivity~\cite{KastrinisS13a} is a variant of object sensitivity that selectively combines call-site and object sensitivity, and the state-of-the-art is the context-tunneled version,  \onesobjHT~\cite{JeJeOh18}. 
We applied \ourtechnique~to \onesobjHT~and transformed it into a call-site-sensitive analysis. 
We compared the performance of \onesobjHT~with the simulated call-site sensitivity without learning (denoted \simonecallH), and the final analysis with learning (denoted \sobjSimLearn). 
Table~\ref{tbl:sobj} shows that, though suboptimal, our simulation technique overall makes hybrid context sensitivity more precise (\simonecallH~vs. \onesobjHT), indicating that hybrid context sensitivity can benefit from our approach. However, the learned analysis (\sobjSimLearn) shows overall worse precision than hybrid context sensitivity.  
The failure of learning comes from the lack of atomic features appropriate for hybrid object sensitivity. We leave adaptation to hybrid object sensitivity as future work. 


\section{Open Question: Is Complete Simulation Possible?}
In this thesis, we showed that it is practically possible to outperform object sensitivity via contexttunneled call-site sensitivity. However, it remains to be seen whether or not call-site sensitivity can
be fundamentally superior to object sensitivity. 
We discuss this in Appendix~\ref{sec:setting}

%\clearpage
%%% Local Variables:
%%% mode: latex
%%% TeX-master: "paper"
%%% End:

%% !TEX root = ./paper.tex
\newcolumntype{Y}{>{\centering\arraybackslash}X}
%\setcopyright{none}
%\usepackage {tikz}
\usetikzlibrary {shapes,positioning}
%\usepackage {bm}
\tikzstyle{block} = [rectangle, draw, fill=white, text width=3.8em,
, text
centered, rounded corners, minimum height=4em]
\tikzstyle{block2} =
[rectangle, draw, fill=white, text width=6em, text centered, rounded
corners, minimum height=4em, minimum width = 7em] \tikzstyle{line} = [draw, -latex']
\tikzstyle{onlyText} = [text width =2em, text centered]
%\setcopyright{rightsretained}

\tikzstyle{block3} =
[rectangle, draw, fill=white, text width=7em, text centered, rounded
corners, minimum height=3em] \tikzstyle{line} = [draw, -latex']

%\setcopyright{rightsretained}


\tikzstyle{blocks} = [rectangle, draw, fill=white, text width=0.7cm, text height = 4em, text
centered,  text width=4em, rounded corners, minimum height=0.6cm]

% \chapter{Open Question: Is Complete Simulation Possible?}\label{sec:counter_example}

% In this paper, we showed that it is practically possible to outperform object sensitivity via context-tunneled call-site sensitivity. 
% %, which supports our claim that call-site sensitivity is empirically superior to object sensitivity in a more general $k$-limited setting. 
% However, it remains to be seen whether or not call-site sensitivity can be fundamentally superior to object sensitivity. 

\myparagraph{Fundamental superiority}Let us define the notion of `superiority' as follows: 
\begin{definition}[Superiority of Call-Site Sensitivity]
Let $\mbp$ be a set of target programs. Let $\mbs$ be a context-tunneling space for the target programs. 
We say call-site sensitivity is superior to object sensitivity with respect to $\mbs$ if 
is always possible to simulate object sensitivity via call-site sensitivity: 
\begin{equation}\label{eq:sup-condition}
\forall P \in \mbp. \forall T_{\it obj} \in \mbs.  \exists T_{\it call} \in \mbs.\forall k \in [0,\infty]. 
\; 
\fix F^{T_{\it call}, U_{\it call}}_{P, k} \moreprecise~\mbox{(more precise than)}~
\fix F^{T_{\it obj}, U_{\it obj}}_{P, k}
%\call_k(P, T_{\it call})~\mbox{is more precise than}~\obj_k(P, T_{\it obj}).
\end{equation}
where $U_{\it call}$ and $U_{\it obj}$ are context-update functions (Eq.~(\ref{eq:objsens}) and (\ref{eq:cfa})) that are naturally given together with the tunneling space $\mbs$, and the precision order $\moreprecise$ is defined in terms of the context-insensitive points-to sets, as follows: 
 \[
   \fix F_{P,k}^{T, \updatectx} \moreprecise \fix
   F_{P,k}^{T', \updatectx'} \iff \forall x\in \Var_P.\; \project(\fix
   F_{P,k}^{T, \updatectx})(x) \subseteq \project(\fix
   F_{P,k}^{T', \updatectx'}) (x)
 \]
where $\project(X, Y, R,\callgraph) = \lambda x.\; \bigsqcup_{\ctx \in \Ctx}
 \myset{h \mid (h, \hctx) \in X(x, \ctx)}$.
\end{definition}
Then, the question is restated as follows: Is there a context-tunneling space $\mbs$ that makes the condition (\ref{eq:sup-condition}) true? 

%Formally answering the question would be challenging, although we conjecture that the answer might be true. Let us explain why with examples.




\myparagraph{Object Sensitivity is Not Fundamentally Superior}
We first show that object sensitivity is not fundamentally superior to call-site sensitivity. 
That is, it is not possible to find a tunneling
abstraction space $\mbs$ that makes $k$-object sensitivity with
tunneling always equal or more precise than $k$-call-site
sensitivity with tunneling. The example code in
Figure~\ref{background:example2} is a universal counter-example for all 
tunneling spaces $\mbs$. By definition of context tunneling
(i.e., selective update of contexts), the method
{\tt id} in the example (Figure~\ref{background:example2}) can have
 \texttt{[D]} (updating context) or \texttt{[$\cdot$]}
(inheriting context) as a context no matter what tunneling abstraction space is
used. Thus, $k$-object sensitivity with tunneling is unable to separate the
three method calls (invoked at the three different invocation sites)
with the two 
available context choices. In summary, we conclude that object sensitivity cannot simulate call-site sensitivity even in the generalized setting with context tunneling. 

%k-call-site sensitivity, however, can separate the
%three method calls with a tunneling abstraction $\emptyset \subseteq
%S$ (always update contexts) for all tunneling abstraction space
%$S$ of context tunneling. 



\myparagraph{The Case for Call-Site Sensitivity}
On the other hand, the situation for call-site sensitivity is nontrivial and it remains to be seen whether or not call-site sensitivity can simulate object sensitivity completely. 
%we conjecture that call-site sensitivity can be fundamentally superior to object sensitivity with a well-designed context tunneling space. 
Let us explain in more detail with examples.



\begin{figure}[t]
\begin{multicols}{2}
%\vfill\null
~\\\\\\\\
\begin{subfigure}[b]{1.2\columnwidth}
\begin{lstlisting}[xleftmargin=13pt,multicols=2,basicstyle={\fontsize{8.5}{10}\selectfont\ttfamily}]
class C {
 C f;
 void m(C v) {
  this.set(v);
  v.set(this);
 }
 void set(C v) {
  this.f = v;
 }
}
void main() {
 C c1 = new C();//C1
 C c2 = new C();//C2

 c1.m(c2);
 c2.m(c1);
 assert(c1.f != c2.f); 
}
\end{lstlisting}
\caption{Example code}
\label{fig:counterexample:code}
\end{subfigure}
\columnbreak~\\


\quad\begin{subfigure}[b]{1.0\columnwidth}
\begin{center}
	\resizebox{0.55\columnwidth}{!}{
		\begin{tikzpicture}
		\tikzstyle{every node}=[font=\LARGE]
		\node [block] (main) {{\tt main} \\ $[\cdot]$};
		\node [block, above right = -0.4cm and
		0.7cm of main] (Dm1) {\tt {m} \\$[$C1$]$};
		\node [block, below = 0.4cm of Dm1] (Dm2) {\tt{m} \\$[$C2$]$};
		\node [block, right = 0.9cm of Dm1] (Cm1) {\tt {set} \\$[$C1$]$};
		\node [block, right = 0.9cm of Dm2] (Cm2) {\tt {set} \\$[$C2$]$};
		
		\path [line] (main) edge node[above] {$15$} (Dm1);
		\path [line] (main) edge node[above] {$16$} (Dm2);
		\path [line] (Dm1) edge node[above] {$4$} (Cm1);
		\path [line] (Dm2) edge node[below] {$4$} (Cm2);
		\path [line] (Dm2) edge node[above] {$5$} (Cm1);
		\path [line] (Dm1) edge node[below] {$5$} (Cm2);
		
		\end{tikzpicture}
	}
\end{center}
\caption{Call-graph by 1-object sensitivity}
\label{fig:counterexample:obj}
\end{subfigure}~\\



\quad\begin{subfigure}[b]{1.0\columnwidth}

\begin{center}
	\resizebox{0.55\columnwidth}{!}{
		\begin{tikzpicture}
		\tikzstyle{every node}=[font=\LARGE]   
		\node [block] (main) {{\tt main} \\ $[\cdot]$};
		\node [block, above right = -0.4cm and
		0.7cm of main] (Dm1) {{\tt m} \\$[15]$};
		\node [block, below = 0.4cm of Dm1] (Dm2) {{\tt m} \\$[16]$};
		\node [block, right = 0.9cm of Dm1] (Cm1) {{\tt set} \\$[4]$};
		\node [block, right = 0.9cm of Dm2] (Cm2) {{\tt set} \\$[5]$};
		
		\path [line] (main) edge node[above] {$15$} (Dm1);
		\path [line] (main) edge node[above] {$16$} (Dm2);
		\path [line] (Dm1) edge node[above] {$4$} (Cm1);
		\path [line] (Dm2) edge node[below] {$5$} (Cm2);
		\path [line] (Dm2) edge node[above left= 0.15cm and 0.02cm] {$5$} (Cm1);
		\path [line] (Dm1) edge node[below left= 0.15cm and 0.02cm] {$4$} (Cm2);	
		\end{tikzpicture}
	}
\end{center}
\caption{Call-graph by 1-call-site sensitivity}
\label{fig:counterexample:call}
\end{subfigure}

\end{multicols}
\vspace{-15pt}
\caption{Example such that call-site sensitivity cannot
  simulate object sensitivity w.r.t. tunneling space $\mbs = \Invo$.}
\vspace{-10pt}
\label{fig:counterexample}
\end{figure}

First, we point out that the condition (\ref{eq:sup-condition}) does not hold w.r.t. the tunneling space considered in this chapter. 
In Section~\ref{sec:setting}, we defined the tunneling space $\mbs$ to be the set of invocation-sites, i.e., $\mbs = \Invo$. 
With this space, there exists a tricky counter-example program that context-tunneled
call-site sensitivity is unable to simulate object sensitivity.
Consider the program in Figure~\ref{fig:counterexample}. 
In the example code, class {\tt C} contains two methods, {\tt set} and {\tt m}.
Method {\tt set} is a setter that stores the parameter value in the
the field of the base object,
and method {\tt m} calls the {\tt set} method twice
with the receiver object and parameter being swapped.
The {\tt main} method creates two objects, {\tt C1} and {\tt C2}, and
stores them in {\tt c1} and {\tt c2}. Method {\tt m} is called on {\tt
  c1} with parameter {\tt c2} at line 15, and it is called again on
{\tt c2} with parameter {\tt c1} at line 16.
At line 17, an assertion asks if
the fields of {\tt c1} and {\tt c2} are not aliased.
In the real execution, the query holds because {\tt c1.f} points to {\tt C2} and
{\tt c2.f} points to {\tt C1}.

The conventional 1-object-sensitive analysis can prove the query but
1-call-site sensitivity cannot do so no matter what tunneling abstraction from the space $\mbs = \Invo$ is
chosen.
Figure~\ref{fig:counterexample:obj} presents
the call-graph of 1-object sensitivity.
The object-sensitive analysis is effectively producing the call-graph above as
each context of {\tt set} determines the value of each parameter.
When the context is {\tt [C1]}, the receiver object is {\tt C1} and the value of the parameter is {\tt C2}.
Otherwise, when the context is {\tt [C2]}, the receiver object and the
parameter value are  {\tt C2} and {\tt C1}, respectively.
On the other hand, conventional 1-call-site sensitivity constructs a similar but
different call-graph in Figure~\ref{fig:counterexample:call}.
Note that the two edges labeled 4 go
to the same method but they are heading to different methods in the call-graph of object sensitivity. 
Thus, it is not possible to simulate object sensitivity with the invocation-site-based context tunneling. 
%However,
%doing so still fails to simulate object sensitivity.


%The code pattern, however, is uncommon in practice.
%To make call-site sensitivity unable to simulate object sensitivity,
%the program should have at least two invocation-sites where a single method
%is called with different parameters and receiver objects.
%The parameters should come from the parameters of the caller methods, and the
%caller method should be called from at least two different invocations with
%different parameters. Satisfying all of these constraints at the same time
%would happen rarely in practice.


%\textcolor{red}{
%\footnote{
%We can define the tunneling abstraction in various ways.
%\citet{JeJeOh18} originally used (pairs of) methods as tunneling abstractions.
%In this paper, we consider more fine-grained program elements, i.e., invocation-sites. 
%} 

\begin{figure}[t]
	\begin{center}
			\resizebox{0.8\columnwidth}{!}{
					\begin{tikzpicture}
					\tikzstyle{every node}=[font=\LARGE]
					\node [block] (main) {{\tt main} \\ $[\cdot]$};
					\node [block, right =
					2.0cm of main] (Dm1) {{\tt m} \\$[15]$};
					\node [block, left = 1.9cm of main] (Dm2) {{\tt m} \\$[16]$};
					\node [block, below right = -0.3cm and 1.9cm of Dm1] (Cm1) {{\tt set} \\$[5]$};
					\node [block, below left = -0.3cm and 1.9cm of Dm2] (Cm2) {{\tt set} \\$[4]$};
	
					\node [block, above = 0.4cm of Cm1] (Cm3) {{\tt set} \\$[15]$};
					\node [block, above = 0.4cm of Cm2] (Cm4) {{\tt set} \\$[16]$};
	
	
					\path [line] (main) edge node[above= 0.15cm] {\large$(C1,15)$} (Dm1);
					\path [line] (main) edge node[above= 0.15cm] {\large$(C2,16)$} (Dm2);
					\path [line] (Dm1) edge node[above left= 0.15cm and -0.7cm] {\large $(C1,4)$} (Cm3);
					\path [line] (Dm2) edge node[above right= 0.15cm and -0.7cm] {\large $(C1,5)$} (Cm4);
					\path [line] (Dm2) edge node[below right = 0.15cm and -0.7cm] {\large $(C2,4)$} (Cm2);
					\path [line] (Dm1) edge node[below left = 0.15cm and -0.7cm] {\large $(C2,5)$} (Cm1);
					\end{tikzpicture}
			}
	\end{center}
	\caption{Call-graph with a fine-grained tunneling space}
	\label{fig:cg-finegrained}
	\vspace{-10pt}
	\end{figure}
However, this counter-example does not mean that it is fundamentally impossible to simulate object sensitivity via call-site sensitivity, as the counter-example becomes no longer valid if we use a more fine-grained tunneling abstraction. 
For example, suppose we define the tunneling space to be pairs of receiver objects and invocation-sites, i.e., $\mbs = \Heap \times \Invo$ (recall that choosing a tunneling abstraction does not affect the analysis soundness). 
With this tunneling space, a context-tunneled 1-call-site-sensitive analysis can now prove the query. 
Suppose we use a tunneling abstraction $T = \myset{(C1,4), (C1,5)}$, which means that we apply context tunneling only when the receiver object is $C1$ and the invocation-site is either $4$ or $5$. 1-call-site sensitivity with $T$ produces the call-graph in Figure~\ref{fig:cg-finegrained}. 
In the call-graph, $\texttt{m[15]} \stackrel{(C1,4)}{\to}
\texttt{set[15]}$ indicates that the caller ({\tt m}) and callee ({\tt
  set}) have the same context 15 where the callee method is called at
invocation-site 4 and its receiver object is $C1$. With this call-graph, we can prove the query 
in Figure~\ref{fig:counterexample:code} as it is strictly more precise than the call-graph produced by object sensitivity
(Figure~\ref{fig:counterexample:obj}). 



%Note that the above counter-example is not valid that 1-call-site
%sensitivity with tunneling can prove the query if we use a generalized
%abstraction space of context tunneling instead of invocation sites. if
%we use pairs of receiver objects and invocation sites as tunneling
%abstraction, 1-call-site-sensitive analysis with tunneling can prove
%the query in the example
%(Figure~\ref{fig:counterexample:call}). Suppose we use a
%tunneling abstraction $T = \{(C1,4),(C1,5)\}$ where it applied context
%tunneling when the receiver object and invocation site are $C1$ and
%$4$, respectively, or the object and the invocation site are $C2$
%and $5$, respectively. It produces
%the following context abstraction:



This way, we conjecture that it would be always possible to find a suitable context-tunneling space $\mbs$ that satisfies the condition (\ref{eq:sup-condition}) w.r.t. the given set of programs ($\mbp$). We leave an in-depth theoretical analysis as future work. 

%}




%%% Local Variables:
%%% mode: latex
%%% TeX-master: "paper"
%%% End:

%%


%the high-level idea of our approach could be adapted for m-CFA and we think it will be an interesting direction for extending our work.
%Our current simulation technique relies on specific properties of k-CFA (e.g., I_2I 2​	
%leverages a unique property of context-tunneled k-CFA). Therefore applying our technique as it is may not be effective for m-CFA [2]. 

